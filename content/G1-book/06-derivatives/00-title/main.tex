\mainChapter{Differenciálszámítás}

\begin{tikzpicture}[
    remember picture,
    overlay,
    thick,
    shift={($(current page.south west) + (\innerMargin+7.5mm, 40mm)$)},
  ]

  % Coordinate system
  \draw[-to, draw=gray, ultra thick, name path = x] (-3.5,0) -- ++(\paperwidth+2cm,0);
  \draw[-to, draw=gray, ultra thick] (-2,-4) -- ++(0,\paperheight-1cm);

  \draw[
    smooth,
    samples = 250,
    domain = -2.5:15.5,
    red!40!black,
    ultra thick,
    name path = f,
  ] plot ({\x*1.2}, {0.0019 * \x^3 + 0.1587 * \x^2 - 2.0811 * \x + 4.3846});

  % \draw[secondaryColor, ultra thick]
  % (9.6,-1.1346) coordinate(A) -- ++(10,8.229)
  % (A) -- ++(-10,-8.229)
  % ;

  % \draw[secondaryColor, ultra thick]
  % (10.8,-0.1055) coordinate(A) -- ++(10,12.372)
  % (A) -- ++(-10,-12.372)
  % ;

  % \draw[secondaryColor, ultra thick]
  % (7.8,-1.91569) coordinate(A) -- ++(10,2.22825)
  % (A) -- ++(-10,-2.22825)
  % ;

  \coordinate (D) at (8.4,-1.7551);
  \coordinate (l1) at ($(D)+(7.5,3.15)$);
  \coordinate (l2) at ($(D)-(5.0,2.10)$);

  \draw[secondaryColor, ultra thick, to-to] (l1) -- (l2);

  \coordinate (t1) at ($(D)+(2.5,0.00)$);
  \coordinate (t2) at ($(D)+(2.5,1.05)$);

  \draw[fill=primaryColor] (D) circle (0.1);
  \draw[dashed, draw=gray] (D)
  -- (t1) node[midway, below] {$1$}
  -- (t2) node[midway, right] {$m = \tan \alpha$}
  ;

  \draw[ultra thick, primaryColor, -to] ($(D)+(1.75,0)$) arc
    [start angle=0, end angle=22.78, radius=1.75cm];

  \node[shift={(1.25,0.2)}] at (D) {$\alpha$};

\end{tikzpicture}

\bgroup
\color{gray!50!black}
\sffamily

A differenciálszámítás a matematika azon területe, amely a változások
vizsgálatával foglalkozik. A természet, a mérnöki, a gazdasági tudományok számos
jelenségének megértéséhez elengedhetetlen eszköz, hiszen lehetővé teszi
számunkra, hogy leírjuk az ott zajló folyamatokat. Alapjait Isaac Newton és
Gottfried Wilhelm Leibniz fektették le a 17. században.

Ebben a fejezetben a differenciálszámítás alapfogalmaival, mint a derivált és a
differenciál fogunk megismerkedni. Vizsgáljuk a differenciálhatóság feltételeit,
bebizonyítjuk a differenciálszámítás alapvető tételeit, mint a Rolle-, a
Lagrange- és a Cauchy-féle középértéktétel, a deriválási szabályok, beleértve a
láncszabályt és az összetett függvények, az inverz függvény deriválását is.
Elemezzük a függvények deriváltjainak geometriai és fizikai jelentését és
megtanuljuk, hogyan alkalmazhatjuk ezeket a fogalmakat a gyakorlatban.

A differenciálszámítás lehetőséget teremt a függvények jellemzésére,
segítségével a korábbi fejezetekben már tárgyalt jellemzők (például monotonitás,
konvexitás) könnyen vizsgálhatók.

A derivált segítségével megérthetjük a sebesség, a gyorsulás, a növekedési ráta
és más változási sebességek matematikai modelljét. A differenciálszámítás a
modern tudomány és mérnöki munka alapvető eszköze, a gazdaságtól kezdve a
fizikán át az informatikáig számos területen alkalmazzák. Megértése nemcsak a
matematikai elmélet megértéséhez vezet el, hanem a valós világ problémáinak
megoldásához is hozzájárul.

% \chaptertoc

% Manual toc to handle the graph
\section*{A fejezetben érintett témakörök}

\hspace{0.0cm}\parbox{14.8cm}{\contentsline {section}{\numberline {6.1}\nameref{section.6.1}}{\pageref*{section.6.1}}{section.6.1}}\par
\hspace{0.3cm}\parbox{14.3cm}{\contentsline {section}{\numberline {6.2}\nameref{section.6.2}}{\pageref*{section.6.2}}{section.6.2}}\par
\hspace{0.6cm}\parbox{13.7cm}{\contentsline {section}{\numberline {6.3}\nameref{section.6.3}}{\pageref*{section.6.3}}{section.6.3}}\par
\hspace{0.8cm}\parbox{13.1cm}{\contentsline {section}{\numberline {6.4}\nameref{section.6.4}}{\pageref*{section.6.4}}{section.6.4}}\par
\hspace{1.0cm}\parbox{12.5cm}{\contentsline {section}{\numberline {6.5}\nameref{section.6.5}}{\pageref*{section.6.5}}{section.6.5}}\par

\egroup

\clearpage
