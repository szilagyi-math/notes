\section{Középértéktételek}\label{section.6.3}

\begin{theorem}[Rolle-tétel]
  Legyen $f$ folytonos $[a; b]$ intervallumon és differenciálható $(a; b)$
  intervallumon, továbbá $f(a) = f(b) = 0$, ekkor létezik $ \xi \in (a; b)$,
  melyre teljesül, hogy
  \[
    f'(\xi) = 0
    \text.
  \]

  \begin{proof}
    \phantom{alma}\vspace{7cm}\phantom{alma}
  \end{proof}
\end{theorem}

\begin{theorem}[Lagrange-féle középértéktétel]
  Legyen $f : I \subset R \to R$ folytonos $[a; b]$ intervallumon és
  differenciálható $(a; b)$ intervallumon, ekkor létezik olyan $\delta \in
    (a; b)$ hogy
  \[
    f'(\delta) = \frac{f(b)-f(a)}{b-a}
    \text.
  \]

  \begin{proof}
    \phantom{alma}\vspace{12cm}\phantom{alma}
  \end{proof}
\end{theorem}

\begin{theorem}
  Ha $f$ folytonos $[a; b]$ intervallumon és differenciálható $(a; b)$
  intervallumon, továbbá $\forall x \in (a; b)$ esetén $f'(x) = 0$, akkor
  $f(x) = c$ (a függvény konstans).
\end{theorem}

\begin{theorem}
  Ha $f$ és $g$ folytonos az $[a; b]$ intervallumon és differenciálható a
  $(a; b)$ intervallumon, továbbá $f'(x) = g'(x) \; \forall x\in (a; b)$-re,
  akkor
  \[
    g(x) = f(x) +c
    \text.
  \]
\end{theorem}

\begin{theorem}[Cauchy-féle középértéktétel]
  Legyen $f$ és $g$ függvények folytonosak $[a; b]$ intervallumon és
  differenciálhatóak $(a; b)$ intervallumon, valamint tegyük fel, hogy $g'(x)
    \neq 0$ bármely $x \in (a; b)$ esetén. Ekkor létezik olyan $\eta \in
    (a; b)$
  hogy
  \[
    \frac{f(b)-f(a)}{g(b)-g(a)} = \frac{f'(\eta)}{g'(\eta)}
    \text.
  \]

  \begin{proof}
    \phantom{alma}\vspace{14cm}\phantom{alma}
  \end{proof}
\end{theorem}

\begin{note}
  $g(x) = x$ választással speciális esetként visszakapjuk a Lagrange-féle
  középértéktételt.
\end{note}

\begin{note}
  A Cauchy-féle középértéktétel nem ekvivalens azzal, hogy külön-külön az
  $f$ és $g$ függvényekre alkalmazzuk a Lagrange-féle középértéktételt,
  majd vesszük ezek hányadosát.
\end{note}