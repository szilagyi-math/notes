\mainChapter{Halmazelmélet}\label{chap-01}

\begin{tikzpicture}[
    remember picture,
    overlay,
    ultra thick,
    shift={($(current page.south west) + (\innerMargin+1.25mm, 26.5mm)$)},
  ]
  % TODO: Do it with ifs
  \foreach \x in {-1,3} {
      \foreach \y in {-1,0,1,2,3,4} {
          \begin{scope}[shift={(\x*6cm,\y*6cm)}]
            % Red: A * B
            \draw[primaryColor] (0,0) circle (.75cm);
            \draw[primaryColor] (0.75,0.5) circle (.75cm);
            \clip[] (0,0) circle (.75cm);
            \fill[primaryColor!50] (0.75,0.5) circle (0.75cm);

            % TODO: Use a more clever option here
            \draw[primaryColor] (0,0) circle (.75cm);
            \draw[primaryColor] (0.75,0.5) circle (.75cm);
          \end{scope}

          \begin{scope}[shift={(\x*6cm + 3cm,\y*6cm)}]
            % Blue: B - A
            \draw[secondaryColor, fill=secondaryColor!50] (0.75,0.5) circle (.75cm);
            \draw[secondaryColor] (0,0) circle (.75cm);
          \end{scope}

          \begin{scope}[shift={(\x*6cm,\y*6cm + 3cm)}]
            % Yellow: A - B
            \draw[ternaryColor, fill=ternaryColor!50] (0,0) circle (.75cm);
            \draw[ternaryColor] (0.75,0.5) circle (.75cm);
          \end{scope}

          \begin{scope}[shift={(\x*6cm + 3cm,\y*6cm + 3cm)}]
            % Green: (A - B) + (B - A)
            \draw[green!40!black, fill=green!40!black!50] (0,0) circle (.75cm);
            \draw[green!40!black, fill=green!40!black!50] (0.75,0.5) circle (.75cm);
            \clip (0,0) circle (.75cm);
            \fill[white] (0.75,0.5) circle (0.75cm);
          \end{scope}

          % TODO: Use a more clever option here
          \begin{scope}[shift={(\x*6cm + 3cm,\y*6cm + 3cm)}]
            \draw[green!40!black] (0,0) circle (.75cm);
            \draw[green!40!black] (0.75,0.5) circle (.75cm);
          \end{scope}
        }
    }
  \foreach \x in {-1,0,1,2,3} {
      \foreach \y in {-1,0,4} {
          \begin{scope}[shift={(\x*6cm,\y*6cm)}]
            % Red: A * B
            \draw[primaryColor] (0,0) circle (.75cm);
            \draw[primaryColor] (0.75,0.5) circle (.75cm);
            \clip[] (0,0) circle (.75cm);
            \fill[primaryColor!50] (0.75,0.5) circle (0.75cm);

            % TODO: Use a more clever option here
            \draw[primaryColor] (0,0) circle (.75cm);
            \draw[primaryColor] (0.75,0.5) circle (.75cm);
          \end{scope}

          \begin{scope}[shift={(\x*6cm + 3cm,\y*6cm)}]
            % Blue: B - A
            \draw[secondaryColor, fill=secondaryColor!50] (0.75,0.5) circle (.75cm);
            \draw[secondaryColor] (0,0) circle (.75cm);
          \end{scope}

          \begin{scope}[shift={(\x*6cm,\y*6cm + 3cm)}]
            % Yellow: A - B
            \draw[ternaryColor, fill=ternaryColor!50] (0,0) circle (.75cm);
            \draw[ternaryColor] (0.75,0.5) circle (.75cm);
          \end{scope}

          \begin{scope}[shift={(\x*6cm + 3cm,\y*6cm + 3cm)}]
            % Green: (A - B) + (B - A)
            \draw[green!40!black, fill=green!40!black!50] (0,0) circle (.75cm);
            \draw[green!40!black, fill=green!40!black!50] (0.75,0.5) circle (.75cm);
            \clip (0,0) circle (.75cm);
            \fill[white] (0.75,0.5) circle (0.75cm);
          \end{scope}

          % TODO: Use a more clever option here
          \begin{scope}[shift={(\x*6cm + 3cm,\y*6cm + 3cm)}]
            \draw[green!40!black] (0,0) circle (.75cm);
            \draw[green!40!black] (0.75,0.5) circle (.75cm);
          \end{scope}
        }
    }
  \foreach \x in {-1,0,1,2,3} {
      \foreach \y in {1} {
          \begin{scope}[shift={(\x*6cm,\y*6cm)}]
            % Red: A * B
            \draw[primaryColor] (0,0) circle (.75cm);
            \draw[primaryColor] (0.75,0.5) circle (.75cm);
            \clip[] (0,0) circle (.75cm);
            \fill[primaryColor!50] (0.75,0.5) circle (0.75cm);

            % TODO: Use a more clever option here
            \draw[primaryColor] (0,0) circle (.75cm);
            \draw[primaryColor] (0.75,0.5) circle (.75cm);
          \end{scope}

          \begin{scope}[shift={(\x*6cm + 3cm,\y*6cm)}]
            % Blue: B - A
            \draw[secondaryColor, fill=secondaryColor!50] (0.75,0.5) circle (.75cm);
            \draw[secondaryColor] (0,0) circle (.75cm);
          \end{scope}
        }
    }
  % \fill[white] (-10mm,23) rectangle ++ (\textwidth+20mm, -12);
\end{tikzpicture}

\bgroup
\color{gray!50!black}
\sffamily
Ebben a fejezetben a halmazelmélet alapfogalmaival ismerkedünk meg, áttekintjük,
rendszerezzük és néhol kibővítjük mindazt, amit eddig a számokról középiskolában
tanultunk.

A halmazok olyan objektumok gyűjteményei, amelyek egy közös tulajdonság vagy
szabály alapján határozhatóak meg. A halmazelmélet lényegében a köztük lévő
kapcsolatokkal foglalkozik és a matematikai érvelés sarokköveként szolgál,
keretet adva a matematikai objektumok rendszerezéséhez és elemzéséhez.

Tanulmányozni fogjuk a számok különböző típusait: a természetes számokat, az
egész számokat és a valós számokat, valamint azt, hogy ezek hogyan viszonyulnak
egymáshoz.

A fejezetben olyan definíciók, tételek kerülnek ismertetésre, amelyek
elengedhetetlenek a további matematikai tanulmányokhoz.

\chaptertoc
\egroup

\clearpage