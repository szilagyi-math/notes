\section{Fogalmak, definíciók}

\begin{definition}[Numerikus sor]
  $!(a_n) : \mathbb{N} \rightarrow \Reals$ numerikus sorozat, amelyből képezzük
  az alábbi sorozatot:
  \begin{align*}
    s_1 & = a_1                                          \\
    s_2 & = a_1 + a_2                                    \\
        & \vdots                                         \\
    s_n & = a_1 + a_2 + \dots + a_n = \sum_{i=1}^{n} a_i \\
  \end{align*}
  Az így képzett $(s_n)$-t az
  $(a_n)$ sorozatból képzett numerikus sornak mondjuk. Jele:
  \[
    s_n = \sum a_n
    \text,
  \]
  ahol $a_n$ a sor $n$-edik/általános tagja, $s_n$ pedig a sor $n$-edik
  részletösszege.

  Azt mondjuk, hogy a $\sum a_n$ sor konvergens, ha az $s_n$ sorozat konvergens,
  továbbá $\sum a_n$ sor divergens, ha $s_n$ sorozat divergens.

  Az $s_n$ sorozat határértékét a $\sum a_n$ sor összegének hívjuk:
  \[
    \lim_{n \rightarrow \infty} s_n = \lim_{n \rightarrow \infty}
    \sum_{i=1}^n a_i=\sum_{i=1}^\infty a_i
    \text.
  \]
\end{definition}

\begin{theorem}[A numerikus sor konvergenciájának szükséges feltétele]
  Ha a $\sum a_n$ numerikus sor konvergens, akkor az $(a_n)$ nullsorozat, azaz
  $\lim_{n\rightarrow\infty} a_n=0$, sőt $a_n+a_{n+1}+...+a_{n+p} \rightarrow
    0$, ha $n\rightarrow \infty$ (ekkor $p\in\mathbb{N}$ rögzített).
\end{theorem}

\begin{note}
  A $\sum_0 a q^n$ végtelen geometriai sor konvergens $\Leftrightarrow$
  ha $|q|<1$, ekkor a sorösszeg $a\cdot \frac{1}{1-q}$.
\end{note}

\begin{theorem}[A numerikus sor konvergenciájának elégséges feltétele]
  $\sum a_n$ numerikus sor akkor és csak akkor konvergens ha $\forall
    \varepsilon > 0 $ esetén
  \[
    \exists N(\varepsilon): |a_{n + 1} + a_{n + 2} + \dots + a_m| < \varepsilon
    \text,
  \]
  ha $n, m > N(\varepsilon)$ és $m > n$.
\end{theorem}

\begin{note}
  Véges sok tag elhagyása/megváltoztatása a sorozatban a konvergenciát nem
  változtatja meg, de a sorösszeget igen.
\end{note}

\begin{note}
  Legyen $\sum a_n$ és $\sum b_n$ konvergens numerikus sor, ekkor $\sum (a_n +
    b_n)$ is konvergens és
  \vspace{-.75em}
  \[
    \sum_{n=1}^\infty (a_n + b_n)
    = \sum_{n=1}^\infty a_n + \sum_{n=1}^\infty b_n
  \]
\end{note}

\begin{note}
  Legyen $\sum a_n$ konvergens numerikus sor és $\lambda$ valós szám, ekkor
  \vspace{-.75em}
  \[
    \sum_{n=1}^\infty \lambda a_n
    = \lambda \sum_{n=1}^\infty a_n
    \text.
  \]
\end{note}

\begin{theorem}[Csoportosított sor konvergenciája]
  Ha $\sum a_n$ konvergens, úgy bármely csoportosított sora is konvergens és a
  két sor összege megegyezik.
\end{theorem}

\begin{note}
  A tétel visszafelé is igaz.
\end{note}

\begin{definition}[Sor abszolút konvergencia]
  A $\sum a_n$-t abszolút konvergensnek hívjuk, ha $\sum |a_n|$ konvergens.
\end{definition}

\begin{note}
  Ha egy nemnegatív tagú sor konvergens, akkor abszolút konvergens. Ha a
  $\sum a_n$  sor konvergens, de nem abszolút konvergens, akkor feltételes
  konvergenciáról beszélünk.
\end{note}

\begin{theorem}[Feltételes konvergencia]
  Abszolút konvergens sor feltételesen is konvergens.
\end{theorem}

\begin{note}
  Az állítás visszafelé nem igaz.
\end{note}

\begin{theorem}[Riemann-tétel]
  Legyen $\sum a_n$ feltételesen konvergens, de nem abszolút konvergens
  numerikus sor és legyen $\alpha$ egy tetszőleges bővített valós szám, ekkor
  $\sum a_n$-nek van olyan átrendezése, hogy az átrendezett sor összege éppen
  $\alpha$.
\end{theorem}

\begin{theorem}[Abszolút konvergens sor átrendezése]
  Abszolút konvergens sor bármely átrendezett sora is abszolút konvergens és a
  sorösszeg azonos.
\end{theorem}

\begin{theorem}[Majoráns (felülről becsül) és minoráns (alulról becsül)
    kritérium]
  Legyenek $\sum a_n$ és $\sum b_n$ nemnegatív tagú sorok, melyekre az $a_n <
    b_n : \forall n \in \mathbb{N}$-re vagy $n_0 < n$ esetén:
  \begin{enumerate}
    \item ha $\sum a_n$ divergens, akkor $\sum b_n$ is az (minoráns kritérium),
    \item ha $\sum b_n$ konvergens, akkor $\sum a_n$ is az (majoráns kritérium).
  \end{enumerate}
\end{theorem}

\begin{theorem}[A hányados vagy D'Alambert-teszt]
  $\sum a_n$ egy pozitív tagú numerikus sor, ha $\exists 0 \leq q < 1$ valós
  szám, hogy $\frac{a_{n+1}}{a_n} \leq q$, ha $n>n_0$ vagy $\forall n$ esetén,
  akkor a $\sum a_n$ konvergens. %(ha 1, akkor nem alkalmazható)
\end{theorem}

\begin{theorem}[Gyök/Cauchy-teszt]
  Legyen $\sum a_n$ egy nemnegatív tagú sor, ha $\exists 0 \leq q < 1$,
  hogy $\sqrt[n]{a_n} \leq q$, ha $n>n_0$ vagy $\forall n$-re, akkor
  $\sum a_n$ konvergens.
\end{theorem}

\begin{note}
  Vegyük észre, hogy a majoráns illetve minoráns kritérium és az előző két
  teszt az abszolút konvergencia eldöntésére szolgál, a feltételes
  konvergenciáról nem ad információt.
\end{note}

\begin{theorem}[Integrál kritérium]
  Ha $x \geq 1$ esetén az $f$ függvény folytonos, nemnegatív és csökkenő, akkor
  a $\sum |f_n|$ numerikus sor konvergens vagy divergens aszerint, hogy
  \[
    \int_1^\infty f(x) \dd x
    \text{ konvergens vagy divergens.}
  \]
\end{theorem}

\begin{definition}[Alternáló sor]
  A $\sum (-1)^{n+1} \cdot b_n$, $b_n > 0$ numerikus sort alternáló sornak
  nevezzük.
\end{definition}

\begin{theorem}[Leibniz sor]
  A $\sum  (-1)^{n+1} \cdot b_n$ alternáló numerikus sor konvergens akkor és
  csakis akkor, ha $(b_n)$ monoton csökkenő nullsorozat, ekkor az $|s - s_n|
    \leq b_{n + 1}$.
\end{theorem}

\begin{definition}[TODO]
  Legyenek $(a_n)$ és $(b_n)$ a nemnegatív egészek halmazán értelmezett
  numerikus sorozatok és
  \vspace{-.75em}
  \[
    c_n:=\sum_{k=0}^n a_k b_{n-k}
    \text,
  \]
  ekkor a $\sum c_n$-t a $\sum a_n$ és $\sum b_n$ numerikus sorok Cauchy-féle
  szorzatsorának hívjuk.
\end{definition}

\begin{theorem}[Cauchy-féle szorzatsorok konvergenciája I]
  Abszolút konvergens sorok Cauchy-féle szorzatsora is abszolút konvergens és
  a szorzatsor összege a tényezősorok összegének szorzata.
\end{theorem}

\begin{theorem}[Cauchy-féle szorzatsorok konvergenciája II]
  Tegyük fel, hogy a $\sum a_n$ abszolút konvergens és sorösszege $A$ és a
  $\sum b_n$ feltételesen konvergens és a sorösszege $B$, ekkor a Cauchy-féle
  szorzatsorok sorösszege $A\cdot B$.
\end{theorem}

% Numerikus sor konvergenciájának vizsgálata:

% 0. lépés: $a_n \rightarrow 0$, ha $n \rightarrow \infty$, ha nem teljesül akkor
% a sor divergens, ha teljesül akkor vagy nevezetes sor, vagy alkalmazzuk a
% teszteket

% % TODO: THIS IS UGLY
% 1. lépés: nevezetes sorok:
% \begin{itemize}
%   \item végtelen sor $\sum a q^n$, ha $|q|<1 \frac{a}{1-q}$,
%   \item alternáló sor $\sum (-1)^{n+1}$.
% \end{itemize}

% % TODO: THIS IS VERY UGLY
% 2. lépés: tesztek, kritériumok: majoráns, minoráns, hányados, gyök, integrál

\begin{theorem}[TODO]
  Legyen $(a_n)$ monoton csökkenő nemnegatív tagú sorozat, a belőle képzett
  numerikus sor akkor és csak akkor konvergens, ha
  \[
    \sum_{k = 0} 2^k a_{2k} =
    a_1 + 2 \cdot a_2 + 2^2 \cdot a_4 + \dots
  \]
  is konvergens.
\end{theorem}

\begin{statement}
  $\displaystyle \sum \dfrac{1}{n^\alpha}$ konvergens, ha $\alpha > 1$ és
  divergens, ha $\alpha \leq 1$.
\end{statement}

\begin{statement}
  $\displaystyle \sum \dfrac{1}{n \cdot (\ln n)^P}$ konvergens, ha $P>1$ és
  divergens, ha $P \leq 1$.
\end{statement}