\section{Alapfogalmak}

\begin{note}
  Középiskolában megismert fogalmak: paritás, korlátosság, monotonitás,
  periodikusság, konvexitás, invertálhatóság, függvénytranszformációk.
\end{note}

\begin{definition}[Függvénygörbe húrja]
  Az $f$ függvény görbéjének $(x_1, f(x_1))$ $(x_2, f(x_2))$ pontjait összekötő
  szakaszt a függvénygörbe húrjának nevezzük, melynek egyenlete:
  \[
    h(x) = f(x_1) + \frac{f(x_2) - f(x_1)}{x_2 - x_1} \cdot (x - x_1)
    \text{, ahol $x_1 \leq x \leq x_2$.}
  \]
\end{definition}

\begin{blueBox}
  \sftitle{Függvénykompozíciók monotonitása:}
  \begin{itemize}
    \item $f$ és $g$ azonos monotonitású: $f \circ g$ monoton növő,
    \item $f$ és $g$ ellentétes monotonitású: $f \circ g$ monoton csökkenő.
  \end{itemize}
\end{blueBox}