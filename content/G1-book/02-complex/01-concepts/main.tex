\section{Fogalmak, definíciók}

\begin{blueBox}
  Tekintsük a következő egyenletrendszert:
  \[
    \begin{cases}
      x + y = -10 \\
      x \cdot y = 40
    \end{cases}
  \]

  Az egyenletrendszert megoldva az alábbi megoldásokat kapjuk:
  \[
    \begin{cases}
      x = -5 + \sqrt{-15} \\
      y = -5 - \sqrt{-15}
    \end{cases}
    \quad
    \begin{cases}
      x = -5 - \sqrt{-15} \\
      y = -5 + \sqrt{-15}
    \end{cases}
  \]

  Mit jelent, ha a négyzetgyökjel alatt negatív számot kapunk?

  Bővítsük a valós számok halmazát! Legyen $\iu^2 = -1$.

  A komplex számokat az úgynevezett Gauss-számsíkon ábrázoljuk.

  \begin{center}
    \begin{tikzpicture}[thick, scale=0.75]
      \draw[->] (-.5, 0) -- (3, 0) node[right] {$\Re$};
      \draw[->] (0, -.5) -- (0, 3) node[above] {$\Im$};

      \draw[draw=gray, dashed]
      (1.5,0) node[below] {$a$}
      -- (1.5, 2.5)
      -- (0, 2.5) node[left] {$b$};

      \draw[fill=primaryColor] (1.5, 2.5) circle (0.1)
      node[above right] {$z = a + b\iu$};
    \end{tikzpicture}
  \end{center}

  A komplex számok halmazának jele: $\mathbb C$.

  \textbf{Algebrai alak}: $z = a + b\iu$, ahol $a, b \in \Reals$.

  A komplex szám \textbf{valós része}: $\iRe z = \{a\}$.

  A komplex szám \textbf{képzetes része} pedig $\iIm \{z\} = b$.
\end{blueBox}

\begin{note}
  A komplex számok halmaza és a valós számok halmazának önmagával vett
  Descartes-szorzata között kölcsönösen egyértelmű megfeleltetés van, vagyis:
  $\mathbb C \cong \Reals \times \Reals$.
\end{note}

\begin{definition}[Két komplex szám egyenlősége]
  Legyenek $z_1 = a_1 + b_1\iu$ és $z_2 = a_2 + b_2\iu$ komplex számok. Ekkor:
  \[
    z_1 = z_2
    \quad \Leftrightarrow \quad
    a_1 = a_2 \quad \text{és} \quad b_1 = b_2
    \text.
  \]
\end{definition}

\begin{definition}[Komplex számok összege]
  Legyenek $z_1 = a_1 + b_1\iu$ és $z_2 = a_2 + b_2\iu$ komplex számok. Ekkor:
  \[
    z_1 + z_2 = (a_1 + a_2) + (b_1 + b_2)\iu
    \text.
  \]
\end{definition}

\begin{definition}[Konjugált]
  Legyen $z = a + b\iu$ egy komplex szám. Ekkor $z$ konjugáltja:
  \[
    \overline{z} = a - b\iu
    \text.
  \]
\end{definition}

\begin{note}
  Komplex szám és konjugáltjának összege valós szám:
  $z + \overline{z} = 2 a \in \Reals$.
  % \end{note}

  % \begin{note}
  Komplex szám és konjugáltjának szorzata valós szám:
  $z \cdot \overline{z} = a^2 + b^2 \in \Reals$.
\end{note}

\begin{definition}[Két komplex szám szorzata]
  \vspace{-1.5em}
  \begin{align*}
    z_1 \cdot z_2
     & = (a_1 + b_1\iu) \cdot (a_2 + b_2\iu)              \\
     & = a_1 a_2 + a_1 b_2\iu + a_2 b_1\iu + b_1 b_2\iu^2 \\
     & = (a_1 a_2 - b_1 b_2) + (a_1 b_2 + a_2 b_1)\iu
  \end{align*}
\end{definition}

\begin{blueBox}
  \sftitle{Áttérés a polárkoordináta-rendszerre:}

  Általában a Descartes-féle koordinátarendszerben dolgozunk, ahol a sík pontjai
  és a számpárok között kölcsönösen egyértelmű megfeleltetés van. Időnként
  azonban célravezető más koordinátarendszerek alkalmazása is.

  \begin{center}
    \begin{tikzpicture}[scale=3/4]
      % Draw the lines at multiples of pi/12
      \foreach \ang in {0,...,31} {
          \draw [lightgray] (0,0) -- (\ang * 180 / 16:4.1);
        }

      % Concentric circles
      \foreach \s in {0, 1, 2, 3} {
          \draw [lightgray] (0,0) circle (\s + 0.5);
          \draw [primaryColor] (0,0) circle (\s);
        }

      % Add the labels at multiples of pi/4
      \foreach \ang/\lab/\dir in {
      % 0/0/right,
      1/{\pi/4}/{above right},
      2/{\pi/2}/above,
      3/{3\pi/4}/{above left},
      4/{\pi}/left,
      5/{5\pi/4}/{below left},
      7/{7\pi/4}/{below right},
      6/{3\pi/2}/below} {
      \draw (0,0) -- (\ang * 180 / 4:4.5);
      \node [fill=cyan!10] at (\ang * 180 / 4:4.35) [\dir] {$\lab$};
      }

      % Outer circle
      \draw (0,0) circle (4);

      % 0 angle line
      \draw [-to, ultra thick] (0,0) -- (6,0) node [above left] {$r$};
      \node [above right] at (4,0) {$0$};

      % Radius labels
      \foreach \lab in {0, 1, 2, 3, 4} {
          \node [fill=cyan!10, inner sep=2pt, outer sep=2pt] at (\lab, 0)
          [below right] {\lab};
        }

      % Example
      \draw[secondaryColor, ultra thick] (0,0) -- (60:3);
      \draw[secondaryColor, thick] (0,0) -- ++(150:.5);
      \draw[secondaryColor, thick] (60:3) -- ++(150:.5);

      \draw[draw=primaryColor, ultra thick, to-to] (150:.35) -- ++(60:3)
      node[midway, above, fill=cyan!10, rotate=60, inner sep=2pt, outer sep=2pt]
      {$r$};

      \draw[ultra thick, primaryColor, -to] (1.4,0) arc
        [start angle=0, end angle=60, radius=1.4cm];

      \node[fill=cyan!10, inner sep=2pt, outer sep=2pt] at (30:.85) {$\varphi$};

      \draw[fill=primaryColor] (60:3) circle (0.1)
      node[below right, fill=cyan!10, inner sep=2pt, outer sep=5pt]
        {$z = r(\cos \varphi + \iu \sin \varphi)$};
    \end{tikzpicture}
  \end{center}

  Ebben az esetben a komplex szám szögfüggvények segítségével fejezhető ki:
  \[
    z = r (\cos \varphi + \iu \sin \varphi)
    \text.
  \]

  Az algebrai és trigonometrikus alak közötti kapcsolat:
  \[
    \begin{cases}
      a = r \cos \varphi \\
      b = r \sin \varphi
    \end{cases}
    \hspace{3cm}
    \begin{cases}
      r = \sqrt{a^2 + b^2} \\
      \varphi = \arctan(b / a)
    \end{cases}
  \]

  Ezek alapján $z_1 = r_1 (\cos \varphi_1 + \iu \sin \varphi_1)$ és $z_2 = r_2 (
    \cos \varphi_2 + \iu \sin \varphi_2)$ komplex számok szorzata trigonometrikus
  azonosságok segítségével:
  \begin{align*}
    z_1 \cdot z_2
     & = r_1 (\cos \varphi_1 + \iu \sin \varphi_1) \cdot
    r_2 (\cos \varphi_2 + \iu \sin \varphi_2)
    \\
     & = r_1 r_2 (\cos \varphi_1 \cos \varphi_2 - \sin \varphi_1 \sin \varphi_2) +
    r_1 r_2 (\cos \varphi_1 \sin \varphi_2 + \sin \varphi_1 \cos \varphi_2) \iu
    \\
     & = r_1 r_2 \left(
    \cos(\varphi_1 + \varphi_2) + \iu \sin(\varphi_1 + \varphi_2)
    \right)
    \text.
  \end{align*}

  A felhasznált azonosságok:
  \begin{align*}
    \sin(\alpha + \beta) & = \cos \alpha \sin \beta + \sin \alpha \cos \beta
    \text,                                                                   \\
    \cos(\alpha + \beta) & =\cos \alpha \cos \beta - \sin \alpha \sin \beta
    \text.
  \end{align*}
\end{blueBox}

\begin{blueBox}
  \sftitle{Komplex számok hatványozása:}
  \[
    z^n
    = (r(\cos \varphi + \iu \sin \varphi))^n
    = r^n (\cos (n \varphi) + \iu \sin (n \varphi))
  \]

  \begin{proof}[Teljes indukció módszerével]
    Vizsgáljuk meg az első pár esetet:
    \begin{align*}
      z^1 & = r^1 (\cos \varphi + \iu \sin \varphi) \text,                                                                                         \\
      z^2 & = r^2 (\cos \varphi + \iu \sin \varphi)^2                                                                                              \\
          & = r^2 (\cos^2 \varphi - \sin^2 \varphi + 2\iu \cos \varphi \sin \varphi)                                                               \\
          & = r^2 (\cos (2\varphi) + \iu \sin (2\varphi)) \text,                                                                                   \\
      z^3 & = z^2 \cdot z                                                                                                                          \\
          & = r^2 (\cos (2\varphi) + \iu \sin (2\varphi)) \cdot r(\cos \varphi + \iu \sin \varphi)                                                 \\
          & = r^3 (\cos (2\varphi) \cos \varphi - \sin (2\varphi) \sin \varphi + \iu (\cos (2\varphi) \sin \varphi + \sin (2\varphi) \cos \varphi) \\
          & = r^3 (\cos (3\varphi) + \iu \sin (3\varphi)) \text.
    \end{align*}

    Tegyük fel, hogy $z^k = r^k (\cos k\varphi + \iu \sin k\varphi)$, majd
    vizsgáljuk meg az $(n + 1)$-edik esetet:
    \begin{align*}
      z^{k + 1}
       & = z^k \cdot z
      = r^k (\cos (k\varphi) + \iu \sin (k\varphi)) \cdot r(\cos \varphi + \iu \sin \varphi)
      \\
       & = r^{k + 1} (\cos (k\varphi) \cos \varphi - \sin (k\varphi) \sin \varphi + \iu \cos (k\varphi) \sin \varphi + \sin (k\varphi) \cos \varphi)
      \\
       & = r^{k + 1} (\cos ((k + 1)\varphi) + \iu \sin ((k + 1)\varphi))
      \text.
    \end{align*}

    Ezzel bebizonyítottuk, hogy $z^n = r^n (\cos n\varphi + \iu \sin n\varphi)$.
  \end{proof}
\end{blueBox}

\begin{blueBox}
  \sftitle{Komplex számok osztása:}
  \[
    \frac{z_1}{z_2}
    = \frac{
      r_1 (\cos \varphi_1 + \iu \sin \varphi_1)
    }{
      r_2 (\cos \varphi_2 + \iu \sin \varphi_2)
    }
    = \frac{r_1}{r_2} (\cos (\varphi_1 - \varphi_2) + \iu \sin (\varphi_1 - \varphi_2))
  \]
\end{blueBox}

\begin{blueBox}
  \sftitle{Komplex számok gyökvonása:}
  \[
    \sqrt[n]{z}
    % = \sqrt[n]{r(\cos \varphi + \iu \sin \varphi)}
    = \sqrt[n]{r} \left(
    \cos \left( \frac{\varphi + 2k\pi}{n} \right) + \iu \sin \left( \frac{\varphi + 2k\pi}{n} \right)
    \right)
    \;
    k \in \{0; 1; \ldots; n-1\}
  \]
\end{blueBox}

\begin{note}
  Tetszőleges komplex szám $n$-edik gyökei egy olyan szabályos $n$-szög csúcsai,
  amelynek középpontja az origó.
  \begin{center}
    \begin{tikzpicture}[ultra thick]
      % COORDINATE SYSTEM
      \draw[-to, draw=primaryColor] (-2, 0) -- (2, 0) node [right] {$\Re$};
      \draw[-to, draw=primaryColor] (0, -2) -- (0, 2) node [above] {$\Im$};

      \coordinate (P) at (318:1.5);
      \foreach \deg in {30,102,174,246,318} {
          \draw[draw=primaryColor] (\deg:1.5) coordinate(C) circle (0.1);
          \draw[dashed, gray] (P) -- (C);
          \coordinate (P) at (C);
        }
    \end{tikzpicture}
  \end{center}
\end{note}

\begin{note}
  A komplex számokat nem tudjuk rendezni, azonban $(\mathbb C, +, \cdot)$
  test.
\end{note}

\begin{theorem}[Az algebra alaptétele]
  Egy $n$-ed fokú komplex együtthatós polinomnak multiplicitással számolva
  pontosan $n$ darab gyöke van.
\end{theorem}

\begin{note}
  Minden valós együtthatós polinom felírható első és másodfokú tényezők
  szorzataként.
\end{note}