\mainChapter{Komplex számok}

\begin{tikzpicture}[
    remember picture,
    overlay,
    ultra thick,
    shift={($(current page.south west) + (\innerMargin-15mm, 7.5mm)$)},
  ]
  % Complex numbers grid
  % \draw[gray, thin] (-0.75,-.75) grid (\paperwidth,13.25cm-\innerMargin);
  % \draw[gray, thin] (-0.75,-.75) grid (\innerMargin-0.75cm,\paperheight);

  \draw[gray!40, thin] (-0.75,-.75) grid (\paperwidth,\paperheight);

  \coordinate (BL) at (\innerMargin-0.75cm,13.25cm-\innerMargin);
  \coordinate (TR) at ($(\textwidth+1cm,\paperheight-\innerMargin-17.25cm)+(BL)$);

  \coordinate (O) at (7, 6);

  \coordinate (L) at ($(O)+(-10.25,0)$);
  \coordinate (R) at ($(O)+(+12.5,0)$);
  \coordinate (T) at ($(O)+(0,+22.5)$);
  \coordinate (B) at ($(O)+(0,-6.25)$);

  \draw[draw=secondaryColor, ultra thick, ->] (L) -- (R) node [above=2mm] {$\Re$};
  \draw[draw=secondaryColor, ultra thick, ->] (B) -- (T) node [right=2mm] {$\Im$};

  \foreach \r in {1,2,...,25} {
      \draw[gray, thin] (O) circle (\r cm);
    }


  \foreach \angle/\mark in {
      10/A,%
      30/B,%
      50/C,%
      70/D,%
      90/E,%
      110/F,%
      130/G,%
      150/H,%
      170/I,%
      190/J,%
      210/K,%
      230/L,%
      250/M,%
      270/N,%
      290/P,%
      310/Q,%
      330/R,%
      350/S%
    } {
      \draw[primaryColor,->, line width=2.5pt, opacity=\angle/500+0.2]
      (O) -- ++(\angle-235:\angle*.375mm+3cm) coordinate (\mark);
    }

  % \draw[smooth, line width=1pt, gray]
  % (A) -- (B) -- (C) -- (D) -- (E) -- (F) -- (G) -- (H) -- (I) -- (J) -- (K) -- (L) -- (M) -- (N) -- (O) -- (P) -- (Q) -- (R);

  \draw[gray, line width=1pt, opacity=.5] plot [smooth, tension=0.5] coordinates {
      (A) (B) (C) (D) (E) (F) (G) (H) (I) (J) (K) (L) (M) (N) (P) (Q) (R) (S)
    };

  \fill[white, fill opacity=.9] (BL) rectangle (TR);
\end{tikzpicture}

\bgroup
\color{gray!50!black}
\sffamily

Amikor a matematikusok évszázadokkal ezelőtt csak a valós számokkal dolgoztak,
találkoztak olyan problémákkal, amelyeket nehezen vagy egyáltalán nem tudtak
megoldani. A sikert az hozta el, mikor bővítették a valós számok halmazát.

A komplex számok bevezetése tehát egyfajta természetes kiterjesztése volt a
valós számoknak A komplex számok segítségével megoldhatunk olyan egyenleteket,
amelyeknek nincsenek valós megoldásai, például az $x^2 + 1 = 0$ egyenlet ilyen
és minden további, aminek negatív a diszkriminánsa.

Ebben a fejezetben megismerkedünk a komplex számok fogalmával, ábrázolásával,
műveleteivel és alkalmazásaival a matematika és a természettudományok különböző
területein. A komplex számok tanulmányozása gazdagítja a matematikai
ismereteinket, mélyíti a matematikai gondolkodásunkat és új lehetőségeket nyit
számunkra a problémamegoldásban.

\chaptertoc
\egroup

\clearpage