\mainChapter{Sorozatok}

\begin{tikzpicture}[
    remember picture,
    overlay,
    ultra thick,
    shift={($(current page.south west) + (\innerMargin+50mm, 8cm)$)},
    % shift={(current page.north east)},
    scale=.04,
  ]
  % \def\spiral#1{%
  %   \pgfmathparse{int(#1)}%
  %   \ifnum\pgfmathresult>0
  %     \draw [help lines] (0,0) rectangle ++(1,1);
  %     \begin{scope}[shift={(1,1)}, rotate=90, scale=1/1.6180339887]
  %       \spiral{#1-1}
  %     \end{scope}
  %     \draw [primaryColor] (0,0) arc (270:360:1);
  %   \fi
  % }

  \def\inverseSpiral#1{%
    \pgfmathsetmacro\currentVal{int(#1)}
    \ifnum\currentVal>0
      \pgfmathsetmacro\currentMod{int(mod(\currentVal, 3))}

      \begin{pgfonlayer}{bg}
        \ifnum\currentMod=0
          \def\fillcolor{primaryColor}
        \fi
        \ifnum\currentMod=1
          \def\fillcolor{secondaryColor}
        \fi
        \ifnum\currentMod=2
          \def\fillcolor{ternaryColor}
        \fi

        \draw [draw=\fillcolor, fill=\fillcolor, fill opacity=.2, thick] (0,0) rectangle ++(1,1);
      \end{pgfonlayer}

      % Then draw the square and arc for the current step
      \draw [ultra thick, primaryColor] (0,0) arc (270:360:1);

      % First, recursively call the macro with one less
      \begin{scope}[scale=1.6180339887, rotate=90, shift={(1/1.6180339887,-1/1.6180339887)}]
        \inverseSpiral{#1-1}
      \end{scope}
    \fi
  }
  \inverseSpiral{15}
\end{tikzpicture}

\bgroup
\color{gray!50!black}
\sffamily

A számsorozatok olyan matematikai objektumok, amelyek a pozitív egész számokhoz
rendelnek valós vagy komplex számokat. Gondolhatunk rájuk függvényként is, ahol
az értelmezési tartomány a pozitív egész számok halmaza, az értékkészlet pedig a
valós vagy a komplex számok egy részhalmaza.

Ebben a fejezetben a számsorozatok definícióját, különböző megadási módjainak
ismertetését követően áttekintjük a sorozatok - már középiskolában megtanult -
tulajdonságait, típusait, majd definiáljuk a határérték, a konvergens és a
divergens sorozat fogalmát.

Részletesen foglalkozunk a sorozatokkal kapcsolatos fontos tételekkel,
fogalmakkal. Az itt megszerzett ismeretek kulcsfontosságúak lesznek a későbbi
tanulmányok során és segítenek abban, hogy mélyebb megértést szerezzünk a
matematika alapvető elveiről és alkalmazásairól.

\chaptertoc
\egroup

\clearpage