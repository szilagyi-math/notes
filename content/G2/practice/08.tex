\documentclass[a4paper, 12pt]{scrartcl}

\usepackage{math-practice}
\usepackage{tikz}
\usepackage{multicol}
\usepackage{amsmath}
\usepackage{arydshln}

\area{Valós analízis}
\title{Taylor-sorok}
\subject{Matematika G2}
\subjectCode{BMETE94BG02}
\date{Utoljára frissítve: \today}
\docno{8}

\begin{document}
\maketitle
\subsection{Elméleti Áttekintő}

\begin{definition}[Taylor-polinom]
  Legyen $f: I \subset \Reals \to \Reals$ függvény, mely az $x_0$ pontban
  legalább $p$-szer differenciálható. Ekkor az $f$ függvény $x_0$ körüli
  $p$-edik Taylor-polinomja:
  $$
    T_p(x) = \sum_{k=0}^{p} \frac{f^{(k)}(x_0)}{k!} (x - x_0)^k
    \text.
  $$
\end{definition}

\begin{theorem}[Taylor-formula Lagrange-féle maradéktaggal]
  Ha az $f$ függvény legalább $(r + 1)$-szer differenciálható az $(x; x_0)$
  intervallumon és $f^{(k)}$ $\forall k \in \{1;2;\dots;r\}$ esetén folytonos
  ay $x$ és $x_0$ pontokban, akkor $\exists \xi \in (x; x_0)$, hogy
  $$
    f(x)
    = \sum_{k=0}^{r} \frac{f^{(k)}(x_0)}{k!} (x - x_0)^k
    + \underbrace{\frac{f^{(r+1)}(\xi)}{(r+1)!} (x - x_0)^{r+1}}_{\text{Lagrange-féle maradéktag}}
  $$
\end{theorem}

\begin{definition}[Taylor-sor]
  Legyen az $f$ függvény az $x_0$ pontban akárhányszor differenciálható. Ekkot a
  $$
    T(x) = \sum_{k=0}^{\infty} \frac{f^{(k)}(x_0)}{k!} (x - x_0)^k
  $$
  hatványsort az $f$ függvény $x_0$ körüli Taylor-sorának nevezzük.
\end{definition}

\begin{note}
  Ha $x_0 = 0$, akkor a Taylor-sorot Maclaurin-sornak nevezzük.
\end{note}

\begin{example}
  \bgroup\sffamily
  Írjuk fel a $p(x) = x^3 + 3x^2 + 2$ függvény $x_0 = 1$ körüli harmadfokú
  Taylor-polinomját!
  \egroup

  \begin{minipage}[t]{.4\textwidth}
    \def\arraystretch{1.3}
    \begin{tabular}{|>{$}l<{$}>{$}r<{$}|}
      \hline
      p^{(n)}(x)            & p^{(n)}(1)
      \\ \hline
      p(x) = x^3 + 3x^2 + 2 & 6
      \\
      p'(x) = 3x^2 + 6x     & 9
      \\
      p''(x) = 6x + 6       & 12
      \\
      p'''(x) = 6           & 6
      \\ \hline
    \end{tabular}
  \end{minipage}\begin{minipage}{.6\textwidth}
    \begin{align*}
      T_3(x)
       & = \frac{6}{0!} + \frac{9}{1!}(x - 1) + \frac{12}{2!}(x - 1)^2 + \frac{6}{3!} (x - 1)^3
      \\
       & = 6 + 9(x - 1) + 6(x - 1)^2 + (x - 1)^3
    \end{align*}
  \end{minipage}
\end{example}

\begin{example}
  \bgroup\sffamily
  Írjuk fel az $f(x) = e^x$ függvény Maclaurin-sorát!
  \egroup

  \hfill\begin{minipage}[t]{.345\textwidth}
    \def\arraystretch{1.2}
    \begin{tabular}{|>{$}l<{$}>{$}r<{$}|}
      \hline
      f^{(n)}(x)       & f^{(n)}(0)
      \\ \hline
      f(x) = e^x       & 1
      \\
      f'(x) = e^x      & 1
      \\
      \vdots           & \vdots
      \\
      f^{(k)}(x) = e^x & 1
      \\ \hline
    \end{tabular}
  \end{minipage}\begin{minipage}{.655\textwidth}
    $$
      T(x)
      = \sum_{k=0}^{\infty} \frac{f^{(k)}(0)}{k!} x^k
      = \sum_{k=0}^{\infty} \frac{1}{k!} x^k
      = \sum_{k=0}^{\infty} \frac{x^k}{k!}
    $$
  \end{minipage}
\end{example}

\begin{example}
  \bgroup\sffamily
  Írjuk fel az $f(x) = \sin x$ függvény Maclaurin-sorát!
  \egroup

  \hfill\begin{minipage}[t]{.345\textwidth}
    \def\arraystretch{1.2}
    \begin{tabular}{|>{$}l<{$}>{$}r<{$}|}
      \hline
      f^{(n)}(x)        & f^{(n)}(0)
      \\ \hline
      f(x) = \sin x     & 0
      \\
      f'(x) = \cos x    & 1
      \\
      f''(x) = -\sin x  & 0
      \\
      f'''(x) = -\cos x & -1
      \\
      \vdots            & \vdots
      \\ \hline
    \end{tabular}
  \end{minipage}\begin{minipage}{.655\textwidth}
    \begin{align*}
      T(x)
       & = \frac{x^1}{1!} - \frac{x^3}{3!} + \frac{x^5}{5!} - \frac{x^7}{7!} + \dots
      = \sum_{k=0}^{\infty} (-1)^k \frac{x^{2k+1}}{(2k+1)!}
    \end{align*}
  \end{minipage}
\end{example}

\begin{blueBox}
  \sftitle{Fontosabb függvények Maclaurin-sorai:}
  \vspace{-3mm}
  \begin{center}
    \setlength\extrarowheight{5pt}
    \renewcommand{\arraystretch}{1.4}
    \begin{tabular}{|c|c|c|}
      \hline
      Függvény         & Taylor-sor                                                         & Konvergencia intervallum
      \\[10pt]
      \hline
      $e^x$            & $\displaystyle\sum_{k=0}^{\infty} \frac{x^k}{k!}$                  & $\Reals$
      \\[10pt]
      \hline
      $\sin x$         & $\displaystyle\sum_{k=0}^{\infty} (-1)^k \frac{x^{2k+1}}{(2k+1)!}$ & $\Reals$
      \\[10pt]
      \hline
      $\cos x$         & $\displaystyle\sum_{k=0}^{\infty} (-1)^k \frac{x^{2k}}{(2k)!}$     & $\Reals$
      \\[10pt]
      \hline
      % $\tan x$         & $\displaystyle\sum_{k=0}^{\infty} \frac{B_{2k} 2^{2k} (2^{2k} - 1)}{(2k)!} x^{2k-1}$ & $(-\pi/2; \pi/2)$
      % \\[10pt]
      % \hline
      % $\arcsin x$      & $\displaystyle\sum_{k=0}^{\infty} \frac{(2k)!}{4^k (k!)^2 (2k+1)} x^{2k+1}$          & $[-1; 1]$
      % \\[10pt]
      % \hline
      % $\arccos x$      & $\displaystyle\sum_{k=0}^{\infty} \frac{(2k)!}{4^k (k!)^2 (2k+1)} x^{2k+1}$          & $[-1; 1]$
      % \\[10pt]
      % \hline
      $\arctan x$      & $\displaystyle\sum_{k=0}^{\infty} (-1)^k \frac{x^{2k+1}}{2k+1}$    & $[-1; 1]$
      \\[10pt]
      \hline
      $\sinh x$        & $\displaystyle\sum_{k=0}^{\infty} \frac{x^{2k+1}}{(2k+1)!}$        & $\Reals$
      \\[10pt]
      \hline
      $\cosh x$        & $\displaystyle\sum_{k=0}^{\infty} \frac{x^{2k}}{(2k)!}$            & $\Reals$
      \\[10pt]
      \hline
      $\arctanh x$     & $\displaystyle\sum_{k=0}^{\infty} \frac{x^{2k+1}}{2k+1}$           & $(-1; 1)$
      \\[10pt]
      \hline
      $\ln(1 + x)$     & $\displaystyle\sum_{k=0}^{\infty} (-1)^{k} \frac{x^{k+1}}{k+1}$    & $(-1; 1]$
      \\[10pt]
      \hline
      $(1 + k)^\alpha$ & $\displaystyle\sum_{k=0}^{\infty} \binom{\alpha}{k} x^k$           & $(-1; 1)$
      \\[10pt]
      \hline
    \end{tabular}
  \end{center}
\end{blueBox}

\clearpage
\subsection{Feladatok}
\begin{enumerate}
  \item Írja fel a $p(x) = (1+x)^3$ függvény Maclauren-sorát!

  \item Határozza meg az alábbi függvények adott pont körüli Taylor-sorát!
        Adja meg az összegfüggvények konvergenciasugarát is!
        $$
          \def\arraystretch{1.25}
          \begin{array}{l<{(x)}@{\;=\;}l<{\text,}>{x_0 = }l}
            f & (1 - x)^3 & 1 \\
            g & e^x       & 1 \\
            h & \ln x     & 1 \\
          \end{array}
        $$

  \item Határozza meg az alábbi függvények Maclauren-sorát!
        Adja meg az összegfüggvények konvergenciasugarát is!
        $$
          \def\arraystretch{1.25}
          \begin{array}{l<{(x)}@{\;=\;}l}
            f & \cos 5x              \\
            g & \sin \sqrt x         \\
            h & \sin^2 x             \\
            i & \sqrt[3]{\exp(-x^2)}
          \end{array}
        $$

  \item Adja meg az alábbi törtfüggvények Taylor-sorát!
        $$
          \begin{array}{l<{(x)}@{\;=\;}l<{\text,}>{x_0 = }l}
            f & \dfrac{x + 1}{x + 3} & -2 \\[5mm]
            g & \dfrac{x + 1}{x + 3} & -1 \\
          \end{array}
        $$

  \item Írja fel az alábbi függvény $x_0 = 2$ pontra illeszkedő Taylor sorát! Mi lesz a konvergenciasugár?
        $$
          f(x) = \cfrac{1}{x^2-3x+2}
        $$

  \item Határozza meg az alábbi függvények Maclauren-sorát!
        Adja meg az összegfüggvények konvergenciasugarát is!
        $$
          \begin{array}{l<{(x)}@{\;=\;}l}
            f & \dfrac{x}{1+x^2}        \\[3mm]
            g & \arctan{x}              \\
            h & \dfrac{1}{\sqrt{2+x^2}} \\
          \end{array}
        $$

  \item Melyik függvény Taylor-sora az alábbi?
        $$
          \sum_{n=0}^{\infty} \cfrac{2n+1}{n!}x^{2n}
        $$

  \item Hanyadfokú taylor polinom közelíti a $\sin(\pi/60)$ értékét 4
        tizedesjegy pontossággal?

  \item Számítsa ki 3 tizedesjegy pontossággal az alábbi integrált!
        $$
          \int_{0}^{0,2} e^{2x} \dd x
        $$
\end{enumerate}




\end{document}