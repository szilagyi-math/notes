\documentclass[a4paper, 12pt]{scrartcl}

\usepackage{math-practice}
\usepackage{tikz}
\usepackage{multicol}
\usepackage{amsmath}
\usepackage{arydshln}

\area{Valós analízis}
\title{Függvény- és hatványsorok}
\subject{Matematika G2}
\subjectCode{BMETE94BG02}
\date{Utoljára frissítve: \today}
\docno{7}

\begin{document}
\maketitle
\subsection{Elméleti Áttekintő}
\begin{definition}[Függvénysor]
    Legyen $f_n(x)$ függvények egy sorozata, ekkor az $\displaystyle S_N(x):= \sum_{n=1}^{N}f(x)$ függvényt rész\-let\-összeg-függvénynek nevezzük. Az $f_n(x)$ függvénysorozathoz tartozó függvénysornak nevezzük a $\displaystyle \sum_{n=1}^{\infty} f_n(x)$ összeget.
\end{definition}

\begin{definition}[Konvergenciatartomány]
    A $\displaystyle \sum_{n=1}^{\infty} f_n(x)$ függvénysor konvergenciatartománya azon $x \in \Reals$-ek halmaza, melyre $S_N$ sorozat konvergens és $F(x) := \displaystyle \lim_{N \to \infty } S_N$ határérték lesz a sor összegfüggvénye.
\end{definition}

\begin{definition}[Egyenletes konvergencia]
    A $S_N$ függvénysor egyenletesen konvergens, ha egyenletesen konvergál $S$-hez. Ha
    \[
    \lim |S-S_N| = \lim_{N \to \infty} \left|\sum_{n = 1}^{\infty} f_n(x)\right| = 0
    \]
\end{definition}

\begin{definition}[Abszolút konvergencia]
    A $\displaystyle \sum_{n=1}^{\infty} f_n(x)$ függvénysor abszolút konvergens, ha $\displaystyle \sum_{n=1}^{\infty} |f_n(x)|$ konvergens.
\end{definition}

\begin{definition}[Weierstrass kritérium]
    Abszolút értelemben egy konvergens numerikus sor tagjaival majorálható függvénysor abszolút és egyenletesen konvergens.
\end{definition}

\begin{blueBox}
    \begin{itemize}
        \item[] Ha $f_n$ és $\displaystyle \sum f_n$ egyenletesen konvergens, akkor az összegfüggvény is folytonos $[a,b]$ intervallumon. Ez azt jelenti, hogy a szumma és a limesz felcserélhető.
    \end{itemize}
\end{blueBox}
\begin{blueBox}
\begin{itemize}
    \item[] Ha $f_n$ folytonos és $\displaystyle \sum f_n$ egyenletesen konvergens, akkor
    \[
    \int_{a}^{b} \sum_{n=0}^{\infty} f_n(x)\, dx = \sum_{n=0}^{\infty} \int_{a}^{b} f_n(x)\, dx 
    \]
    \item[] Ha $f_n$ foyltonos, $f_n'$ folytonos, $\displaystyle \sum f_n = s(x)$ pontonként konvergens és $\displaystyle \sum f_n'$ egyenletesen konvergens, akkor $s$ deriválható és $s' = g$
\end{itemize}
\end{blueBox}

\subsection{Hatványsorok}

\begin{definition}[Hatványsor]
    A $\displaystyle \sum_{n=0}^{\infty} a_n(x-x_0)^n$ alakú függvénysorokat hatványsornak nevezzük.Az $x_0$ a hatványsor középpontjaés $a_n$ 
    az együtthatósorozat.
\end{definition}

\begin{theorem}[Cauchy-Hadamard tétel]
    A $\displaystyle \sum_{n=0}^{\infty} a_n(x-x_0)^n$ hatványsor konvergenciasugara
    \[
    r = \lim_{n \to \infty}\cfrac{1}{\sqrt[n]{\mid a_n \mid}} = \lim_{n \to \infty} \left|\cfrac{a_n }{a_{n+1}}\right|
    \]
    A hatványsor az $(x_0-r,x_0+r)$ intervallumon konvergens, míg a $[x_0-r, x_0+r]$ intervallumon kívül divergens.
\end{theorem}

\begin{theorem}[Hatványsor deriválása]
    Az $f(x) = \displaystyle \sum_{n=0}^{\infty}a_n(x-x_0)^n$ hatványsor kon\-ver\-gen\-ci\-a\-tar\-to\-má\-nyá\-nak belsejében a tagonkénti deriválásával kapott $\displaystyle \sum_{n=0}^{\infty}a_n n (x-x_0)^{n-1}$ hatványsor is konvergens ugyanazon intervallumon és $f'(x)$-el egyezik meg.
\end{theorem}

\begin{theorem}[Hatványsor integrálása]
    Az $f(x) = \displaystyle \sum_{n=0}^{\infty}a_n(x-x_0)^n$ hatványsor kon\-ver\-gen\-ci\-a\-tar\-to\-má\-nyá\-nak bármely belső $[a,b]$ intervallumban tagonként integrálható, azaz
    \[
\int_a^b f(x) \, dx = \sum_{n=0}^\infty \int_a^b a_n (x - x_0)^n \, dx 
= \sum_{n=0}^\infty \frac{a_n}{n+1} \left[(x - x_0)^{n+1}\right]_a^b.
\]
    
\end{theorem}
\clearpage
\subsection{Feladatok}
\begin{enumerate}
    \item Vizsgáljuk meg a következő függvénysorok konvergenciatartományát, értelmezési tartományát ás összegfüggvényét!
    \begin{multicols}{3}
        \begin{enumerate}
            \item $\displaystyle \sum_{n=1}^{\infty} \sqrt[n]{x}$
            \item $\displaystyle \sum_{n=1}^{\infty} \left(\cfrac{x-1}{x+1}\right)^n$
            \item $\displaystyle \sum_{n=1}^{\infty} \sin^{2n}x$
        \end{enumerate}
    \end{multicols}
    \item Határozzuk meg az alábbi függvénysorok értelmezési tartományát, konvergenciátartományát és hogy a konvergenciatartományon belül abszolút konvergens-e!
    \begin{multicols}{2}
        \begin{enumerate}
            \item $\displaystyle \sum_{n=1}^{\infty} \cfrac{1}{1+x^{2n}}$
            \item $\displaystyle \sum_{n=1}^{\infty} \cfrac{\left(\left|\frac{z-i-1}{3}\right|\right)^n}{z}$
            \item $\displaystyle \sum_{n=1}^{\infty} (-1)^n n^{-x}$
            \item $\displaystyle \sum_{n=1}^{\infty} \cfrac{\cos (3x^3 +(\pi/3)nx^2)}{3^n +x^4n^4}$
        \end{enumerate}
    \end{multicols}
    \item El lehet-e végezni a következő függvénysorok tagonkénti integrálását?
    \[
    \int_0^2 \sum_{n=1}^{\infty} \cfrac{x^n}{e^{nx}}\, dx \overset{?}{=} \sum_{n=1}^{\infty} \int_0^2 \cfrac{x^n}{e^{nx}}\, dx 
    \]
    \item El lehet-e végezni a következő függvénysorok tagonkénti deriválását?
    \[
    \sum_{n=1}^{\infty} \cfrac{\arctan(x/n)}{n^2}
    \]
    \item Határozza meg az alábbi hatványsorok konvergenciatartományát!
    \begin{multicols}{2}
        \begin{enumerate}
            \item $\displaystyle \sum_{n=1}^{\infty} n(x-2)^n$
            \item $\displaystyle \sum_{n=2}^{\infty} \cfrac{x^n}{2^n(n-1)}$
            \item $\displaystyle \sum_{n=1}^{\infty} (-1)^n\cfrac{(x-2)^n}{3}$
            \item $\displaystyle \sum_{n=1}^{\infty} \left(4-\cfrac{1}{n}\right)^n x^n$
            \item $\displaystyle \sum_{n=1}^{\infty} \left(\cfrac{4n(-1)^n+n+2}{2n}\right)^nx^n$
        \end{enumerate}
    \end{multicols}
    \item Határozza meg az alábbi komplex hatványsorok konvergenciatartományát!
    \begin{multicols}{3}
        \begin{enumerate}
            \item $\displaystyle \sum_{n=1}^{\infty} \cfrac{(n!)^2}{(2!)}z^n$
            \item $\displaystyle \sum_{n=1}^{\infty} \cfrac{x^{2n-1}}{2n-1}$
            \item $\displaystyle \sum_{n=1}^{\infty} (n+1)(n+2)x^n$
        \end{enumerate}
    \end{multicols}
\end{enumerate}



\end{document}