\documentclass[a4paper, 12pt]{scrartcl}

\usepackage{math-practice}

\area{Valós analízis}
\title{Függvény- és hatványsorok}
\subject{Matematika G2}
\subjectCode{BMETE94BG02}
\date{Utoljára frissítve: \today}
\docno{7}

\begin{document}
\maketitle
\subsection{Elméleti Áttekintő}

\begin{definition}[Függvénysor]
  Legyen $f_n : I \subset \Reals \to \Reals$ függvénysorozat. Képezzük az
  alábbi függvénysorozatot:
  \begin{align*}
    s_1(x) & := f_1(x)
    \text,                              \\
    s_2(x) & := f_1(x) + f_2(x)
    \text,                              \\
           & \phantom{:} \vdots
    \\
    s_j(x) & := \sum_{i = 1}^{j} f_i(x)
    \\
           & \phantom{:} \vdots
  \end{align*}

  Az így előálló $(s_n)$ függvénysorozatot az $(f_n)$ függvénysorozatból képzett
  függvénysornak hívjuk és $\sum f_n$-nel jelöljük.
\end{definition}

\begin{definition}[Függvénysor pontbeli konvergenciája]
  A $\sum f_n$ függvénysor konvergens az $x_0 \in I$ pontban, ha az $(s_n)$
  függvénysorozat konvergens az $x_0$ pontban.
\end{definition}

\begin{definition}[Függvénysor konvergenciahalmaza]
  A $\sum f_n$ függvénysor konvergens a $H \subset I$ halmazon, ha az
  $(s_n)$ függvénysorozat konvergens a $H$-n.
\end{definition}

\begin{definition}[Függvénysor egyenletes konvergenciája]
  A $\sum f_n$ függvénysor egyenletesen konvergens az $E \subset H$
  halmazon, ha az $(s_n)$ függvénysorozat egyenletesen konvergens az $E$-n.
\end{definition}

\begin{definition}[Függvénysor összegfüggvénye]
  A $\sum f_n$ függvénysorozat összegfüggvénye az $s(x) := \lim\limits_{n \to
      \infty} s_n(x)$ függvény, ahol $x \in H$.
\end{definition}

\begin{definition}[Abszolút konvergencia]
  A $\sum f_n$ függvénysor abszolút konvergens, ha a $\sum \left| f_n \right|$
  függvénysor konvergens.
\end{definition}

\begin{theorem}[%
    Cauchy-féle konvergencia kritérium egyenletes konvergenciára
  ]
  A $\sum f_n$ akkor és csak akkor egyenletesen konvergens az
  $E \subset H$ halmazon, ha $\forall \varepsilon > 0$ esetén
  $\exists N(\varepsilon)$ úgy, hogy ha $n; m > N(\varepsilon)$,
  akkor $\forall x \in E$ esetén ${|s_n(x) - s_m(x)| < \varepsilon}$.
\end{theorem}

\begin{theorem}[Weierstrass-tétel függvénysorok egyenletes konvergenciájára]
  Legyen $f_n : I \subset \Reals \to \Reals$ függvénysorozat és $\sum f_n$
  a belőle képzett függvénysor, továbbá $\sum a_n$ olyan konvergens numerikus
  sor, melyre $\forall x \in I$ esetén $|f_n(x)| \leq a_n$
  $\forall n \in \mathbb N$-re $n > n_0 \in \mathbb N$ esetén.

  Ekkor a $\sum f_n$ függvénysor egyenletesen konvergens.
\end{theorem}

\clearpage

\begin{definition}[Hatványsor]
  Legyen $f_n(x) := a_n \, (x - x_0)^n$. A belőle képzett
  $$
    \sum f_n(x) = \sum a_n \, (x - x_0)^n
  $$
  függvénysort hatványsornak nevezzük, ahol $a_n$ a hatványsor $n$-edik
  együtthatója, $x_0$ pedig a sorfejtés centruma.
\end{definition}

\begin{definition}[Hatványsor konvergenciasugara]
  A $\sum a_n \, (x - x_0)^n$ hatványsor konvergenciasugara:
  $$
    r = \frac{1}{\limsup\limits_{n \to \infty} \sqrt[n]{|a_n|}} \in \Reals_b
    \text.
  $$
\end{definition}

\begin{theorem}[Cauchy-Hadamard-tétel]
  Legyen $r$ a $\sum a_n x^n$ hatványsor konvergenciasugara. Ha \dots
  \begin{enumerate}
    \item $r = 0$, akkor a hatványsor csak az $x_0 = 0$ pontban konvergens,
    \item $r = \infty$, akkor a hatványsor $\forall x \in \Reals$ esetén
          konvergens,
    \item $0 < r < \infty$, akkor a hatványsor konvergens, ha $|x| < r$ és
          divergens, ha $|x| > r$.
  \end{enumerate}

  \begin{center}
    \begin{tikzpicture}[thick]
      \draw[very thick, -to, draw=secondaryColor]
      (0,0) -- (6,0) node[below left] {$x$};

      \foreach \x/\l in {1/-r,3/0,5/+r}{
          \draw[draw=primaryColor] (\x, 3pt) -- (\x, -3pt) node[below]{$\l$};
        }
      \draw [
        draw=primaryColor,
        decorate,
        decoration={brace,amplitude=5pt,mirror,raise=4ex}
      ]
      (1,0) -- (5,0) node[midway,yshift=-3em] {abszolút konvergencia};
    \end{tikzpicture}
  \end{center}
\end{theorem}

\begin{theorem}[Tagonkénti integrálhatóság]
  Legyenek a $\sum f_n$ függvénysor tagjai integrálhatóak az $[a; b]$ zárt
  intervallumon. Tegyük fel, hogy a sor egyenletesen konvergens az $[a; b]$-n
  és összegfüggvénye folytonos. Ekkor
  $$
    \int_a^b f(x) \dd x = \sum_{n = 1}^{\infty} \int_a^b f_n(x) \dd x
    \text.
  $$
\end{theorem}

\begin{note}
  Nem korlátos intervallum esetén nem igaz az állítás.
\end{note}

\begin{theorem}[Tagonkénti differenciálhatóság]
  Legyenek az $f_n$ függvénysorozat tagjai differenciálhatóak a $J$ intervallumon,
  $f'_n$ függvények folytonosak a $J$-n, valamint a $\sum f'_n$ és a $\sum f_n$
  függvénysorok egyenletesen konvergensek a $J$-n. Ekkor
  $$
    f'(x) = \sum_{n = 1}^{\infty} f'_n(x)
    \text.
  $$
\end{theorem}

\clearpage
\subsection{Feladatok}
\begin{enumerate}
  \item Vizsgálja meg a következő függvénysorok konvergenciatartományát,
        értelmezési tartományát és adja meg az összegfüggvényüket!
        \begin{multicols}{3}
          \begin{enumerate}
            \item $\displaystyle
                    \sum_{n=1}^{\infty} \sqrt[n]{x}
                  $

            \item $\displaystyle
                    \sum_{n=1}^{\infty} \left(\cfrac{x-1}{x+1}\right)^n
                  $

            \item $\displaystyle
                    \sum_{n=0}^{\infty} \sin^{2n}x
                  $
          \end{enumerate}
        \end{multicols}

  \item Határozza meg az alábbi függvénysorok értelmezési tartományát,
        konvergenciátartományát és hogy a konvergenciatartományon belül abszolút
        konvergensek-e!
        \begin{multicols}{2}
          \begin{enumerate}
            \item $\displaystyle
                    \sum_{n=1}^{\infty} \cfrac{1}{1+x^{2n}}
                  $

            \item $\displaystyle
                    \sum_{n=1}^{\infty} \cfrac{\left(\left|\frac{z-i-1}{3}\right|\right)^n}{z}
                  $

            \item $\displaystyle
                    \sum_{n=1}^{\infty} (-1)^n n^{-x}
                  $

            \item $\displaystyle
                    \sum_{n=1}^{\infty} \cfrac{\cos (3x^3 +(\pi/3)nx^2)}{3^n +x^4n^4}
                  $
          \end{enumerate}
        \end{multicols}

  \item El lehet-e végezni a következő függvénysor tagonkénti integrálását?
        $$
          \int_0^2 \sum_{n=1}^{\infty} \cfrac{x^n}{e^{nx}}\, dx \overset{?}{=} \sum_{n=1}^{\infty} \int_0^2 \cfrac{x^n}{e^{nx}}\, dx
        $$

  \item El lehet-e végezni a következő függvénysor tagonkénti deriválását?
        $$
          \sum_{n=1}^{\infty} \cfrac{\arctan(x/n)}{n^2}
        $$

  \item Határozza meg az alábbi hatványsorok konvergenciatartományát!
        \begin{multicols}{2}
          \begin{enumerate}
            \item $\displaystyle
                    \sum_{n=1}^{\infty} n(x-2)^n
                  $

            \item $\displaystyle
                    \sum_{n=2}^{\infty} \cfrac{x^n}{2^n(n-1)}
                  $

            \item $\displaystyle
                    \sum_{n=1}^{\infty} (-1)^n\cfrac{(x-2)^n}{3}
                  $

            \item $\displaystyle
                    \sum_{n=1}^{\infty} \left(4-\cfrac{1}{n}\right)^n x^n
                  $

            \item $\displaystyle
                    \sum_{n=1}^{\infty} \left(\cfrac{4n(-1)^n+n+2}{2n}\right)^nx^n
                  $
          \end{enumerate}
        \end{multicols}

  \item Határozza meg az alábbi komplex hatványsorok konvergenciatartományát!
        \begin{multicols}{3}
          \begin{enumerate}
            \item $\displaystyle
                    \sum_{n=1}^{\infty} \cfrac{(n!)^2}{(2!)}z^n
                  $

            \item $\displaystyle
                    \sum_{n=1}^{\infty} \cfrac{x^{2n-1}}{2n-1}
                  $

            \item $\displaystyle
                    \sum_{n=1}^{\infty} (n+1)(n+2)x^n
                  $
          \end{enumerate}
        \end{multicols}
\end{enumerate}
\end{document}