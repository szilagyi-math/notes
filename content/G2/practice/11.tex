\documentclass[a4paper, 12pt, fleqn]{scrartcl}

\usepackage{math-practice}

\area{Többváltozós analízis}
\title{Differenciálás I}
\subject{Matematika G2}
\subjectCode{BMETE94BG02}
\date{Utoljára frissítve: \today}
\docno{11}

\begin{document}
\maketitle
\subsection{Elméleti Áttekintő}

\begin{definition}[Iránymenti derivált]
  Legyen $I \in \Reals^n$ nyílt halmaz, $f: I \to \Reals$ függvény és
  legyen adva egy $\rvec v \in \Reals^n$ egységvektor. Ha létezik a
  $$
    \lim_{\lambda \to 0^+} \frac{
      f(\rvec x - \lambda \rvec v) - f(\rvec x)
    }{
      \lambda
    }
  $$
  határérték és ez egy valós szám, akkor ezt az $f$ függvény $\rvec a$
  pontbeli $\rvec v$ irányú, iránymenti deriváltjának nevezzük.
\end{definition}

\begin{definition}[Gradiens]
  Legyen $f: \Reals^n \to \Reals$. Az $f$ függvény
  $\rvec a(a_{1}; a_{2}; \ldots; a_{n})$ pontbeli gradiense:
  \def\arraystretch{1.5}
  $$
    \grad f(\rvec a) = \nabla f(\rvec a) = \begin{bmatrix}
      \partial_1 f(\rvec a) \\
      \vdots                \\
      \partial_n f(\rvec a)
    \end{bmatrix} = \begin{pmatrix}
      \displaystyle\pdv{f(\rvec x_0)}{x_1} &
      \cdots                               &
      \displaystyle\pdv{f(\rvec x_0)}{x_n}
    \end{pmatrix}^\T
  $$
\end{definition}

\begin{note}
  A gyakrolatban az iránymenti deriváltakat a gradiens segítségével számítjuk:
  $$
    \partial_{\rvec v} f(\rvec a) = \grad f(\rvec a) \cdot \rvec v
    \text.
  $$
\end{note}

\begin{example}
  Számítsuk ki az $f(x; y) = x^3 + 5x^2y + 3xy^2 - 12y^3 + 5x - 6y + 7$
  függvény $\rvec v(3;4)$ irányú deriváltját az $(1;2)$ pontban!

  A gradiens az előző példában számolt parciális deriváltak alapján:
  $$
    \grad f(1;2) = \begin{bmatrix}
      40 \\ -133
    \end{bmatrix}
    \text.
  $$
  Az iránymenti derivált számításához még szükségünk van az $\rvec v$ irányú
  egységvektorra:
  $$
    \|\rvec v\| = \sqrt{3^2 + 4^2} = 5
    \quad\Rightarrow\quad
    \uvec e_v = \frac{\rvec v}{\|\rvec v\|} = \frac{1}{5} \begin{bmatrix}
      3 \\ 4
    \end{bmatrix} = \begin{bmatrix}
      3/5 \\ 4/5
    \end{bmatrix}
    \text.
  $$
  Az iránymenti derivált:
  $$
    \partial_{\rvec v} f(1;2) = \grad f(1;2) \cdot \uvec e_v = \begin{bmatrix}
      40 \\ -133
    \end{bmatrix} \cdot \begin{bmatrix}
      3/5 \\ 4/5
    \end{bmatrix} = 40 \cdot 3/5 - 133 \cdot 4/5 = -82,4
    \text.
  $$
\end{example}


\begin{note}
  Kétváltozós függvény esetén az gyakran az irányvektort az $x$-tengellyel
  bezárt szög ($\alpha$) segítségével adjuk meg. Ekkor az egységvektor:
  $$
    \uvec e = \begin{bmatrix}
      \cos \alpha \\
      \sin \alpha
    \end{bmatrix}
    \text.
  $$
\end{note}

\begin{blueBox}
  \sftitle{Magasabb rendű iránymenti deriváltak:}

  Az elsőrendű iránymenti derivált:
  $$
    \pdv{f}{\uvec e} = \grad f \cdot \uvec e
    \text.
  $$
  A másodrendű iránymenti derivált:
  $$
    \pdv[order=2]{f}{\uvec e}
    = \grad \left(\pdv{f}{\uvec e}\right) \cdot \uvec e
    = \grad \left(\grad f \cdot \uvec e\right) \cdot \uvec e
    \text.
  $$
\end{blueBox}

\begin{blueBox}
  \sftitle{Adott irányú érintő egyenes kétváltozós függvények esetén:}

  A $z = f(x; y)$ függvény segítségével a 3D-s térben egy felületet adhatunk
  meg. Ezen felület egy $P_0(x_0; y_0)$ pontbeli, $\uvec e(e_x; e_y)$ irányú
  érintő egyenese:
  $$
    \frac{x - x_0}{e_x}
    = \frac{y - y_0}{e_y}
    = \frac{z - z_0}{\partial_{\uvec e}f(x_0; y_0)}
    \text{, \quad ahol $z_0 = f(x_0; y_0)$.}
  $$
  Amennyiben az irány egybeesik az $x$-tengellyel, akkor:
  $$
    x - x_0 = \frac{z - z_0}{\partial_x f(x_0; y_0)}
    \quad \Rightarrow \quad
    \partial_x f(x_0; y_0) (x - x_0) = z - z_0
    \text.
  $$
  Amennyiben az irány egybeesik az $y$-tengellyel, akkor:
  $$
    y - y_0 = \frac{z - z_0}{\partial_y f(x_0, y_0)}
    \quad \Rightarrow \quad
    \partial_y f(x_0; y_0) (y - y_0) = z - z_0
    \text.
  $$
\end{blueBox}

\begin{blueBox}
  \sftitle{Érintősík megadása kétváltozós függvény esetében:}

  Az érintősík független az iránytól, azt csak a felületből kimutató normális,
  vagyis a gradiens adja meg. Az $\rvec x_0$ pontban felírt normálvektor:
  $$
    \rvec n_{\text{be}} = \begin{bmatrix}
      \partial_x f \\
      \partial_y f \\
      -1
    \end{bmatrix}_{|\rvec x = \rvec x_0}
    \qquad
    \rvec n_{\text{ki}} = \begin{bmatrix}
      -\partial_x f \\
      -\partial_y f \\
      1
    \end{bmatrix}_{|\rvec x = \rvec x_0}
    \text,
  $$
  ahol $\rvec n_{\text{be}}$ a befele, $\rvec n_{\text{ki}}$ pedig a kifele
  mutató normálvektor. Ezek alapján az érintősík egyenlete:
  $$
    \rvec n (\rvec x - \rvec x_0) = 0
    \quad \Rightarrow \quad
    \left.\pdv{f}{x}\right|_{\rvec x = \rvec x_0} (x - x_0)
    + \left.\pdv{f}{y}\right|_{\rvec x = \rvec x_0} (y - y_0)
    = z - z_0
    \text.
  $$
\end{blueBox}

\begin{blueBox}
  \sftitle{Implicit függvény parciális deriváltjai:}

  Amennyiben a változók közötti kapcsolat nem írható fel explicit
  ($z = f(x; y)$) módon, akkor az implicit függvénymegadási módszerhez tudunk
  fordulni. Ez többváltozós esetben $F(x; y; z) = 0$ alakban tudjuk megtenni.

  Ilyen esetben a $z$-től függő tagokat összetett függvényként kell kezelnünk,
  a parciális deriváltak pedig a következőek:
  $$
    \pdv{f}{x} = \pdv{z}{x}
    \qquad \text{és} \qquad
    \pdv{f}{y} = \pdv{z}{y}
    \text.
  $$
\end{blueBox}

\begin{blueBox}
  \sftitle{Teljes differenciál:}

  Egy $f: \Reals^n \to \Reals$ függvény teljes differenciálja:
  $$
    Df = \sum_{i = 1}^n \pdv{f}{x_i} \dd x_i
    \text.
  $$
  Kétváltozós esetben:
  $$
    Df
    = \pdv{f}{x} \dd x
    + \pdv{f}{y} \dd y
    \text.
  $$
\end{blueBox}

\begin{definition}[Jacobi-mátrix]
  Legyen $\rvec f: \Reals^n \to \Reals^k$ leképezés.
  Ekkor $\rvec f'(\rvec a) = \rmat J \rvec f(\rvec a)$,
  ahol $\rmat J \in \mathscr M_{k \times n}$. A $\rmat J$ mátrixot az
  $\rvec f$ függvény Jacobi-mátrixának nevezzük, melynek elemei:
  \def\arraystretch{1.5}
  $$
    \rmat J(\rvec a) = \begin{bmatrix}
      \displaystyle\pdv{f_1(\rvec x)}{x_1} & \displaystyle\pdv{f_1(\rvec x)}{x_2} & \cdots & \displaystyle\pdv{f_1(\rvec x)}{x_n} \\
      \displaystyle\pdv{f_2(\rvec x)}{x_1} & \displaystyle\pdv{f_2(\rvec x)}{x_2} & \cdots & \displaystyle\pdv{f_2(\rvec x)}{x_n} \\
      \vdots                               & \vdots                               & \ddots & \vdots                               \\
      \displaystyle\pdv{f_k(\rvec x)}{x_1} & \displaystyle\pdv{f_k(\rvec x)}{x_2} & \cdots & \displaystyle\pdv{f_k(\rvec x)}{x_n}
    \end{bmatrix}_{|\rvec x = \rvec a} = \begin{bmatrix}
      \grad^\T f_1(\rvec a) \vphantom{\displaystyle\pdv{f_1}{x_1}} \\
      \grad^\T f_2(\rvec a) \vphantom{\displaystyle\pdv{f_1}{x_1}} \\
      \vdots                                                       \\
      \grad^\T f_k(\rvec a) \vphantom{\displaystyle\pdv{f_1}{x_1}}
    \end{bmatrix}
  $$
\end{definition}

\clearpage
\subsection{Feladatok}

\begin{enumerate}
  \item Határozza meg az alábbi függvények iránymenti deriváltját az adott
        pontban és irányszög vagy irányvektor mentén!
        \begin{enumerate}
          \item $f(x;y)=x^3-3xy^2+4y^4$,
                \tabto{7cm} $P_a(1,1)$,
                \tabto{10cm} $\alpha=45^{\circ}$,

          \item $g(x;y;z) = e^{-\left(x^2+y^2\right)}$,
                \tabto{7cm} $P_b(1;0;1)$,
                \tabto{10cm} $\rvec v_b=[3,2,-5]^\T$.
        \end{enumerate}

  \item Határozza meg azon pontok halmazát, amin nem létezik az
        $f(x;y)=\ln\left(x^2+xy\right)$ függvény $\alpha=150^\circ$-os irányhoz
        tartozó iránymenti deriváltja!

  \item Határozza meg az alábbi függvények első és második iránymenti
        deriváltjait az adott pontban és irányrányszög vagy irányvektor mentén!
        \begin{enumerate}
          \item $f(x;y)=4x^4y+y^3x^2$,
                \tabto{7cm} $P_a(1;1)$,
                \tabto{10cm} $\alpha=45^\circ$,

          \item $g(x;y;z)=\sqrt{14}\,xyz+\sqrt{14}\,x^2y^2z^2$,
                \tabto{7cm} $P_b(1;1;1)$,
                \tabto{10cm} $\rvec v_b=[1;1;1]^\T$.
        \end{enumerate}

  \item Határozza meg az $f(x; y)=\ln\left(x^2+y^2\right)$ függvény $P(0;1)$
        ponthoz és $\alpha=60^\circ$ irányhoz tartozó érintőegyenesének
        egyenletét!

  \item Határozza meg a $f(x; y)=\sin xy$ függvény érintősíkjának egyenletét a
        $P(\sfrac{\pi}{3};2)$ pontban!

  \item Határozza meg azon pontok halmazát, ahol a $z=2x^2+5y^2-3x+2y-1$ felület
        érintősíkja párhuzamos a $2x-y+5z-25=0$ síkkal!

  \item Határozza meg az $f(x;y)=\arctan xy$ függvény teljes differenciálját
        paraméteresen a $P_7(x_0; y_0)$ és a $Q_7(1; 2)$ pontban!

  \item Határozza meg egy henger térfogat mérésének a relatív hibáját, ha
        ismert, hogy a sugár 1\%-os és a magasság 2\%-os hibával lett mérve!

  \item Határozza meg az $\rvec f: \Reals^3 \to \Reals^2$ függvény Jakobi
        mátrixát!
        \begin{equation*}
          \rvec f(x;y;z)=\begin{bmatrix}
            x^2 + y^2 +z \sin x \\
            z^2 + z \sin y
          \end{bmatrix}
        \end{equation*}
\end{enumerate}


\end{document}