\documentclass[a4paper, 12pt]{scrartcl}

\usepackage{math-practice}

\area{Valós analízis}
\title{Függvénysorozatok}
\subject{Matematika G2}
\subjectCode{BMETE94BG02}
\date{Utoljára frissítve: \today}
\docno{6}

\begin{document}
\maketitle

\subsection{Elméleti Áttekintő}

\begin{definition}[Numerikus sor]
  Legyen $(a_n) : \mathbb{N} \rightarrow \Reals$ numerikus sorozat, amelyből
  képezzük az alábbi sorozatot:
  $$
    s_n = a_1 + a_2 + \dots + a_n = \sum_{i=1}^{n} a_i
  $$
  Az így képzett $(s_n)$-t az $(a_n)$ sorozatból képzett numerikus sornak
  mondjuk.

  Azt mondjuk, hogy a $\sum a_n$ sor konvergens, ha az $s_n$ sorozat konvergens,
  továbbá $\sum a_n$ sor divergens, ha $s_n$ sorozat divergens.

  Az $s_n$ sorozat határértékét a $\sum a_n$ sor összegének hívjuk:
  $$
    \lim_{n \rightarrow \infty} s_n = \lim_{n \rightarrow \infty}
    \sum_{i=1}^n a_i=\sum_{i=1}^\infty a_i
    \text.
  $$
\end{definition}

\begin{blueBox}
  \sftitle{Numerikus sorozat konvergencia tesztek:}
  \begin{itemize}
    \item \sftitle{Majoráns kritérium:}
          \begin{center}
            ha $\sum a_n < \sum b_n$ és $\sum b_n$ konvergens,
            akkor $\sum a_n$ is konvergens.
          \end{center}

    \item \sftitle{Minoráns kritérium:}
          \begin{center}
            ha $\sum a_n > \sum b_n$ és $\sum b_n$ divergens,
            akkor $\sum a_n$ is divergens.
          \end{center}

    \item \sftitle{Hányadosteszt:}
          \begin{center}
            ha $\lim\limits_{n \to \infty} \left|\dfrac{a_{n+1}}{a_n}\right| = q$ és $q < 1$,
            akkor $\sum a_n$ konvergens.
          \end{center}

    \item \sftitle{Gyökteszt:}
          \begin{center}
            ha $\lim\limits_{n \to \infty} \sqrt[n]{|a_n|} = q$ és $q < 1$,
            akkor $\sum a_n$ konvergens.
          \end{center}

    \item \sftitle{Integrálkritérium:}
          ha $x \geq 1$ esetén $f(x)$ nemnegatív és csökkenő, akkor
          \begin{center}
            $\sum |f_n|$ konvergens,
            ha $\displaystyle\int_{1}^{\infty} f(x) \dd x$ konvergens.
          \end{center}

    \item \sftitle{Leibniz-sor:}
          \begin{center}
            $\sum (-1)^n a_n$ konvergens, ha $(a_n)$ monoton csökkenő nullsorozat.
          \end{center}
  \end{itemize}
\end{blueBox}

\begin{note}
  A $\sum a_n$ sorozat abszolút konvergens, ha $\sum |a_n|$ is konvergens.

  A $\sum a_n$ sorozat feltételesen konvergens, ha $\sum a_n$ konvergens, de
  $\sum |a_n|$ divergens.
\end{note}

\begin{definition}[Függvénysorozat]
  Az $f_n : I \subset \mathbb R \to \mathbb R$ sorozatot függvénysorozatnak
  nevezzük.
\end{definition}

\begin{note}
  Egy függvénysor értelmezése tartománya azon halmaz, ahol az összes $f_n$
  tagfüggvény értelmezve van:
  $$
    \Domain_f = \bigcap_{n=0}^\infty \Domain_{f_n}
    \text.
  $$
\end{note}

% \begin{example}
%   \sftitle{Példák függvénysorozatokra:}

%   \def\arraystretch{1.33}
%   \begin{tabular}{ll}
%     \bullet \; $f_n: \mathbb R \to [-1; 1]$     & $f_n(x) = \sin nx$ \\
%     \bullet \; $g_n: [0; \infty] \to \mathbb R$ & $g_n(x) = x^n$     \\
%     \bullet \; $h_n: \mathbb R \to \mathbb R$   & $h_n(x) = e^{nx}$  \\
%   \end{tabular}
% \end{example}

\begin{definition}[Függvénysorozat pontbeli konvergenciája]
  Ha az $x_0 \in I$ pontban az $(f_n(x_0))$ számsorozat konvergens, akkor azt
  mondjuk, hogy az $(f_n)$ függvénysorozat konvergens az $x_0$-ban. A
  konvergenciahalmaz:
  $$
    K := \big\{\;
    x \mid x \in I \land (f_n) \text{ konvergens az } x \text{ pontban}
    \;\big\}
    \text.
  $$
\end{definition}

% \begin{example}
%   \sftitle{Példák konvergenciahalmazra}:

%   \def\arraystretch{1.33}
%   \begin{tabular}{ll}
%     \bullet \; $f_n(x) = \sin nx$ & $H_{f_n} = \{ k\pi; k \in \mathbb Z \}$ \\
%     \bullet \; $g_n(x) = x^n$     & $H_{g_n} = [0; 1]$                      \\
%     \bullet \; $h_n(x) = e^{nx}$  & $H_{h_n} = \{0\}$                       \\
%   \end{tabular}
% \end{example}

\begin{definition}[Függvénysorozat határfüggvénye]
  Az $f$ függvényt az $(f_n)$ függvénysorozat határfüggvényének nevezzük:
  $$
    f(x) := \lim_{n \to \infty} f_n(x)
    \text,\quad
    x \in K
    \text.
  $$
  Azt mondjuk, hogy az $(f_n)$ függvénysorozat pontonként konvergál az $f$
  határfüggvényhez a $K$-n, ha $\forall \varepsilon > 0$ esetén
  $\exists N(\varepsilon; x)$, hogy $|f_n(x) - f(x)| < \varepsilon$, ha
  $n > N(\varepsilon; x)$.
\end{definition}

\begin{definition}[Függvénysorozat egyenletes konvergenciája]
  Az  $(f_n)$ egyenletesen konvergens az $E \subset H$ halmazon, ha
  $\forall\varepsilon > 0$ esetén létezik $N(\varepsilon)$ úgy, hogy
  $|f_n(x) - f(x)| < \varepsilon$, ha $n > N(\varepsilon)$ minden $x \in E$
  esetén.
\end{definition}

\begin{blueBox}
  Ha az $(f_n)$ függvénysorozat folytonos és egyenletesen konvergens, akkor
  $$
    \lim_{n \to \infty} \int_a^b f_n(x) \dd x
    = \int_a^b \lim_{n \to \infty} f_n(x) \dd x
    \text.
  $$
\end{blueBox}

\begin{blueBox}
  Ha az $(f_n)$ függvénysorozat folytonos és az $(f_n')$ függvénysorozat
  is folytonos és egyenletesen konvergens, valamint az $(f_n)$ függvénysorozat
  pontonként konvergens, akkor
  $$
    \lim_{n \to \infty} f_n'(x)
    = \left( \lim_{n \to \infty} f_n(x) \right)'
    \text.
  $$
\end{blueBox}

\clearpage
\subsection{Feladatok}
\begin{enumerate}
  \item Konvergensek-e az alábbi numerikus sorok?
        \begin{multicols}{2}
          \begin{enumerate}
            \item $\displaystyle
                    \sum_{n= 1}^{\infty} \frac{(\cos^n (\sfrac{\pi}{2}) )^{4n}}{n^n + 1}
                  $

            \item $\displaystyle
                    \sum_{n = 1}^{\infty} \frac{2n^2}{\left(2 + \sfrac{1}{n}\right)^n}
                  $

            \item $\displaystyle
                    \sum_{n = 1}^{\infty} \frac{1}{\sqrt{n}}\left(1-\frac{1}{n}\right)^n
                  $

            \item $\displaystyle
                    \sum_{n= 0}^{\infty} \frac{n!}{2^n + 1}
                  $

            \item $\displaystyle
                    \sum_{n=1}^{\infty} \frac{n (-1)^{n+1} }{n^2-1}
                  $

            \item $\displaystyle
                    \sum_{n=1}^{\infty} \frac{n}{e^n}
                  $
          \end{enumerate}
        \end{multicols}

  \item Határozza meg az alábbi függvénysorozatok értelmezési tartományát,
        konvergencia tartományát és határfüggvényét!
        \begin{multicols}{2}
          \begin{enumerate}
            \item $
                    f_n(x) = x^n
                  $

            \item $\displaystyle
                    f_n(x) = \frac{x^{n+2}+1}{x^n}
                  $

            \item $\displaystyle
                    f_n(x) = \frac{\sin nx}{n}
                  $

            \item $
                    f_n(x) = (\ln x)^n
                  $

            \item $\displaystyle
                    f_n(x) = n\sin\left(\frac{x}{n}\right)
                  $

            \item $\displaystyle
                    f_n(x) = n\cos\left(\frac{x}{n}\right)
                  $
          \end{enumerate}
        \end{multicols}

  \item Egyenletesen konvergens-e az alábbi függvénysorozat a $(2; 5)$
        intervallumon?
        $$
          f_n(x) = \frac{2x^3n^2}{x^2n^2+5}
        $$

  \item Bizonyítsa be, hogy
        $$
          \lim_{n \to \infty} \int_{0}^{2\pi} \frac{\sin(n^4x^2+3)}{x^2+n^3} \dd x = 0
        $$

  \item Létezik-e az alábbi függvénysorozat deriváltja?
        $$
          f_n(x) = x^2 + \frac{1}{n}\sin\left[n\left(x+\frac{\pi}{2}\right)\right]
        $$

  \item Adja meg az $f_n$ függvénysorozat összegfüggvényét a $[0; 2]$
        intervallumon! Egyenletesen konvergens-e az összegfüggvény a
        konvergencia-intervallumon?
        $$
          f_n = \begin{cases}
            n^2 x \text,        & \text{ha} \quad 0 \leq x \leq \sfrac{1}{n} \; \land \; n \in \mathbb N^+ \\
            \sfrac{1}{x} \text, & \text{ha} \quad \sfrac{1}{n} \leq x \leq 2 \; \land \; n \in \mathbb N^+
          \end{cases}
        $$
\end{enumerate}




\end{document}