\documentclass[a4paper, 12pt, fleqn]{scrartcl}

\usepackage{math-practice}

\area{Többváltozós analízis}
\title{Integrálás II}
\subject{Matematika G2}
\subjectCode{BMETE94BG02}
\date{Utoljára frissítve: \today}
\docno{14}

\begin{document}
\maketitle
\subsection{Elméleti Áttekintő}

\begin{blueBox}
  \sftitle{Integráltranszformációk:}

  Előfordulhat olyan eset, amikor egy adott $A$ tartományon nehéz elvégezni az
  integrálást, de létezik egy olyan koordináta-transzformáció, amely által az
  integrálás egyszerűbbé válik. Ha lézezik egy egyértelmű $\varphi$ leképezés
  az $A$ tartományról az $A'$ tartományra, akkor az integrálás a következő
  módon történik:
  $$
    \iint_A f(\rvec x) \dd \rmat A
    =
    \iint_{A'} f(\rvec x') \left| \rmat J_{\varphi} \right| \dd \rmat A'
    \text,
  $$
  ahol $\rmat J_{\varphi}$ a leképezés Jacobi-mátrixa.

  \sftitle{2D polárkoordináták:}

  \begin{minipage}{.3\textwidth}
    \begin{tikzpicture}[thick]
      % Coordinate system
      \draw[draw = primaryColor, -to]
      (-0.25,0) -- (2.75,0) node[below] {$x$};
      \draw[draw = primaryColor, -to]
      (0,-0.25) -- (0,1.75) node[left] {$y$};

      \coordinate (O) at (0,0);
      \coordinate (X) at (1,0);
      \draw[draw=secondaryColor] (30:2)
      coordinate(P) -- (0,0) node[above, midway, rotate=30] {$r$}
      ;

      \draw[fill=primaryColor] (P) circle(2pt);

      \draw pic[
          "$\varphi$",
          draw = secondaryColor,
          ->,
          angle radius=1.5cm,
          angle eccentricity=.7,
        ] {angle = X--O--P};
    \end{tikzpicture}
  \end{minipage}\begin{minipage}{.25\textwidth}
    $\displaystyle
      \begin{aligned}
        x & = r \cos \varphi \\
        y & = r \sin \varphi
      \end{aligned}
    $
  \end{minipage}\begin{minipage}{.45\textwidth}
    $\displaystyle
      |\rmat J_{\varphi}| = \begin{vmatrix}
        \cos \varphi & -r \sin \varphi \\
        \sin \varphi & r \cos \varphi
      \end{vmatrix} = r
    $
  \end{minipage}

  \sftitle{3D hengerkoordináták:}

  \begin{minipage}{.3\textwidth}
    \begin{tikzpicture}[scale=.8, thick]
      % Source: https://tikz.net/bloch-sphere/
      % Define radius
      \def\r{2}

      % Bloch vector
      \draw (0, 0)
      node[circle, fill, inner sep=1] (orig) {} -- (\r/2, 2*\r/3) coordinate (a)
      ;

      \draw[dashed] (orig) -- (\r/2, -\r/7) coordinate (phi) -- (a);

      \draw[fill=primaryColor] (a) circle(2.5pt);


      % Sphere
      \draw[dashed] (orig) ellipse (\r{} and \r/3);

      % Axes
      % x
      \draw[-to, draw = primaryColor]
      (orig) -- ++(-1.5*\r/5, -1.5*\r/3)
      coordinate (x)
      node[right] {$x$}
      ;
      % y
      \draw[-to, draw = primaryColor]
      (orig) -- ++(1.25*\r, 0)
      coordinate (y)
      node[above] {$y$}
      ;
      % z
      \draw[-to, draw = primaryColor]
      (orig) -- ++(0, 1.25*\r)
      coordinate (z)
      node[right] {$z$}
      ;

      % Angles
      \pic[draw=secondaryColor, text=secondaryColor, -to, "$\varphi$"]
      {angle = x--orig--phi};

      % Radius
      \draw[draw=secondaryColor] (orig) -- (phi) node[above, midway] {$r$};
    \end{tikzpicture}
  \end{minipage}\begin{minipage}{.25\textwidth}
    $\displaystyle
      \begin{aligned}
        x & = r \cos \varphi \\
        y & = r \sin \varphi \\
        z & = z
      \end{aligned}
    $
  \end{minipage}\begin{minipage}{.45\textwidth}
    $\displaystyle
      |\rmat J_{\varphi}| = \begin{vmatrix}
        \cos \varphi & -r \sin \varphi & 0 \\
        \sin \varphi & r \cos \varphi  & 0 \\
        0            & 0               & 1
      \end{vmatrix} = r
    $
  \end{minipage}

  \sftitle{3D gömbkoordináták:}

  \begin{minipage}{.3\textwidth}
    \begin{tikzpicture}[scale=.8, thick]
      % Source: https://tikz.net/bloch-sphere/
      % Define radius
      \def\r{2}

      % Bloch vector
      \draw (0, 0)
      node[circle, fill, inner sep=1] (orig) {} -- (\r/3, \r/2) coordinate (a);

      \draw[dashed] (orig) -- (\r/3, -\r/5) node (phi) {} -- (a);

      % Sphere
      \draw (orig) circle (\r);
      \draw[dashed] (orig) ellipse (\r{} and \r/3);

      \draw[fill=primaryColor] (a) circle(2.5pt);

      % Axes
      % x
      \draw[-to, draw = primaryColor]
      (orig) -- ++(-1.5*\r/5, -1.5*\r/3)
      coordinate (x)
      node[right] {$x$}
      ;
      % y
      \draw[-to, draw = primaryColor]
      (orig) -- ++(1.25*\r, 0)
      coordinate (y)
      node[above] {$y$}
      ;
      % z
      \draw[-to, draw = primaryColor]
      (orig) -- ++(0, 1.25*\r)
      coordinate (z)
      node[right] {$z$}
      ;

      % Angles
      \pic[draw=secondaryColor, text=secondaryColor, -to, "$\varphi$"]
      {angle = x--orig--phi};
      \pic[draw=secondaryColor, text=secondaryColor, to-, "$\vartheta$", angle radius=8mm]
      {angle = a--orig--z};
    \end{tikzpicture}
  \end{minipage}\begin{minipage}{.25\textwidth}
    $\displaystyle
      \begin{aligned}
        x & = r \sin \vartheta \cos \varphi \\
        y & = r \sin \vartheta \sin \varphi \\
        z & = r \cos \vartheta
      \end{aligned}
    $
  \end{minipage}\begin{minipage}{.45\textwidth}
    $\displaystyle
      |\rmat J_{\varphi}| = \begin{vmatrix}
        s_\vartheta c_\varphi & -r s_\vartheta s_\varphi & c_\vartheta c_\varphi \\
        s_\vartheta s_\varphi & r s_\vartheta c_\varphi  & c_\vartheta s_\varphi \\
        c_\vartheta           & 0                        & r
      \end{vmatrix} = r^2 \sin \vartheta
    $

  \end{minipage}
\end{blueBox}

\begin{note}
  Fontos, hogy a Jacobi-mátrix determinánsával való szorzást ne felejtsük el!
\end{note}

\begin{blueBox}
  \sftitle{Görbék megadása:}

  Görbék esetén egy futó változót használunk.

  \sftitle{Egyenes:}
  $$
    \rvec r(t) = \rvec r_0 + t \rvec v
    \text,
  $$
  ahol $r_0$ az egyenes egy pontja, $v$ az egyenes irányvektora, $t \in \Reals$
  pedig a futó paraméter.

  \sftitle{Kör:}
  $$
    \rvec r(t) = \begin{bmatrix}
      R \cdot \cos \varphi \\
      R \cdot \sin \varphi
    \end{bmatrix}
    \text,
  $$
  ahol $r$ a kör sugara, $\varphi \in [0; 2\pi]$ pedig a futó paraméter.
\end{blueBox}

\begin{blueBox}
  \sftitle{Felületek megadása:}

  Felületek esetén két futó paraméterre van szükségünk.

  \sftitle{Körlap:}
  $$
    \rvec \rho (r; \varphi) = \begin{bmatrix}
      r \cos \varphi \\
      r \sin \varphi
    \end{bmatrix}
    \text,
  $$
  ahol $r \in [0; R]$, $\varphi \in [0; 2\pi]$.

  \sftitle{Hengerfelület:}
  $$
    \rvec \rho(t; s) = \rvec r_0(r) + s \rvec v
    \text,
  $$
  ahol $\rvec r_0(t)$ az alapgörbe, $\rvec v$ pedig az irányvektor, $t$,
  $s \in \Reals$ pedig a futó paraméterek.

  \sftitle{Forgásfelület:}
  $$
    \begin{cases}
      \; x(t) = f(t) \\
      \; z(t) = g(t)
    \end{cases}
    \quad \Rightarrow \quad
    \rvec \rho(t;s) = \begin{bmatrix}
      f(t) \cos s \\
      f(t) \sin s \\
      g(t)
    \end{bmatrix}
    \text.
  $$
\end{blueBox}

\clearpage
\subsection{Feladatok}

\begin{enumerate}
  \item Határozza meg az alábbi felületi integrálokat!
        \newcounter{enumTemp}
        \setcounter{enumTemp}{0}
        \newcommand{\nextEnum}{%
          % Step the counter
          \stepcounter{enumTemp}%
          % Negative space
          \hspace{-7mm}%
          % Display the counter with a letter
          \text{\alph{enumTemp})\;\;\;}%
        }

        \vspace{-3mm}
        \begin{minipage}{.6\textwidth}
          \begin{gather*}
            \nextEnum
            \iint_T \frac{1}{\sqrt{x^2 + y^2 + 1}} \dd T
            \\[5mm]
            T = \left\{
            (x; y)
            \mid
            x^2 + y^2 \leq 1
            \;\land\;
            x; y \geq 0
            \right\}
          \end{gather*}
        \end{minipage}\hfill\begin{minipage}{.3\textwidth}
          \begin{center}
            \begin{tikzpicture}[thick]
              % Coordinate system
              \draw[draw = primaryColor, -to]
              (-1.85,0) -- (1.85,0) node[below] {$x$};
              \draw[draw = primaryColor, -to]
              (0,-1.35) -- (0,1.35) node[left] {$y$};

              % r = 1 circle
              \draw[ternaryColor] (0,0) circle (1);

              % First quadrant area dashed
              \draw[
                draw = secondaryColor,
                pattern = north east lines,
                pattern color = secondaryColor,
              ] (0,0) -- (1,0) arc (0:90:1) -- cycle;

              % T area
              \node[
                fill = white,
                fill opacity=.5,
                text opacity=1
              ] at (0.4,0.4) {$T$};

              % Dimentions
              \node[above right=-.5mm] at (1,0) {\scriptsize$1$};
              \node[above right=-.5mm] at (0,1) {\scriptsize$1$};
            \end{tikzpicture}
          \end{center}
        \end{minipage}

        \vspace{-3mm}
        \begin{minipage}{.6\textwidth}
          \begin{gather*}
            \nextEnum
            \iint_T x^2 + y^2 \dd T
            \\[5mm]
            T = \left\{
            (x; y)
            \mid
            (x-3)^2 + (y-2)^2 \leq 1
            \right\}
          \end{gather*}
        \end{minipage}\hfill\begin{minipage}{.3\textwidth}
          \begin{center}
            \begin{tikzpicture}[thick]
              % Coordinate system
              \draw[draw = primaryColor, -to]
              (-0.35,0) -- (3.35,0) node[below] {$x$};
              \draw[draw = primaryColor, -to]
              (0,-0.35) -- (0,2.35) node[left] {$y$};

              % Dimentions
              \draw[dashed, draw=ternaryColor]
              (0, 1.33) node[left] {$2$} -- (1.33, 1.33)
              (2, 0) node[below] {$3$} -- (2, 0.67)
              ;

              % r = 1 circle
              \draw[
                secondaryColor,
                pattern = north east lines,
                pattern color = secondaryColor,
              ] (2, 1.33) circle (.67);

              % T area
              \node[
                fill = white,
                fill opacity=.5,
                text opacity=1
              ] at (2, 1.33) {$T$};
            \end{tikzpicture}
          \end{center}
        \end{minipage}

        \vspace{-3mm}
        \begin{minipage}{.6\textwidth}
          \begin{gather*}
            \nextEnum
            \iint_T |2xy| \dd T
            \\[5mm]
            T = \left\{
            (x; y)
            \;\middle|\;
            \frac{x^2}{9} + \frac{y^2}{4} \leq 1
            \;\land\;
            x; y \geq 0
            \right\}
          \end{gather*}
        \end{minipage}\hfill\begin{minipage}{.3\textwidth}
          \begin{center}
            \begin{tikzpicture}[thick]
              % Coordinate system
              \draw[draw = primaryColor, -to]
              (-1.85,0) -- (1.85,0) node[below] {$x$};
              \draw[draw = primaryColor, -to]
              (0,-1.35) -- (0,1.35) node[left] {$y$};

              % x = 3, y = 2 ellipse
              \draw[
                ternaryColor,
              ] (0,0) ellipse (1.2 and 0.8);

              % Fill the first quadrant
              \draw[
                draw = secondaryColor,
                pattern = north east lines,
                pattern color = secondaryColor,
              ] (0,0) -- (1.2,0) arc (0:90:1.2 and 0.8) -- cycle;

              % T area
              \node[
                fill = white,
                fill opacity=.5,
                text opacity=1
              ] at (0.45, 0.3) {$T$};

              % Dimentions
              \node[above right=-.5mm] at (1.2,0) {\scriptsize$3$};
              \node[above right=-.5mm] at (0,0.8) {\scriptsize$2$};
            \end{tikzpicture}
          \end{center}
        \end{minipage}

        \vspace{-3mm}
        \begin{minipage}{.631\textwidth}
          \begin{gather*}
            \nextEnum
            \iint_T 4xy^3 \dd T
            \\[5mm]
            T = \left\{
            (x; y)
            \;\middle|\;
            1 \leq x^2 + y^2 \leq 4
            \;\land\;
            x \geq 0
            \;\land\;
            y \geq \frac{x}{\sqrt3}
            \right\}
          \end{gather*}
        \end{minipage}\hfill\begin{minipage}{.3\textwidth}
          \begin{center}
            \begin{tikzpicture}[thick]
              % Coordinate system
              \draw[draw = primaryColor, -to]
              (-1.85,0) -- (1.85,0) node[below] {$x$};
              \draw[draw = primaryColor, -to]
              (0,-1.35) -- (0,1.35) node[left] {$y$};

              % r = 1 circle
              \draw[ternaryColor] (0,0) circle (1) circle (.5);

              % First quadrant area dashed
              \draw[
                draw = secondaryColor,
                pattern = north east lines,
                pattern color = secondaryColor,
              ] (30:.5) -- (30:1) arc (30:90:1) -- (90:.5) arc (90:30:.5) -- cycle;

              % T area
              \node[
                fill = white,
                fill opacity=.5,
                text opacity=1,
                inner sep=.5mm,
              ] at (60:.75) {\scriptsize$T$};

              % Dimentions
              \node[above left=-.5mm] at (.5,0) {\scriptsize$1$};
              \node[below right=-.5mm] at (0,.5) {\scriptsize$1$};
              \node[above right=-.5mm] at (1,0) {\scriptsize$2$};
              \node[above right=-.5mm] at (0,1) {\scriptsize$2$};
            \end{tikzpicture}
          \end{center}
        \end{minipage}

        \vspace{-3mm}
        \begin{minipage}{.6\textwidth}
          \begin{gather*}
            \nextEnum
            \iint_T \sin(x^2 + y^2) \dd T
            \\[5mm]
            T = \left\{
            (x; y)
            \;\middle|\;
            x^2 + y^2 \leq 2^2
            \;\land\;
            x; y \geq 0
            \right\}
          \end{gather*}
        \end{minipage}\hfill\begin{minipage}{.3\textwidth}
          \begin{center}
            \begin{tikzpicture}[thick]
              % Coordinate system
              \draw[draw = primaryColor, -to]
              (-1.85,0) -- (1.85,0) node[below] {$x$};
              \draw[draw = primaryColor, -to]
              (0,-1.35) -- (0,1.35) node[left] {$y$};

              % r = 2 circle
              \draw[ternaryColor] (0,0) circle (1);

              % First quadrant area dashed
              \draw[
                draw = secondaryColor,
                pattern = north east lines,
                pattern color = secondaryColor,
              ] (0,0) -- (1,0) arc (0:90:1) -- cycle;

              % T area
              \node[
                fill = white,
                fill opacity=.5,
                text opacity=1
              ] at (0.4,0.4) {$T$};

              % Dimentions
              \node[above right=-.5mm] at (1,0) {\scriptsize$2$};
              \node[above right=-.5mm] at (0,1) {\scriptsize$2$};
            \end{tikzpicture}
          \end{center}
        \end{minipage}

  \item Határozza meg az alábbi improprius integrálok értékét!
        \begin{multicols}{2}
          \begin{enumerate}
            \item $\displaystyle
                    \int\limits_{-\infty}^\infty
                    \int\limits_{-\infty}^\infty
                    2^{-x^2-y^2}
                    \dd x \dd y
                  $

            \item $\displaystyle
                    \int\limits_0^\infty e^{-x^2} \dd x
                  $
          \end{enumerate}
        \end{multicols}

  \item Határozza meg a gömb térfogatát egy hármas-integrál segítségével:

        \begin{minipage}{.6\textwidth}
          \begin{gather*}
            V = \iiint_V \dd V
            \\[5mm]
            V = \left\{
            (x; y; z)
            \;\middle|\;
            x^2 + y^2 + z^2 \leq 1
            \right\}
          \end{gather*}
        \end{minipage}\hfill\begin{minipage}{.3\textwidth}
          \begin{center}
            \begin{tikzpicture}[scale=.75, thick]
              % Source: https://tikz.net/bloch-sphere/
              % Define radius
              \def\r{2}

              % Bloch vector
              \draw (0, 0)
              node[circle, fill, inner sep=1] (orig) {} -- (\r/3, \r/2)
              node[circle, fill, inner sep=0.7, label=above:$\rvec x$] (a) {};

              \draw[dashed] (orig) -- (\r/3, -\r/5) node (phi) {} -- (a);

              % Sphere
              \draw (orig) circle (\r);
              \draw[dashed] (orig) ellipse (\r{} and \r/3);

              % Axes
              % x
              \draw[-to, draw = primaryColor]
              (orig) -- ++(-1.5*\r/5, -1.5*\r/3)
              coordinate (x)
              node[right] {$x$}
              ;
              % y
              \draw[-to, draw = primaryColor]
              (orig) -- ++(1.25*\r, 0)
              coordinate (y)
              node[above] {$y$}
              ;
              % z
              \draw[-to, draw = primaryColor]
              (orig) -- ++(0, 1.25*\r)
              coordinate (z)
              node[right] {$z$}
              ;

              % Angles
              \pic[draw=secondaryColor, text=secondaryColor, -to, "$\varphi$"]
              {angle = x--orig--phi};
              \pic[draw=secondaryColor, text=secondaryColor, to-, "$\vartheta$", angle radius=8mm]
              {angle = a--orig--z};
            \end{tikzpicture}
          \end{center}
        \end{minipage}

  \item Adja meg az alábbi integrál értékét, ha $V$ az $r = 1$ sugarú, $z = 0$
        és $z = 2$ síkok által határolt $z$-tengellyel párhuzamos
        szimmetriavonalú henger a tér első térnyolcadában elhelyezkedő része!

        \begin{minipage}{.65\textwidth}
          \begin{gather*}
            \phantom{V =} \iiint_V  z \cdot \sqrt{x^2 + y^2} \dd V
            \\[5mm]
            V = \left\{
            (x; y; z)
            \;\middle|\;
            x^2 + y^2 \leq 1
            \;\land\;
            0 \leq z \leq 2
            \;\land\;
            x; y \geq 0
            \right\}
          \end{gather*}
        \end{minipage}\hfill\begin{minipage}{.275\textwidth}
          \begin{center}
            \begin{tikzpicture}[scale=.75, thick]
              % Source: https://tikz.net/bloch-sphere/
              % Define radius
              \def\r{2}

              % Bloch vector
              \draw (0, 0)
              node[circle, fill, inner sep=1] (orig) {} -- (\r/2, 2*\r/3) coordinate (a)
              ;

              \draw[dashed] (orig) -- (\r/2, -\r/7) coordinate (phi) -- (a);

              \draw[fill=primaryColor] (a) circle(2.5pt);


              % Sphere
              \draw[dashed] (orig) ellipse (\r{} and \r/3);

              % Axes
              % x
              \draw[-to, draw = primaryColor]
              (orig) -- ++(-1.5*\r/5, -1.5*\r/3)
              coordinate (x)
              node[right] {$x$}
              ;
              % y
              \draw[-to, draw = primaryColor]
              (orig) -- ++(1.25*\r, 0)
              coordinate (y)
              node[above] {$y$}
              ;
              % z
              \draw[-to, draw = primaryColor]
              (orig) -- ++(0, 1.25*\r)
              coordinate (z)
              node[right] {$z$}
              ;

              % Angles
              \pic[draw=secondaryColor, text=secondaryColor, -to, "$\varphi$"]
              {angle = x--orig--phi};

              % Radius
              \draw[draw=secondaryColor] (orig) -- (phi) node[above, midway] {$r$};
            \end{tikzpicture}
          \end{center}
        \end{minipage}

  \item Határozza meg az $f(x; y; z) = \sqrt{x^2 + y^2 + z^2}$ függvény
        egységgömbön vett integrálját!

  \item Határozza meg az $\rvec r(t)$ vezérgörbéjű $\rvec v$ irányvektorú
        hengerefelületet!
        $$
          \rvec v(t) = \begin{bmatrix}
            t^2 + 1 \\
            5t      \\
            3t-2
          \end{bmatrix}
          \qquad
          \rvec v = \begin{bmatrix}
            1 \\
            2 \\
            3
          \end{bmatrix}
        $$

  \item Adja meg a $2$ sugarú, $z$ tengellyel párhuzamos szimmetriavonalú,
        origó központú körhenger felületének egyenletét!

  \item Adja meg azon tórusz felületének egyenletét, amelyet az $xz$ síkban
        elhelyezkedő, $(x; z) = (5; 0)$ középpontú, $r = 3$ sugarú kör $z$
        tengely körül való forgatásával kapunk!
\end{enumerate}


\end{document}
