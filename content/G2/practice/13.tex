\documentclass[a4paper, 12pt, fleqn]{scrartcl}

\usepackage{math-practice}
\usepackage{tabto}

\area{Többváltozós analízis}
\title{Integrálás I}
\subject{Matematika G2}
\subjectCode{BMETE94BG02}
\date{Utoljára frissítve: \today}
\docno{13}

\pgfdeclarelayer{ft}
\pgfdeclarelayer{bg}
\pgfsetlayers{bg,main,ft}

\begin{document}
\maketitle
\subsection{Elméleti Áttekintő}

\begin{blueBox}
  \sftitle{Integrálás téglatartományon:}

  Téglatartomány esetén az integrálás sorrendje tetszőleges.

  \begin{minipage}{.4\textwidth}
    \begin{center}
      \begin{tikzpicture}[thick, scale=5/4]
        % Coordinate system
        \draw[draw = primaryColor, -to]
        (-0.35,0) -- (3.35,0) node[below] {$x$};
        \draw[draw = primaryColor, -to]
        (0,-0.35) -- (0,2.35) node[left] {$y$};

        % Rectangle
        \draw[
          draw=secondaryColor,
          pattern = north east lines,
          pattern color = secondaryColor,
        ] (.75,.5) rectangle (2.75,1.75);

        % Dimentions
        \draw[draw=ternaryColor, dashed]
        (.75,.5) -- (0,.5) node[left] {$y_1$}
        (.75,1.75) -- (0,1.75) node[left] {$y_2$}
        (.75,.5) -- (.75,0) node[below] {$x_1$}
        (2.75,.5) -- (2.75,0) node[below] {$x_2$}
        ;
      \end{tikzpicture}
    \end{center}
  \end{minipage}\begin{minipage}{.6\textwidth}
    \begin{equation*}
      I
      = \int\limits_{x_1}^{x_2} \int\limits_{y_1}^{y_2} f(x;y) \dd y \dd x
      = \int\limits_{y_1}^{y_2} \int\limits_{x_1}^{x_2} f(x;y) \dd x \dd y
    \end{equation*}
  \end{minipage}
\end{blueBox}

\begin{blueBox}
  \sftitle{Integrálás normáltartományon:}

  \begin{minipage}{.4\textwidth}
    \begin{center}
      \begin{tikzpicture}[thick, scale=5/4]
        % Coordinate system
        \draw[draw = primaryColor, -to]
        (-0.35,0) -- (3.35,0) node[below] {$x$};
        \draw[draw = primaryColor, -to]
        (0,-0.35) -- (0,2.35) node[left] {$y$};

        % Dimentions
        \draw[draw=ternaryColor, dashed]
        (.75,2) -- (.75,0) node[below] {$x_1$}
        (2.75,2) -- (2.75,0) node[below] {$x_2$}
        ;

        % Lower boundary
        \draw[draw = secondaryColor, name path = f1]
        plot[domain=0.75:2.75, smooth, samples=150]
        ({\x}, {0.25+0.125*sin(540*\x)+\x*0.125})
        node[right] {$g_1(x)$}
        ;

        % Upper boundary
        \draw[draw = secondaryColor,name path = f2]
        plot[domain=0.75:2.75, smooth, samples=150]
        ({\x}, {1.75+0.125*sin(540*\x)-\x*0.125})
        node[right] {$g_2(x)$}
        ;

        % Fill between boundaries
        \tikzfillbetween[
          of=f1 and f2,
          on layer = bg,
        ]{
          pattern = north east lines,
          pattern color = secondaryColor,
        }
      \end{tikzpicture}
    \end{center}
  \end{minipage}\begin{minipage}{.6\textwidth}
    \begin{equation*}
      I
      = \int\limits_{x_1}^{x_2} \int\limits_{g_1(x)}^{g_2(x)} f(x;y) \dd y \dd x
      = \int\limits_{y_1}^{y_2} \int\limits_{g^{-1}_1(y)}^{g^{-1}_2(y)} f(x;y) \dd x \dd y
    \end{equation*}
  \end{minipage}

  \begin{minipage}{.4\textwidth}
    \begin{center}
      \begin{tikzpicture}[thick, scale=5/4]
        % Coordinate system
        \draw[draw = primaryColor, -to]
        (-0.35,0) -- (3.35,0) node[below] {$x$};
        \draw[draw = primaryColor, -to]
        (0,-0.35) -- (0,2.35) node[left] {$y$};

        % Dimentions
        \draw[draw=ternaryColor, dashed]
        (3,.5) -- (0,.5) node[left] {$y_1$}
        (3,1.75) -- (0,1.75) node[left] {$y_2$}
        ;

        % Lower boundary
        \draw[draw = secondaryColor, name path = f1]
        plot[domain=1.75:0.5, smooth, samples=150]
        ({0.75+0.125*sin(540*\x)+\x*0.125}, {\x})
        ;
        \node[above] at (1,1.75) {$g_1(y)$};

        % Upper boundary        
        \draw[draw = secondaryColor,name path = f2]
        plot[domain=0.5:1.75, smooth, samples=150]
        ({2.75+0.125*sin(540*\x)-\x*0.125}, {\x})
        node[above] {$g_2(y)$}
        ;

        % Fill between boundaries
        \tikzfillbetween[
          of=f1 and f2,
          on layer = bg,
        ]{
          pattern = north east lines,
          pattern color = secondaryColor,
        }
      \end{tikzpicture}
    \end{center}
  \end{minipage}\begin{minipage}{.6\textwidth}
    \begin{equation*}
      I
      = \int\limits_{y_1}^{y_2} \int\limits_{g_1(y)}^{g_2(y)} f(x;y) \dd x \dd y
      = \int\limits_{x_1}^{x_2} \int\limits_{g^{-1}_1(x)}^{g^{-1}_2(x)} f(x;y) \dd y \dd x
    \end{equation*}
  \end{minipage}
\end{blueBox}

\clearpage
\subsection{Feladatok}

\begin{enumerate}
  \item Számolja ki az alábbi függvények integrálját a megadott tégla
        tartományokon!
        \begin{enumerate}
          \item $f(x; y) = 2x^2 + 3xy + 4y^2$
                \tabto{6cm} $1 \leq x \leq 2$
                \tabto{9cm} $0 \leq y \leq 3$

          \item $g(x; y) = xy\sin(x^2 + y^2)$
                \tabto{6cm} $0 \leq x \leq \sfrac{\pi}{2}$
                \tabto{9cm} $0 \leq y \leq \sfrac{\pi}{2}$

          \item $h(x; y) = y\cos(2xy)$
                \tabto{6cm} $1 \leq x \leq 2$
                \tabto{9cm} $1 \leq y \leq 3$
        \end{enumerate}

  \item Határozza meg az alábbi hármasintergrált!
        \begin{equation*}
          \int_{1}^{2}
          \int_{0}^{3}
          \int_{0}^{1}
          z \, x \, \sqrt{x^2+y}
          \dd x \dd y \dd z
        \end{equation*}

  \item Határozza meg az alábbi integrálokat, majd írja fel az integrációs
        határokat, ha először az $x$, majd a $y$ változó szerint integrálnánk!
        \begin{multicols}{2}
          \begin{enumerate}
            \item $\displaystyle
                    \int_{0}^{1} \int_{x^2}^{\sqrt{x}}
                    \left(x^2 + y^2\right)
                    \dd y \dd x
                  $

            \item $\displaystyle
                    \int_{1}^{3} \int_{0}^{1/x}
                    \left(2y + x + 2\right)
                    \dd y \dd x
                  $
          \end{enumerate}
        \end{multicols}

  \item Határozza meg az alábbi felületi integrálok értékét!
        \newcounter{enumTemp}
        \setcounter{enumTemp}{0}
        \newcommand{\nextEnum}{%
          % Step the counter
          \stepcounter{enumTemp}%
          % Negative space
          \hspace{-7mm}%
          % Display the counter with a letter
          \text{\alph{enumTemp})\;\;\;}%
        }

        \vspace{-3mm}
        \begin{minipage}{.6\textwidth}
          \begin{gather*}
            \nextEnum
            \iint_T x^2 + y^2 \dd T
            \\[5mm]
            T = \left\{
            (x;y)
            \;\middle|\;
            0 \leq y \leq 1
            \; \land \;
            y \leq x \leq 3 - y
            \right\}
          \end{gather*}
        \end{minipage}\hfill\begin{minipage}{.3\textwidth}
          \begin{center}
            \begin{tikzpicture}[thick]
              % Coordinate system
              \draw[draw = primaryColor, -to]
              (-0.35,0) -- (3.35,0) node[below] {$x$};
              \draw[draw = primaryColor, -to]
              (0,-0.85) -- (0,1.85) node[left] {$y$};

              % Dimentions
              \draw[draw=ternaryColor, dashed]
              (1,1) -- (0,1) node[left] {\scriptsize$1$}
              (1,1) -- (1,0) node[below] {\scriptsize$1$}
              (2,1) -- (2,0) node[below] {\scriptsize$2$}
              ;
              \node[below] at (3,0) {\scriptsize$3$};
              \node[below right] at (0,0) {\scriptsize$0$};

              % Area
              \draw[
                draw = secondaryColor,
                pattern = north east lines,
                pattern color = secondaryColor,
              ] (0,0) -- (1,1) -- (2,1) -- (3,0) -- cycle;

              % T area
              \node[
                fill = white,
                fill opacity=.5,
                text opacity=1
              ] at (1.5,0.5) {$T$};
            \end{tikzpicture}
          \end{center}
        \end{minipage}

        \vspace{-3mm}
        \begin{minipage}{.6\textwidth}
          \begin{gather*}
            \nextEnum
            \iint_T x y \dd T
            \\[5mm]
            T = \left\{
            (x; y)
            \;\middle|\;
            x^2 + y^2 \leq R^2
            \; \land \;
            x; y \geq 0
            \right\}
          \end{gather*}
        \end{minipage}\hfill\begin{minipage}{.3\textwidth}
          \begin{center}
            \begin{tikzpicture}[thick]
              % Coordinate system
              \draw[draw = primaryColor, -to]
              (-1.85,0) -- (1.85,0) node[below] {$x$};
              \draw[draw = primaryColor, -to]
              (0,-1.35) -- (0,1.35) node[left] {$y$};

              % r = 1 circle
              \draw[ternaryColor] (0,0) circle (1);

              % First quadrant area dashed
              \draw[
                draw = secondaryColor,
                pattern = north east lines,
                pattern color = secondaryColor,
              ] (0,0) -- (1,0) arc (0:90:1) -- cycle;

              % T area
              \node[
                fill = white,
                fill opacity=.5,
                text opacity=1
              ] at (0.4,0.4) {$T$};

              % Dimentions
              \node[above right=-.5mm] at (1,0) {\scriptsize$R$};
              \node[above right=-.5mm] at (0,1) {\scriptsize$R$};
            \end{tikzpicture}
          \end{center}
        \end{minipage}

        \vspace{-3mm}
        \begin{minipage}{.634\textwidth}
          \begin{gather*}
            \nextEnum
            \iint_T y \, e^{(x-1)^2} \dd T
            \\[5mm]
            T = \left\{
            (x; y)
            \;\middle|\;
            x \leq 1 + y^2
            \; \land \;
            y \geq 0
            \; \land \;
            x \leq 5
            \right\}
          \end{gather*}
        \end{minipage}\hfill\begin{minipage}{.3\textwidth}
          \begin{center}
            \begin{tikzpicture}[thick]
              % Coordinate system
              \draw[draw = primaryColor, -to]
              (-0.85,0) -- (2.85,0) node[below] {$x$};
              \draw[draw = primaryColor, -to]
              (0,-0.85) -- (0,1.85) node[left] {$y$};

              % r = 1 circle
              % \draw[ternaryColor] (0,0) circle (1);

              % First quadrant area dashed
              \draw[
                draw = secondaryColor,
                pattern = north east lines,
                pattern color = secondaryColor,
                scale=1/2,
              ]
              plot[domain=0:2, smooth, samples=150]
              ({1+\x*\x}, {\x}) coordinate(T) |- (0,0);

              % T area
              \node[
                fill = white,
                fill opacity=.5,
                text opacity=1
              ] at (1.7,0.35) {$T$};

              % Dimentions
              \node[below] at (2.5,0) {\scriptsize$5$};
              \node[below] at (0.5,0) {\scriptsize$1$};
              \draw[draw=ternaryColor, dashed]
              (T) -- ++(-2.5,0) node[left] {\scriptsize$2$};
            \end{tikzpicture}
          \end{center}
        \end{minipage}

  \item Adja meg az integrációs intervallumokat, ha az alábbi felületeken
        kell integrálni:
        \begin{enumerate}
          \item $R$ sugarú körfelület,
          \item ha az $x = 0$, $y = x^2$ és $y = 2 - x$ görbék által határolt
                felület!
        \end{enumerate}

  \item Adja meg a $f(x; y) = xy$ függvény a $P_1(1;1)$, $P_2(4;5)$ és
        $P_3(4;2)$ pontok által meghatározott háromszög terület fölötti
        integrálját!
\end{enumerate}


\end{document}
