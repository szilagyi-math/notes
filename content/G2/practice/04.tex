\documentclass[a4paper, 12pt]{scrartcl}

\usepackage{math-practice}

\area{Lineáris Algebra}
\title{Lineáris leképzések I}
\subject{Matematika G2}
\subjectCode{BMETE94BG02}
\date{Utoljára frissítve: \today}
\docno{4}

\begin{document}
\maketitle
\subsection{Elméleti Áttekintő}

\begin{definition}[Lineáris leképezés]
  Legyenek $V_1$ és $V_2$ ugyanazon $T$ test feletti vektorterek. Legyen
  $\varphi: V_1 \rightarrow V_2$ leképezés, melyet lineáris leképezésnek
  nevezünk, ha tetszőleges két $V_1$-beli vektor ($\forall \rvec a; \rvec b \in
    V_1$) és $T$-beli skalár ($\lambda \in T$) esetén teljesülnek az alábbiak:

  \def\arraystretch{1.5}
  \begin{tabular}{>{\bullet\;}l>{$\quad \sim \quad$}l}
    $\varphi(\rvec a + \rvec b) = \varphi(\rvec a) + \varphi(\rvec b)$
     & additív (összegre tagonként hat), \\
    $\varphi(\lambda \rvec a) = \lambda \varphi(\rvec a)$
     & homogén (skalár kiemelhető).
  \end{tabular}
\end{definition}

\begin{definition}[Leképezés magtere]
  Legyen $\varphi: V_1 \rightarrow V_2$ lineáris leképezés, ekkor a
  $$
    \ker \varphi := \big\{\;
    \rvec v \; \big| \; \rvec v \in V_1 \land \varphi(\rvec v) = \nvec
    \;\big\}
  $$
  halmazt a leképezés magterének nevezzük.
\end{definition}

\begin{definition}[Leképezés defektusa]
  A magtér dimenzióját defektusnak nevezzük, és $\operatorname{def} \varphi$-vel
  jelöljük.
\end{definition}

\begin{note}
  Nem létezik olyan vektortér, melynek magtere az üreshalmaz
  (a nullvektor mindig benne van, mert a nullvektor képe mindig nullvektor).
\end{note}

\begin{note}
  Invertálható lineáris leképezés magtere a nullvektor.
\end{note}

\begin{definition}[Lineáris leképezés rangja]
  Egy lineáris leképezés rangjának nevezzük a képtér dimenzióját.
  $\rg \varphi = \dim \varphi(V_1)$.
\end{definition}

\begin{theorem}[Rang-nullitás tétele]
  Legyen $V_1$ véges dimenziós vektortér, $\varphi: V_1 \rightarrow V_2$
  lineáris leképezés, ekkor
  $$
    \rg \varphi + \operatorname{def} \varphi = \dim V_1
    \text.
  $$
\end{theorem}

\begin{blueBox}
  \sftitle{Lineáris leképezések mátrixreprezentációja:}

  Legyenek $V_1$ és $V_2$ ugyanazon test feletti vektorterek, és $\dim V_1 = n$,
  valamint $\dim V_2 = k$. Legyen $\{ \rvec a_1; \rvec a_2; \ldots; \rvec a_n \}$
  bázis $V_1$-ben, és $\{ \rvec b_1; \rvec b_2; \ldots; \rvec b_n \}$ bázis
  $V_2$-ben. Legyen $\varphi: V_1 \rightarrow V_2$ lineáris leképezés, ekkor
  $$
    \varphi(\rvec a_i)
    = \alpha_{1i} \rvec b_1
    + \alpha_{2i} \rvec b_2
    + \ldots
    + \alpha_{ki} \rvec b_k
    = \sum_{j=1}^{k} \alpha_{ji} \rvec b_j
    \quad\Rightarrow\quad
    \rmat A := \begin{bmatrix}
      \alpha_{11} & \alpha_{2i} & \cdots & \alpha_{1n} \\
      \alpha_{21} & \alpha_{2i} & \cdots & \alpha_{2n} \\
      \vdots      & \vdots      & \ddots & \vdots      \\
      \alpha_{k1} & \alpha_{ki} & \cdots & \alpha_{kn}
    \end{bmatrix}_{k \times n}
    \text.
  $$
  Az $\rmat A$ mátrixot $\varphi$ leképezést reprezentáló mátrixnak hívjuk,
  segítségével tetszőleges $\rvec x \in V_1$ képét meghatározhatjuk. Legyenek
  $(\xi_1, \xi_2, \ldots, \xi_n)$ az $\rvec x$ koordinátái, ekkor a képét az
  alábbi módon számíthatjuk:
  $$
    \varphi(\rvec x)
    = \varphi\left( \sum_{i=1}^{n} \xi_i \rvec a_i \right)
    = \sum_{i=1}^{n} \xi_i \varphi(\rvec a_i)
    = \begin{bmatrix}
      \alpha_{11} & \alpha_{2i} & \cdots & \alpha_{1n} \\
      \alpha_{21} & \alpha_{2i} & \cdots & \alpha_{2n} \\
      \vdots      & \vdots      & \ddots & \vdots      \\
      \alpha_{k1} & \alpha_{ki} & \cdots & \alpha_{kn}
    \end{bmatrix} \begin{bmatrix}
      \xi_1  \\
      \xi_2  \\
      \vdots \\
      \xi_n
    \end{bmatrix}
    \text.
  $$
\end{blueBox}

\begin{definition}[Bázistranszformáció]
  Legyenek $\{ \rvec b_1; \rvec b_2; \ldots; \rvec b_n \}$ és
  $\{ \hat{\rvec b}_1; \hat{\rvec b}_2; \ldots; \hat{\rvec b}_n \}$ bázisok
  $V$-ben. Ekkor a $\{ \rvec b_1; \rvec b_2; \ldots; \rvec b_n \} \rightarrow
    \{ \hat{\rvec b}_1; \hat{\rvec b}_2; \ldots; \hat{\rvec b}_n \}$
  bázistranszformáció $\rmat T$ mátrixa a következőképpen írható fel:
  $$
    \left.\begin{array}{rl}
      \hat{\rvec b}_1 & = t_{11} \rvec b_1 + t_{21} \rvec b_2 + \ldots + t_{n1} \rvec b_n \\
      \hat{\rvec b}_2 & = t_{12} \rvec b_1 + t_{22} \rvec b_2 + \ldots + t_{n2} \rvec b_n \\
                      & \vdots                                                            \\
      \hat{\rvec b}_j & = t_{1j} \rvec b_1 + t_{2j} \rvec b_2 + \ldots + t_{nj} \rvec b_n \\
                      & \vdots                                                            \\
      \hat{\rvec b}_n & = t_{1n} \rvec b_1 + t_{2n} \rvec b_2 + \ldots + t_{nn} \rvec b_n
    \end{array}\right\}
    \quad\Rightarrow\quad
    \rmat T = \begin{bmatrix}
      t_{11} & t_{12} & \cdots & t_{1n} \\
      t_{21} & t_{22} & \cdots & t_{2n} \\
      \vdots & \vdots & \ddots & \vdots \\
      t_{n1} & t_{n2} & \cdots & t_{nn}
    \end{bmatrix}
  $$
\end{definition}

\begin{note}
  A $\rmat T$ bázistranszformációs mátrix segítségével a régi és új bázisban
  felírt vektorok kordinátái közötti kapcsolat mátrixosan:
  $$
    \rvec x = \rmat T \rvec x'
    \quad\text{és}\quad
    \rvec x' = \rmat T^{-1} \rvec x
    \text.
  $$
\end{note}

\begin{theorem}[Lineáris leképezés mátrixa új bázisban]
  Legyen $\varphi: V \rightarrow V$ lineáris leképezés,
  $\{ \rvec b_1; \rvec b_2; \ldots; \rvec b_n \}$ és $\{ \hat{\rvec b}_1;
    \hat{\rvec b}_2; \ldots; \hat{\rvec b}_n \}$ bázisok $V$-ben. A
  $\varphi$ $\{ \rvec b_1; \rvec b_2; \ldots \rvec b_n \}$ bázisra vonatkozó
  mátrixa $\rmat A$, a $\varphi$ $\{ \hat{\rvec b}_1; \hat{\rvec b}_2; \ldots;
    \hat{\rvec b}_n \}$ bázisra vonatkozó mátrixa $\hat{\rmat A}$. Jelölje
  $\rmat T$ a $\{ \rvec b_1; \rvec b_2; \ldots; \rvec b_n \}$ bázisról a
  $\{ \hat{\rvec b}_1; \hat{\rvec b}_2; \ldots; \hat{\rvec b}_n \}$ bázisra
  való áttérés mátrixát, ekkor
  $$
    \hat{\rmat A} = \rmat T^{-1} \rmat A \rmat T
    \text.
  $$
\end{theorem}

\begin{note}
  A $\rmat A$ és $\hat{\rmat A}$ mátrix hasonló.
\end{note}

\begin{blueBox}
  \sftitle{Alap geometriai leképezések:}

  \begin{itemize}
    \item \textbf{Tükrözés} valamely tengelyre:
          \vspace{-11mm}
          \begin{multicols}{3}
            \centering
            $$
              \rmat T_x = \begin{bmatrix}
                1 & 0  & 0  \\
                0 & -1 & 0  \\
                0 & 0  & -1
              \end{bmatrix}
              \phantom{= \rmat T_x}
            $$
            $x$-tengelyre tükrözés

            $$
              \rmat T_y = \begin{bmatrix}
                -1 & 0 & 0  \\
                0  & 1 & 0  \\
                0  & 0 & -1
              \end{bmatrix}
              \phantom{= \rmat T_y}
            $$
            $y$-tengelyre tükrözés

            $$
              \rmat T_z = \begin{bmatrix}
                -1 & 0  & 0 \\
                0  & -1 & 0 \\
                0  & 0  & 1
              \end{bmatrix}
              \phantom{= \rmat T_z}
            $$
            $z$-tengelyre tükrözés
          \end{multicols}
          \vspace{-3mm}

    \item \textbf{Vetítés} valamely tengelyre:
          \vspace{-11mm}
          \begin{multicols}{3}
            \centering
            $$
              \rmat T_x = \begin{bmatrix}
                1 & 0 & 0 \\
                0 & 0 & 0 \\
                0 & 0 & 0
              \end{bmatrix}
              \phantom{= \rmat T_x}
            $$
            $x$-tengelyre vetítés

            $$
              \rmat T_y = \begin{bmatrix}
                0 & 0 & 0 \\
                0 & 1 & 0 \\
                0 & 0 & 0
              \end{bmatrix}
              \phantom{= \rmat T_y}
            $$
            $y$-tengelyre vetítés

            $$
              \rmat T_z = \begin{bmatrix}
                0 & 0 & 0 \\
                0 & 0 & 0 \\
                0 & 0 & 1
              \end{bmatrix}
              \phantom{= \rmat T_z}
            $$
            $z$-tengelyre vetítés
          \end{multicols}
          \vspace{-3mm}

    \item \textbf{Tükrözés} valamely síkra:
          \vspace{-11mm}
          \begin{multicols}{3}
            \centering
            $$
              \rmat T_{xy} = \begin{bmatrix}
                1 & 0 & 0  \\
                0 & 1 & 0  \\
                0 & 0 & -1
              \end{bmatrix}
              \phantom{= \rmat T_{xy}}
            $$
            $xy$ síkra tükrözés

            $$
              \rmat T_{yz} = \begin{bmatrix}
                -1 & 0 & 0 \\
                0  & 1 & 0 \\
                0  & 0 & 1
              \end{bmatrix}
              \phantom{= \rmat T_{yz}}
            $$
            $yz$ síkra tükrözés

            $$
              \rmat T_{xz} = \begin{bmatrix}
                1 & 0  & 0 \\
                0 & -1 & 0 \\
                0 & 0  & 1
              \end{bmatrix}
              \phantom{= \rmat T_{xz}}
            $$
            $xz$ síkra tükrözés
          \end{multicols}
          \vspace{-3mm}

    \item \textbf{Vetítés} valamely síkra:
          \vspace{-11mm}
          \begin{multicols}{3}
            \centering
            $$
              \rmat T_{xy} = \begin{bmatrix}
                1 & 0 & 0 \\
                0 & 1 & 0 \\
                0 & 0 & 0
              \end{bmatrix}
              \phantom{= \rmat T_{xy}}
            $$
            $xy$ síkra vetítés

            $$
              \rmat T_{yz} = \begin{bmatrix}
                0 & 0 & 0 \\
                0 & 1 & 0 \\
                0 & 0 & 1
              \end{bmatrix}
              \phantom{= \rmat T_{yz}}
            $$
            $yz$ síkra vetítés

            $$
              \rmat T_{xz} = \begin{bmatrix}
                1 & 0 & 0 \\
                0 & 0 & 0 \\
                0 & 0 & 1
              \end{bmatrix}
              \phantom{= \rmat T_{xz}}
            $$
            $xz$ síkra vetítés
          \end{multicols}
          \vspace{-3mm}

    \item $\lambda$-szoros \textbf{nyújtás} valamely irányban:
          \vspace{-11mm}
          \begin{multicols}{3}
            \centering
            $$
              \rmat T_{x} = \begin{bmatrix}
                \lambda & 0 & 0 \\
                0       & 1 & 0 \\
                0       & 0 & 1
              \end{bmatrix}
              \phantom{= \rmat T_{x}}
            $$
            $x$ irányba

            $$
              \rmat T_{y} = \begin{bmatrix}
                1 & 0       & 0 \\
                0 & \lambda & 0 \\
                0 & 0       & 1
              \end{bmatrix}
              \phantom{= \rmat T_{y}}
            $$
            $y$ irányba

            $$
              \rmat T_{z} = \begin{bmatrix}
                1 & 0 & 0       \\
                0 & 1 & 0       \\
                0 & 0 & \lambda
              \end{bmatrix}
              \phantom{= \rmat T_{z}}
            $$
            $z$ irányba
          \end{multicols}
          \vspace{-3mm}

    \item \textbf{Forgatás} $+\alpha$ szöggel:
          \begin{alignat*}{9}
            \rmat R_{x}(\alpha)
             & = \begin{bmatrix}
                   1 & 0           & 0            \\
                   0 & \cos \alpha & -\sin \alpha \\
                   0 & \sin \alpha & \cos \alpha
                 \end{bmatrix}
             & \quad \sim \quad
             & \text{$x$ tengely körüli forgatás} \hspace{3.25cm}
            \\
            \rmat R_{y}(\alpha)
             & = \begin{bmatrix}
                   \cos \alpha  & 0 & \sin \alpha \\
                   0            & 1 & 0           \\
                   -\sin \alpha & 0 & \cos \alpha
                 \end{bmatrix}
             & \quad \sim \quad
             & \text{$y$ tengely körüli forgatás}
            \\
            \rmat R_{z}(\alpha)
             & = \begin{bmatrix}
                   \cos \alpha & -\sin \alpha & 0 \\
                   \sin \alpha & \cos \alpha  & 0 \\
                   0           & 0            & 1
                 \end{bmatrix}
             & \quad \sim \quad
             & \text{$z$ tengely körüli forgatás}
          \end{alignat*}
  \end{itemize}
\end{blueBox}

\begin{note}
  Ha egymás után több transzformációt kell végrehajtani $\rmat A$,
  $\rmat B$, $\rmat C$ sorrendben, akkor:
  $$
    \rvec x' = \rmat C \rmat B \rmat A \rvec x
    \text.
  $$
\end{note}

\begin{definition}[Ortogonális transzformáció]
  Az $n$ dimenziós  euklideszi tér $\mathcal A: V \rightarrow V$ lineáris
  transzformációját ortogonálisnak mondjuk, ha $\langle \mathcal A \rvec x;
    \mathcal A \rvec y \rangle = \langle \rvec x; \rvec y \rangle$, minden $\rvec
    x; \rvec y \in V$ esetén.
\end{definition}

\begin{note}
  Egy ortogonális transzformáció $\rmat Q$ mátrixának inverze megegyezik a
  transzponáltjával.

  Amennyiben $\det \rmat Q = 1$, akkor a transzformáció orientációtartó.

  Amennyiben $\det \rmat Q = -1$, akkor a transzformáció orientációváltó.
\end{note}

\begin{example}
  A két dimenziós térben való forgatás orientációtartó, hiszen
  $$
    \det \rmat Q
    = \begin{vmatrix}
      \cos \alpha & -\sin \alpha \\
      \sin \alpha & \cos \alpha
    \end{vmatrix}
    = \cos^2 \alpha + \sin^2 \alpha
    = 1
    \text.
  $$
\end{example}

\clearpage
\subsection{Feladatok}
\begin{enumerate}
  \item Állapítsa meg, hogy az alábbi leképezések lineárisak-e?
        $$
          \varphi: \Reals^2 \to \Reals^2;
          \quad
          \begin{bmatrix}
            x \\
            y
          \end{bmatrix} \mapsto \begin{bmatrix}
            x+y \\
            5xy
          \end{bmatrix}
          \hspace{2cm}
          \psi: \Reals^2 \to \Reals^3;
          \quad
          \begin{bmatrix}
            x \\
            y
          \end{bmatrix} \mapsto \begin{bmatrix}
            x \\
            y \\
            x+y
          \end{bmatrix}
        $$

  \item Határozza meg a $P(5; -4; -1)$ pont koordinátáit az $\rvec a_1(2; 1;0)$,
        $\rvec a_2(0; 2; 1)$ és $\rvec a_3(1; 0; 2)$ vektorok által
        meghatározott bázisban!

  \item Írja fel az $\{ \rvec i, \rvec j, \rvec k \}$ és a $\{ \rvec z_1,
          \rvec z_2, \rvec z_3\}$ ortonormált bázisok közti báziscsere mátrixát!

  \item A harmadik feladatban meghatározott báziscsere mátrixát felhasználva
        oldja meg a második feladatot!

  \item Írja fel a 2D Descartes koordinátarendszer $\alpha$ fokos elforgatásával
        nyert új koordinátarendszerbe mutató báziscsere mátrixát!

  \item Adjuk meg annak a lineáris leképezésnek a mátrixát, amely az alábbi
        vektorba viszi át a bázisodat:
        $$
          \rvec i \mapsto \begin{bmatrix}
            2 \\ 1 \\ 3
          \end{bmatrix}
          \text, \quad
          \rvec j \mapsto \begin{bmatrix}
            5 \\ 5 \\ 5
          \end{bmatrix}
          \text, \quad
          \rvec k \mapsto \begin{bmatrix}
            0 \\ 0 \\ -1
          \end{bmatrix}
          \text.
        $$
        Mi lesz a $P(1; 1; 1)$ pont képe?

  \item Adja meg az első feladatban szereplő leképezések mátrixait!

  \item Határozza meg az origón áthaladó $\rvec u (a,b,c)$ normálisú
        $(a^2+b^2+c^2 = 1)$ síkra vonatkozó tükrözés mátrixát!

  \item Adott egy lineáris leképezés a szokásos $\{ \rvec i, \rvec j \}$
        bázisban. Írja fel a leképezés mátrixát az $\{ \rvec f_1; \rvec f_2 \}$
        bázisban, ha $\rvec f_1(2; 1)$ és $\rvec f_2(1; 1)$.

  \item Adott két lineáris leképezés mátrixa $\rmat A$ és $\rmat B$.
        Mit ad eredményül\dots
        \begin{multicols}{4}
          \begin{enumerate}
            \item $(\rmat A + \rmat B) \rvec r$,
            \item $\rmat A \rmat B \rvec r$,
            \item $\rmat A^2 \rvec r$,
            \item $\rmat A^{-1} \rvec r$?
          \end{enumerate}
        \end{multicols}

  \item Egy $\varphi$ leképezés mátrixa $\rmat A$. Döntsük el, hogy:\\[-2mm]
        \begin{minipage}[b]{.25\textwidth}
          $$
            \rmat A = \begin{bmatrix}
              1  & 2  & 3  \\
              -1 & 0  & -2 \\
              3  & -1 & 4  \\
            \end{bmatrix}
          $$
        \end{minipage}\hfill
        \begin{minipage}[c]{.35\textwidth}
          \begin{enumerate}
            \item $P(2,0,1) \in \ker \varphi$,
            \item mi $Q'(1,4,0)$ ősképe,
          \end{enumerate}
        \end{minipage}
        \begin{minipage}[c]{.3\textwidth}
          \begin{enumerate}
            \setcounter{enumii}{2}
            \item $\dim \varphi = ?$
            \item $\operatorname{def} \varphi = ?$
          \end{enumerate}
        \end{minipage}

  \item Mennyi a leképezés defektusa\dots
        \begin{multicols}{2}
          \begin{enumerate}
            \item $x$ tengelyre való vetítés esetén,
            \item $yz$ síkra való vetítés esetén?
          \end{enumerate}
        \end{multicols}

  \item Írja fel annak a leképezésnek a mátrixát amely $z$ körül $\alpha$
        szöggel forgat, majd tükröz az $xy$ síkra, végül $x$ irányba $2$-szeres,
        $z$ irányba $3$-szoros nagyítást végez!

  \item Írja fel azt a leképezést, amely az $y=x$ és $z=0$ egyenletrendszerű
        egyenesre tükröz!

  \item Írja fel az $e$ egyenes körül pozizív $y$ irányból $90^\circ$-os
        forgatás mátrixát a szokásos, illetve a $\rvec v_1(1; 0; 0)$,
        $\rvec v_2(1; 1; 0)$ és $\rvec v_3 = (1; 1; 1)$ bázisokban, ha az
        egyenes egyenletrendszere:
        $$
          e:
          \frac{-1}{2} x + \frac{\sqrt{3}}{2} y = 0
          \quad \text{és} \quad
          z = 0
          \text.
        $$

  \item Adja meg a $\alpha$ és $\beta$ paramétereket, hogy a $\varphi$ leképezés
        $\rmat A$ mátrixa orientciótartó és skalárisszorzattartó legyen
        (ortogonális)!
        $$
          \rmat A = \begin{bmatrix}
            \alpha & \beta & 0 \\
            1      & 0     & 0 \\
            0      & 0     & 1
          \end{bmatrix}
        $$
\end{enumerate}



\end{document}