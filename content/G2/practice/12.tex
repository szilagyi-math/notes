\documentclass[a4paper, 12pt]{scrartcl}

\usepackage{math-practice}

\area{Többváltozós analízis}
\title{Differenciálás II}
\subject{Matematika G2}
\subjectCode{BMETE94BG02}
\date{Utoljára frissítve: \today}
\docno{12}

\begin{document}
\maketitle
\subsection{Elméleti Áttekintő}

\begin{blueBox}
  \sftitle{Többváltozós Taylor-formula:}

  Az egyváltozós függvényekhet hasonló módon, az $n$-edrendű Taylor-polinom:
  \begin{align*}
    T_0 & = f(\rvec x_0)
    \\
    T_1 & = T_0
    + \pdv{f(\rvec x_0)}{x}(x - x_0)
    + \pdv{f(\rvec x_0)}{y}(y - y_0)
    \\
    T_2 & = T_1 + \frac{1}{2} \left[
      \pdv[order=2]{f(\rvec x_0)}{x}(x - x_0)^2
      + \pdv[order=2]{f(\rvec x_0)}{y}(y - y_0)^2
      + 2 \pdv{f(\rvec x_0)}{x,y}(x - x_0)(y - y_0)
    \right]
    \\
        & \vdots
  \end{align*}
\end{blueBox}

\begin{blueBox}
  \sftitle{Többváltozós függvények szélsőérték-számítása:}

  Szélsőérték létezésének \textbf{szükséges} feltétele, hogy az adott pontban a
  parciális deriváltak eltűnjenek, vagyis
  $$
    \forall i:
    \pdv{f(\rvec x_0)}{x_i} = 0
    \text.
  $$

  Szélsőérték létezésének elégséges feltétele, hogy az adott pontban felírt
  Hesse-mátrix pozitív definit legyen. Ezen mátrix az alábbi módon írható fel:
  $$
    \rmat H = \begin{bmatrix}
      \partial_1^2 f(\rvec x_0)          & \partial_1 \partial_2 f(\rvec x_0) & \cdots & \partial_1 \partial_n f(\rvec x_0) \\
      \partial_2 \partial_1 f(\rvec x_0) & \partial_2^2 f(\rvec x_0)          & \cdots & \partial_2 \partial_n f(\rvec x_0) \\
      \vdots                             & \vdots                             & \ddots & \vdots                             \\
      \partial_n \partial_1 f(\rvec x_0) & \partial_n \partial_2 f(\rvec x_0) & \cdots & \partial_n^2 f(\rvec x_0)
    \end{bmatrix}
    \text.
  $$
  A determinánsa alapján:
  \begin{itemize}
    \item ha $\det \rmat H > 0$, akkor lokális szélsőérték van,
    \item ha $\det \rmat H < 0$, akkor nincsen szélsőérték,
    \item ha $\det \rmat H = 0$, akkor nem lehet eldönteni.
  \end{itemize}
  Amennyiben $\det \rmat H > 0$, akkor a szélsőérték jellege
  \begin{itemize}
    \item lokális maximum, ha $\rmat H$ főátlóra feszített aldeterminánsai
          váltakozó előjelűek,
    \item lokális minimun, ha $\rmat H$ főátlóra feszített aldeterminánsai
          pozitívak.
  \end{itemize}
\end{blueBox}

\begin{blueBox}
  \sftitle{Feltételes szélsőérték egy görbe mentén:}

  Amennyiben egy $f$ függvény egy adott $g = 0$ görbe menti szélsőértékeit
  szeretnénk megkeresni, akkor az
  $$
    F = f + \lambda g
  $$
  függvényre kell megoldanunk a szélsőérték-feladatot.
\end{blueBox}

\begin{blueBox}
  \sftitle{Feltételes szélsőérték egy görbe által bezárt területen:}

  Amennyiben egy $f$ függvény szélsőértékeit egy adott $g = 0$ görbe által
  bezárt tartományán szeretnénk megkeresni, akkor:
  \begin{itemize}
    \item először megkeressük a lokális szélsőértékeket, és ezekből kiválasztjuk
          azokat, amelyek a tartomyán belsejébe esnek.
    \item utána pedig megkeressük a görbe peremére eső szélsőértékeket az előző
          módszer alapján.
  \end{itemize}
\end{blueBox}


\clearpage
\subsection{Feladatok}

\begin{enumerate}
  \item Határozza meg az $f(x; y)=2x^2-xy-y^2-6x-3y+5$ függvény első és második
        Taylor polinomját a $P(1; -2)$ pontban!

  \item Határozza meg az $f(x; y; z)=\sin x \sin y \sin z$ függvény első és
        második Taylor polinomjait a $P_2(\sfrac{\pi}{4}; \sfrac{\pi}{4};
          \sfrac{\pi}{6})$ pontban!

  \item Végezzen szélsőérték vizsgálatot az $f(x; y)=x^2-xy+y^2+3x-2y+1$
        függvényen!

  \item Határozza meg az $f(x; y)=x^2+xy+y^2+\sfrac{8}{x}+\sfrac{8}{y}$ függvény
        egy darab szélsértékét!

  \item Ellenőrizze, hogy az $f(x; y)=\cos x \cos y \cos (x+y)$ függvénynek a
        $P_1(\sfrac{\pi}{2}; \sfrac{\pi}{2})$ és $P_2(0; 0)$ pontokban
        szélsőérték helye van!

  \item Határozza meg azt a csomagméretet, amely esetén egy
        $V=4,5\;\mathrm{dm}^3$ térfogatú téglatest alakú csomag a legkevesebb
        zsineg felhasználásával az alábbi ábrán látható módón bekötözhető!
        \begin{center}
          \begin{tikzpicture}[thick]
            \draw
            (0, 1.5) -- (4, 1.5) -- (5.5, 2.5)
            (4, 1.5) -- (4, 0)
            (0, 0) -- (4, 0) -- (5.5, 1) -- (5.5, 2.5) -- (1.5,2.5) -- (0, 1.5) -- cycle
            ;

            \draw[ultra thick, primaryColor]
            (1.333, 0) -- ++(0,  1.5) -- ++(1.5, 1)
            (2.667, 0) -- ++(0,  1.5) -- ++(1.5, 1)
            (0.75,2) -- ++(4, 0) -- ++(0,-1.5)
            ;
          \end{tikzpicture}
        \end{center}

  \item Határozza meg annak a derékszögű hasábnak a minimális térfogatát, amely
        éleinek összege $l$ hosszú!

  \item Határozza meg a $f(x; y)=x^2y^2$ függvény szélőértékét az
        $x^2+y^2=1$ egyenletű görbe mentén!

  \item Határozza meg az $f(x; y)=x^2+y^2+xy-x$ függvény globális maximumát és
        minimumát az $x^2+y^2\leq1$ tartományon!
\end{enumerate}


\end{document}
