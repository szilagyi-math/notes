\documentclass[a4paper, 12pt]{scrartcl}

\usepackage{math-practice}
\usepackage{tikz}
\usepackage{multicol}
\usepackage{amsmath}
\usepackage{arydshln}

\area{Lineáris Algebra}
\title{Lineáris leképzések II}
\subject{Matematika G2}
\subjectCode{BMETE94BG02}
\date{Utoljára frissítve: \today}
\docno{5}

\begin{document}
\maketitle
\subsection{Elméleti Áttekintő}

\begin{definition}[Ortogonális transzformáció]
  Az $n$ dimenziós  euklideszi tér $\mathcal A: V \rightarrow V$ lineáris
  transzformációját ortogonálisnak mondjuk, ha $\langle \mathcal A \rvec x;
    \mathcal A \rvec y \rangle = \langle \rvec x; \rvec y \rangle$, minden $\rvec
    x; \rvec y \in V$ esetén.
\end{definition}

\begin{note}
  Egy ortogonális transzformáció $\rvec Q$ mátrixának inverze megegyezik a
  transzponáltjával.

  Amennyiben $\det \rvec Q = 1$, akkor a transzformáció orientciótartó.

  Amennyiben $\det \rvec Q = -1$, akkor a transzformáció orientációváltó.
\end{note}

\begin{example}
  A két dimenziós térben való forgatás orientációtartó, hiszen
  $$
    \det \rmat Q
    = \begin{vmatrix}
      \cos \alpha & -\sin \alpha \\
      \sin \alpha & \cos \alpha
    \end{vmatrix}
    = \cos^2 \alpha + \sin^2 \alpha
    = 1
    \text.
  $$
\end{example}

\begin{definition}[Sajátértékek és sajátvektorok]
  Legyen $V$ a $T$ test feletti vektortér, $\rvec v \in V$, $\rvec v \neq
    \nvec$. $\rvec v$-t a $\varphi: V \rightarrow V$ lineáris leképezés
  sajátvektorának mondjuk, ha önmaga skalárszorosába megy át a leképezés
  során, azaz $\varphi(\rvec v) = \lambda \rvec v$,  $\lambda \in T$.
  $\lambda$-t a $\rvec v$ sajátvektorhoz tartozó sajátértéknek mondjuk.
\end{definition}

\begin{note}
  Ha a $\rvec v$ sajátvektora a $\varphi$-nek, akkor annak skalárszorosa is.
\end{note}

\begin{theorem}[Sajátértékek számítása]
  Az $\rmat A \in \mathcal M_{n \times n}$ mátrix sajátértékei a
  $$
    \det(\rmat A - \lambda \imat) = 0
  $$
  karakterisztikus egyenlet gyökei.
\end{theorem}

\begin{note}
  A $\det(\rmat A - \lambda \imat) = 0$ egyenletet \textbf{karakterisztikus
    egyenlet}nek nevezzük.

  A $\det(\rmat A - \lambda \imat)$ polinomot \textbf{karakterisztikus
    polinom}nak nevezzük.
\end{note}

\begin{note}
  Szimmetrikus mátrix sajátvektorai ortogonálisak és a sajátértékek mindig valósak.
\end{note}

\begin{note}
  Antiszimmetrikus mátrix sajátvektorai páronként konjugáltak és a sajátértékek mindig tisztán képzetesek.
\end{note}

\begin{definition}[Sajátaltér]
  Legyen \( \textbf{A} \) egy \( n \times n \)-es mátrix, és legyen \( \lambda \) az \( \textbf{A} \) egy sajátértéke. A \( \lambda \)-hoz tartozó sajátvektortér az alábbi halmaz:
  \[
    E_\lambda = \{ \rvec{v} \mid \textbf{A}\rvec{v} = \lambda\rvec{v} \}.
  \]
  Ez az \( \mathbb{R}^n \) egy altere.
\end{definition}

Itt meg kellene az algebrai es a geometriai multiplicitast is emliteni, de nem vagyok benne mi a helyes megfogalmazás, zavaros a kézzel írt jegyzet.

\begin{definition}[Karakterisztikus polinom]
  Legyen \( \textbf{A} \) egy négyzetes mátrix. Az alábbi kifejezést:
  \[
    p(\lambda) = \det(\textbf{A} - \lambda \mathbb{I})
  \]
  \( \textbf{A} \) karakterisztikus polinomjának nevezzük.
\end{definition}

\begin{blueBox}
  Legyen \( \textbf{A} \) egy \( n \times n \)-es mátrix. Az \( \textbf{A} \)-hoz tartozó sajátértékek és sajátvektorok meghatározásának lépései a következők:
  \begin{enumerate}
    \item Számítsd ki a karakterisztikus polinomot: \( \det(\textbf{A} - \lambda \mathbb{I}) \).
    \item A sajátértékek a karakterisztikus polinom gyökei.
    \item Minden sajátértékre (\( \lambda \)) határozz meg egy bázist a sajátvektorokhoz az alábbi homogén egyenletrendszer megoldásával:
          \[
            (\textbf{A} - \lambda \mathbb{I})\rvec{v} = 0.
          \]
  \end{enumerate}
\end{blueBox}

\begin{note}
  Legyen \( \textbf{A} \) egy felső vagy alsó háromszögmátrix. Ekkor az \( \textbf{A} \)-hoz tartozó sajátértékek a főátló elemei.
\end{note}

\begin{definition}[Diagonalizálható mátrix]
  Legyen \( \textbf{A} \) egy \( n \times n \)-es mátrix. Az \( \textbf{A} \)-t diagonalizálhatónak nevezzük, ha létezik egy invertálható \( \textbf{P} \) mátrix és egy \( \textbf{D} \) diagonális mátrix, amelyekre teljesül, hogy
  \[
    \textbf{P}^{-1}\textbf{AP} = \textbf{D}.
  \]
\end{definition}

\begin{theorem}[Diagonizálhatóság és sajátvektorok]
  Egy \( n \times n \)-es mátrix \( A \) akkor és csak akkor diagonalizálható, ha \( A \)-nak \( n \) lineárisan független sajátvektora van.

  Továbbá, ebben az esetben legyen \( P \) az az invertálható mátrix, amelynek oszlopai \( A \) \( n \) lineárisan független sajátvektorai, és legyen \( D \) az a diagonális mátrix, amelynek főátlójában a megfelelő sajátértékek szerepelnek. Ekkor
  \[
    P^{-1}AP = D.
  \]
\end{theorem}

\begin{definition}[Mátrixfüggvények]
  Ide kell egy definíció, mert nincs rendes a füzetben!
\end{definition}

\begin{blueBox}
  \textbf{Kvadratikus formák}\\
  Csupa másodfokú tagot tartalmazó polinomokat átírhatóak mátrixos alakba.
  \[
    ax^2 + 2bxy + cy^2 = \begin{bmatrix}
      x & y
    \end{bmatrix} \begin{bmatrix}
      a & b \\
      b & c
    \end{bmatrix} \begin{bmatrix}
      x \\
      y
    \end{bmatrix}
  \]
  \textbf{Másodrendű görbe}
  \[
    ax^2 + 2bxy + cy^2 + dx + ey + f = 0
  \]
  \[
    \begin{bmatrix}
      x & y
    \end{bmatrix} \begin{bmatrix}
      a & b \\
      b & c
    \end{bmatrix} \begin{bmatrix}
      x \\
      y
    \end{bmatrix} + \begin{bmatrix}
      d & e
    \end{bmatrix} \begin{bmatrix}
      x \\
      y
    \end{bmatrix} + f = 0
  \]

  Másodrendű görbék lehetnek ellipszisek, hiperbólák és parabolák.

  IDE KÉNE EGY KIS rendezett leírás, arra, hogy mikor is lehet kvadratikus alakot áttranszformálni sajátértékek koordinata rendszerébe, hogyan alakulnak az egyenletek es mikor is lesz semi deficit deficit és indeficit.
\end{blueBox}

\clearpage
\subsection{Feladatok}
\begin{enumerate}
  \item Írja fel az $e$ egyenes körül pozizív $y$ irányból $90^\circ$-os
        forgatás mátrixát a szokásos, illetve a $\rvec v_1(1; 0; 0)$,
        $\rvec v_2(1; 1; 0)$ és $\rvec v_3 = (1; 1; 1)$ bázisokban, ha az
        egyenes egyenletrendszere:
        $$
          e:
          \frac{-1}{2} x + \frac{\sqrt{3}}{2} y = 0
          \quad \text{és} \quad
          z = 0
          \text.
        $$

  \item Adja meg a $\alpha$ és $\beta$ paramétereket, hogy a $\varphi$ leképezés
        $\rmat A$ mátrixa orientciótartó és skalárisszorzattartó legyen
        (ortogonális)!
        $$
          \rmat A = \begin{bmatrix}
            \alpha & \beta & 0 \\
            1      & 0     & 0 \\
            0      & 0     & 1
          \end{bmatrix}
        $$

  \item Adja meg az alábbi mátrixok sajátvektorait és sajátértékeit!
        \begin{alignat*}{9}
          \rmat A & =
          \begin{bmatrix}
            4 & 3 \\
            1 & 2
          \end{bmatrix}
                  & \rmat B & =
          \begin{bmatrix}
            -2 & -8 & -12 \\
            1  & 4  & 4   \\
            0  & 0  & 1
          \end{bmatrix}
          \hspace{15mm}
                  & \rmat C & =
          \begin{bmatrix}
            0  & 1 \\
            -1 & 0
          \end{bmatrix}
          \\
          \rmat D & =
          \begin{bmatrix}
            1 & 0 & 0 & 0 \\
            0 & 1 & 0 & 0 \\
            0 & 0 & 2 & 1 \\
            0 & 0 & 0 & 2
          \end{bmatrix}
          \hspace{15mm}
                  & \rmat E & =
          \begin{bmatrix}
            4 & 1 & 0 \\
            0 & 4 & 1 \\
            0 & 0 & 4
          \end{bmatrix}
                  & \rmat F & =
          \begin{bmatrix}
            2 & 2 & 1 \\
            1 & 3 & 1 \\
            1 & 2 & 2
          \end{bmatrix}
        \end{alignat*}

  \item A leképezés mátrixainak felírása nélkül adja meg a lehető legtöbb
        sajátértélet és sajátvektort!
        \begin{enumerate}
          \item $z$-tengely körüli $45^\circ$-os forgatás,
          \item $xy$ síkra vetítés,
          \item $xy$ síkra tükrözés.
        \end{enumerate}

  \item Diagonizálhatóak-e a harmadik feladatban szereplő $\rmat E$, $\rmat D$
        és $\rmat B$ mátrixok?

  \item A sajátértékek kiszámítása nélkül mondjuk meg a lehető legtöbb
        sajátér\-ték-sa\-ját\-vek\-tor párt!
        $$
          \rmat A =
          \begin{bmatrix}
            1 & 0 & 0 \\
            2 & 4 & 0 \\
            3 & 0 & 5
          \end{bmatrix}
          \hspace{2cm}
          \rmat B =
          \begin{bmatrix}
            5 & 0 & 0 \\
            0 & 7 & 0 \\
            0 & 0 & 9
          \end{bmatrix}
        $$

  \item Határozzuk meg a harmadik feladatban szereplő $\rmat B$ mátrix tizedik
        hatványát!

  \item Határozzuk meg az $e^{10^{\rmat B}}$ függvényt, ha $\rmat B$ a harmadik
        feladatban szereplő mátrix!

  \item Milyen alakzatot írnak le az alábbi másodrendű görbék?
        Írja fel a kanonikus egyenletüket!
        \begin{enumerate}
          \item $
                  -3x^2 + 23y^2 + 26\sqrt{3}xy = 144
                $

          \item $
                  57x^2 + 43y^2 + 14\sqrt{3}xy = 576
                $

          \item $
                  2x^2 - 5 = 0
                $
        \end{enumerate}
\end{enumerate}




\end{document}