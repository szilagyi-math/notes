\documentclass[a4paper, 12pt]{scrartcl}

\usepackage{math-practice}

\area{Mátrixok}
\title{Mátrixok I}
\subject{Matematika G2}
\subjectCode{BMETE94BG02}
\date{Utoljára frissítve: \today}
\docno{1}

\begin{document}
\maketitle

\subsection{Elméleti Áttekintő}

\begin{blueBox}
  Az előző félévben az $\Reals^3$-at, az $3$-dimenziós oszlopvektorok
  vektorterét vizsgáltuk. Idén egy általánosabb vektor fogalmat vezetünk be.

  A vektorok olyan objektumok, amelyeket össze lehet adni és skálázni lehet.
  Bár egyes vektorok számok listáinak tűnnek (például oszlopvektorok,
  sorvektorok), más típusú vektorok egyáltalán nem hasonlítanak számok listáira
  (például függvények, polinomok). A lineáris algebra erejének egyik forrása az
  a képességünk, hogy sok váratlan helyen találunk vektortereket.

  A félév elején átismételjük azokat a fogalmakat, amelyeket már az előző
  félévben az $\Reals^n$ kontextusában megismertünk. Például szó lesz lineáris
  kombinációkról, lineáris függetlenségről, lineáris egyenletrendszerekről és
  így tovább. Ezeket a fogalmakat eredetileg az $\Reals^n$ vektortér keretében
  vezettük be, hogy könnyebben megérthetők legyenek. Most azonban látni fogjuk,
  hogy valójában minden vektortérre alkalmazhatóak.
\end{blueBox}

\begin{definition}[Vektortér]
  Legyen $V$ nemüres halmaz, és $\circ, +$ két művelet, $T$ test.
  A $(V; +, \circ)$ a $T$ test feletti vektortér, ha teljesülnek az alábbiak:
  \begin{enumerate}
    \item $(V; +)$ Abel-csoport,

    \item $\forall \lambda; \mu \in T \; \land \; \forall \rvec x \in V:
            (\lambda \circ \mu) \circ \rvec x
            = \lambda \circ (\mu \circ \rvec x)$,

    \item ha $\varepsilon$ a $T$-beli egységelem, akkor
          $\forall \rvec x \in V: \varepsilon \circ \rvec x = \rvec x$,

    \item teljesül a disztributivitás:
          \begin{itemize}
            \item $\forall \lambda; \mu \in T \; \land \; \forall \rvec x \in V:
                    \lambda \circ (\rvec x + \rvec y)
                    = \lambda \circ \rvec x + \lambda \circ \rvec y$,

            \item $\forall \lambda; \mu \in T \; \land \; \forall \rvec x \in V:
                    (\lambda + \mu) \circ \rvec x
                    = \lambda \circ \rvec x + \mu \circ \rvec x$.
          \end{itemize}
  \end{enumerate}
\end{definition}

\begin{definition}[Lineáris függetlenség]
  A $(V; +; \lambda)$ vektortér $\rvec v_1, \rvec v_2, \ldots, \rvec v_n$
  vektorait lineárisan függetlennek mondjuk, ha a
  $$
    \lambda_1 \rvec v_1
    + \lambda_2 \rvec v_2
    + \ldots
    + \lambda_n \rvec v_n
    = \nvec
  $$
  vektoregyenletnek \textbf{csak a triviális megoldása} létezik, azaz
  $\lambda_1 = \lambda_2 = \ldots = \lambda_n = 0$.

  Ha az egyenletnek nem csak a triviális megoldása létezik, akkor a vektorok
  lineárisan függők.
\end{definition}

\begin{definition}[Altér]
  Legyen $(V; +; \lambda)$ $\mathbb R$ feletti vektortér, valamint
  $\emptyset \neq L \subset V$. $L$-t altérnek nevezzük a $V$-ben, ha
  $(L; +; \lambda)$ ugyancsak vektortér.
\end{definition}

\begin{definition}[Generátorrendszer]
  Legyen $V$ vektortér, valamint $\emptyset \neq G \subset V$. $G$ által
  generált altérnek nevezzük azt a legszűkebb alteret, amely tartalmazza $G$-t.
  Jele: $\mathcal L(G)$.

  $G$ generátorrendszere $V$-nek, ha $\mathcal L(G) = V$.
\end{definition}

\begin{definition}[Bázis]
  A $V$ vektortér egy lineárisan független generátorrendszerét a $V$
  bázisának nevezzük.
\end{definition}

\begin{definition}[Vektortér dimenziója]
  Végesen generált vektortér dimenzióján a bázisainak közös tagszámát értjük.
\end{definition}

\begin{definition}[Mátrix]
  Egy mátrix vízszintes vonalban elhelyezkedő elemei \textbf{sorok}at,
  míg függőlegesen elhelyezkedő elemei \textbf{oszlop}okat alkotnak.

  Egy $m$ sorból és $n$ oszlopból álló mátrix jelölése:
  $$
    \rmat A = \begin{bmatrix}
      a_{11} & a_{12} & \ldots & a_{1n} \\
      a_{21} & a_{22} & \ldots & a_{2n} \\
      \vdots & \vdots & \ddots & \vdots \\
      a_{m1} & a_{m2} & \ldots & a_{mn}
    \end{bmatrix}
    \text.
  $$

  Mátrixok jelölése nyomtatott szövegben: $\rmat A$.

  Mátrixok jelölése írásban: $\underline{\underline A}$.

  Az $m \times n$-es mátrixok halmazának jelölései: $\mathcal M_{m \times n}
    = \mathbb R^m \times \mathbb R^n = \mathbb R^{m \times n}$.

  A mátrix $i$-edik sorában és $j$-edik oszlopában található elemet
  $a_{ij}$-vel jelöljük.
\end{definition}

\begin{definition}[Mátrix transzponáltja]
  Egy $\rmat A \in \mathcal M_{m \times n}$ mátrix transzponáltja a főátlójára
  vett tükörképe. Jele: $\rmat A^\T \in \mathcal M_{n \times m}$.
  $$
    \rmat A = \begin{bmatrix}
      a_{11} & \cdots & a_{1n} \\
      a_{21} & \cdots & a_{2n} \\
      \vdots & \ddots & \vdots \\
      a_{m1} & \cdots & a_{mn}
    \end{bmatrix}
    \quad \Rightarrow \quad
    \rmat A^\T = \begin{bmatrix}
      a_{11} & a_{21} & \cdots & a_{m1} \\
      \vdots & \vdots & \ddots & \vdots \\
      a_{1n} & a_{2n} & \cdots & a_{mn}
    \end{bmatrix}
  $$
\end{definition}

\begin{blueBox}
  \sftitle{Speciális mátrixstruktúrák:}
  \newenvironment{tmatrix}{%
    \begin{bmatrix}
      \hphantom{a_{n1}} & \hphantom{a_{n2}} & \hphantom{\ldots} & \hphantom{a_{nn}} \\[-14pt]
      }{%
    \end{bmatrix}
  }
  $$
    \begin{array}{rc>{\in\;}c>{\sim\;\;}l}
       & \begin{bmatrix}
           a_{11} \\ a_{21} \\ \vdots \\ a_{n1}
         \end{bmatrix}
       & \mathcal M_{n \times 1}
       & \text{oszlopvektor / oszlopmátrix}
      \\[12mm]
       & \begin{bmatrix}
           \;a_{11} & a_{12} & \ldots & a_{1n}\;
         \end{bmatrix}
       & \mathcal M_{1 \times n}
       & \text{sorvektor / sormátrix}
      \\[4mm]
       & \begin{bmatrix}
           a_{11} & a_{12} & \ldots & a_{1n} \\
           a_{21} & a_{22} & \ldots & a_{2n} \\
           \vdots & \vdots & \ddots & \vdots \\
           a_{n1} & a_{n2} & \ldots & a_{nn}
         \end{bmatrix}
       & \mathcal M_{n \times n}
       & \text{kvadratikus / négyzetes mátrix}
      \\[12mm]
      \imat =
       & \begin{tmatrix}
           1      & 0      & \ldots & 0      \\
           0      & 1      & \ldots & 0      \\
           \vdots & \vdots & \ddots & \vdots \\
           0      & 0      & \ldots & 1
         \end{tmatrix}
       & \mathcal M_{n \times n}
       & \text{egységmátrix}
      \\[12mm]
      \nmat =
       & \begin{tmatrix}
           0      & 0      & \ldots & 0      \\
           0      & 0      & \ldots & 0      \\
           \vdots & \vdots & \ddots & \vdots \\
           0      & 0      & \ldots & 0
         \end{tmatrix}
       & \mathcal M_{m \times n}
       & \text{nullmátrix}
      \\[12mm]

       & \begin{bmatrix}
           a_{11} & 0      & \ldots & 0      \\
           0      & a_{22} & \ldots & 0      \\
           \vdots & \vdots & \ddots & \vdots \\
           0      & 0      & \ldots & a_{nn}
         \end{bmatrix}
       & \mathcal M_{n \times n}
       & \text{diagonális mátrix}
    \end{array}
  $$
\end{blueBox}

\begin{definition}[Mátrixok összege]
  Két mátrix összegén azt a mátrixot értjük, melyet a két mátrix elemenkénti
  összeadásával kapunk, azaz, ha $\rmat A, \rmat B \in \mathcal M_{m \times n}$,
  akkor $\rmat C := \rmat A + \rmat B \in \mathcal M_{m \times n}$, ahol
  $c_{ij} := a_{ij} + b_{ij}$.
\end{definition}

\begin{definition}[Mátrix és skalár szorzata]
  Egy mátrix és egy skalár szorzata olyan mátrix, melynek minden eleme
  skalárszorosa az eredeti mátrix elemeinek, azaz ha
  $\rmat A \in \mathcal M_{m \times n}$ és $\lambda \in \mathbb R$, akkor
  $\rmat C := \lambda \rmat A$, ahol $c_{ij} := \lambda a_{ij}$.
\end{definition}

\begin{definition}[Mátrixok szorzata]
  Legyen $\rmat A \in \mathcal M_{m \times n}$ és
  $\rmat B \in \mathcal M_{n \times p}$. Ekkor a két mátrix szorzata
  $$
    \rmat C := \rmat A \cdot \rmat B
    \text{, ahol }
    c_{ij}
    = \sum_{k=1}^{n} a_{ik} \cdot b_{kj}
    = a_{i1} \cdot b_{1j} + a_{i2} \cdot b_{2j} + \ldots + a_{in} \cdot b_{nj}
    \text.
  $$
\end{definition}

\begin{blueBox}
  \sftitle{A mátrixszorzás vizualizálása:}
  \newcolumntype{x}[1]{>{\centering\arraybackslash\hspace{0pt}}p{#1}}
  \newcolumntype{F}[1]{>{$}x{#1}<{$}}
  \def\arraystretch{1.1}
  \begin{align*}
     & \left[\begin{array}{F{2cm}cF{2cm}}
                 b_{11} & \dots  & b_{1p} \\
                 b_{21} & \dots  & b_{2p} \\
                 \vdots & \ddots & \vdots \\
                 b_{n1} & \dots  & b_{np}
               \end{array}\right]
    \\
    \left[\begin{array}{cccc}
              a_{11} & a_{12} & \dots  & a_{1n} \\
              a_{21} & a_{22} & \dots  & a_{2n} \\
              \vdots & \vdots & \ddots & \vdots \\
              a_{m1} & a_{m2} & \dots  & a_{mn}
            \end{array}\right]
     & \left[\begin{array}{F{2cm}cF{2cm}}
                 \sum a_{1i} b_{i1} & \dots  & \sum a_{1i} b_{ip} \\
                 \sum a_{2i} b_{i1} & \dots  & \sum a_{2i} b_{ip} \\
                 \vdots             & \ddots & \vdots             \\
                 \sum a_{mi} b_{i1} & \dots  & \sum a_{mi} b_{ip}
               \end{array}\right]
  \end{align*}
\end{blueBox}

\begin{definition}[Szimmetrikus mátrix]
  Egy $\rmat A \in \mathcal M_{n \times n}$ mátrix szimmetrikus, ha
  $\rmat A = \rmat A^\T$.
\end{definition}

\begin{definition}[Antiszimmetrikus mátrix]
  Egy $\rmat A \in \mathcal M_{n \times n}$ mátrix antiszimmetrikus, ha
  $\rmat A = -\rmat A^\T$.
\end{definition}

\begin{blueBox}
  \sftitle{Kvadratikus mátrix felbontása szimmetrikus és antiszimmetrikus részekre:}

  $$
    \textbf{A} = \underbrace{\frac{1}{2}(\textbf{A} + \textbf{A}^{\text{T}})}_{\text{Szimmetrikus}} + \underbrace{\frac{1}{2}(\textbf{A} - \textbf{A}^{\text{T}})}_{\text{Antiszimmetrikus}}
  $$
\end{blueBox}

% \begin{blueBox}
%   \sftitle{Speciális tulajdonságok:}

%   Legyen $\rmat A$ és $\rmat B$ megfelelő méretű mátrixok, és
%   $\lambda \in \Reals$. Ekkor az alábbiak teljesülnek:
%   \begin{enumerate}
%     \item $\left(\rmat A^\T\right)^\T = \rmat A$,
%     \item $\left(\rmat A + \rmat B\right)^\T = \rmat A^\T + \rmat B^\T$
%     \item $\left(\lambda \rmat A\right)^\T = \lambda \rmat A^\T$,
%     \item $\left(\rmat A \rmat B\right)^\T = \rmat B^\T \rmat A^\T$,
%   \end{enumerate}
% \end{blueBox}

\begin{definition}[Determináns]
  \newcommand\noskp{\vspace{-3mm}}
  \newcommand{\edet}[1]{\det \begin{pmatrix} \phantom{i}\cdots & #1 & \cdots\phantom{i} \end{pmatrix}}
  Legyen $\rmat A \in \mathcal M_{n \times n}$ kvadratikus mátrix, és
  $\det: \mathcal M_{n \times n} \rightarrow \mathbb R$ függvény. A mátrix
  $i$-edik oszlopának elemeit tartalmazó oszlopvektorokat $\rvec a_i$-vel
  jelöljük. Az $\rmat A$ determinánsának nevezzük $\det \rmat A$-t, a
  hozzárendelést pedig az alábbi négy axióma írja le:
  \begin{enumerate}
    \item homogén:
          $$
            \edet{\lambda \rvec a_i} = \lambda \edet{\rvec a_i}
            \text,
          $$
    \item \noskp additív:\noskp
          $$
            \edet{\rvec a_i + \rvec b_i} =
            \edet{\rvec a_i} + \edet{\rvec b_i}
            \text,
          $$
    \item \noskp alternáló:\noskp
          $$
            \edet{\rvec a_i & \dots & \rvec a_j} =
            - \edet{\rvec a_j & \dots & \rvec a_i}
            \text,
          $$
    \item \noskp $\imat$ determinánsa:
          $$
            \det \imat =\det \begin{pmatrix}
              \uvec e_1 & \uvec e_2 & \cdots & \uvec e_n
            \end{pmatrix} = 1
            \text,
          $$
  \end{enumerate}
\end{definition}

\begin{note}
  Ha egy mátrix determinánsa nem zérus, akkor a az oszlopaiból, vagy soriból
  képzett vektorok lineárisan függetlenek.

  Ellenkező esetben lineárisan összefüggőek.
\end{note}

\clearpage
\subsection{Feladatok}

\begin{enumerate}
  \item Vizsgálja meg, hogy vektorteret alkotnak-e a szokásos műveletekre\dots
        \begin{enumerate}
          \item $\Reals^3$ azon vektorai, amelyek első koordinátája $1$,
          \item $\Reals^3$ azon vektorai, amelyek második koordinátája $0$,
          \item a harmadfokú, valós együtthatós polinomok.
        \end{enumerate}

  \item Döntse el, hogy az alábbi vektorok $\Reals^2$-ben bázist vagy
        generátorrendszert alkotnak-e!
        \begin{center}
          \begin{tikzpicture}[very thick]
            \node at (-.75, 2) {a)};

            \draw[->, secondaryColor] (0,0) -- (0,2);
            \draw[->, primaryColor] (0,0) -- (3,0);
            \draw[->, ternaryColor] (0,0) -- (1.5,1.5);

            \begin{scope}[xshift=6.5cm]
              \node at (-2.25, 2) {b)};

              \draw[->, primaryColor] (0,0) -- (2,0);
              \draw[->, secondaryColor] (0,0) -- (-1.5,1.5);
            \end{scope}

            \begin{scope}[xshift=11cm]
              \node at (-.75, 2) {c)};

              \coordinate (O) at (0,0);
              \draw[->, secondaryColor] (0,0) -- (0,2) coordinate (A);
              \draw[->, primaryColor] (0,0) -- (2,0) coordinate (B);

              % Right angle
              \draw pic["$\cdot$", draw, angle eccentricity=.5, angle radius=4mm, thick]
                {angle=B--O--A}
              ;
            \end{scope}

            \begin{scope}[yshift=-3cm]
              \node at (-.75, 1.5) {d)};

              \draw[->, primaryColor] (0,0) -- (2,0);
              \draw[->, secondaryColor] (2,0) -- (4,0);
            \end{scope}

            \begin{scope}[yshift=-3cm, xshift=11cm]
              \node at (-2.75, 1.5) {e)};

              \draw[->, primaryColor] (0,0) -- (2,0);
              \draw[->, secondaryColor] (0,0) -- (-2,0);
              \draw[->, ternaryColor] (0,0) -- (1.5,1.5);
            \end{scope}
          \end{tikzpicture}
        \end{center}

  \item Vizsgálja meg, hogy az $\rvec v_1(1; 2; 3)$, $\rvec v_2(1; 3; -1)$ és
        $\rvec v_3 (1, 0, 0)$ vektorok $\Reals^3$-ban bázist vagy
        generátorrendszert alkotnak-e!

  \item Írja fel az $\rmat A$ mátrix transzponáltját!
        $$
          \rmat A = \begin{bmatrix}
            1 & 7 & 3 & 5 & 10 \\
            0 & 5 & 0 & 4 & 0  \\
            2 & 1 & 0 & 2 & 1
          \end{bmatrix}
        $$

  \item Adottak az $\rmat A$, $\rmat B$ és $\rmat C$ mátrixok. Végezze el az
        alábbi műveleteket!
        $$
          \rmat A =\begin{bmatrix}
            2 & 5 & 1 \\
            7 & 0 & 3
          \end{bmatrix}
          \qquad
          \rmat B =\begin{bmatrix}
            6 & 8 & 2 \\
            1 & 2 & 3
          \end{bmatrix}
          \qquad
          \rmat C =\begin{bmatrix}
            2  & 2 \\
            -3 & 7 \\
            1  & 3
          \end{bmatrix}
        $$
        \begin{multicols}{3}
          \begin{enumerate}
            \item $\rmat A + \rmat B$

            \item $2\rmat A + 3\rmat B$

            \item  $3\rmat A + \rmat C$

            \item $\rmat B \cdot \rmat A$

            \item $\rmat B \cdot \rmat C$

            \item $2\rmat A + 3\rmat B\rmat C$
          \end{enumerate}

        \end{multicols}

  \item Adottak az $\rmat A$ és $\rmat B$ mátrixok. Végezze el
        $\rmat A \cdot \rmat B$ és $\rmat B \cdot \rmat A$ műveleteket!
        $$
          \rmat A = \begin{bmatrix}
            1 \\
            2 \\
            4
          \end{bmatrix}
          \qquad
          \rmat B = \begin{bmatrix}
            2 & 0 & 1
          \end{bmatrix}
        $$


  \item Bontsa fel az $\rmat A$ mátrixot szimmetrikus és antiszimetrikus
        összetevőkre!
        $$
          \rmat A = \begin{bmatrix}
            3  & -5 & 2 \\
            0  & 4  & 7 \\
            10 & 8  & 1
          \end{bmatrix}
        $$

  \item Számolja ki az $\rmat A$, $\rmat B$, $\rmat C$ és $\rmat D$ mátrixok
        determinánsát!
        $$
          \rmat A = \begin{bmatrix}
            2 & 3 \\
            1 & 0
          \end{bmatrix}
          \qquad
          \rmat B = \begin{bmatrix}
            4 & 9 & 2 \\
            3 & 5 & 7 \\
            8 & 1 & 6
          \end{bmatrix}
          \qquad
          \rmat C = \begin{bmatrix}
            3 & 8 & 6  & 3 \\
            1 & 2 & 0  & 1 \\
            1 & 1 & -1 & 2 \\
            2 & 5 & 1  & 5
          \end{bmatrix}
          \qquad
          \rmat D = \begin{bmatrix}
            1 & 5 & 3 & -7 \\
            0 & 1 & 4 & -2 \\
            0 & 0 & 1 & 4  \\
            0 & 0 & 0 & 1
          \end{bmatrix}
        $$

  \item  Mutassa meg hogy $\rmat A$ determinánsa osztható $7$-tel!
        $$
          \rmat A = \begin{bmatrix}
            6 & 3 & 7 \\
            3 & 4 & 3 \\
            7 & 3 & 5
          \end{bmatrix}
        $$
\end{enumerate}
\end{document}