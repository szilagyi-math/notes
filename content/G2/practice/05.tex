\documentclass[a4paper, 12pt]{scrartcl}

\usepackage{math-practice}

\pgfkeys{/pgf/plot/gnuplot call={cd build && gnuplot}}

\area{Lineáris Algebra}
\title{Lineáris leképzések II}
\subject{Matematika G2}
\subjectCode{BMETE94BG02}
\date{Utoljára frissítve: \today}
\docno{5}

\begin{document}
\maketitle
\subsection{Elméleti Áttekintő}

\begin{definition}[Sajátértékek és sajátvektorok]
  Legyen $V$ a $T$ test feletti vektortér, $\rvec v \in V$, $\rvec v \neq
    \nvec$. $\rvec v$-t a $\varphi: V \rightarrow V$ lineáris leképezés
  sajátvektorának mondjuk, ha önmaga skalárszorosába megy át a leképezés
  során, azaz $\varphi(\rvec v) = \lambda \rvec v$,  $\lambda \in T$.
  $\lambda$-t a $\rvec v$ sajátvektorhoz tartozó sajátértéknek mondjuk.
\end{definition}

\begin{note}
  Ha a $\rvec v$ sajátvektora a $\varphi$-nek, akkor annak skalárszorosa is az.
\end{note}

\begin{theorem}[Sajátértékek számítása]
  Az $\rmat A \in \mathcal M_{n \times n}$ mátrix sajátértékei a
  $$
    \det(\rmat A - \lambda \imat) = 0
  $$
  karakterisztikus egyenlet gyökei.
\end{theorem}

\begin{example}
  Határozzuk meg az $\rmat A$ mátrix sajátértékeit és sajátvektorait!

  $$
    \rmat A = \begin{bmatrix}
      2  & -1 \\
      -1 & 2
    \end{bmatrix}
    \qquad
    \rmat A - \lambda \imat = \begin{bmatrix}
      2  & -1 \\
      -1 & 2
    \end{bmatrix} - \lambda \begin{bmatrix}
      1 & 0 \\
      0 & 1
    \end{bmatrix} = \begin{bmatrix}
      2 - \lambda & -1          \\
      -1          & 2 - \lambda
    \end{bmatrix}
  $$

  A karakterisztikus egyenlet, és ennek alapján a sajátértékek:
  $$
    \det(\rmat A - \lambda \imat) = (2 - \lambda)^2 - 1 = 0
    \quad \Rightarrow \quad
    \lambda_1 = 1 \text, \quad \lambda_2 = 3 \text.
  $$

  A sajátvektorokat az $(\rmat A - \lambda_i \imat) \rvec v_i = 0$
  egyenlet segítségével számíthatjuk ki:

  \begin{enumerate}
    \item A $\lambda_1 = 1$ sajátértékhez tartozó sajátvektor:
          $$
            \begin{bmatrix}
              1  & -1 \\
              -1 & 1
            \end{bmatrix} \begin{bmatrix}
              x \\
              y
            \end{bmatrix} = \begin{bmatrix}
              0 \\
              0
            \end{bmatrix}
            \quad \Rightarrow \quad
            x = y
            \quad \Rightarrow \quad
            \rvec v_1 = t_1 \begin{bmatrix}
              1 \\
              1
            \end{bmatrix}
          $$

    \item A $\lambda_2 = 3$ sajátértékhez tartozó sajátvektor:
          $$
            \begin{bmatrix}
              -1 & -1 \\
              -1 & -1
            \end{bmatrix} \begin{bmatrix}
              x \\
              y
            \end{bmatrix} = \begin{bmatrix}
              0 \\
              0
            \end{bmatrix}
            \quad \Rightarrow \quad
            x = -y
            \quad \Rightarrow \quad
            \rvec v_2 = t_2 \begin{bmatrix}
              1 \\
              -1
            \end{bmatrix}
          $$
  \end{enumerate}
\end{example}

\begin{note}
  A $\det(\rmat A - \lambda \imat) = 0$ egyenletet \textbf{karakterisztikus
    egyenlet}nek nevezzük.

  A $\det(\rmat A - \lambda \imat)$ polinomot \textbf{karakterisztikus
    polinom}nak nevezzük.
\end{note}

\begin{theorem}[Főtengelytétel]
  Szimmetrikus mátrix sajátvektorai ortogonálisak és a sajátértékek mindig
  valósak.

  Antiszimmetrikus mátrix sajátvektorai páronként konjugáltak és a sajátértékek
  mindig tisztán képzetesek.
\end{theorem}

\begin{statement}
  Ha $\rmat A$ háromszögmátrix, akkor a sajátértékek a főátlóbeli elemek.
\end{statement}

\begin{definition}[Hasonlóság]
  Azt mondjuk, hogy az $\rmat A$ és $\hat{\rmat A}$ mátrixok hasonlóak, ha
  létezik olyan $\rmat T$ invertálható mátrix, hogy
  $$
    \hat{\rmat A} = \rmat T^{-1} \rmat A \rmat T
    \text.
  $$
  Jele: $\rmat A \sim \hat{\rmat A}$.
\end{definition}

\begin{definition}[Diagonalizálhatóság]
  Az $n \times n$-es $\rmat A$ mátrix diagonalizálható, ha hasonló egy
  diagonális mátrixhoz, azaz ha létezik olyan $\rmat Λ$ diagonális mátrix és egy
  $\rmat T$ invertálható mátrix, hogy
  $$
    \rmat Λ = \rmat T^{-1} \rmat A \rmat T
    \text.
  $$
\end{definition}

\begin{theorem}[Diagonizálhatóság szükséges és elégséges feltétele]
  Legyen $\rmat A$ egy $n \times n$-es mátrix. Az $\rmat A$ mátrix akkor és
  csak akkor diagonalizálható, ha létezik $n$ darab lineárisan független
  sajátvektora. Ekkor a diagonális mátrix az $\rmat A$ sajátértékeiből, míg
  a $\rmat T$ transzformációs mátrix $\rmat A$ sajátvektoraiból áll:
  $$
    \rmat Λ
    = \rmat T^{-1} \rmat A \rmat T
    = \begin{bmatrix}
      \lambda_1 & 0         & \cdots & 0         \\
      0         & \lambda_2 & \cdots & 0         \\
      \vdots    & \vdots    & \ddots & \vdots    \\
      0         & 0         & \cdots & \lambda_n
    \end{bmatrix}
    \quad \text{és} \quad
    \rmat T = \begin{bmatrix}
      \rvec v_1 & \rvec v_2 & \cdots & \rvec v_n
    \end{bmatrix}
    \text.
  $$
\end{theorem}

\begin{blueBox}
  \sftitle{Sajátvektorok koordinátarendszere:}

  Egy $\varphi$ leképezés mátrixa tetszőlegesen sok bázisban felírható. Ha
  $\varphi$-t egy a leképezés sajátvektorával párhuzamos vektorra hattatjuk,
  akkor az nyújtásnak felel meg. Ha $\dim \varphi = n$, és $n$ darab lineárisan
  független sajátvektorral rendelkezik, akkor $\varphi$ mátrixreprezentációja
  a sajátvektorok által felírt bázisban diagonális lesz.
\end{blueBox}

\begin{blueBox}
  \sftitle{Invariáns mennyiségek:}

  Legyen $\rmat A$ egy $3 \times 3$-as mátrix, amelynek sajátértékei $\lambda_1$,
  $\lambda_2$ és $\lambda_3$. Ekkor az alábbi mennyiségek bármely $\rmat A$-hoz
  hasonló mátrix esetén invariánsak:
  \begin{itemize}
    \item $
            I_1
            = \operatorname{tr} \rmat A
            = \lambda_1 + \lambda_2 + \lambda_3
          $,

    \item $
            I_2
            = \dfrac{1}{2} \left(
            (\operatorname{tr} \rmat A)^2 - \operatorname{tr} (\rmat A^2)
            \right)
            = \lambda_1 \lambda_2 + \lambda_2 \lambda_3 + \lambda_3 \lambda_1$,

    \item $
            I_3
            = \det \rmat A
            = \lambda_1 \lambda_2 \lambda_3
          $.
  \end{itemize}
\end{blueBox}

\begin{blueBox}
  \sftitle{Mátrixfüggvények:}

  Legyen $\rmat A$ egy $n \times n$-es, $\rmat T$ mátrix segítségével
  diagonizálható mátrix, amelyre szeretnénk hattatni az $f$ függvényt.
  Ekkor az $f(\rmat A)$ mátrixot a következő módon számíthatjuk ki:
  $$
    f(\rmat A)
    = \rmat T f(\rmat Λ) \rmat T^{-1}
    = \rmat T \begin{bmatrix}
      f(\lambda_1) & 0            & \cdots & 0            \\
      0            & f(\lambda_2) & \cdots & 0            \\
      \vdots       & \vdots       & \ddots & \vdots       \\
      0            & 0            & \cdots & f(\lambda_n)
    \end{bmatrix} \rmat T^{-1}
    \text.
  $$
\end{blueBox}

\begin{example}
  Számítsuk ki az $\rmat A = \begin{bmatrix}
      2  & -1 \\
      -1 & 2
    \end{bmatrix}$ mátrix tizedik hatványát!

  Az $\rmat A$ mátrix sajátértékei $\lambda_1 = 1$ és $\lambda_2 = 3$.
  A hozzájuk tartozó sajátvektorok pedig $\rvec v_1(1; 1)$ és $\rvec v_2(1; -1)$.
  Legyen $\rmat T = \begin{bmatrix} \rvec v_1 & \rvec v_2 \end{bmatrix}$.
  Ekkor az $\rmat A$ mátrix diagonális alakja:
  $$
    \rmat Λ
    = \rmat T^{-1} \rmat A \rmat T
    = \begin{bmatrix}
      1/2 & 1/2  \\
      1/2 & -1/2
    \end{bmatrix} \begin{bmatrix}
      2  & -1 \\
      -1 & 2
    \end{bmatrix} \begin{bmatrix}
      1 & 1  \\
      1 & -1
    \end{bmatrix} = \begin{bmatrix}
      1 & 0 \\
      0 & 3
    \end{bmatrix}
    \text.
  $$
  Ezek alapján:
  $$
    \rmat A^{10}
    = \begin{bmatrix}
      1 & 1  \\
      1 & -1
    \end{bmatrix} \begin{bmatrix}
      1^{10} & 0      \\
      0      & 3^{10}
    \end{bmatrix} \begin{bmatrix}
      1/2 & 1/2  \\
      1/2 & -1/2
    \end{bmatrix} = \begin{bmatrix}
      29525  & -29524 \\
      -29524 & 29525
    \end{bmatrix}
    \text.
  $$
\end{example}

\begin{definition}[Sajátaltér]
  Legyen $\rmat A \in \mathcal M_{n \times n}$ és legyen $\lambda_i$ az $\rmat A$
  egy sajátértéke. A $\lambda_i$-hez tartozó sajátaltér az alábbi halmaz:
  $$
    E_{\lambda_i} = \{ \rvec v \mid \textbf{A}\rvec{v} = \lambda_i \rvec v \}.
  $$
  Ez az $\Reals^n$ egy altere.
\end{definition}

\begin{blueBox}
  \sftitle{Algebrai és geometriai multiplicitás:}

  Ha a $\det(\rmat A - \lambda \imat) = 0$ karakterisztikus egyenletnek
  $\lambda_i$ $k$-szoros gyöke, akkor azt mondjuk, hogy a $\lambda_i$-nek az
  algebrai multiplicitása $k$. Ebben az esetben a $\lambda_i$ sajátértékhez
  tartozó sajátaltér $d$ dimenziója (geometriai multiplicitása)
  $1 \leq d \leq k$.
\end{blueBox}

\begin{note}
  A geometriai multiplicitás sosem nagyobb az algebrainál.

  A diagonalizálhatóság ekvivalens azzal, hogy a geometriai és algebrai
  multiplicitások minden sajátérték esetén megegyeznek.
\end{note}

\begin{example}
  Diagonalizálható-e az $\rmat A = \begin{bmatrix}
      1 & 2 & 2 \\
      2 & 1 & 2 \\
      3 & 3 & 2
    \end{bmatrix}$ mátrix?

  Először határozzuk meg az $\rmat A$ mátrix sajátértékeit:
  $$
    \det(\rmat A - \lambda \imat)
    =
    \begin{vmatrix}
      1 - \lambda & 2           & 2           \\
      2           & 1 - \lambda & 2           \\
      3           & 3           & 2 - \lambda
    \end{vmatrix}
    = -\lambda^3 + 4\lambda^2 + 11\lambda + 6
    = -(\lambda + 1)^2(\lambda - 6) = 0
    \text.
  $$

  Láthatjuk, hogy a $\lambda_{12} = -1$ sajátérték algebrai multiplicitása $2$.
  Keressük meg a hozzá tartozó sajátvektort/sajátvektorokat. Oldjuk meg az
  $(\rmat A - 1 \imat) \rvec v = \nvec$ egyenletet:
  $$
    \begin{bmatrix}
      2 & 2 & 2 \\
      2 & 2 & 2 \\
      3 & 3 & 3
    \end{bmatrix}
    \begin{bmatrix}
      v_1 \\
      v_2 \\
      v_3
    \end{bmatrix}
    =
    \begin{bmatrix}
      0 \\
      0 \\
      0
    \end{bmatrix}
    \quad \Rightarrow \quad
    v_1 = -v_2 - v_3
    \quad \Rightarrow \quad
    \begin{bmatrix}
      -t_1 - t_2 \\
      t_1        \\
      t_2
    \end{bmatrix}
    = \underbrace{t_1
      \begin{bmatrix}
        -1 \\
        1  \\
        0
      \end{bmatrix}}_{\rvec v_1}
    + \underbrace{t_2
      \begin{bmatrix}
        -1 \\
        0  \\
        1
      \end{bmatrix}}_{\rvec v_2}
  $$
  Láthatjuk, hogy a $\lambda_{12}$ sajátértékhez tartozó sajátaltér dimenziója
  $2$, amely megegyezik az algebrai multiplicitással, tehát az $\rmat A$ mátrix
  diagonalizálható.
\end{example}

\begin{example}
  Diagonizálható-e az $\rmat A = \begin{bmatrix} 2 & 1 \\ 0 & 2\end{bmatrix}$
  mátrix?

  Először határozzuk meg az $\rmat A$ mátrix sajátértékeit:
  $$
    \det(\rmat A - \lambda \imat) = \begin{vmatrix}
      2 - \lambda & 1           \\
      0           & 2 - \lambda
    \end{vmatrix} = (2 - \lambda)^2 = 0
    \text.
  $$
  Láthatjuk, hogy a $\lambda = 2$ sajátérték algebrai multiplicitása $2$.

  A sajátvektorok meghatározása az $(\rmat A - \lambda \imat) \rvec v = \nvec$
  egyenlet segítségével:
  $$
    \begin{bmatrix}
      0 & 1 \\
      0 & 0
    \end{bmatrix}
    \begin{bmatrix}
      v_1 \\
      v_2
    \end{bmatrix}
    =
    \begin{bmatrix}
      0 \\
      0
    \end{bmatrix}
    \quad \Rightarrow \quad
    \rvec v = t \begin{bmatrix}
      1 \\
      0
    \end{bmatrix}
  $$
  A $\rmat A$ mátrix nem diagonalizálható, mivel a sajátértékhez tartozó
  geometriai multiplicitás (vagyis a sajátaltér dimenziója) $1$.
\end{example}

\begin{note}
  Ha egy $2 \times 2$-es mátrix $\lambda$ sajátértékéhez tartozó algebrai és
  geometriai multiplicitás is $2$, akkor a mátrix diagonális.
\end{note}

\begin{blueBox}
  \sftitle{Kvadratikus formák és másodrendű görbék:}

  Egy csupa másodfokú tagot tartalmazó kétváltozós polinom átírható mátrixos
  alakba:
  $$
    \rho(x;y) = a x^2 + 2b x y + c y^2 = \begin{bmatrix}
      x & y
    \end{bmatrix} \begin{bmatrix}
      a & b \\
      b & c
    \end{bmatrix} \begin{bmatrix}
      x \\
      y
    \end{bmatrix} = \rvec x^\T \rmat A \rvec x
    \text.
  $$
  Ha $\rho(x; y) = 0$, akkor az egyenlet egy origó középpontú másodrendű görbét
  ír le. Az $\rmat A$ mátrix definitsége alapján a görbe lehet
  \begin{itemize}
    \item ellipszis, ha $\rmat A$ definit,
          \hfill (sajátértékek azonos előjelűek)

    \item parabola, ha $\rmat A$ szemidefinit,
          \hfill (egyik sajátérték nulla)

    \item hiperbola, ha $\rmat A$ indefinit.
          \hfill (sajátértékek ellentétes előjelűek)
  \end{itemize}
  Amennyiben a görbe egyenlete nem csak másodfokú tagokat tartalmaz, akkor azzal
  egy általános másodrendű görbét írunk le:
  \begin{alignat*}{9}
    a x^2 + 2b x y + c y^2 \;\;
     & + \;\;
    d x + e y
     & \; + \; f = 0
    \text,
    \\
    \underbrace{
      \vphantom{\begin{bmatrix} x \\ y \end{bmatrix}}
      \begin{bmatrix}
        x & y
      \end{bmatrix}
    }_{\rvec x^\T}
    \;
    \underbrace{
      \begin{bmatrix}
        a & b \\
        b & c
      \end{bmatrix}
    }_{\rmat A}
    \;
    \underbrace{
      \begin{bmatrix}
        x \\
        y
      \end{bmatrix}
    }_{\rvec x}
     & +
    \underbrace{
      \vphantom{\begin{bmatrix} x \\ y \end{bmatrix}}
      \begin{bmatrix}
        d & e
      \end{bmatrix}
    }_{\rvec k^\T}
    \underbrace{
      \;
      \begin{bmatrix}
        x \\
        y
      \end{bmatrix}
      \;
    }_{\rvec x}
     & \; + \; f = 0
    \text.
    % \\
    % \rvec x^\T \rmat A \rvec x
    %  & +
    % \rvec b^\T \rvec x
    %  & \; +
    %  & \; f
    %  & = 0
  \end{alignat*}

  Legyenek $\rmat A$ mátrix sajátértékei $\lambda_1$ és $\lambda_2$, és legyen
  $\rvec v_1$ a $\lambda_1$-hez, $\rvec v_2$ pedig a $\lambda_2$-höz tartozó
  egységhosszúságú sajátvektor. Képezzük a $\rmat T$ transzformációs mátrixot,
  amelynek oszlopai tartalmazzák a $\rvec v_1$ és $\rvec v_2$ vektorokat,
  vagyis $\rmat T = \begin{bmatrix} \rvec v_1 & \rvec v_2 \end{bmatrix}$.
  A mátrix segítségével az általános másodrendű görbe egyenlete átírható
  kanonikus alakra:
  $$
    \underbrace{
      \rvec x^\T
      \rmat T
    }_{\rvec \xi^\T}
    \cdot
    \underbrace{
      \rmat T^{-1}
      \rmat A
      \rmat T
    }_{\rmat Λ}
    \cdot
    \underbrace{
      \rmat T^{-1}
      \rvec x
    }_{\rvec \xi}
    +
    \underbrace{
      \rvec k^\T
      \rmat T
    }_{\rvec \kappa^\T}
    \cdot
    \underbrace{
      \rmat T^{-1}
      \rvec x
    }_{\rvec \xi}
    + f
    = 0
    \text,
  $$
  ahol $\rvec \xi = \begin{bmatrix} \xi & \eta \end{bmatrix}^\T$ az $\rmat A$
  mátrix sajátkoordinátái, $\rmat Λ$ diagonális mátrix, melynek főátlójában az
  $\rmat A$ mátrix sajátértékei szerepelnek, $\rvec \kappa$ pedig tartalmazza a
  $\xi$ és $\eta$ irányba való eltolást.

  \begin{multicols}{3}
    \begin{center}
      \begin{tikzpicture}[very thick, scale=2/3]
        \begin{scope}[xshift=3.43cm, yshift=2.845cm]
          \draw[-to] (-3,0) -- (3,0) node[below left] {$x$};
          \draw[-to] (0,-3) -- (0,3) node[below right] {$y$};

          \begin{scope}[rotate=60, draw=primaryColor]
            \draw[-to] (-3,0) -- (3,0) node[below right] {$\xi$};
            \draw[-to] (0,-3) -- (0,3) node[above right] {$\eta$};
          \end{scope}
        \end{scope}
        \begin{axis}[
            xticklabels={,,},
            yticklabels={,,},
            axis line style={draw=none},
            tick style={draw=none},
            ymin=-50/6.86, xmin=-50/5.69,
            ymax=50/6.86, xmax=50/5.69,
          ]
          \addplot +[
          no markers,
          raw gnuplot,
          ultra thick,
          empty line = jump,
          secondaryColor,
          ] gnuplot {
              set contour base;
              set cntrparam levels discrete 0.003;
              unset surface;
              set view map;
              set isosamples 250;
              splot -3 * x**2 + 23 * y**2 + 45.033321 * x * y - 144;
            };
        \end{axis}
      \end{tikzpicture}

      Hiperbola
    \end{center}

    \begin{center}
      \begin{tikzpicture}[very thick, scale=2/3]
        \begin{scope}[xshift=3.43cm, yshift=2.845cm]
          \draw[-to] (-3,0) -- (3,0) node[below left] {$x$};
          \draw[-to] (0,-3) -- (0,3) node[below right] {$y$};

          \begin{scope}[rotate=30, draw=primaryColor]
            \draw[-to] (-3,0) -- (3,0) node[below] {$\xi$};
            \draw[-to] (0,-3) -- (0,3) node[right] {$\eta$};
          \end{scope}
        \end{scope}
        \begin{axis}[
            xticklabels={,,},
            yticklabels={,,},
            axis line style={draw=none},
            tick style={draw=none},
            ymin=-50/6.86, xmin=-50/5.69,
            ymax=50/6.86, xmax=50/5.69,
          ]
          \addplot +[
          no markers,
          raw gnuplot,
          ultra thick,
          empty line = jump,
          secondaryColor,
          ] gnuplot {
              set contour base;
              set cntrparam levels discrete 0.003;
              unset surface;
              set view map;
              set isosamples 250;
              splot 57 * x**2 + 43 * y**2 + 24.24871131 * x * y - 864;
            };
        \end{axis}
      \end{tikzpicture}

      Ellipszis
    \end{center}

    \begin{center}
      \begin{tikzpicture}[very thick, scale=2/3]
        \begin{scope}[xshift=3.43cm, yshift=2.845cm]
          \draw[-to] (-3,0) -- (3,0) node[below left] {$x$};
          \draw[-to] (0,-3) -- (0,3) node[below right] {$y$};

          \begin{scope}[rotate=45, draw=primaryColor]
            \draw[-to] (-3,0) -- (3,0) node[below right] {$\xi$};
            \draw[-to] (0,-3) -- (0,3) node[above right] {$\eta$};
          \end{scope}
        \end{scope}
        \begin{axis}[
            xticklabels={,,},
            yticklabels={,,},
            axis line style={draw=none},
            tick style={draw=none},
            ymin=-50/6.86, xmin=-50/5.69,
            ymax=50/6.86, xmax=50/5.69,
          ]
          \addplot +[
          no markers,
          raw gnuplot,
          ultra thick,
          empty line = jump,
          secondaryColor,
          ] gnuplot {
              set contour base;
              set cntrparam levels discrete 0.003;
              unset surface;
              set view map;
              set isosamples 250;
              splot 1.41421356 * (x**2 + y**2) - 2.82842712 * x * y - 3 * (x + y);
            };
        \end{axis}
      \end{tikzpicture}

      Parabola
    \end{center}
  \end{multicols}
\end{blueBox}

\begin{note}
  A $\rmat Λ$ mátrix főátlójába a sajátértékeket olyan sorrendben kell beírni,
  amilyen sorrendben a sajátvektorokat a $\rmat T$ mátrixba rendeztük.
\end{note}

\clearpage
\subsection{Feladatok}
\begin{enumerate}
  \item Adja meg az alábbi mátrixok sajátvektorait és sajátértékeit!
        \begin{alignat*}{9}
          \rmat A & =
          \begin{bmatrix}
            4 & 3 \\
            1 & 2
          \end{bmatrix}
                  & \rmat B & =
          \begin{bmatrix}
            -2 & -8 & -12 \\
            1  & 4  & 4   \\
            0  & 0  & 1
          \end{bmatrix}
          \hspace{15mm}
                  & \rmat C & =
          \begin{bmatrix}
            0  & 1 \\
            -1 & 0
          \end{bmatrix}
          \\
          \rmat D & =
          \begin{bmatrix}
            1 & 0 & 0 & 0 \\
            0 & 1 & 0 & 0 \\
            0 & 0 & 2 & 1 \\
            0 & 0 & 0 & 2
          \end{bmatrix}
          \hspace{15mm}
                  & \rmat E & =
          \begin{bmatrix}
            4 & 1 & 0 \\
            0 & 4 & 1 \\
            0 & 0 & 4
          \end{bmatrix}
                  & \rmat F & =
          \begin{bmatrix}
            2 & 2 & 1 \\
            1 & 3 & 1 \\
            1 & 2 & 2
          \end{bmatrix}
        \end{alignat*}

  \item A leképezés mátrixainak felírása nélkül adja meg a lehető legtöbb
        sajátértélet és sajátvektort!
        \begin{enumerate}
          \item $z$-tengely körüli $45^\circ$-os forgatás,
          \item $xy$ síkra vetítés,
          \item $xy$ síkra tükrözés.
        \end{enumerate}

  \item Diagonizálhatóak-e a harmadik feladatban szereplő $\rmat E$, $\rmat D$
        és $\rmat B$ mátrixok?

  \item A sajátértékek kiszámítása nélkül mondjuk meg a lehető legtöbb
        sajátér\-ték-sa\-ját\-vek\-tor párt!
        $$
          \rmat A =
          \begin{bmatrix}
            1 & 0 & 0 \\
            2 & 4 & 0 \\
            3 & 0 & 5
          \end{bmatrix}
          \hspace{2cm}
          \rmat B =
          \begin{bmatrix}
            5 & 0 & 0 \\
            0 & 7 & 0 \\
            0 & 0 & 9
          \end{bmatrix}
        $$

  \item Határozzuk meg a harmadik feladatban szereplő $\rmat B$ mátrix tizedik
        hatványát!

  \item Határozzuk meg az $e^{10^{\rmat B}}$ függvényt, ha $\rmat B$ a harmadik
        feladatban szereplő mátrix!

  \item Milyen alakzatot írnak le az alábbi másodrendű görbék?
        Írja fel a kanonikus egyenletüket!
        \begin{enumerate}
          \item $
                  -3x^2 + 23y^2 + 26\sqrt{3}xy = 144
                $

          \item $
                  57x^2 + 43y^2 + 14\sqrt{3}xy = 576
                $

          \item $
                  2x^2 - 5 = 0
                $
        \end{enumerate}
\end{enumerate}




\end{document}