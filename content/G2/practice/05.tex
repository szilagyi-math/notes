\documentclass[a4paper, 12pt]{scrartcl}

\usepackage{math-practice}
\usepackage{tikz}
\usepackage{multicol}
\usepackage{amsmath}
\usepackage{arydshln}

\area{Lineáris Algebra}
\title{Lineáris leképzések II}
\subject{Matematika G2}
\subjectCode{BMETE94BG02}
\date{Utoljára frissítve: \today}
\docno{5}

\begin{document}
\maketitle
\subsection{Elméleti Áttekintő}
\begin{definition}[Ortogonalitás]
  Egy mátrix ortogonális, ha
  \[
    \textbf{Q}^{-1} = \textbf{Q}^T.
  \]
  Az ortogonalitás azt jelenti, hogy ha $<\rvec v, \rvec w>=0$, akkor $<\textbf{A}\rvec v, \textbf{A}\rvec{w}> =0$, azaz két merőleges vektor a transzforméció után is merőleges marad. Bázistranszformáció esetén ha 2 ortonormált bázisunk van, akkor a bázistrafó mátrixa ortogonális.
\end{definition}
\begin{blueBox}
  A bázis orientációja változhat, ha a mátrix determinánsa negatív.\\
  \textbf{Példa:} $\alpha$ szöggel forgatás mátrixa $\begin{bmatrix}
      \cos \alpha & -\sin \alpha \\
      \sin \alpha & \cos \alpha
    \end{bmatrix}$, ami ortogonális, mivel $\operatorname{det}\textbf{A}$ = $\cos^2 \alpha + \sin^2 \alpha = 1$.
\end{blueBox}

\begin{definition}[Sajátérték, sajátvektor]
  Tegyük fel, hogy adott egy $\textbf{A}_{n \times n}$ és egy nem zérus $n$ hosszúságú vektor $\rvec{v}$. Ha az $\textbf{A}$-szal történő szorzás $\rvec{v}$-vel (jelölve: $\textbf{A}\rvec{v}$) egyszerűen $\rvec{v}$-t egy $\lambda$ skalár faktorral méretezi át, ahol $\lambda$ egy skalár, akkor $\rvec{v}$-t $\textbf{A}$ sajátvektorának nevezzük, és $\lambda$ az ehhez tartozó sajátérték. Ez a kapcsolat az alábbi módon fejezhető ki:
  \[
    \textbf{A}\rvec{v} = \lambda \rvec{v}.
  \]
\end{definition}

\begin{note}
  Szimmetrikus mátrix sajátvektorai ortogonálisak és a sajátértékek mindig valósak.
\end{note}

\begin{note}
  Antiszimmetrikus mátrix sajátvektorai páronként konjugáltak és a sajátértékek mindig tisztán képzetesek.
\end{note}

\begin{definition}[Sajátaltér]
  Legyen \( \textbf{A} \) egy \( n \times n \)-es mátrix, és legyen \( \lambda \) az \( \textbf{A} \) egy sajátértéke. A \( \lambda \)-hoz tartozó sajátvektortér az alábbi halmaz:
  \[
    E_\lambda = \{ \rvec{v} \mid \textbf{A}\rvec{v} = \lambda\rvec{v} \}.
  \]
  Ez az \( \mathbb{R}^n \) egy altere.
\end{definition}

Itt meg kellene az algebrai es a geometriai multiplicitast is emliteni, de nem vagyok benne mi a helyes megfogalmazás, zavaros a kézzel írt jegyzet.

\begin{definition}[Karakterisztikus polinom]
  Legyen \( \textbf{A} \) egy négyzetes mátrix. Az alábbi kifejezést:
  \[
    p(\lambda) = \det(\textbf{A} - \lambda \mathbb{I})
  \]
  \( \textbf{A} \) karakterisztikus polinomjának nevezzük.
\end{definition}

\begin{blueBox}
  Legyen \( \textbf{A} \) egy \( n \times n \)-es mátrix. Az \( \textbf{A} \)-hoz tartozó sajátértékek és sajátvektorok meghatározásának lépései a következők:
  \begin{enumerate}
    \item Számítsd ki a karakterisztikus polinomot: \( \det(\textbf{A} - \lambda \mathbb{I}) \).
    \item A sajátértékek a karakterisztikus polinom gyökei.
    \item Minden sajátértékre (\( \lambda \)) határozz meg egy bázist a sajátvektorokhoz az alábbi homogén egyenletrendszer megoldásával:
          \[
            (\textbf{A} - \lambda \mathbb{I})\rvec{v} = 0.
          \]
  \end{enumerate}
\end{blueBox}

\begin{note}
  Legyen \( \textbf{A} \) egy felső vagy alsó háromszögmátrix. Ekkor az \( \textbf{A} \)-hoz tartozó sajátértékek a főátló elemei.
\end{note}

\begin{definition}[Diagonalizálható mátrix]
  Legyen \( \textbf{A} \) egy \( n \times n \)-es mátrix. Az \( \textbf{A} \)-t diagonalizálhatónak nevezzük, ha létezik egy invertálható \( \textbf{P} \) mátrix és egy \( \textbf{D} \) diagonális mátrix, amelyekre teljesül, hogy
  \[
    \textbf{P}^{-1}\textbf{AP} = \textbf{D}.
  \]
\end{definition}

\begin{theorem}[Diagonizálhatóság és sajátvektorok]
  Egy \( n \times n \)-es mátrix \( A \) akkor és csak akkor diagonalizálható, ha \( A \)-nak \( n \) lineárisan független sajátvektora van.

  Továbbá, ebben az esetben legyen \( P \) az az invertálható mátrix, amelynek oszlopai \( A \) \( n \) lineárisan független sajátvektorai, és legyen \( D \) az a diagonális mátrix, amelynek főátlójában a megfelelő sajátértékek szerepelnek. Ekkor
  \[
    P^{-1}AP = D.
  \]
\end{theorem}

\begin{definition}[Mátrixfüggvények]
  Ide kell egy definíció, mert nincs rendes a füzetben!
\end{definition}

\begin{blueBox}
  \textbf{Kvadratikus formák}\\
  Csupa másodfokú tagot tartalmazó polinomokat átírhatóak mátrixos alakba.
  \[
    ax^2 + 2bxy + cy^2 = \begin{bmatrix}
      x & y
    \end{bmatrix} \begin{bmatrix}
      a & b \\
      b & c
    \end{bmatrix} \begin{bmatrix}
      x \\
      y
    \end{bmatrix}
  \]
  \textbf{Másodrendű görbe}
  \[
    ax^2 + 2bxy + cy^2 + dx + ey + f = 0
  \]
  \[
    \begin{bmatrix}
      x & y
    \end{bmatrix} \begin{bmatrix}
      a & b \\
      b & c
    \end{bmatrix} \begin{bmatrix}
      x \\
      y
    \end{bmatrix} + \begin{bmatrix}
      d & e
    \end{bmatrix} \begin{bmatrix}
      x \\
      y
    \end{bmatrix} + f = 0
  \]

  Másodrendű görbék lehetnek ellipszisek, hiperbólák és parabolák.

  IDE KÉNE EGY KIS rendezett leírás, arra, hogy mikor is lehet kvadratikus alakot áttranszformálni sajátértékek koordinata rendszerébe, hogyan alakulnak az egyenletek es mikor is lesz semi deficit deficit és indeficit.
\end{blueBox}

\clearpage
\subsection{Feladatok}
\begin{enumerate}
  \item Írjuk fel az $\frac{-1}{2}x + \frac{\sqrt{3}}{2}y = 0$ és $z=0$ egyenes körül pozizív $y$ irányból $90^\circ$-os forgatás mátrixát a szokásos, illetve a $\rvec{v}_1 = (1,0,0)$, $\rvec{v}_2 =(1,1,0)$ és $\rvec{v}_3 = (1,1,1)$ bázisokban!
  \item Adjuk meg a $\beta$ és $\alpha$ paramétereket, hogy az alábbi leképezés mátrixa orientciótartó és skalárisszorzattartó legyen (ortogonális)!
  \item Számolja ki az alábbi mátrix sajátvektorait és sajátértékeit!
        \begin{multicols}{3}
          \begin{enumerate}
            \item $\begin{bmatrix}
                      4 & 3 \\
                      1 & 2
                    \end{bmatrix}$
            \item $\begin{bmatrix}
                      -2 & -8 & -12 \\
                      1  & 4  & 4   \\
                      0  & 0  & 1
                    \end{bmatrix}$
            \item $\begin{bmatrix}
                      0  & 1 \\
                      -1 & 0
                    \end{bmatrix}$
            \item $\begin{bmatrix}
                      1 & 0 & 0 & 0 \\
                      0 & 1 & 0 & 0 \\
                      0 & 0 & 2 & 1 \\
                      0 & 0 & 0 & 2
                    \end{bmatrix}$

            \item $ \begin{bmatrix}
                      4 & 1 & 0 \\
                      0 & 4 & 1 \\
                      0 & 0 & 4
                    \end{bmatrix}$

            \item $\begin{bmatrix}
                      2 & 2 & 1 \\
                      1 & 3 & 1 \\
                      1 & 2 & 2
                    \end{bmatrix}$
          \end{enumerate}
        \end{multicols}
  \item A leképezés mátrixával felírása nélkül adjuk meg, hogy a lehető legtöbb sajátvektor/sajátértéket!
        \begin{enumerate}
          \item $z$ körüli $45^{\circ}$-os forgatás
          \item $xy$-ra való vetítés
          \item $xy$-ra való tükrözés
        \end{enumerate}
  \item Diagonizálható-e az alábbi mátrixok feladatbeli mátrix?
        \begin{multicols}{3}
          \begin{enumerate}
            \item $\begin{bmatrix}
                      4 & 3 \\
                      1 & 2
                    \end{bmatrix}$
            \item $\begin{bmatrix}
                      -2 & -8 & -12 \\
                      1  & 4  & 4   \\
                      0  & 0  & 1
                    \end{bmatrix}$
            \item $\begin{bmatrix}
                      0  & 1 \\
                      -1 & 0
                    \end{bmatrix}$
            \item $\begin{bmatrix}
                      1 & 0 & 0 & 0 \\
                      0 & 1 & 0 & 0 \\
                      0 & 0 & 2 & 1 \\
                      0 & 0 & 0 & 2
                    \end{bmatrix}$

            \item $ \begin{bmatrix}
                      4 & 1 & 0 \\
                      0 & 4 & 1 \\
                      0 & 0 & 4
                    \end{bmatrix}$

            \item $\begin{bmatrix}
                      2 & 2 & 1 \\
                      1 & 3 & 1 \\
                      1 & 2 & 2
                    \end{bmatrix}$
          \end{enumerate}
        \end{multicols}

  \item A sajátértékek kiszámítása nélkül mondjuk meg a lehető legtöbb sajátérték-sajátvektor párt!
        \[
          \text{a)}
          \begin{bmatrix}
            1 & 0 & 0 \\
            2 & 4 & 0 \\
            3 & 0 & 5
          \end{bmatrix} \quad
          \text{b)}
          \begin{bmatrix}
            5 & 0 & 0 \\
            0 & 7 & 0 \\
            0 & 0 & 9
          \end{bmatrix}
        \]

  \item Határozzuk meg az alábbi mátrix tizedik hatványát!
        \[
          \begin{bmatrix}
            -2 & -8 & -12 \\
            1  & 4  & 4   \\
            0  & 0  & 1
          \end{bmatrix}.
        \]

  \item Határozzuk meg az \( e^{10^{\textbf{A}}} \) függvényt!

  \item Milyen alakzatot ír le az alábbi másodrendű görbe? Írjuk fel a kanonikus egyenletét!
        \[
          -3x^2 + 23y^2 + 26\sqrt{3}xy = 144.
        \]

  \item Milyen alakzatot ír le az alábbi másodrendű görbe? Írjuk fel a kanonikus egyenletét!
        \[
          57x^2 + 43y^2 + 14\sqrt{3}xy = 576.
        \]

  \item Milyen alakzatot ír le az alábbi másodrendű görbe? Írjuk fel a kanonikus egyenletét!
        \[
          2x^2 - 5 = 0.
        \]
\end{enumerate}




\end{document}