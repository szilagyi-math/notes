\documentclass[a4paper, 12pt]{scrartcl}

\usepackage{math-practice}

\area{Valós analízis}
\title{Fourier-sorok}
\subject{Matematika G2}
\subjectCode{BMETE94BG02}
\date{Utoljára frissítve: \today}
\docno{9}

\begin{document}
\maketitle
\subsection{Elméleti Áttekintő}

% \begin{definition}[Trigonometrikus polinom]
%   Az alábbi alakú függvényt trigonometrikus polinopmnak nevezzük:
%   $$
%     t_n(x)
%     = a_0
%     + a_1 \cos x + b_1 \sin x
%     + a_2 \cos 2x + b_2 \sin 2x
%     + \ldots
%     + a_n \cos nx + b_n \sin nx
%     \text.
%   $$
% \end{definition}

% \begin{definition}[Trigonometrikus sor]
%   Az alábbi alakó összeget trigonometrikus sornak nevezzük:
%   \begin{align*}
%     t(x)
%      & = a_0
%     + a_1 \cos x + b_1 \sin x
%     + \ldots
%     + a_n \cos nx + b_n \sin nx
%     + \ldots
%     =
%     \\
%      & = a_0
%     + \sum_{k=1}^{\infty} a_k \cos kx + b_k \sin kx
%   \end{align*}
% \end{definition}

\begin{blueBox}
  \sftitle{A $2\pi$ szerint periodikus függvények vektortere:}

  A $2\pi$ szerint periodikus függvények az összeadásra és a skalárral való
  szorzásra egy végtelen dimenziós vektorteret alkoznak.

  A vektortér egy bázisa:
  $$
    \{\;
    1;
    \cos x; \sin x;
    \cos 2x; \sin 2x;
    \dots;
    \cos kx; \sin kx;
    \dots;
    \;\}
    \text.
  $$
  A vektortéren belül az alábbi módon definiáljuk a skaláris szorzatot:
  $$
    \langle f; g \rangle
    = \int_{-\pi}^{\pi} f(x) g(x) \dd x
    \text.
  $$
  Az előbb definiált bázis vektorai lineárisan függetlenek, vagyis egymással
  vett skaláris szorzatuk mindig nulla: ($k \neq l$)
  \begin{align*}
    \langle 1; \sin kx \rangle
     & = \int_{-\pi}^{\pi} \sin kx \dd x = 0
    \text,
    \\
    \langle 1; \cos kx \rangle
     & = \int_{-\pi}^{\pi} \cos kx \dd x = 0
    \text,
    \\
    \langle \sin kx; \cos lx \rangle
     & = \int_{-\pi}^{\pi} \sin kx \cos lx \dd x
    = \frac{1}{2} \int_{-\pi}^{\pi} \sin[(k-l)x] + \sin[(k+l)x] \dd x
    = 0
    \text,
    \\
    \langle \sin kx; \sin lx \rangle
     & = \int_{-\pi}^{\pi} \sin kx \sin lx \dd x
    = \frac{1}{2} \int_{-\pi}^{\pi} \cos[(k-l)x] - \cos[(k+l)x] \dd x
    = 0
    \text,
    \\
    \langle \cos kx; \cos lx \rangle
     & = \int_{-\pi}^{\pi} \cos kx \cos lx \dd x
    = \frac{1}{2} \int_{-\pi}^{\pi} \cos[(k-l)x] + \cos[(k+l)x] \dd x
    = 0
    \text.
  \end{align*}
  A bázisvektorok hosszai a skaláris szorzat segítségével:
  \begin{align*}
    \langle 1; 1 \rangle
     & = \int_{-\pi}^{\pi} 1 \cdot 1 \dd x = 2\pi
    \text,
    \\
    \langle \sin kx; \sin kx \rangle
     & = \int_{-\pi}^{\pi} \sin^2 kx \dd x =
    \int_{-\pi}^{\pi} \frac{1 - \cos 2kx}{2} \dd x
    = \pi
    \text,
    \\
    \langle \cos kx; \cos kx \rangle
     & = \int_{-\pi}^{\pi} \cos^2 kx \dd x =
    \int_{-\pi}^{\pi} \frac{1 + \cos 2kx}{2} \dd x
    = \pi
    \text.
  \end{align*}
  % \begin{align*}
  %   \langle 1; 1 \rangle
  %    & = \int_{-\pi}^{\pi} 1 \cdot 1 \dd x = 2\pi
  %   \text,
  %   \\
  %   \langle 1; \sin kx \rangle
  %    & = \int_{-\pi}^{\pi} \sin kx \dd x = 0
  %   \text,
  %   \\
  %   \langle 1; \cos kx \rangle
  %    & = \int_{-\pi}^{\pi} \cos kx \dd x = 0
  %   \text,
  %   \\
  %   \langle \sin kx; \sin lx \rangle
  %    & = \int_{-\pi}^{\pi} \sin kx \cos lx \dd x
  %   = 0
  %   \text,
  %   \\
  %   \langle \sin kx; \sin lx \rangle
  %    & = \int_{-\pi}^{\pi} \sin kx \sin lx \dd x =
  %   \\
  %    & = \frac{1}{2} \int_{-\pi}^{\pi} \cos[(k - l)x] - \cos[(k + l)x] \dd x
  %   = \begin{cases}
  %       0,   & \text{ha } k \neq l \\
  %       \pi, & \text{ha } k = l
  %     \end{cases}
  %   \text,
  %   \\
  %   \langle \cos kx; \cos lx \rangle
  %    & = \int_{-\pi}^{\pi} \cos kx \cos lx \dd x =
  %   \\
  %    & = \frac{1}{2} \int_{-\pi}^{\pi} \cos[(k - l)x] + \cos[(k + l)x] \dd x
  %   = \begin{cases}
  %       0,   & \text{ha } k \neq l \\
  %       \pi, & \text{ha } k = l
  %     \end{cases}
  % \end{align*}
\end{blueBox}

\begin{blueBox}
  Láthatjuk, hogy az előbb leírt bázis nem ortonormált, azonban ha az egyes
  vektorokat normáljuk, akkor ortonormált bázist kapunk:
  $$
    \left\{\;
    \frac{1}{\sqrt{2\pi}};
    \frac{1}{\sqrt{\pi}} \cos x; \frac{1}{\sqrt{\pi}} \sin x;
    \frac{1}{\sqrt{\pi}} \cos 2x; \frac{1}{\sqrt{\pi}} \sin 2x;
    \dots;
    \frac{1}{\sqrt{\pi}} \cos nx; \frac{1}{\sqrt{\pi}} \sin nx;
    \dots
    \;\right\}
  $$
  Ezek alapján tetszőleges $2\pi$ szerint periodikus függvény felírható az
  alábbi alakban:
  $$
    f(x) = a_0 + \left.\sum_{k=1}^{\infty} \middle( a_k \cos kx + b_k \sin kx \right)
    \text,
  $$
  ahol az együtthatók a következő módon számíthatóak:
  \begin{alignat*}{9}
    a_0
     & = \frac{\langle f; 1 \rangle}{\langle 1; 1 \rangle}
     & \;=\;
     & \frac{1}{2\pi} \int_{-\pi}^{\pi} f(x) \dd x
    \text,
    \\
    a_k
     & = \frac{\langle f; \cos kx \rangle}{\langle \cos kx; \cos kx \rangle}
     & \;=\;
     & \frac{1}{\pi} \int_{-\pi}^{\pi} f(x) \cos kx \dd x
    \text,
    \\
    b_k
     & = \frac{\langle f; \sin kx \rangle}{\langle \sin kx; \sin kx \rangle}
     & \;=\;
     & \frac{1}{\pi} \int_{-\pi}^{\pi} f(x) \sin kx \dd x
    \text.
  \end{alignat*}
\end{blueBox}

\begin{note}
  Ezzel analóg módon, ha tekintjük az $\Reals^3$ egy standard normális
  bázisban $\{\rvec i; \rvec j; \rvec k\}$ felírt $\rvec v(v_1; v_2; v_3)$
  vektorát és egy másik olyan $\{\rvec b_1; \rvec b_2; \rvec b_3\}$
  bázist, amely bázisvektorai merőlegesek egymásra, akkor az $\rvec v$ vektor
  koordinátái az új bázisban:
  $$
    v_i = \frac{
      \langle \rvec v; \rvec b_i \rangle
    }{
      \langle \rvec b_i; \rvec b_i \rangle
    }
    \text.
  $$
\end{note}

\begin{definition}[Fourier-sor]
  Legyen $f : \mathbb R \rightarrow \mathbb R$ egy $2p$ szerint periodikus
  függvény, amely a $[0;2p]$ intervallumon Riemann-integrálható ($f \in
    \mathcal R [0; 2p]$). Ekkor $f$ Fourier-során az alábbi trigonometrikus
  sort értjük:
  $$
    F(x)
    = a_0
    + \sum_{k = 1}^\infty a_k \cos \left( \frac{k \pi x}{p} \right)
    + \sum_{k = 1}^\infty b_k \sin \left( \frac{k \pi x}{p} \right)
    \text.
  $$
  % ahol az együtthatók a következők:
  % \begin{align*}
  %   a_0 & = \frac{1}{2p} \int_{0}^{2p} f(x) \dd x
  %   \text,                                                                          \\
  %   a_k & = \frac{1}{p} \int_{0}^{2p} f(x) \cos \left(\frac{k\pi x}{p}\right) \dd x
  %   \text,                                                                           \qquad
  %   b_k = \frac{1}{p} \int_{0}^{2p} f(x) \sin \left(\frac{k\pi x}{p}\right) \dd x
  % \end{align*}
  Ha a függvény $2 \pi$ szerint periodikus:
  $$
    F(x)
    = a_0
    + \sum_{n = 1}^\infty a_n \cos (nx)
    + \sum_{n = 1}^\infty b_n \sin (nx)
  $$
  % Az együtthatók pedig:
  % $$
  %   a_0 = \frac{1}{2p} \int_{0}^{2p} f(x) \dd x
  %   \text, \quad
  %   a_k  = \frac{1}{\pi} \int_{0}^{2\pi} f(x) \cos(nx) \dd x
  %   \text, \quad
  %   b_k  = \frac{1}{\pi} \int_{0}^{2\pi} f(x) \sin(nx) \dd x
  % $$
\end{definition}

\begin{statement}
  Ha egy függvény páros, akkor a Fourier-sorában csak $a_0$ és koszinusz tagok
  szerepelnek, vagyis $b_k \equiv 0$.
\end{statement}

\begin{statement}
  Ha egy függvény páratlan, akkor a Fourier-sorában csak szinusz tagok
  szerepelnek, vagyis $a_0 \equiv 0$ és $a_k \equiv 0$.
\end{statement}

\begin{blueBox}
  \sftitle{Fourier együtthatók számítása:} ($2\pi$ / $2p$ periodicitás esetén)
  \begin{alignat*}{9}
    a_0 & =
    \frac{1}{2\pi} \int_0^{2\pi} f(x) \dd x
        &   & a_0
        &   & =
    \frac{1}{2p} \int_0^{2p} f(x) \dd x
    \\
    a_k & =
    \frac{1}{\pi} \int_0^{2\pi} f(x) \cos(k x) \dd x
    \hspace{2.2cm}
        &   & a_k
        &   & =
    \frac{1}{p} \int_0^{2p} f(x) \cos \left( \frac{k \pi x}{p} \right) \dd x
    \\
    b_k & =
    \frac{1}{\pi} \int_0^{2\pi} f(x) \sin(k x) \, \mathrm d x
        &   & b_k
        &   & =
    \frac{1}{p} \int_0^{2p} f(x) \sin \left( \frac{k \pi x}{p} \right) \dd x
  \end{alignat*}
\end{blueBox}

\begin{note}
  Ha az $f$ függvény $2p$ szerint periodikus, akkor mindegy, hogy a $[0; 2p]$
  intervallumon, vagy egy skalárral eltolva az $[a; a+2p]$ intervallumon
  integrálunk, vagyis:
  $$
    \int_0^{2p} f(x) \dd x = \int_a^{a+2p} f(x) \dd x
  $$
\end{note}

\begin{theorem}
  Ha a $2\pi$ szerint periodikus $f$ függvénynek létezik az $x_0$ pontban a
  jobb- és baloldali határértéke, továbbá az $f$ függvény Fourier-sora ebben a
  pontban konvergens, akkor a Fourier-sor összege ezen pontokban a függvény
  bal- és jobboldali határértékeinek számtani középe.
\end{theorem}

\begin{blueBox}
  \textbf{Hasznos trigonometrikus összefüggések:}
  \begin{center}
    \def\arraystretch{1.1}
    \newcolumntype{x}[1]{>{\centering\arraybackslash\hspace{0pt}$}p{#1}<{$}}
    \begin{tabular}[t]{ x{5cm} x{5cm} }
      \cos^2 x + \sin^2 x = 1                    & e^{\iu x} = \cos x + \iu \sin x            \\[2mm]
      \sin x = \dfrac{e^{\iu x} - e^{-\iu x}}{2} & \cos x = \dfrac{e^{\iu x} + e^{-\iu x}}{2} \\[4mm]
      \sin 2x = 2 \sin x \cos x                  & \cos 2x = \cos^2 x - \sin^2 x              \\[2mm]
      \sin^2 x = \dfrac{1 - \cos 2x}{2}          & \cos^2 x = \dfrac{1 + \cos 2x}{2}          \\[4mm]
      \multicolumn{2}{ >{$}c<{$} }{\sin (t \pm s) = \sin t \cos s \pm \cos t \sin s}          \\[1mm]
      \multicolumn{2}{ >{$}c<{$} }{\cos (t \pm s) = \cos t \cos s \mp \sin t \sin s}          \\[1mm]
      \multicolumn{2}{ >{$}c<{$} }{2 \sin t \sin s = \cos(t - s) - \cos(t + s)}               \\[1mm]
      \multicolumn{2}{ >{$}c<{$} }{2 \cos t \cos s = \cos(t - s) + \cos(t + s)}               \\[1mm]
      \multicolumn{2}{ >{$}c<{$} }{2 \sin t \cos s = \sin(t - s) + \sin(t + s)}               \\[1mm]
    \end{tabular}
  \end{center}
\end{blueBox}

\clearpage
\subsection{Feladatok}
\begin{enumerate}
  \item Határozza meg az alábbi, $2\pi$ szerint periodikus függvények
        Fourier-sorát!
        $$
          f(x)
          =
          \begin{cases}
            0,   & \text{ha } -\pi < x \leq 0            \\
            \pi, & \text{ha } \phantom-\, 0 < x \leq \pi \\
          \end{cases}
          \qquad \land \qquad
          f(x) = f(x + 2k\pi)
        $$

  \item Határozzuk meg az alábbi, $2\pi$ szerint periodikus függvények
        Fourier-sorát! Adja meg a sorösszeget az $x_0 = \pi$ pontban!
        $$
          f(x) = x \text{,\quad{ha}\quad{}} x \in (-\pi; \pi)
          \qquad \land \qquad
          f(x) = f(x + 2k\pi)
        $$

  \item Írja fel az alábbi $2\pi$ szerint periodikus függvény Fourier-sorát!
        $$
          f(x) = \begin{cases}
            \sin x, & 0 \leq x < \pi    \\
            0,      & \pi \leq x < 2\pi
          \end{cases}
          \qquad \land \qquad
          f(x) = f(x + 2k\pi)
        $$

  \item Adja meg az alábbi trigonometrikus függvények Fourier-sorát!
        \begin{multicols}{2}
          \begin{enumerate}
            \item $f(x) = \sin^4 x$
            \item $g(x) = \cos 3x \cdot \sin^2 x$
          \end{enumerate}
        \end{multicols}

  \item Írja fel az alábbi $2p = 2\pi$ szerint periodikus függvény
        Fourier-sorát!
        $$
          f(x) = x^2
          \text{,\quad{ha}\quad{}} x \in (-2; 2)
          \qquad \land \qquad
          f(x) = f(x + 2k)
        $$

        % \item Határozzuk meg az alábbi függvény Fourier-sorát!
        %       \[
        %         f(x) = \begin{cases}
        %           x, & -\pi < x < \pi \\
        %         \end{cases}
        %       \]
        % \item Adjuk meg az alábbi függvény Fourier-sorát!
        %       \[
        %         f(x) = \sin^4 x
        %       \]
        % \item Írjuk fel az alábbi függvény Fourier-sorát!
        %       \[
        %         f(x) = \cos 3x \sin^2 x
        %       \]
        % \item Írjuk fel az alábbi függvény Fourier-sorát!
        %       \[
        %         f(x) = x^2
        %       \]
\end{enumerate}


\end{document}