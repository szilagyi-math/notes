\clearpage
\section{Iránymenti és parciális deriváltak}\label{sec-03-02}

\begin{definition}[Iránymenti derivált]
  Legyen $I \in \Reals^n$ nyílt halmaz, $f: I \to \Reals$ függvény és
  legyen adva egy $\rvec v \in \Reals^n$ egységvektor. Ha létezik a
  $$
    \lim_{\lambda \to 0^+} \frac{
      f(\rvec x + \lambda \rvec v) - f(\rvec x)
    }{
      \lambda
    }
  $$
  határérték és ez egy valós szám, akkor ezt az $f$ függvény $\rvec a$
  pontbeli $\rvec v$ irányú, iránymenti deriváltjának nevezzük. Jele:
  $$
    \partial_{\rvec v} f(\rvec x) = \lim_{\lambda \to 0^+} \frac{
      f(\rvec x + \lambda \rvec v) - f(\rvec x)
    }{
      \lambda
    }
    \text.
  $$
\end{definition}

\begin{note}
  Amennyiben $\rvec v$ az $n$-dimenziós téren az $i$-edik irányba mutat,
  akkor azt parciális deriváltnak nevezzük, jelölései:
  $$
    \pdv{f(\rvec x)}{x_i}
    = \partial_i f(\rvec x)
    = \partial_{x_i} f(\rvec x)
    = f'_{x_i}(\rvec x)
    = \lim_{\lambda \to 0^+} \frac{
      f(x_1, \ldots, x_{i-1}, x_i - \lambda, x_{i+1}, \ldots, x_n) - f(\rvec x)
    }{
      \lambda
    }
    \text.
  $$
\end{note}

\begin{example}
  Adjuk meg az $f(x; y) = x^3 + 5x^2y + 3xy^2 - 12y^3 + 5x - 6y + 7$ függvény
  parciális deriváltjait az $(1;2)$ pontban!

  \hdashrule[.8ex][x]{\dimexpr\textwidth}{1pt}{2mm 3pt}

  Először határozzuk meg a parciális deriváltakat parametrikusan, majd
  számoljuk ki az $(1;2)$ pontbeli értékeket:
  \begin{alignat*}{9}
    \pdv{f(x; y)}{x}                      & = 3x^2 + 10xy + 3y^2 + 5
                                          & \qquad\Rightarrow\quad
    \left.\pdv{f(x; y)}{x}\right|_{(1;2)} & = 3 + 20 + 12 + 5 = 40
    \text,
    \\
    \pdv{f(x; y)}{y}                      & = 5x^2 + 6xy - 36y^2 - 6
                                          & \qquad\Rightarrow\quad
    \left.\pdv{f(x; y)}{y}\right|_{(1;2)} & = 5 + 12 - 144 - 6 = -133
    \text.
  \end{alignat*}
\end{example}

\begin{definition}[Gradiens]
  Legyen $f: \Reals^n \to \Reals$. Az $f$ függvény
  $\rvec a(a_{1}; a_{2}; \ldots; a_{n})$ pontbeli gradiensén az alábbi
  oszlopvektort értjük:
  \def\arraystretch{1.5}
  $$
    \grad f(\rvec a) = \nabla f(\rvec a) = \begin{bmatrix}
      \partial_1 f(\rvec a) \\
      \partial_2 f(\rvec a) \\
      \vdots                \\
      \partial_n f(\rvec a)
    \end{bmatrix} = \begin{pmatrix}
      \displaystyle\pdv{f(\rvec x_0)}{x_1} &
      \displaystyle\pdv{f(\rvec x_0)}{x_2} &
      \cdots                               &
      \displaystyle\pdv{f(\rvec x_0)}{x_n}
    \end{pmatrix}^\T
  $$
\end{definition}

\begin{note}
  A gyakrolatban az iránymenti deriváltakat a gradiens segítségével számítjuk:
  $$
    \partial_{\rvec v} f(\rvec a) = \grad f(\rvec a) \cdot \rvec v
    \text.
  $$
\end{note}

\begin{example}
  Számítsuk ki az $f(x; y) = x^3 + 5x^2y + 3xy^2 - 12y^3 + 5x - 6y + 7$
  függvény $\rvec v(3;4)$ irányú deriváltját az $(1;2)$ pontban!

  \hdashrule[.8ex][x]{\dimexpr\textwidth}{1pt}{2mm 3pt}

  A gradiens az előző példában számolt parciális deriváltak alapján:
  $$
    \grad f(1;2) = \begin{bmatrix}
      40 \\ -133
    \end{bmatrix}
    \text.
  $$
  Az iránymenti derivált számításához még szükségünk van az $\rvec v$ irányú
  egységvektorra:
  $$
    \|\rvec v\| = \sqrt{3^2 + 4^2} = 5
    \quad\Rightarrow\quad
    \uvec e_v = \frac{\rvec v}{\|\rvec v\|} = \frac{1}{5} \begin{bmatrix}
      3 \\ 4
    \end{bmatrix} = \begin{bmatrix}
      3/5 \\ 4/5
    \end{bmatrix}
    \text.
  $$
  Az iránymenti derivált:
  $$
    \partial_{\rvec v} f(1;2) = \grad f(1;2) \cdot \rvec v = \begin{bmatrix}
      40 \\ -133
    \end{bmatrix} \cdot \begin{bmatrix}
      3/5 \\ 4/5
    \end{bmatrix} = 40 \cdot 3/5 - 133 \cdot 4/5 = -82,4
    \text.
  $$
\end{example}

\begin{definition}[Jacobi-mátrix]
  Legyen $\rvec f: \Reals^n \to \Reals^k$ leképezés.
  Ekkor $\rvec f'(\rvec a) = \rmat J \rvec f(\rvec a)$,
  ahol $\rmat J \in \mathscr M_{k \times n}$. A $\rmat J$ mátrixot az
  $\rvec f$ függvény Jacobi-mátrixának nevezzük, melynek elemei:
  \def\arraystretch{1.5}
  $$
    \rmat J(\rvec a) = \begin{bmatrix}
      \displaystyle\pdv{f_1(\rvec x)}{x_1} & \displaystyle\pdv{f_1(\rvec x)}{x_2} & \cdots & \displaystyle\pdv{f_1(\rvec x)}{x_n} \\
      \displaystyle\pdv{f_2(\rvec x)}{x_1} & \displaystyle\pdv{f_2(\rvec x)}{x_2} & \cdots & \displaystyle\pdv{f_2(\rvec x)}{x_n} \\
      \vdots                               & \vdots                               & \ddots & \vdots                               \\
      \displaystyle\pdv{f_k(\rvec x)}{x_1} & \displaystyle\pdv{f_k(\rvec x)}{x_2} & \cdots & \displaystyle\pdv{f_k(\rvec x)}{x_n}
    \end{bmatrix}_{|\rvec x = \rvec a} = \begin{bmatrix}
      \grad^\T f_1(\rvec a) \vphantom{\displaystyle\pdv{f_1}{x_1}} \\
      \grad^\T f_2(\rvec a) \vphantom{\displaystyle\pdv{f_1}{x_1}} \\
      \vdots                                                       \\
      \grad^\T f_k(\rvec a) \vphantom{\displaystyle\pdv{f_1}{x_1}}
    \end{bmatrix}
  $$
\end{definition}

\begin{theorem}[Young-tétel]
  Legyen adott az $I \subset \Reals^n$ nyílt halmaz, $f: H \to \Reals$ és
  $\rvec a \in I$, továbbá $\rvec a$-nak létezik olyan környezete, amelyben
  $f$ összes $p$-edrendű parciális deriváltja létezik és folytonos. Ekkor
  $$
    \partial_i \partial_j f(\rvec a) = \partial_j \partial_i f(\rvec a)
    \text,\quad
    i;j \in \{1;2;\ldots;n\}
    \text,
  $$
  azaz a parciális deriváltak sorrendje $p$-ed rendig felcserélhető.

  \begin{proof}
    \vspace{7em}
  \end{proof}
\end{theorem}