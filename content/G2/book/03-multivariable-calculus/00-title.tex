\mainChapter{Többváltozós analízis}\label{chap-03}

\bgroup
\color{gray!50!black}
\sffamily

Eddigi tanulmányaink során olyan függvényekkel foglalkoztunk, amelyek egyetlen
változótól függtek. Ebben a fejezetben az eddig vizsgált egyváltozós függvények
síkbeli görbéi után a többdimenziós terek világába lépünk. Célunk, hogy a
differenciál- és integrálszámítás jól ismert eszközeit átültessük és kibővítsük
olyan függvényekre, amelyek kettő vagy több független változótól függenek. Ez a
lépés lehetővé teszi számunkra, hogy a valós világ összetett, több tényezős
jelenségeit matematikailag is precízebben tudjuk modellezni.

Először az egyváltozós kalkulusból már ismert alapfogalmakat általánosítjuk. A
tárgyalást a topológiai alapfogalmak, a határérték és a folytonosság
általánosításával kezdjük. Mivel egy pontot már nem csak balról vagy jobbról,
hanem végtelen sok irányból megközelíthetünk, ez a konvergencia vizsgálatát
lényegesen komplexebbé teszi.

Ezt követően bevezetjük a differenciálhatóság többváltozós koncepcióját. A
derivált egyváltozós fogalmát az iránymenti derivált általánosítja, mely a
függvény lokális változását írja le tetszőleges irány mentén. Egy speciális
iránymenti derivált a parciális derivált, amely esetén a koordináta-tengelyek
irányában elemezzük a változást. A totális differenciálhatóság precíz
definíciója a függvény lineáris approximálhatóságán alapul, amelynek
mátrixreprezentációja a Jacobi-mátrix.

A differenciálszámítás elméletét a szélsőérték-feladatok megoldására
alkalmazzuk. Kidolgozunk egy szisztematikus eljárást a függvények lokális
maximum- és minimumhelyeinek azonosítására. A lokális extrémumok létezésének
szükséges feltétele a stacionárius pontok megtalálása, az elégséges feltételét
pedig a függvény második parciális deriváltjaiból képzett Hesse-mátrix
definitségének vizsgálatával adjuk meg. A gyakorlati problémákban a
szélsőértéket egy vagy több feltétel teljesülése mellett keressük. Az ilyen, ún.
feltételes szélsőérték problémák megoldására a Lagrange-multiplikátoros módszert
ismertetjük.

Végül, az integrálás fogalmát is általánosítjuk. A görbe alatti terület helyett
immár felületek alatti térfogatokat és más, magasabb dimenziós tartományokhoz
rendelt mennyiségeket számítunk ki a többszörös integrálok segítségével. Ez az
eszköz kulcsfontosságú lesz sík- és térbeli alakzatok geometriai és fizikai
jellemzőinek (pl. terület, térfogat, tömegközéppont, tehetetlenségi nyomaték)
meghatározásában.

\chaptertoc
\egroup

\clearpage