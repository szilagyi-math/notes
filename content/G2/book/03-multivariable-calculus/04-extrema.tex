\clearpage
\section{Szélsőértékszámítás}\label{sec-03-04}

\begin{definition}[Többváltozós függvény maximuma]
  Legyen $f: H \subset \Reals^n \to \Reals$. Azt mondjuk, hogy az $f$-nek
  az $\rvec a \in \inner H$ pontban lokális maximuma van, ha létezik az
  $\rvec a \in U \subset H$ környetzete, hogy $f(\rvec a) \geq f(\rvec x)$
  $\;\forall \rvec x \in U$-ra.
\end{definition}

\begin{definition}[Többváltozós függvény minimuma]
  Legyen $f: H \subset \Reals^n \to \Reals$. Azt mondjuk, hogy az $f$-nek
  az $\rvec a \in \inner H$ pontban lokális minimuma van, ha létezik az
  $\rvec a \in U \subset H$ környetzete, hogy $f(\rvec a) \leq f(\rvec x)$
  $\;\forall \rvec x \in U$-ra.
\end{definition}

\begin{definition}[Stacionárius pont]
  Legyen $f: H \subset \Reals^n \to \Reals$, $\rvec a \in \inner H$ és
  léteznek az $f$ parciális deriváltjai az $\rvec a$ pontban. Ha
  $\forall i \in \{1;2;\ldots;n\}$-re $\partial_i f(\rvec a) = 0$, akkor
  az $\rvec a$ a függvény stacionárius pontja.
\end{definition}

\begin{theorem}
  Legyen $f: H \subset \Reals^n \to \Reals$. Ha az $f$ függvény az összes
  változója szerint parciálisan differenciálható az $\rvec a \in \inner H$
  pontban, és ott lokális szelsőértéke van, akkor az $\rvec a$ stacionárius
  pontja az $f$ függvénynek.

  \begin{proof}
    \vspace{7em}
  \end{proof}
\end{theorem}

\begin{definition}[Lineáris forma]
  Legyen a $V$ a $T$ test feletti vektortér. Ha a $\varphi: V \to \Reals$
  leképezés lineáris, vagyis
  $$
    \varphi(\lambda \rvec x + \mu \rvec y)
    = \lambda \varphi(\rvec x) + \mu \varphi(\rvec y)
    \qquad \forall \rvec x; \rvec y \in V, \lambda; \mu \in \Reals
    \text,
  $$
  akkor a $\varphi$-t lineáris formának is hívjuk.
\end{definition}

\begin{definition}[Billineáris forma]
  Legyen $\psi: V \times V \to \Reals$ mindkét változójában lineáris, azaz
  \begin{alignat*}{9}
    \psi(\lambda_1 \rvec x_1 + \lambda_2 \rvec x_2; \rvec y)
     & = \lambda_1 \psi(\rvec x_1; \rvec y) + \lambda_2 \psi(\rvec x_2; \rvec y)           &
     & \qquad \forall \rvec x_1; \rvec x_2; \rvec y \in V, \lambda_1; \lambda_2 \in \Reals
    \text,
    \\
    \psi(\rvec x; \mu_1 \rvec y_1 + \mu_2 \rvec y_2)
     & = \mu_1 \psi(\rvec x; \rvec y_1) + \mu_2 \psi(\rvec x; \rvec y_2)                   &
     & \qquad \forall \rvec x; \rvec y_1; \rvec y_2 \in V, \mu_1; \mu_2 \in \Reals
    \text.
  \end{alignat*}
  Ekkor a $\psi$-t bilineáris formának mondjuk.
\end{definition}

\begin{note}
  A $\psi$ bilineáris forma szimmetrikus, ha
  $\psi(\rvec x; \rvec y) = \psi(\rvec y; \rvec x)
    \quad \forall \rvec x; \rvec y \in V$-re.

  A $\psi$ bilineáris forma antiszimmetrikus, ha
  $\psi(\rvec x; \rvec y) = -\psi(\rvec y; \rvec x)
    \quad \forall \rvec x; \rvec y \in V$-re.
\end{note}

\begin{definition}[Kvadratikus forma]
  Legyen $\eta: V \to \Reals$. Ha létezik olyan $\psi: V \times V \to \Reals$
  szimmetrikus bilineáris forma, hogy $\eta(\rvec x) = \psi(\rvec x; \rvec x)
    \quad \forall \rvec x \in V$-re, akkor az $\eta$-t kvadratikus formának,
  vagy kvadratikus alaknak nevezzük.

  Az $\eta$ kvadratikus forma\dots\\[.33em]
  \def\arraystretch{1.33}
  \begin{tabular}{>{\bullet\;}l<{,}l}
    pozitív definit      & ha $\eta(\rvec x) > 0 \quad \forall \rvec x \in V \setminus \{\nvec\}$,    \\
    negatív definit      & ha $\eta(\rvec x) < 0 \quad \forall \rvec x \in V \setminus \{\nvec\}$,    \\
    pozitív szemidefinit & ha $\eta(\rvec x) \geq 0 \quad \forall \rvec x \in V \setminus \{\nvec\}$, \\
    negatív szemidefinit & ha $\eta(\rvec x) \leq 0 \quad \forall \rvec x \in V \setminus \{\nvec\}$. \\
  \end{tabular}\\[.33em]
  Ha ezek egyike sem teljesül, indefinit kvadratikus formáról beszélünk.
\end{definition}

\begin{theorem}
  Legyen $f: H \subset \Reals^n \to \Reals$. Tegyük fel, hogy az $f$ függvény
  az összes változója szerint parciálisan differenciálható az
  $\rvec a \in \inner H$ pont valamely környezetében. Legyen továbbá az
  $\rvec a$ stacionárius pontja az $f$-nek és $Q: \Reals^n \to \Reals$ olyan
  kvadratikus forma, melynek mátrixa:
  \def\arraystretch{1.5}
  $$
    \rmat Q(\rvec a) = \begin{bmatrix}
      \partial_1^2 f(\rvec a)          & \partial_1 \partial_2 f(\rvec a) & \cdots & \partial_1 \partial_n f(\rvec a) \\
      \partial_2 \partial_1 f(\rvec a) & \partial_2^2 f(\rvec a)          & \cdots & \partial_2 \partial_n f(\rvec a) \\
      \vdots                           & \vdots                           & \ddots & \vdots                           \\
      \partial_n \partial_1 f(\rvec a) & \partial_n \partial_2 f(\rvec a) & \cdots & \partial_n^2 f(\rvec a)
    \end{bmatrix}
    \in \mathscr M_{n \times n}
    \text.
  $$

  \def\arraystretch{1.33}
  \setlength{\tabcolsep}{0pt}
  \begin{tabular}{>{\bullet\;}l<{,\;}l}
    Ha $Q$ pozitív definit & akkor az $\rvec a$ pontban az $f$-nek lokális minimuma van. \\
    Ha $Q$ negatív definit & akkor az $\rvec a$ pontban az $f$-nek lokális maximuma van. \\
    Ha $Q$ indefinit       & akkor az $\rvec a$ pontban az $f$-nek nincs szélsőértéke.   \\
  \end{tabular}
\end{theorem}

\begin{theorem}
  Legyen $n = 2$ és teljesüljenek az előző tétel feltételei. Ekkor
  $$
    \mathscr D(\rvec a)
    = \det \rmat Q(\rvec a)
    = \begin{vmatrix}
      \partial_1^2 f(\rvec a)          & \partial_1 \partial_2 f(\rvec a) \\
      \partial_2 \partial_1 f(\rvec a) & \partial_2^2 f(\rvec a)
    \end{vmatrix}
    \quad\text{ és }\quad
    \mathscr S(\rvec a)
    = \trace \rmat Q(\rvec a)
    = \partial_1^2 f(\rvec a) + \partial_2^2 f(\rvec a)
  $$

  \def\arraystretch{1.33}
  \setlength{\tabcolsep}{0pt}
  \begin{tabular}{>{\bullet\;}l<{,\;}l}
    Ha $\mathscr D(\rvec a) > 0$ és $\mathscr S(\rvec a) > 0$ & akkor az $\rvec a$ pontban az $f$-nek lokális minimuma van. \\
    Ha $\mathscr D(\rvec a) > 0$ és $\mathscr S(\rvec a) < 0$ & akkor az $\rvec a$ pontban az $f$-nek lokális maximuma van. \\
    Ha $\mathscr D(\rvec a) < 0$                              & akkor az $\rvec a$ pontban az $f$-nek nincs szélsőértéke.   \\
  \end{tabular}
\end{theorem}

\begin{example}
  Vizsgáljuk az $f(x,y)=x^{3}-3x+y^{2}$ függvény lokális szélsőértékeit!

  \hdashrule[.8ex][x]{\dimexpr\textwidth}{1pt}{2mm 3pt}

  \textbf{1.~Stacionárius pontok, ahol a gradiens nullvektor:}
  \begin{alignat*}{9}
    \partial_x f(x,y) & = 3x^{2}-3 = 0
    \qquad            & \Rightarrow \qquad & x=\pm 1
    \text,
    \\
    \partial_y f(x,y) & = 2y = 0
    \qquad            & \Rightarrow \qquad & y = 0
    \text.
  \end{alignat*}

  \textbf{2.~Hesse-mátrix:}
  \[
    \rmat Q(x,y)=
    \begin{bmatrix}
      \partial_{1}^{2}f         & \partial_{1}\partial_{2}f \\[2pt]
      \partial_{2}\partial_{1}f & \partial_{2}^{2}f
    \end{bmatrix}
    =
    \begin{bmatrix}
      6x & 0 \\
      0  & 2
    \end{bmatrix}.
  \]

  \begin{center}
    \def\arraystretch{1.3}
    \begin{tabular}{c|c|c|c}
      pont        & $\mathscr D$         & $\mathscr S$ & következtetés     \\\hline
      $P_1(1,0)$  & $6\cdot2-0=12>0$     & $6+2=8>0$    & lokális minimum   \\
      $P_2(-1,0)$ & $(-6)\cdot2-0=-12<0$ & $-6+2=-4$    & nincs szélsőérték
    \end{tabular}
  \end{center}

  \medskip
  Tehát az $f$ függvénynek a $P_1(1,0)$ pontban lokális minimuma van, míg a
  $P_2(-1,0)$ pontban nincs szélsőértéke.
\end{example}

\begin{definition}
  Legyen $m;n \in \mathbb Z^+$, $m > n$, $H \in \Reals^m$ nyílt halmaz,
  $f: H \to \Reals$ és $\rvec g : H \to \Reals^n$ leképezések, továbbá
  $H_0 := \{ \rvec x \mid \rvec x \in H \land \rvec g (\rvec x) = \nvec\}$.
  Ha az $f$ függvény $H_0$-ra való leszűkítésének ($f|_{H_0}$) az
  $\rvec a$ pontban lokális szélsőértéke van, akkor azt mondjuk, hogy az $f$-nek
  az $\rvec a$ pontban feltételes szélsőértéke van a $\rvec g(\rvec a) = \nvec$
  feltétellel.
\end{definition}

\begin{definition}[Lagrange féle multiplikátor]
  Legyen $m;n \in \mathbb Z^+$, $m > n$, $H \in \Reals^m$ nyílt halmaz,
  $f: H \to \Reals$ és $\rvec g : H \to \Reals^n$ leképezések, továbbá
  $H_0 := \{ \rvec x \mid \rvec x \in H \land \rvec g (\rvec x) = \nvec\}$.
  Tegyük fel, hogy az $f$ és $\rvec g_i$ függvények minden parciális deriváltja
  folytonos a $H$ halmazon. Ha az $f$ függvény az $\rvec a \in H_0$ pontban
  feltételes szélsőértéke van a $\rvec g(\rvec a) = \nvec$ feltétellel és a
  \def\arraystretch{1.5}
  $$
    \rg \begin{bmatrix}
      \partial_1 g_1 (\rvec a) & \partial_2 g_1 (\rvec a) & \cdots & \partial_n g_1 (\rvec a) \\
      \partial_1 g_2 (\rvec a) & \partial_2 g_2 (\rvec a) & \cdots & \partial_n g_2 (\rvec a) \\
      \vdots                   & \vdots                   & \ddots & \vdots                   \\
      \partial_1 g_n (\rvec a) & \partial_2 g_n (\rvec a) & \cdots & \partial_n g_n (\rvec a)
    \end{bmatrix} = n
    \text{ (maximális)},
  $$
  akkor léteznek olyan $\lambda_1; \lambda_2; \ldots \lambda_n$ skalárok, hogy
  $$
    \partial_i f(\rvec a)
    + \sum_{k = 1}^{n} \lambda_k \,\partial_i g_k(\rvec a) = 0
    \quad
    \forall i \in \{1;2;\ldots;m\}
    \text.
  $$
  A $\lambda_1; \lambda_2; \ldots \lambda_n$ skalárokat Lagrange-féle
  multiplikátoroknak nevezzük.
\end{definition}