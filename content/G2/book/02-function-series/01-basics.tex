\clearpage
\section{Alapfogalmak}\label{sec-02-01}

\begin{definition}[Függvénysorozat]
  Az $f_n : I \subset \mathbb R \to \mathbb R$ sorozatot függvénysorozatnak
  nevezzük.
\end{definition}

\begin{example}
  \sftitle{Példák függvénysorozatokra:}

  % \begin{itemize}
  %   \item $f_n : \mathbb R \to [-1; 1]$, \quad $f_n(x) = \sin nx$
  %   \item $g_n: [0; \infty] \to \mathbb R$, \quad $g_n(x) = x^n$
  % \end{itemize}

  \def\arraystretch{1.33}
  \begin{tabular}{ll}
    \bullet \; $f_n: \mathbb R \to [-1; 1]$     & $f_n(x) = \sin nx$ \\
    \bullet \; $g_n: [0; \infty] \to \mathbb R$ & $g_n(x) = x^n$     \\
    \bullet \; $h_n: \mathbb R \to \mathbb R$   & $h_n(x) = e^{nx}$  \\
  \end{tabular}
\end{example}

\begin{definition}[Függvénysorozat pontbeli konvergenciája]
  Ha az $x_0 \in I$ pontban az $(f_n(x_0))$ számsorozat konvergens, akkor azt
  mondjuk, hogy az $(f_n)$ függvénysorozat konvergens az $x_0$-ban. A
  konvergenciahalmaz:
  $$
    H := \big\{\;
    x \mid x \in I \land (f_n) \text{ konvergens az } x \text{ pontban}
    \;\big\}
    \text.
  $$
\end{definition}

\begin{example}
  \sftitle{Példák konvergenciahalmazra}:

  % \begin{itemize}
  %   \item $f_n(x) = \sin nx$, \quad $H_{f_n} = \{ k\pi; k \in \mathbb Z \}$,
  %   \item $g_n(x) = x^n$, \quad $H_{g_n} = [0; 1]$.
  % \end{itemize}

  \def\arraystretch{1.33}
  \begin{tabular}{ll}
    \bullet \; $f_n(x) = \sin nx$ & $H_{f_n} = \{ k\pi; k \in \mathbb Z \}$ \\
    \bullet \; $g_n(x) = x^n$     & $H_{g_n} = [0; 1]$                      \\
    \bullet \; $h_n(x) = e^{nx}$  & $H_{h_n} = \{0\}$                       \\
  \end{tabular}
\end{example}

\begin{definition}[Függvénysorozat határfüggvénye]
  Az $f$ függvényt az $(f_n)$ függvénysorozat határfüggvényének nevezzük:
  $$
    f(x) := \lim_{n \to \infty} f_n(x)
    \text,\quad
    x \in H
    \text.
  $$
  Azt mondjuk, hogy az $(f_n)$ függvénysorozat pontonként konvergál az $f$
  határfüggvényhez a $H$-n, ha $\forall \varepsilon > 0$ esetén
  $\exists N(\varepsilon; x)$, hogy $|f_n(x) - f(x)| < \varepsilon$, ha
  $n > N(\varepsilon; x)$.
\end{definition}

\begin{definition}[Függvénysorozat egyenletes konvergenciája]
  Az  $(f_n)$ egyenletesen konvergens az $E \subset H$ halmazon, ha
  $\forall\varepsilon > 0$ esetén létezik $N(\varepsilon)$ úgy, hogy
  $|f_n(x) - f(x)| < \varepsilon$, ha $n > N(\varepsilon)$ minden $x \in E$
  esetén.
\end{definition}

\begin{note}
  Az egyenletes konvergenciából következik a pontonkénti konvergencia.

  Az állítás azonban megfordítva nem igaz.
\end{note}

\clearpage
\begin{theorem}[Cauchy-kritérium függvénysorozatok konvergenciájára]
  \begin{itemize}
    \item Az $(f_n)$ akkor és csak akkor konvergens az $x_0 \in H$ pontban,
          ha $\forall \varepsilon > 0$ esetén $\exists N(\varepsilon)$, hogy
          $|f_n(x_0) - f_m(x_0)| < \varepsilon$, ha $n; m > N(\varepsilon)$.

    \item Az $(f_n)$ akkor és csak akkor konvergens az $H \subset I$ halmazon,
          ha $\forall \varepsilon > 0$ esetén $\exists N(\varepsilon, x)$, hogy
          ha $n; m > N(\varepsilon, x)$, akkor $\forall x \in H$ esetén
          $|f_n(x) - f_m(x)| < \varepsilon$.

    \item Az $(f_n)$ akkor és csak akkor egyenletesen konvergens az
          $E \subset H$ halmazon, ha $\forall \varepsilon > 0$ esetén
          $\exists N(\varepsilon)$, hogy ha $n; m > N(\varepsilon)$, akkor
          $\forall x \in E$ esetén ${|f_n(x) - f_m(x)| < \varepsilon}$.
  \end{itemize}

  \begin{proof}
    Az első két eset bizonyítása a numerikus sorozatoknál tanultak szerint
    történik.

    A harmadik eset bizonyítása:
    \begin{itemize}
      \item[$(\Rightarrow)$] Ha $(f_n)$ egyenletesen konvergens $E$-n, akkor
            $\forall \varepsilon > 0$ esetén $\exists N(\varepsilon)$ úgy, hogy:
            \begin{alignat*}{9}
              |f_n(x) - f(x)| & < \frac{\varepsilon}{2}
              \text, \quad \text{ha} \quad
              n               & \;>\;                   & N\left(\frac{\varepsilon}{2}\right), \forall x \in E
              \text,
              \\
              |f_m(x) - f(x)| & < \frac{\varepsilon}{2}
              \text, \quad \text{ha} \quad
              m               & \;>\;                   & N\left(\frac{\varepsilon}{2}\right), \forall x \in E
              \text.
            \end{alignat*}

            Továbbá, ha $n; m > N(\varepsilon / 2)$ és a
            háromszög-egyenlőtlenséget felhasználva:
            \begin{align*}
              |f_n(x) - f_m(x)|
               & = |f_n(x) - f(x) + f(x) - f_m(x)| \leq
              \\
               & \leq \underbrace{|f_n(x) - f(x)|}_{< \varepsilon / 2}
              + \underbrace{|f(x) - f_m(x)|}_{< \varepsilon / 2}
              < \frac{\varepsilon}{2} + \frac{\varepsilon}{2}
              = \varepsilon
              \text.
            \end{align*}

      \item[$(\Leftarrow)$] $|f_n(x) - f_m(x)| < \varepsilon / 2$,
            ha $n; m > N(\varepsilon)$ és $x \in E$. Ekkor $f_n$ Cauchy-sorozat,
            $f_m$ is az, azaz $f_m \rightarrow(x) \to f(x)$, ha $m \to \infty$,
            Ekkor:
            $$
              |f_n(x) - f(x)| < \frac{\varepsilon}{2} < \varepsilon
              \text, \quad \text{ha} \quad
              n > N(\varepsilon)
              \text, \quad
              \forall x \in E.
            $$
    \end{itemize}
  \end{proof}
\end{theorem}

\begin{definition}[Függvénysor]
  Legyen $f_n : I \subset \Reals \to \Reals$ függvénysorozat. Képezzük az
  alábbi függvénysorozatot:
  \begin{align*}
    s_1(x) & := f_1(x)
    \text,                              \\
    s_2(x) & := f_1(x) + f_2(x)
    \text,                              \\
           & \phantom{:} \vdots
    \\
    s_j(x) & := \sum_{i = 1}^{j} f_i(x)
    \\
           & \phantom{:} \vdots
  \end{align*}

  Az így előálló $(s_n)$ függvénysorozatot az $(f_n)$ függvénysorozatból képzett
  függvénysornak hívjuk és $\sum f_n$-nel jelöljük.
\end{definition}

\begin{definition}[Függvénysor pontbeli konvergenciája]
  A $\sum f_n$ függvénysor konvergens az $x_0 \in I$ pontban, ha az $(s_n)$
  függvénysorozat konvergens az $x_0$ pontban.
\end{definition}

\begin{definition}[Függvénysor konvergenciahalmaza]
  Az $\sum f_n$ függvénysor konvergens a $H \subset I$ halmazon, ha az
  $(s_n)$ függvénysorozat konvergens a $H$-n.
\end{definition}

\begin{definition}[Függvénysor egyenletes konvergenciája]
  Az $\sum f_n$ függvénysor egyenletesen konvergens az $E \subset H$
  halmazon, ha az $(s_n)$ függvénysorozat egyenletesen konvergens az $E$-n.
\end{definition}

\begin{definition}[Függvénysor összegfüggvénye]
  Az $\sum f_n$ függvénysorozat összegfüggvénye az $s(x) := \lim\limits_{n \to
      \infty} s_n(x)$ függvény, ahol $x \in H$.
\end{definition}

\begin{theorem}[%
    Cauchy-féle konvergencia kritérium egyenletes konvergenciára
  ]
  A $\sum f_n$ akkor és csak akkor egyenletesen konvergens az
  $E \subset H$ halmazon, ha $\forall \varepsilon > 0$ esetén
  $\exists N(\varepsilon)$ úgy, hogy ha $n; m > N(\varepsilon)$,
  akkor $\forall x \in E$ esetén ${|s_n(x) - s_m(x)| < \varepsilon}$.

  \begin{proof}
    \vspace{5em}
  \end{proof}
\end{theorem}

\begin{theorem}[Weierstrass-tétel függvénysorok egyenletes konvergenciájára]
  Legyen $f_n : I \subset \Reals \to \Reals$ függvénysorozat és $\sum f_n$
  a belőle képzett függvénysor, továbbá $\sum a_n$ olyan konvergens numerikus
  sor, melyre $\forall x \in I$ esetén $|f_n(x)| \leq a_n$
  $\forall n \in \mathbb N$-re $n > n_0 \in \mathbb N$ esetén.

  Ekkor a $\sum f_n$ füffvénysor egyenletesen konvergens.

  \begin{proof}
    \vspace{5em}
  \end{proof}
\end{theorem}

\begin{definition}[Függvénysor abszolút konvergenciája]
  A $\sum f_n$ függvénysort abszolút konvergensnek mondjuk, ha a $\sum |f_n|$
  függvénysor konvergens.
\end{definition}

\begin{note}
  A Weierstrass-tételbeli konvergencia abszolút konvergencia is.
\end{note}

\begin{definition}[Hatványsor]
  Legyen $f_n(x) := a_n \, (x - x_0)^n$. A belőle képzett
  $$
    \sum f_n(x) = \sum a_n \, (x - x_0)^n
  $$
  függvénysort hatványsornak nevezzük, ahol $a_n$ a hatványsor $n$-edik
  együtthatója, $x_0$ pedig a sorfejtés centruma.
\end{definition}

\begin{note}
  Ha $x_0 = 0$, akkor a hatványsor az alábbi alakra egyszerűsödik:
  $$
    \sum a_n \cdot x^n
    \text.
  $$
\end{note}

\begin{definition}[Hatványsor konvergenciasugara]
  A $\sum a_n \, (x - x_0)^n$ hatványsor konvergenciasugara:
  $$
    r = \frac{1}{\limsup\limits_{n \to \infty} \sqrt[n]{|a_n|}} \in \Reals_b
    \text.
  $$
\end{definition}

\begin{theorem}[Függvénysor konvergenciája]
  Legyen a $\sum f_n$ függvénysor egyenletesen konvergens az $x_0$ pontot
  tartalmazó környezetben, továbbá legyenek a sor tagjai az $x_0$-ban
  folytonosak. Ekkor az összegfüggvény is folytonos az $x_0$ pontban.

  \begin{proof}
    Tudjuk, hogy a $\sum f_n$ folytonos és egyenletesen konvergens. Azt akarjuk
    belátni, hogy $|f(x) - f(x_0)|$ tetszőlegesen kicsivé tehető, mert ekkor az
    összegfüggvény folytonos.
    \begin{align*}
      |f(x) - f(x_0)|
       & = |f(x) - f_n(x) + f_n(x) - f_n(x_0) + f_n(x_0) - f(x_0)| \leq
      \\
       & \leq \underbrace{|f(x) - f_n(x)|}_{< \varepsilon / 3 \; (i)}
      + \underbrace{|f_n(x) - f_n(x_0)|}_{< \varepsilon / 3 \; (ii)}
      + \underbrace{|f_n(x_0) - f(x_0)|}_{< \varepsilon / 3 \; (iii)}
      < 3 \cdot \frac{\varepsilon}{3} = \varepsilon
      \text,
    \end{align*}
    ha $n > N(\varepsilon / 3) := \max \left\{
      N_1(\varepsilon / 3); N_2(\varepsilon / 3); N_3(\varepsilon / 3)
      \right\}$.
    \begin{itemize}
      \item[$(i)$] egyenletes konvergencia miatt,
      \item[$(ii)$] folytonosság miatt,
      \item[$(iii)$] egyenletes konvergencia miatt.
    \end{itemize}
  \end{proof}
\end{theorem}

\begin{statement}
  \begin{enumerate}
    \item Folytonos függvények egyenletesen konvergens sorozatának
          határfüggvénye is \\folytonos.

    \item Folytonos függvények egyenletesen konvergens függvénysorának
          összegfüggvénye is folytonos, ha a függvénysor tagjai folytonosak.
  \end{enumerate}

  \begin{proof}
    \vspace{6em}
  \end{proof}
\end{statement}

\begin{theorem}[Tagonkénti integrálhatóság]
  Legyenek a $\sum f_n$ függvénysor tagjai integrálhatóak az $[a; b]$ zárt
  intervallumon. Tegyük fel, hogy a sor egyenletesen konvergens az $[a; b]$-n
  és összegfüggvénye folytonos. Ekkor
  $$
    \int_a^b f(x) \dd x = \sum_{n = 1}^{\infty} \int_a^b f_n(x) \dd x
    \text.
  $$

  \begin{proof}
    \vspace{7em}
  \end{proof}
\end{theorem}

\begin{note}
  Nem korlátos intervallum esetén nem igaz az állítás.
\end{note}

\begin{theorem}[Tagonkénti differenciálhatóság]
  Legyenek az $f_n$ függvénysor tagjai differenciálhatóak a $J$ intervallumon,
  $f'_n$ függvények folytonosak a $J$-n, valamint a $\sum f'_n$ és a $\sum f_n$
  függvénysorok egyenletesen konvergensek a $J$-n. Ekkor
  $$
    f'(x) = \sum_{n = 1}^{\infty} f'_n(x)
    \text.
  $$

  \begin{proof}
    \vspace{7em}
  \end{proof}
\end{theorem}

\begin{theorem}[Hatványsor konvergenciája]
  Ha a $\sum a_n x^n$ hatványsor konvergens az $x_0$ pontban, akkor az $x < x_0$
  helyeken abszolút és egyenletesen konvergens.

  \begin{center}
    \begin{tikzpicture}[very thick, scale=3/4]
      \draw[draw=secondaryColor, -to] (-3,0) -- ++(6,0) node[below left] {$x$};

      \draw[primaryColor] (-1.5,.5)
      .. controls (-1.7,.5) and (-1.7,-.5) ..
      (-1.5,-.5);

      \draw[primaryColor] (1.5,.5)
      .. controls (1.7,.5) and (1.7,-.5) ..
      (1.5,-.5);

      \draw[primaryColor] (0,.1) -- (0,-.1);

      \node[below] at (-1.65,-.5) {$-x_0\vphantom{0}$};
      \node[below] at (1.65,-.5) {$x_0\vphantom{0}$};
      \node[below] at (0,-.5) {$0\vphantom{x_0}$};
    \end{tikzpicture}
  \end{center}

  \begin{proof}
    Ha a $\sum a_n x^n$ hatványsor konvergens az $x_0$ pontban, akkor
    $a_n x^n \to 0$, ha $n \to \infty$. Ekkor tehát korlátos is, azaz
    $\exists K \in \Reals$, hogy $|a_n x^n| \leq K$.
    $$
      \left| a_n x^n \right|
      = \frac{|a_n|^{\phantom{n}}}{|x_0|^n} |x_0|^n |x|^n
      = \underbrace{|a_n x_0^n|}_{\leq K}
      \underbrace{\left|\frac{x}{x_0}\right|^n}_{< 1 \text{ ha } |x| < |x_0|}
      \rightarrow 0
    $$
    Ekkor $|a_n x^n| \leq K \cdot q^n$,
    tehát $\sum a_n x^n \leq \sum K \cdot q^n = K \cdot \sum q^n$. Láthatjuk,
    hogy így konvergens geometriai sorral becsülhetjük a hatványsort. Alkalmazva
    a Weierstrass-tételt, $\sum a_n x^n$ abszolút és egyenletesen is konvergens,
    ha $|x| < |x_0|$.
  \end{proof}
\end{theorem}

\begin{theorem}[Cauchy-Hadamard-tétel]
  Legyen $r$ a $\sum a_n x^n$ hatványsor konvergenciasugara. Ha \dots
  \begin{enumerate}
    \item $r = 0$, akkor a hatványsor csak az $x_0 = 0$ pontban konvergens,
    \item $r = \infty$, akkor a hatványsor $\forall x \in \Reals$ esetén
          konvergens,
    \item $0 < r < \infty$, akkor a hatványsor konvergens, ha $|x| < r$ és
          divergens, ha $|x| > r$.
  \end{enumerate}

  \begin{center}
    \begin{tikzpicture}[thick]
      \draw[very thick, -to, draw=secondaryColor]
      (0,0) -- (6,0) node[below left] {$x$};

      \foreach \x/\l in {1/-r,3/0,5/+r}{
          \draw[draw=primaryColor] (\x, 3pt) -- (\x, -3pt) node[below]{$\l$};
        }
      \draw [
        draw=primaryColor,
        decorate,
        decoration={brace,amplitude=5pt,mirror,raise=4ex}
      ]
      (1,0) -- (5,0) node[midway,yshift=-3em] {abszolút konvergencia};
    \end{tikzpicture}
  \end{center}

  \begin{proof}
    Az első két eset az előző tételek alapján könnyen adódik.

    A harmadik esetben a hatványsor konvergens, ha $|x_0| < r$:
    $$
      \limsup \sqrt[n]{|a_n x_0^n|}
      = |x_0| \limsup \sqrt[n]{|a_n|}
      = \frac{|x_0|}{r}
      < 1.
    $$
    Azaz létezik $q < 1$, hogy a $\sum a_n x^n$ hatványsor a gyökteszt miatt
    konvergens. Mivel $x_0$ tetszőleges volt ($|x_0| < r$), így minden $|x| < r$
    esetén igaz, hogy a $\sum a_n x^n$ hatványsor konvergens. Ugyancsak a
    gyökteszt miatt a hatványsor divergens, ha $|x| > r$.
  \end{proof}
\end{theorem}

\begin{note}
  Ha a $\lim\limits_{n \to \infty} \sqrt[n]{|a_n|}$ határérték létetik, akkor ez
  megegyezik a $\limsup \sqrt[n]{|a_n|}$ határértékkel.

  Gyakran így számolunk:
  $$
    r = \frac{1}{\lim\limits_{n \to \infty} \sqrt[n]{|a_n|}}
    \text.
  $$
\end{note}

\begin{theorem}[Abel második tétele]
  Tegyük fel, hogy a $\sum a_n x^n$ hatványsor konvergenciasugara $r$ és ez a
  hatványsor az $x = r$ pontban konvergens. Ekkor a hatványsor a $[0; r]$
  intervallumon egyenletesen konvergens, így az összegfüggvény is folytonos a
  $[0; r]$ intervallumon. Ha a hatványsor az $x = -r$ pontban konvergens, akkor
  a hatványsor a $[-r; 0]$ intervallumon egyenletesen konvergens, így az
  összegfüggvény is folytonos a $[-r; 0]$ intervallumon.

  \begin{proof}
    \vspace{20em}
  \end{proof}
\end{theorem}

\begin{note}
  \sftitle{Abel második tételének következményei:}

  \begin{enumerate}
    \item A hatványsor összegfüggvénye a konvergencia intervallum belsejében
          folytonos.

    \item A hatványsor konvergenciaintervallum tetszőleges részintervallumán
          tagonként integrálható, azaz ha $[a; b] \subset (-r; r)$ és
          $s(x) := \sum a_n x^n$ összegfüggvény, akkor
          $$
            \int_a^b s(x) \dd x = \sum_{n = 0}^{\infty} \int_a^b a_n x^n \dd x
            \text.
          $$

    \item Ha $s(x)$ a hatványsor összegfüggvénye, akkor
          $$
            s'(x)
            = \sum_{n = 0}^{\infty} \left( a_n x^n \right)'
            = \sum_{n = 0}^{\infty} n \, a_n \, x^{n - 1}
            \text.
          $$
  \end{enumerate}
\end{note}