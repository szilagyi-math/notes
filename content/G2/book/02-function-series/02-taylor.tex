\clearpage
\section{Taylor-sorok}\label{sec-02-02}

\begin{blueBox}
  Tegyük fel, hogy az $f$ függvény $\sum a_n x^n$ hatványsor alakban
  előállítható.
  \begin{align*}
    f(x)    & = a_0 + a_1 x + a_2 x^2 + a_3 x^3 + \ldots + a_n x^n
    \\
    f'(x)   & = a_1 + 2 a_2 x + 3 a_3 x^2 + \ldots + n \, a_n x^{n-1}
    \\
    f''(x)  & = 2 a_2 + 3 \cdot 2 a_3 x + \ldots + n \, (n-1) \, a_n x^{n-2}
    \\
    f'''(x) & = 3 \cdot 2 a_3 + \ldots + n \, (n-1) \, (n-2) \, a_n x^{n-3}
    % \\
    %            & \vdots
    % \\
    % f^{(n)}(x) & = n! \, a_n
  \end{align*}
  Vizsgáljuk azt az esetet, amikor $x = 0$.
  \begin{align*}
    f(0)       & = a_0
    \\
    f'(0)      & = a_1
    \\
    f''(0)     & = 2 a_2
    \\
    f'''(0)    & = 3 \cdot 2 a_3
    \\
               & \vdots
    \\
    f^{(n)}(0) & = n! \, a_n
  \end{align*}
  Ebből felírva a függvényt:
  $$
    f(x)
    = \frac{f(0)}{0!}
    + \frac{f'(0)}{1!} x
    + \frac{f''(0)}{2!} x^2
    + \frac{f'''(0)}{3!} x^3
    + \dots
    + \frac{f^{(n)}(0)}{n!} x^n
    \text.
  $$
  Zárt alakra hozva:
  $$
    f(x)
    = \sum_{n=0}^{\infty} \frac{f^{(n)}(0)}{n!} x^n
    \text, \quad \text{ahol} \quad
    f^{(0)}(x) = f(x)
    \text.
  $$
\end{blueBox}

\begin{definition}[Taylor-polinom]
  Legyen $f: I \subset \Reals \to \Reals$ függvény, mely az $x_0$ pontban
  legalább $p$-szer differenciálható. Ekkor az $f$ függvény $x_0$ körüli
  $p$-edik Taylor-polinomja:
  $$
    T_p(x) = \sum_{k=0}^{p} \frac{f^{(k)}(x_0)}{k!} (x - x_0)^k
    \text.
  $$
\end{definition}

\begin{example}
  \bgroup\sffamily
  Írjuk fel a $p(x) = x^3 + 3x^2 + 2$ függvény $x_0 = 1$ körüli harmadfokú
  Taylor-polinomját!
  \egroup

  \hdashrule[.8ex][x]{\dimexpr\textwidth}{1pt}{2mm 3pt}

  \begin{minipage}[t]{.4\textwidth}
    \def\arraystretch{1.2}
    \begin{tabular}{|>{$}l<{$}>{$}r<{$}|}
      \hline
      p^{(n)}(x)            & p^{(n)}(1)
      \\ \hline
      p(x) = x^3 + 3x^2 + 2 & 6
      \\
      p'(x) = 3x^2 + 6x     & 9
      \\
      p''(x) = 6x + 6       & 12
      \\
      p'''(x) = 6           & 6
      \\ \hline
    \end{tabular}
  \end{minipage}\begin{minipage}{.6\textwidth}
    \begin{align*}
      T_3(x)
       & = \frac{6}{0!} + \frac{9}{1!}(x - 1) + \frac{12}{2!}(x - 1)^2 + \frac{6}{3!} (x - 1)^3
      \\
       & = 6 + 9(x - 1) + 6(x - 1)^2 + (x - 1)^3
    \end{align*}
  \end{minipage}
\end{example}

\begin{theorem}[Taylor-formula Lagrange-féle maradéktaggal]
  Ha az $f$ függvény legalább $(r + 1)$-szer differenciálható az $(x; x_0)$
  intervallumon és $f^{(k)}$ $\forall k \in \{1;2;\dots;r\}$ esetén folytonos
  ay $x$ és $x_0$ pontokban, akkor $\exists \xi \in (x; x_0)$, hogy
  $$
    f(x)
    = \sum_{k=0}^{r} \frac{f^{(k)}(x_0)}{k!} (x - x_0)^k
    + \underbrace{\frac{f^{(r+1)}(\xi)}{(r+1)!} (x - x_0)^{r+1}}_{\text{Lagrange-féle maradéktag}}
  $$

  \begin{proof}
    Definiáljuk a következő függvényt:
    $$
      F(t)
      = f(t)
      + \frac{f'(t)}{1!}
      + \frac{f''(t)}{2!}
      + \dots
      + \frac{f^{(r)}(t)}{r!} (x - t)^r
      + \frac{c_{r+1}}{(r+1)!} (x-t)^{(r+1)}
      \text.
    $$
  \end{proof}

  Válasszuk meg a $c_{r+1}$ együtthatót úgy, hogy $F(x) = f(x)$. Differenciáljuk
  az $F(t)$ függvényt!
  \begin{align*}
    F'(t)
    = & f'(t)
    + \left( \frac{f''(t)}{1!}(x - t) - \frac{f'(t)}{1!} \right)
    + \left( \frac{f'''(t)}{2!}(x - t)^2 - \frac{f''(t)}{2!} 2 (x - t) \right)
    +
    \\
      & + \dots
    + \left( \frac{f^{(r + 1)(t)}}{r!}(x - t)^r - \frac{f^{(r)}(t)}{r!} r (x-t)^{r-1} \right)
    - \frac{c_{r+1}}{r!} (r + 1) (x - t)^{r}
  \end{align*}

  Ha felbontjuk a zárójeleket, akkor láthatjuk, hogy az egyes tagok páronként
  kiejtik egymást. Az egyenlőség az alábbi alakra egyszerűsödik:
  $$
    F'(t) = \frac{(x - t)^r}{r!} \left( f^{(r + 1)} - c_{r + 1} \right)
    \text.
  $$

  A Rolle-tétel alapján $\exists \xi \in (x; x_0)$, hogy $F'(\xi) = 0$. Ekkor:
  $$
    0
    = \underbrace{\frac{(x - \xi)^r}{r!}}_{\neq 0}
    \underbrace{\left( f^{(r + 1)} - c_{r + 1} \right)\vphantom{\frac12}}_{= 0}
    \quad \Rightarrow \quad
    f^{(r + 1)}(\xi) = c_{r + 1}
    \text.
  $$
\end{theorem}

\begin{definition}[Taylor-sor]
  Legyen az $f$ függvény az $x_0$ pontban akárhányszor differenciálható. Ekkot a
  $$
    T(x) = \sum_{k=0}^{\infty} \frac{f^{(k)}(x_0)}{k!} (x - x_0)^k
  $$
  hatványsort az $f$ függvény $x_0$ körüli Taylor-sorának nevezzük.
\end{definition}

\begin{note}
  Ha $x_0 = 0$, akkor a Taylor-sorot Maclaurin-sornak nevezzük.
\end{note}

\begin{theorem}
  Az előbb definiált Taylor-sor akkor és csak akkor állítja elő a függvényt
  az $x = x_0$ pontban, ha a maradéktag a nullához tart $n \to \infty$ esetén.

  \begin{proof}
    \vspace{10em}
  \end{proof}
\end{theorem}

\begin{blueBox}
  \sftitle{Fontosabb függvények Maclaurin-sorai:}

  \begin{center}
    \setlength\extrarowheight{5pt}
    \renewcommand{\arraystretch}{1.5}
    \begin{tabular}{|c|c|c|}
      \hline
      Függvény         & Taylor-sor                                                         & Konvergencia intervallum
      \\[10pt]
      \hline
      $e^x$            & $\displaystyle\sum_{k=0}^{\infty} \frac{x^k}{k!}$                  & $\Reals$
      \\[10pt]
      \hline
      $\sin x$         & $\displaystyle\sum_{k=0}^{\infty} (-1)^k \frac{x^{2k+1}}{(2k+1)!}$ & $\Reals$
      \\[10pt]
      \hline
      $\cos x$         & $\displaystyle\sum_{k=0}^{\infty} (-1)^k \frac{x^{2k}}{(2k)!}$     & $\Reals$
      \\[10pt]
      \hline
      $\arctan x$      & $\displaystyle\sum_{k=0}^{\infty} (-1)^k \frac{x^{2k+1}}{2k+1}$    & $[-1; 1]$
      \\[10pt]
      \hline
      $\sinh x$        & $\displaystyle\sum_{k=0}^{\infty} \frac{x^{2k+1}}{(2k+1)!}$        & $\Reals$
      \\[10pt]
      \hline
      $\cosh x$        & $\displaystyle\sum_{k=0}^{\infty} \frac{x^{2k}}{(2k)!}$            & $\Reals$
      \\[10pt]
      \hline
      $\arctanh x$     & $\displaystyle\sum_{k=0}^{\infty} \frac{x^{2k+1}}{2k+1}$           & $(-1; 1)$
      \\[10pt]
      \hline
      $\ln(1 + x)$     & $\displaystyle\sum_{k=0}^{\infty} (-1)^{k} \frac{x^{k+1}}{k+1}$    & $(-1; 1]$
      \\[10pt]
      \hline
      $(1 + k)^\alpha$ & $\displaystyle\sum_{k=0}^{\infty} \binom{\alpha}{k} x^k$           & $(-1; 1)$
      \\[10pt]
      \hline
    \end{tabular}
  \end{center}
\end{blueBox}