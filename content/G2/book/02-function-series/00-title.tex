\mainChapter{Függvénysorozatok, függvénysorok}\label{chap-02}

\bgroup
\color{gray!50!black}
\sffamily

Ebben a fejezetben a számsorozatokról (numerikus sorozatokról) és a sorokról
tanult ismereteinket bővítjük, emeljük magasabb szintre: függvényekből alkotott
sorozatokkal és sorokkal foglalkozunk. Míg egy számsorozat esetén egyetlen
határértékét létezését vizsgáltuk, addig egy függvénysorozat értelmezési
tartományának pontjaiban vizsgáljuk a konvergenciát. Azon pontok halmazát, ahol
a függvénysorozat konvergens, konvergenciahalmaznak hívjuk. Ezekben a pontokban
a függvénysorozat a határfüggvényhez tart. Ahogy numerikus sornál is külön nevet
adtunk a határértéknek (sorösszeg), úgy függvénysorok esetén összegfüggvényről
beszélünk. Megvizsgáljuk, hogy ez a konvergencia hogyan valósulhat meg, és
bevezetjük a pontonkénti és az egyenletes konvergencia kulcsfontosságú
fogalmait. Látni fogjuk, hogy az utóbbi egy erős tulajdonság, amely lehetővé
teszi, hogy a függvények olyan kellemes tulajdonságai, mint a folytonosság vagy
az integrálhatóság, "öröklődjenek" a határfüggvényre, összegfüggvényre.
Megismerkedünk a hatványsorokkal, amelyek esetén a konvergencia könnyebben
vizsgálható, nem egyszer a numerikus sorok esetén megismert eljárásoknak
köszönhetően.

A fejezet második felében fontos és széles körben alkalmazott függvénysorokkal
foglalkozunk. A Taylor-sorok speciális hatványsorok, amelyek a korábban
megismert Taylor-polinomok általánosításaiként is tekinthetők. Használatukkal
bonyolult függvényeket közelíthetünk polinomokkal. Ez a technika a mérnöki
számításoktól a fizikai modellezésig számtalan területen elengedhetetlen.

A Fourier-sorok használata a periodikus függvényekkel való számolásokat
könnyítheti meg, azok szinuszok és koszinuszok végtelen összegeként való
előállításával foglalkoznak. A Fourier-sorok alapvető eszközt jelentenek a
jelenségek frekvenciakomponensekre való bontásához, így kulcsfontosságúak a
jelfeldolgozásban, az akusztikában és a képfeldolgozásban.

Ez a fejezet tehát ajtót nyit a függvények egy új, dinamikusabb szemléletére,
amely elengedhetetlen a modern matematika és alkalmazásainak megértéséhez.

\chaptertoc
\egroup