\clearpage
\section{Fourier-sorok}\label{sec-02-03}

\begin{definition}[Trigonometrikus polinom]
  Az alábbi alakú függvényt trigonometrikus polinopmnak nevezzük:
  $$
    t_n(x)
    = a_0
    + a_1 \cos x + b_1 \sin x
    + a_2 \cos 2x + b_2 \sin 2x
    + \ldots
    + a_n \cos nx + b_n \sin nx
    \text.
  $$
\end{definition}

\begin{definition}[Trigonometrikus sor]
  Az alábbi alakó összeget trigonometrikus sornak nevezzük:
  \begin{align*}
    t(x)
     & = a_0
    + a_1 \cos x + b_1 \sin x
    + \ldots
    + a_n \cos nx + b_n \sin nx
    + \ldots
    =
    \\
     & = a_0
    + \sum_{k=1}^{\infty} a_k \cos kx + b_k \sin kx
  \end{align*}
\end{definition}

\begin{note}
  \begin{enumerate}
    \item Az összegfüggvény, ha létezik $2\pi$ szerint periodikus.

    \item Ha folytonos a függvény és egyenletesen konvergens, akkor az
          összegfüggvény is konvergens.
  \end{enumerate}
\end{note}

\begin{definition}[Fourier-sor]
  Legyen $f : \Reals \to \Reals$ egy $2p$ szerint periodikus függvény, amely a
  $[0; 2p]$ intervallumon Riemann-integrálható. Ekkor a függvény Fourier-során
  az alábbi trigonometrikus sort értjük:
  $$
    F(x)
    = a_0
    + \sum_{k=1}^{\infty} \left[
      a_k \cos \left(\frac{k\pi x}{p}\right) + b_k \sin \left(\frac{k\pi x}{p}\right)
    \right]
    \text,
  $$
  ahol az együtthatók a következők:
  \begin{align*}
    a_0 & = \frac{1}{2p} \int_{0}^{2p} f(x) \dd x
    \text,                                                                          \\
    a_k & = \frac{1}{p} \int_{0}^{2p} f(x) \cos \left(\frac{k\pi x}{p}\right) \dd x
    \text,                                                                           \qquad
    b_k = \frac{1}{p} \int_{0}^{2p} f(x) \sin \left(\frac{k\pi x}{p}\right) \dd x
  \end{align*}
\end{definition}

\begin{note}
  Ha az $f$ függvény előáll a fenti típusú összegként, akkor az együtthatók
  csak ilyenek lehetnek.
\end{note}

\begin{note}
  Ha az $f$ függvény $2p$ szerint periodikus, akkor mindegy, hogy a $[0; 2p]$
  intervallumon, vagy egy skalárral eltolva az $[a; a+2p]$ intervallumon
  integrálunk, vagyis:
  $$
    \int_0^{2p} f(x) \dd x = \int_a^{a+2p} f(x) \dd x
  $$
\end{note}

\begin{theorem}
  Ha a $2\pi$ szerint periodikus $f$ függvénynek létezik az $x_0$ pontban a
  jobb- és baloldali határértéke, továbbá az $f$ függvény Fourier-sora ebben a
  pontban konvergens, akkor a Fourier-sor összege ezen pontokban a függvény
  bal- és jobboldali határértékeinek számtani középe.

  \begin{proof}
    \vspace{8em}
  \end{proof}
\end{theorem}

\begin{note}
  Ha az $f$ függvény folytonos az $x_0$ pontban, akkor a Fourier-sor összege
  ebben a pontban a függvény határértékével egyezik meg.
\end{note}

\begin{theorem}
  Ha a $2\pi$ szerint periodikus, integrálható $f$ függvény szakaszonként
  monoton és az $x_0$ pontban differenciálható, akkor az $f$ függvény
  Fourier-sora ebben a pontban konvergens.

  \begin{proof}
    \vspace{8em}
  \end{proof}
\end{theorem}

\begin{statement}
  Ha egy függvény páros, akkor a Fourier-sorában csak $a_0$ és koszinusz tagok
  szerepelnek, vagyis $b_k \equiv 0$.

  \begin{proof}
    \vspace{8em}
  \end{proof}
\end{statement}

\begin{statement}
  Ha egy függvény páratlan, akkor a Fourier-sorában csak szinusz tagok
  szerepelnek, vagyis $a_0 \equiv 0$ és $a_k \equiv 0$.

  \begin{proof}
    \vspace{8em}
  \end{proof}
\end{statement}

\begin{example}
  \bgroup\sffamily
  Állítsuk elő az $f$ függvény Fourier-sorát, $f(x) = x^2$, ha
  $x \in [-\pi; \pi]$ és $f(x) = f(x + 2k\pi)$, ahol $k \in \mathbb Z$.
  \egroup

  \hdashrule[.8ex][x]{\dimexpr\textwidth}{1pt}{2mm 3pt}

  Mivel a függvény páros, ezért a Fourier-sorában csak $a_0$ és $a_k$ tagok
  szerepelnek. Az integrációs intervallumot $[-\pi; \pi]$-re választjuk.

  Az $a_0$ együttható:
  $$
    a_0 = \frac{1}{2\pi} \int_{-\pi}^{\pi} x^2 \dd x
    = \frac{1}{2\pi} \left[ \frac{x^3}{3} \right]_{-\pi}^{\pi}
    = \frac{1}{2\pi} \left( \frac{\pi^3}{3} - \frac{(-\pi)^3}{3} \right)
    = \frac{\pi^3}{3}
    \text.
    $$

      Az $a_k$ meghatározása:
      \begin{align*}
        a_k
         & = \frac{1}{\pi} \int_{-\pi}^{\pi} x^2 \cos kx \dd x
        = \frac{1}{\pi} \left(
        \left[\frac{x^2 \sin kx}{k}\right]_{-\pi}^{\pi}
        - \left[\frac{2x \cos kx}{k^2}\right]_{-\pi}^{\pi}
        + \left[\frac{2 \sin kx}{k^3}\right]_{-\pi}^{\pi}
        \right) =
        \\
         & = \frac{1}{\pi} \left(
        \frac{2\pi \cos k\pi}{k^2} + \frac{2\pi \cos (-k\pi)}{k^2}
        \right) = \frac{1}{\pi} \left(
        \frac{4\pi \cos k\pi}{k^2}
        \right) = \frac{4}{k^2} \cos k\pi
        = \frac{4}{k^2} (-1)^k
        \text.
      \end{align*}

      A Fourier-sor tehát:
    $$
    F(x) = \frac{\pi^3}{3} + \sum_{k=1}^{\infty} \frac{4(-1)^k}{k^2} \cos kx
    \text.
  $$
\end{example}