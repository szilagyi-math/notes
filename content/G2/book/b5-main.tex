\documentclass[b5paper, 10pt, twoside]{scrbook}

\usepackage[
  inner=25mm,
  outer=15mm,
  top=20mm,
  bottom=20mm,
  headsep=10mm,
  headheight=15pt,
  footskip=10mm,
  % showframe,
]{geometry}

\usepackage[
  headsepline=1mm:\textwidth,
  footsepline=1mm:\textwidth,
  olines,
]{scrlayer-scrpage}

\usepackage[stixtwo]{fontsetup}
\usepackage[magyar]{babel}

\usepackage{amsmath}
\usepackage{unicode-math}

\begin{document}
\frontmatter

\begin{titlepage}
  Titlepage goes here
\end{titlepage}

\tableofcontents

\mainmatter

% ~~~~~~~~~~~~~~~~~~~~~~~~~~~~~~~~~~~~~~~~~~~~~~~~~~~~~~~~~~~~~~~~~~~~~~~~~~~~~~
% ~~~~~~~~~~~~~~~~~~~~~~~~~~~~~~~~~~~~~~~~~~~~~~~~~~~~~~~~~~~~~~~~~~~~~~~~~~~~~~
% ~~~~~~~~~~~~~~~~~~~~~~~~~~~~~~~~~~~~~~~~~~~~~~~~~~~~~~~~~~~~~~~~~~~~~~~~~~~~~~
\clearpage
\chapter{Lineáris algebra}

\clearpage
\section{Mátrixalgebra}

\clearpage
\section{Lineáris egyenletrendszerek}

\clearpage
\section{Lineáris leképezések}



% ~~~~~~~~~~~~~~~~~~~~~~~~~~~~~~~~~~~~~~~~~~~~~~~~~~~~~~~~~~~~~~~~~~~~~~~~~~~~~~
% ~~~~~~~~~~~~~~~~~~~~~~~~~~~~~~~~~~~~~~~~~~~~~~~~~~~~~~~~~~~~~~~~~~~~~~~~~~~~~~
% ~~~~~~~~~~~~~~~~~~~~~~~~~~~~~~~~~~~~~~~~~~~~~~~~~~~~~~~~~~~~~~~~~~~~~~~~~~~~~~
\clearpage
\chapter{Függvénysorozatok, függvénysorok}

\clearpage
\section{Alapfogalmak}

\clearpage
\section{Taylor-sorok}

\clearpage
\section{Fourier-sorok}



% ~~~~~~~~~~~~~~~~~~~~~~~~~~~~~~~~~~~~~~~~~~~~~~~~~~~~~~~~~~~~~~~~~~~~~~~~~~~~~~
% ~~~~~~~~~~~~~~~~~~~~~~~~~~~~~~~~~~~~~~~~~~~~~~~~~~~~~~~~~~~~~~~~~~~~~~~~~~~~~~
% ~~~~~~~~~~~~~~~~~~~~~~~~~~~~~~~~~~~~~~~~~~~~~~~~~~~~~~~~~~~~~~~~~~~~~~~~~~~~~~
\clearpage
\chapter{Többváltozós analízis}

\clearpage
\section{Alapfogalmak}

\clearpage
\section{Többváltozós függvények differenciálása}

\clearpage
\section{Többváltozós függvények szélsőértékei}

\clearpage
\section{Többváltozós függvények integrálása}


\end{document}