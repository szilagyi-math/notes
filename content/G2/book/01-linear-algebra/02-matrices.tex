\clearpage
\section{Mátrixalgebra}\label{sec-01-02}

\begin{definition}[Mátrix]
  Egy mátrix vízszintes vonalban elhelyezkedő elemei \textbf{sorok}at,
  míg függőlegesen elhelyezkedő elemei \textbf{oszlop}okat alkotnak.

  Egy $m$ sorból és $n$ oszlopból álló mátrix jelölése:
  $$
    \rmat A = \begin{bmatrix}
      a_{11} & a_{12} & \ldots & a_{1n} \\
      a_{21} & a_{22} & \ldots & a_{2n} \\
      \vdots & \vdots & \ddots & \vdots \\
      a_{m1} & a_{m2} & \ldots & a_{mn}
    \end{bmatrix}
    \text.
  $$

  Mátrixok jelölése nyomtatott szövegben: $\rmat A$.

  Mátrixok jelölése írásban: $\underline{\underline A}$.

  Az $m \times n$-es mátrixok halmazának jelölései: $\mathcal M_{m \times n}
    = \mathbb R^m \times \mathbb R^n = \mathbb R^{m \times n}$.

  A mátrix $i$-edik sorában és $j$-edik oszlopában található elemet
  $a_{ij}$-vel jelöljük.
\end{definition}

\begin{note}
  A mátrix dimenzióit mindig először a sorok számával, majd azt követően az
  oszlopok számával adják meg.
\end{note}

\begin{blueBox}
  \sftitle{Speciális mátrixstruktúrák:}
  \newenvironment{tmatrix}{%
    \begin{bmatrix}
      \hphantom{a_{n1}} & \hphantom{a_{n2}} & \hphantom{\ldots} & \hphantom{a_{nn}} \\[-14pt]
      }{%
    \end{bmatrix}
  }
  $$
    \begin{array}{rc>{\in\;}c>{\sim\;\;}l}
       & \begin{bmatrix}
           a_{11} \\ a_{21} \\ \vdots \\ a_{n1}
         \end{bmatrix}
       & \mathcal M_{n \times 1}
       & \text{oszlopvektor / oszlopmátrix}
      \\[12mm]
       & \begin{bmatrix}
           \;a_{11} & a_{12} & \ldots & a_{1n}\;
         \end{bmatrix}
       & \mathcal M_{1 \times n}
       & \text{sorvektor / sormátrix}
      \\[4mm]
       & \begin{bmatrix}
           a_{11} & a_{12} & \ldots & a_{1n} \\
           a_{21} & a_{22} & \ldots & a_{2n} \\
           \vdots & \vdots & \ddots & \vdots \\
           a_{n1} & a_{n2} & \ldots & a_{nn}
         \end{bmatrix}
       & \mathcal M_{n \times n}
       & \text{kvadratikus / négyzetes mátrix}
      \\[12mm]
      \imat =
       & \begin{tmatrix}
           1      & 0      & \ldots & 0      \\
           0      & 1      & \ldots & 0      \\
           \vdots & \vdots & \ddots & \vdots \\
           0      & 0      & \ldots & 1
         \end{tmatrix}
       & \mathcal M_{n \times n}
       & \text{egységmátrix}
      \\[12mm]
      \nmat =
       & \begin{tmatrix}
           0      & 0      & \ldots & 0      \\
           0      & 0      & \ldots & 0      \\
           \vdots & \vdots & \ddots & \vdots \\
           0      & 0      & \ldots & 0
         \end{tmatrix}
       & \mathcal M_{m \times n}
       & \text{nullmátrix}
    \end{array}
  $$
\end{blueBox}

\begin{definition}[Mátrix transzponáltja]
  Egy $\rmat A \in \mathcal M_{m \times n}$ mátrix transzponáltja a főátlójára
  vett tükörképe. Jele: $\rmat A^\T \in \mathcal M_{n \times m}$.
\end{definition}

\begin{example}
  Határozzuk meg az $\rmat A = \begin{bmatrix}
      1 & 2 & 3 \\
      4 & 5 & 6
    \end{bmatrix}$ mátrix transzponáltját!

  \hdashrule[.8ex][x]{\dimexpr\textwidth}{1pt}{2mm 3pt}
  $$
    \rmat A = \begin{bmatrix}
      1 & 2 & 3 \\
      4 & 5 & 6
    \end{bmatrix} \in \mathcal M_{2 \times 3}
    \quad \Rightarrow \quad
    \rmat A^\T = \begin{bmatrix}
      1 & 4 \\
      2 & 5 \\
      3 & 6
    \end{bmatrix} \in \mathcal M_{3 \times 2}
  $$
\end{example}

\begin{definition}[Szimmertikus mátrix]
  Egy $\rmat A \in \mathcal M_{n \times n}$ mátrix szimmetrikus, ha
  $\rmat A = \rmat A^\T$.
\end{definition}

\begin{definition}[Antiszimmertikus mátrix]
  Egy $\rmat A \in \mathcal M_{n \times n}$ mátrix antiszimmertikus, ha
  $\rmat A = -\rmat A^\T$.
\end{definition}

\begin{note}
  Antiszimmertikus mátrixok főátlójában csak nullák szerepelnek.
\end{note}

\begin{definition}[Mátrixok egyenlősége]
  Két mátrix akkor és csak akkor egyenlő, ha a megfelelő helyeken álló elemei
  egyenlőek.
  $$
    !A, B \in \mathcal M_{m \times n}: A = B
    \quad \Longleftrightarrow \quad
    \forall i \in \{1, 2, \ldots, m\}
    \; \land \;
    \forall j \in \{1, 2, \ldots, n\}:
    a_{ij} = b_{ij}
  $$
\end{definition}

% ~~~~~~~~~~~~~~~~~~~~~~~~~~~~~~~~~~~~~~~~~~~~~~~~~~~~~~~~~~~~~~~~~~~~~~~~~~~~~~
% ~~~~~~~~~~~~~~~~~~~~~~~~~~~~~~~ Old Break ~~~~~~~~~~~~~~~~~~~~~~~~~~~~~~~~~~~~
% ~~~~~~~~~~~~~~~~~~~~~~~~~~~~~~~~~~~~~~~~~~~~~~~~~~~~~~~~~~~~~~~~~~~~~~~~~~~~~~

\begin{definition}[Mátrixok összege]
  Két mátrix összegén azt a mátrixot értjük, melyet a két mátrix elemenkénti
  összeadásával kapunk, azaz, ha $\rmat A, \rmat B \in \mathcal M_{m \times n}$,
  akkor $\rmat C := \rmat A + \rmat B \in \mathcal M_{m \times n}$, ahol
  $c_{ij} := a_{ij} + b_{ij}$.
\end{definition}

\begin{example}
  Határozzuk meg az $\rmat A = \begin{bmatrix}
      1 & 2 & 3 \\
      4 & 5 & 6
    \end{bmatrix}$ és a $\rmat B = \begin{bmatrix}
      6 & 5 & 4 \\
      3 & 2 & 1
    \end{bmatrix}$ mátrixok összegét!

  \hdashrule[.8ex][x]{\dimexpr\textwidth}{1pt}{2mm 3pt}
  $$
    \begin{bmatrix}
      1 & 2 & 3 \\
      4 & 5 & 6
    \end{bmatrix}
    +
    \begin{bmatrix}
      6 & 5 & 4 \\
      3 & 2 & 1
    \end{bmatrix}
    =
    \begin{bmatrix}
      1 + 6 & 2 + 5 & 3 + 4 \\
      4 + 3 & 5 + 2 & 6 + 1
    \end{bmatrix}
    =
    \begin{bmatrix}
      7 & 7 & 7 \\
      7 & 7 & 7
    \end{bmatrix}
  $$
\end{example}

\begin{definition}[Mátrix és skalár szorzata]
  Egy mátrix és egy skalár szorzata olyan mátrix, melynek minden eleme
  skalárszorosa az eredeti mátrix elemeinek, azaz ha
  $\rmat A \in \mathcal M_{m \times n}$ és $\lambda \in \mathbb R$, akkor
  $C := \lambda \rmat A$, ahol $c_{ij} := \lambda a_{ij}$.
\end{definition}

\begin{example}
  Határozzuk meg a $\lambda = 2$ skalár és az $\rmat A = \begin{bmatrix}
      1 & 2 & 3 \\
      4 & 5 & 6
    \end{bmatrix}$ mátrix szorzatát!

  \hdashrule[.8ex][x]{\dimexpr\textwidth}{1pt}{2mm 3pt}
  $$
    \lambda \cdot \rmat A
    = 2 \cdot
    \begin{bmatrix}
      1 & 2 & 3 \\
      4 & 5 & 6
    \end{bmatrix}
    =
    \begin{bmatrix}
      2 \cdot 1 & 2 \cdot 2 & 2 \cdot 3 \\
      2 \cdot 4 & 2 \cdot 5 & 2 \cdot 6
    \end{bmatrix}
    =
    \begin{bmatrix}
      2 & 4  & 6  \\
      8 & 10 & 12
    \end{bmatrix}
  $$
\end{example}

\begin{definition}[Mátrixok szorzata]
  Legyen $\rmat A \in \mathcal M_{m \times n}$ és
  $\rmat B \in \mathcal M_{n \times p}$. Ekkor a két mátrix szorzata
  $$
    \rmat C := \rmat A \cdot \rmat B
    \text{, ahol }
    c_{ij}
    = \sum_{k=1}^{n} a_{ik} \cdot b_{kj}
    = a_{i1} \cdot b_{1j} + a_{i2} \cdot b_{2j} + \ldots + a_{in} \cdot b_{nj}
    \text.
  $$
\end{definition}

\begin{blueBox}
  \sftitle{A mátrixszorzás vizualizálása:}
  \newcolumntype{x}[1]{>{\centering\arraybackslash\hspace{0pt}}p{#1}}
  \newcolumntype{F}[1]{>{$}x{#1}<{$}}
  \def\arraystretch{1.1}
  \begin{align*}
     & \left[\begin{array}{F{2cm}cF{2cm}}
                 b_{11} & \dots  & b_{1p} \\
                 b_{21} & \dots  & b_{2p} \\
                 \vdots & \ddots & \vdots \\
                 b_{n1} & \dots  & b_{np}
               \end{array}\right]
    \\
    \left[\begin{array}{cccc}
              a_{11} & a_{12} & \dots  & a_{1n} \\
              a_{21} & a_{22} & \dots  & a_{2n} \\
              \vdots & \vdots & \ddots & \vdots \\
              a_{m1} & a_{m2} & \dots  & a_{mn}
            \end{array}\right]
     & \left[\begin{array}{F{2cm}cF{2cm}}
                 \sum a_{1i} b_{i1} & \dots  & \sum a_{1i} b_{ip} \\
                 \sum a_{2i} b_{i1} & \dots  & \sum a_{2i} b_{ip} \\
                 \vdots             & \ddots & \vdots             \\
                 \sum a_{mi} b_{i1} & \dots  & \sum a_{mi} b_{ip}
               \end{array}\right]
  \end{align*}
\end{blueBox}

\begin{example}
  Határozzuk meg az $\rmat A = \begin{bmatrix}
      1  & 0  \\
      -2 & -1 \\
      3  & 4
    \end{bmatrix}$ és a $\rmat B = \begin{bmatrix}
      1  & 0 & 2 \\
      -1 & 2 & 0
    \end{bmatrix}$ mátrixok szorzatát!

  \newcolumntype{x}[1]{>{\centering\arraybackslash\hspace{0pt}}p{#1}}
  \newcolumntype{F}[1]{>{$}x{#1}<{$}}
  \begin{align*}
     & \left[\begin{array}{F{3.5cm}F{3.5cm}}
                 1  & 0  \\
                 -2 & -1 \\
                 3  & 4
               \end{array}\right]
    \\
    \left[\begin{array}{ccc}
              1  & 0 & 2 \\
              -1 & 2 & 0
            \end{array}\right]
     & \left[\begin{array}{F{3.5cm}F{3.5cm}}
                 1 \cdot 1 + 0 \cdot (-2) + 2 \cdot 3  & 1 \cdot 0 + 0 \cdot 2 + 2 \cdot 4     \\
                 -1 \cdot 1 + 2 \cdot (-2) + 0 \cdot 3 & -1 \cdot 0 + 2 \cdot (-1) + 0 \cdot 4
               \end{array}\right] = \begin{bmatrix}
                                    7  & 8  \\
                                    -5 & -2
                                  \end{bmatrix}
  \end{align*}
\end{example}

\begin{statement}
  Ha $\rmat A$, $\rmat B$ és $\rmat C$ olyan mátrixok, hogy létezik az
  $(\rmat A \cdot \rmat B) \cdot \rmat C$ mátrixszorzat, akkor az
  $\rmat A \cdot (\rmat B \cdot \rmat C)$ mátrixszorzat is létezik, és
  ezek egyenlőek.

  A mátrixszorzás tehát \textbf{asszociativív}.

  \begin{proof}
    \vspace{4em}
  \end{proof}
\end{statement}

\begin{statement}
  Ha $\rmat A$, $\rmat B$ és $\rmat C$ olyan mátrixok, hogy létezik az
  $\rmat A \cdot (\rmat B + \rmat C)$ szorzat, akkor az $\rmat A \cdot \rmat B$
  és az $\rmat A \cdot \rmat C$ mátrixszorzatok is léteznek, valamint
  $\rmat A \cdot (\rmat B + \rmat C)
    = \rmat A \cdot \rmat B + \rmat A \cdot \rmat C$.

  Teljesül tehát a \textbf{disztributivitás}.

  \begin{proof}
    \vspace{6em}
  \end{proof}
\end{statement}

% ~~~~~~~~~~~~~~~~~~~~~~~~~~~~~~~~~~~~~~~~~~~~~~~~~~~~~~~~~~~~~~~~~~~~~~~~~~~~~~
% ~~~~~~~~~~~~~~~~~~~~~~~~~~~~~~~ Old Break ~~~~~~~~~~~~~~~~~~~~~~~~~~~~~~~~~~~~
% ~~~~~~~~~~~~~~~~~~~~~~~~~~~~~~~~~~~~~~~~~~~~~~~~~~~~~~~~~~~~~~~~~~~~~~~~~~~~~~

\begin{definition}[Determináns]
  \newcommand\noskp{\vspace{-3mm}}
  \newcommand{\edet}[1]{\det \begin{pmatrix} \phantom{i}\cdots & #1 & \cdots\phantom{i} \end{pmatrix}}
  Legyen $\rmat A \in \mathcal M_{n \times n}$ kvadratikus mátrix, és
  $\det: \mathcal M_{n \times n} \rightarrow \mathbb R$ függvény. A mátrix
  $i$-edik oszlopának elemeit tartalmazó oszlopvektorokat $\rvec a_i$-vel
  jelöljük. Az $\rmat A$ determinánsának nevezzük $\det \rmat A$-t, a
  hozzárendelést pedig az alábbi négy axióma írja le:
  \begin{enumerate}
    \item homogén:
          $$
            \edet{\lambda \rvec a_i} = \lambda \edet{\rvec a_i}
            \text,
          $$
    \item \noskp additív:\noskp
          $$
            \edet{\rvec a_i + \rvec b_i} =
            \edet{\rvec a_i} + \edet{\rvec b_i}
            \text,
          $$
    \item \noskp alternáló:\noskp
          $$
            \edet{\rvec a_i & \dots & \rvec a_j} =
            - \edet{\rvec a_j & \dots & \rvec a_i}
            \text,
          $$
    \item \noskp $\imat$ determinánsa:
          $$
            \det \imat =\det \begin{pmatrix}
              \uvec e_1 & \uvec e_2 & \cdots & \uvec e_n
            \end{pmatrix} = 1
            \text,
          $$
  \end{enumerate}
\end{definition}

\begin{statement}
  Ha egy mátrixban van két azonos oszlop, akkor a determinánsa nulla.

  \begin{proof}
    \vspace{6em}
  \end{proof}
\end{statement}

\begin{statement}
  Egy mátrix determinánsa nem változik, ha az egyik oszlopához hozzáadjuk egy
  másik oszlopának skalárszorosát.

  \begin{proof}
    \vspace{6em}
  \end{proof}
\end{statement}

\begin{theorem}[Kifejtési tétel]
  \begin{align*}
    \det \rmat A
     & = \det \left( \rvec a_1; \rvec a_2; \dots; \rvec a_n \right)
    = \begin{vmatrix}
        a_{11} & a_{12} & \cdots & a_{1n} \\
        a_{21} & a_{22} & \cdots & a_{2n} \\
        \vdots & \vdots & \ddots & \vdots \\
        a_{n1} & a_{n2} & \cdots & a_{nn}
      \end{vmatrix}
    = \det \left(
    \sum_{j = 1}^{n} a_{j1} \uvec e_j; \rvec a_2; \dots; \rvec a_n
    \right)
    \\[1mm]
     & = a_{11} \det \left( \uvec e_1; \rvec a_2; \dots; \rvec a_n \right)
    + a_{21} \det \left( \uvec e_2; \rvec a_2; \dots; \rvec a_n \right)
    + \ldots
    + a_{n1} \det \left( \uvec e_n; \rvec a_2; \dots; \rvec a_n \right)
    \\[1mm]
     & = a_{11}\; \scalebox{.825}{$\begin{vmatrix}
                                         1      & 0      & \cdots & 0      \\
                                         0      & a_{22} & \cdots & a_{2n} \\
                                         \vdots & \vdots & \ddots & \vdots \\
                                         0      & a_{n2} & \cdots & a_{nn}
                                       \end{vmatrix}$}
    + a_{21}\; \scalebox{.825}{$\begin{vmatrix}
                                      0      & a_{12} & \cdots & a_{1n} \\
                                      1      & 0      & \cdots & 0      \\
                                      \vdots & \vdots & \ddots & \vdots \\
                                      0      & a_{n2} & \cdots & a_{nn}
                                    \end{vmatrix}$}
    + \dots
    + a_{n1}\; \scalebox{.825}{$\begin{vmatrix}
                                      0      & a_{12} & \cdots & a_{1n} \\
                                      0      & a_{22} & \cdots & a_{2n} \\
                                      \vdots & \vdots & \ddots & \vdots \\
                                      1      & 0      & \cdots & 0
                                    \end{vmatrix}$}
    \\[1mm]
     & = a_{11}\; \scalebox{.825}{$\begin{vmatrix}
                                         1      & 0      & \cdots & 0      \\
                                         0      & a_{22} & \cdots & a_{2n} \\
                                         \vdots & \vdots & \ddots & \vdots \\
                                         0      & a_{n2} & \cdots & a_{nn}
                                       \end{vmatrix}$}
    - a_{21}\; \scalebox{.825}{$\begin{vmatrix}
                                      a_{12} & 0      & \cdots & a_{1n} \\
                                      0      & 1      & \cdots & 0      \\
                                      \vdots & \vdots & \ddots & \vdots \\
                                      a_{n2} & 0      & \cdots & a_{nn}
                                    \end{vmatrix}$}
    + \dots
    + (-1)^{n-1} a_{n1}\; \scalebox{.825}{$\begin{vmatrix}
                                                 a_{12} & \cdots & a_{1n} & 0      \\
                                                 a_{22} & \cdots & a_{2n} & 0      \\
                                                 \vdots & \ddots & \vdots & \vdots \\
                                                 a_{n2} & \cdots & a_{nn} & 1
                                               \end{vmatrix}$}
  \end{align*}

  Jelölje a $k$. sor és $j$. oszlop kitakarásával kapott aldeterminánst
  $\rmat A_{kj}$, ekkor az egyenlőség a következőképpen írható át:
  $$
    a_{11} \rmat A_{11}
    - a_{21} \rmat A_{21}
    + \ldots
    + (-1)^{n-1} a_{n1} \rmat A_{n1}
    \text.
  $$

  Vezessük be a következő jelölést:
  $\rmat{\overline A}_{kj} = (-1)^{k-1} \rmat A_{kj}$. Így:
  $$
    a_{11} \rmat{\overline A}_{11}
    + a_{21} \rmat{\overline A}_{21}
    + \ldots
    + a_{n1} \rmat{\overline A}_{n1}
    =
    \sum_{j=1}^{n} a_{j1} \rmat A_{j1}
    =
    \cdots
    =
    \sum_{k = 1}^{n} (-1)^{\varepsilon} a_{k1} a_{k2} \ldots a_{kn} \det \imat
    \text,
  $$
  ahol $\varepsilon$ az inverziók száma és $\imat$ az egységmátrix.
\end{theorem}

\begin{note}
  \sftitle{A kifejtési tételből következményei:}
  \begin{itemize}
    \item Nem lényeges, hogy sorról vagy oszlopról beszélünk a determinánssal
          kapcsolatban:
          $$
            \det \rmat A = \det \rmat A^\T
            \text.
          $$

    \item A determináns bármely sora vagy oszlopa alapján kifejthető:
          $$
            \det \rmat A
            = \underbrace{
              \sum_{k = 1}^n a_{kj} \rmat{\overline A}_{kj}
            }_{\text{$j$-edik oszlop szerint}}
            = \underbrace{
              \sum_{k = 1}^n a_{ik} \rmat{\overline A}_{ik}
            }_{\text{$i$-edik sor szerint}}
          $$
  \end{itemize}
\end{note}

\begin{example}
  \bgroup\sffamily
  Adjuk meg az $\rmat A = \begin{bmatrix}
      1 & 2 \\ 6 & 7
    \end{bmatrix}$ mátrix determinánsát!
  \egroup

  \hdashrule[.8ex][x]{\dimexpr\textwidth}{1pt}{2mm 3pt}
  $$
    \det A = \begin{vmatrix}
      1 & 2 \\ 6 & 7
    \end{vmatrix} = 1 \cdot 7 - 2 \cdot 6 = 7 - 12 = -5
  $$
\end{example}

% \begin{definition}[Mátrix inverze]
%   Az $\rmat A \in \mathcal M_{n \times n}$ mátrix inverzén egy olyan
%   $\rmat A^{-1} \in \mathcal M_{n \times n}$ mátrixot értünk, melyre
%   $$
%     \rmat A \cdot \rmat A^{-1} = \rmat A^{-1} \cdot \rmat A = \imat
%     \text.
%   $$
% \end{definition}

\begin{theorem}[Lineárisan független vektorok]
  Az $\{\rvec a_1; \rvec a_2; \ldots; \rvec a_n\}$ vektorok lineárisan
  függetlenek, ha $\det(\rvec a_1; \rvec a_2; \ldots; \rvec a_n) \neq 0$.

  \begin{proof}
    \vspace{10em}
  \end{proof}
\end{theorem}

\begin{definition}[Mátrix rangja]
  A mátrix rangjának nevezzük az oszlopvektorai közül a lineárisan függetlenek
  maximális számát.
\end{definition}

\begin{example}
  \bgroup\sffamily
  Határozzuk meg az $\rmat A = \begin{bmatrix}
      1 & 2 & 3 \\
      4 & 5 & 6 \\
      7 & 8 & 9
    \end{bmatrix}$ mátrix rangját!
  \egroup

  \hdashrule[.8ex][x]{\dimexpr\textwidth}{1pt}{2mm 3pt}
  Az $\rmat A$ mátrix rangja 2, mivel a harmadik oszlop a második oszlop
  skalárszorosaként áll elő.
\end{example}

\begin{theorem}[Mátrixok rangszámának tétele]
  Egy mátrix rangja megegyezik maximális el nem tűnő aldeterminánsának
  rendjével.

  \begin{proof}
    \vspace{10em}
  \end{proof}
\end{theorem}

\begin{definition}[Mátrix elemi átalakításai]
  Egy mátrix elemi átalakításainak nevezzük a következőket:
  \begin{itemize}
    \item A mátrix egy tetszőleges sorát vagy oszlopát egy 0-tól különböző
          számmal megszorozzuk.

    \item A mátrix egy tetszőleges sorát vagy oszlopát felcseréljük.

    \item A mátrix egy tetszőleges sorához vagy oszlopához egy másik tetszőleges
          sorát vagy oszlopát adjuk.
  \end{itemize}
\end{definition}

\begin{statement}
  Egy mátrix rangja elemi átalakítások során nem változik.

  \begin{proof}
    A determináns axiómáit figyelembe véve látható, hogy az elemi
    átalakítások nem változtatják meg a determináns 0 voltát.
  \end{proof}
\end{statement}

\begin{definition}[Kvadratikus és szinguláris mátrix]
  Egy kvadratikus (négyzetes) mátrixot \textbf{regulárisnak} mondunk, ha
  determinánsa nem zérus.

  Ha a kvadratikus mátrix determinánsa 0, \textbf{szinguláris} mátrixról
  beszélünk.
\end{definition}

\begin{theorem}[A determinánsok szorzástétele]
  Legyen $\rmat A; \rmat B \in \mathcal M_{n \times n}$ mátrix, ekkor
  $\det(\rmat A \cdot \rmat B) = \det \rmat A \cdot \det \rmat B$.

  \begin{proof}
    \vspace{8em}
  \end{proof}
\end{theorem}

\begin{statement}
  $(\mathcal M_{n \times n}; +; \cdot)$ egységelemes gyűrű, mert\dots
  \begin{itemize}
    \item  $(\mathcal M_{n \times n}; +)$ Abel-csoport,
    \item $(\mathcal M_{n \times n}; \cdot)$ asszociatív,
    \item teljesül a disztributivitás,
    \item létezik a szorzás egységeleme, amely maga az egységmátrix.
  \end{itemize}
\end{statement}

\begin{definition}[Mátrix inverze]
  Az $\rmat A \in \mathcal M_{n \times n}$ mátrix inverzét az $\rmat A^{-1}$
  jelöli, és az a mátrix, melyre $\rmat A \cdot \rmat A^{-1} = \imat$
  teljesül.
\end{definition}

\begin{theorem}[Ferde kifejtési tétel]
  Legyen $\rmat A \in \mathcal M_{n \times n}$, ekkor
  $$
    \sum_{i=1}^{n} a_{ji} \rmat A_{ki} = 0
    \text{, ha}
    j \neq k
    \text.
  $$

  \begin{proof}
    \vspace{6em}
  \end{proof}
\end{theorem}

\begin{statement}
  Egy szinguláris mátrixnak nem létezik inverze.

  \begin{proof}[Indirekt módon]
    Legyen $\rmat A \in \mathcal M_{n \times n}$ szinguláris mátrix. Tegyük fel,
    hogy létezik az inverze. Ekkor igaz, hogy
    $\rmat A \cdot \rmat A^{-1} = \imat$.
    Vigzgáljuk meg a következő egyenlőséget:
    $$
      \underbrace{\det \rmat A}_{=0} \cdot \det \rmat A^{-1}
      = \det \left( \rmat A \cdot \rmat A^{-1} \right)
      = \underbrace{\det \imat}_{=1}
      \text.
    $$
    Látjuk, hogy ezzel ellenmondásra jutunk.
  \end{proof}
\end{statement}

\begin{statement}
  Reguláris mátrix inverze egyértelmű. Ha $\rmat A \in \mathcal M_{n \times n}$,
  akkor
  $$
    \rmat A^{-1} := \frac{\adj \rmat A}{\det \rmat A}
    \text.
  $$

  \begin{proof}
    \vspace{6em}
  \end{proof}
\end{statement}

\begin{blueBox}
  Egy $3 \times 3$-as mátrix adjugáltja:
  $$
    \rmat A = \begin{bmatrix}
      a_{11} & a_{12} & a_{13} \\
      a_{21} & a_{22} & a_{23} \\
      a_{31} & a_{32} & a_{33}
    \end{bmatrix}
    \quad
    \Rightarrow
    \quad
    \adj \rmat A = \begin{bmatrix}
      + \begin{vmatrix}
          a_{22} & a_{23} \\
          a_{32} & a_{33}
        \end{vmatrix}
       &
      - \begin{vmatrix}
          a_{12} & a_{13} \\
          a_{32} & a_{33}
        \end{vmatrix}
       &
      +\begin{vmatrix}
         a_{12} & a_{13} \\
         a_{22} & a_{23}
       \end{vmatrix}
      \\
      - \begin{vmatrix}
          a_{21} & a_{23} \\
          a_{31} & a_{33}
        \end{vmatrix}
       &
      + \begin{vmatrix}
          a_{11} & a_{13} \\
          a_{31} & a_{33}
        \end{vmatrix}
       &
      - \begin{vmatrix}
          a_{11} & a_{13} \\
          a_{21} & a_{23}
        \end{vmatrix}
      \\
      + \begin{vmatrix}
          a_{21} & a_{22} \\
          a_{31} & a_{32}
        \end{vmatrix}
       &
      - \begin{vmatrix}
          a_{11} & a_{12} \\
          a_{31} & a_{32}
        \end{vmatrix}
       &
      +\begin{vmatrix}
         a_{11} & a_{12} \\
         a_{21} & a_{22}
       \end{vmatrix}
    \end{bmatrix}
  $$
\end{blueBox}