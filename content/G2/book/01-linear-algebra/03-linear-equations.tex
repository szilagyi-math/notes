\clearpage
\section{Lineáris egyenletrendszerek}\label{sec-01-03}

\begin{definition}[Lineáris egyenletrendszer]
  Véges sok elsőfokú egyenletet és véges sok ismeretlent tartalmazó
  egyenletrendszert lineáris egyenletrendszernek nevezünk.

  Az $m$ egyenletből és $n$ ismeretlenből álló lineáris egyenletrendszer
  általános alakja:
  $$
    \begin{array}{*{9}{c}}
      a_{11} x_{1} & + & a_{12} x_{2} & + & \dots  & + & a_{1n} x_{n} & = & b_{1}\text, \\[1mm]
      a_{21} x_{1} & + & a_{22} x_{2} & + & \dots  & + & a_{2n} x_{n} & = & b_{2}\text, \\[1mm]
      \vdots       &   & \vdots       &   & \vdots &   & \vdots       &   & \vdots      \\[1mm]
      a_{m1} x_{1} & + & a_{m2} x_{2} & + & \dots  & + & a_{mn} x_{n} & = & b_{m}\text,
    \end{array}
  $$
  ahol $a_{ij}$ együtthatók, $b_{j}$ konstansok, $x_{j}$ ismeretlenek.
\end{definition}

\begin{blueBox}
  \sftitle{Lineáris egyenletrendszer csoportosítása}:

  \begin{itemize}
    \item A lineáris egyenletrendszert \textbf{megoldható}nak nevezzük,
          ha létezik megoldása.

    \item A lineáris egyenletrendszert \textbf{ellentmondó}nak nevezzük,
          ha nincs megoldása.

    \item A lineáris egyenletrendszert \textbf{határozott}nak nevezzük,
          ha csupán egyetlen megoldása van.

    \item A lineáris egyenletrendszert \textbf{határozatlan}nak nevezzük,
          ha végtelen sok megoldása van.
  \end{itemize}
\end{blueBox}

\begin{definition}
  Két lineáris egyenletrendszer ekvivalens, ha a megoldáshalmazuk megegyezik.
\end{definition}

\begin{note}
  Az ekvivalencia szemponjából az egyenletek és az ismeretlenek sorrendje nem
  számít.
\end{note}

\begin{statement}
  Az eredetivel ekvivalens lineáris egyenletrendszert kapunbk, ha az
  egyenletrendszer valamelyik egyenletét egy nemnulla számmal szorozzuk, vagy
  valamelyik egyenlethet a lineáris egyenletrendszer egy másik egyenletét
  hozzáadjuk.

  \begin{proof}
    \vspace{10em}
  \end{proof}
\end{statement}

\clearpage
\begin{blueBox}
  \sftitle{Lineáris egyenletrendszer mátrixos alakja:}

  Egy lineáris egyenletrendszer felírható $\rmat A \rvec x = \rmat b$
  alakban, ahol $\rmat A$ az együttható mátrix, $\rvec x$ az ismeretlenek
  vektora, $\rvec b$ pedig a konstans vektor.
  $$
    \underbrace{\begin{bmatrix}
        a_{11} & a_{12} & \cdots & a_{1n} \\
        a_{21} & a_{22} & \cdots & a_{2n} \\
        \vdots & \vdots & \ddots & \vdots \\
        a_{m1} & a_{m2} & \cdots & a_{mn}
      \end{bmatrix}}_{\rmat A} \underbrace{\begin{bmatrix}
        x_{1} \\ x_{2} \\ \vdots \\ x_{n}
      \end{bmatrix}}_{\rvec x} = \underbrace{\begin{bmatrix}
        b_{1} \\ b_{2} \\ \vdots \\ b_{n}
      \end{bmatrix}}_{\rvec b}
  $$
\end{blueBox}

\begin{theorem}[LER megoldhatóságának szükséges és elégséges feltétele]
  Az $\rmat A \rvec x = \rvec b$ lineáris egyenletrendszer akkor és csak
  akkor oldható meg, ha $\rg(\rmat A) = \rg(\rmat A | \rvec b)$, ahol az
  $(\rmat A | \rvec b)$ mátrixot kibővített mátrixnak nevezzük.

  A feltétel mátrixosan:
  \[
    \rg \begin{bmatrix}
      a_{11} & a_{12} & \cdots & a_{1n} \\
      a_{21} & a_{22} & \cdots & a_{2n} \\
      \vdots & \vdots & \ddots & \vdots \\
      a_{m1} & a_{m2} & \cdots & a_{mn}
    \end{bmatrix} = \rg \left[\begin{array}{cccc|c}
        a_{11} & a_{12} & \cdots & a_{1n} & b_1    \\
        a_{21} & a_{22} & \cdots & a_{2n} & b_2    \\
        \vdots & \vdots & \ddots & \vdots & \vdots \\
        a_{m1} & a_{m2} & \cdots & a_{mn} & b_n
      \end{array}\right]\text.
  \]

  \begin{proof}
    \vspace{8em}
  \end{proof}
\end{theorem}

\begin{definition}[Homogén lineáris egyenletrendszer]
  Az $\rmat A \rvec x = \rvec b$ lineáris egyenletrendszer homogénnek mondjuk,
  ha $\rvec b = \nvec$.

  Ha $\rvec b \neq \nvec$, akkor a lineáris egyenletrendszer inhomogén.
\end{definition}

\begin{note}
  A feltételből következik, hogy homogén lineáris egyenletrendszer
  ($\rvec b = \nvec$) mindig megoldható, hiszen az együttható mátrixból és egy
  nullvektorból képzett kibővített mátrix rangja mindig meg fog egyezni az
  együttható mátrix rangjával.
\end{note}

\begin{note}
  Tekintsük az $n$ egyenletből és $n$ ismeretlenből álló homogén lineáris
  egyenletrendszert. Ekkor ha az $\rmat A$ mátrix reguláris, akkor az
  egyenletrendszernek csak a triviális megoldása létezik. Ha az $\rmat A$
  szinguláris, akkor létezik nemtriviális megoldás is.
\end{note}

\clearpage
\begin{blueBox}
  \sftitle{Megoldási módszerek}:

  \begin{enumerate}
    \item Ha az $\rmat A$ mátrix reguláris, akkor invertálható és az
          $\rvec x = \rmat A^{-1} \rvec b$.

    \item Cramer-szabály: ha az $\rmat A$ mátrix reguláris, akkor az
          együtthatók az alábbi módon számíthatóak:
          $$
            x_i = \frac{\det \rmat A_i}{\det \rmat A}
            \text,
          $$
          ahol az $\rmat A_i$ mátrixot úgy képezzük, hogy az $i$-edik oszlopába
          $\rvec b$ vektort írjuk be.

    \item Gauss-elimináció: sorműveletekkel alakítjuk a kibővített mátrixot:
          $$
            \left[\begin{array}{cccc|c}
                a_{11} & a_{12} & \cdots & a_{1n} & b_1    \\
                a_{21} & a_{22} & \cdots & a_{2n} & b_2    \\
                \vdots & \vdots & \ddots & \vdots & \vdots \\
                a_{m1} & a_{m2} & \cdots & a_{mn} & b_n
              \end{array}\right]
            \quad\sim\quad
            \left[\begin{array}{cccc|c}
                \square & \square & \cdots & \square & \square \\
                0       & \square & \cdots & \square & \square \\
                \vdots  & \vdots  & \ddots & \vdots  & \vdots  \\
                0       & 0       & \cdots & \circ   & \circ
              \end{array}\right]
          $$
  \end{enumerate}
\end{blueBox}