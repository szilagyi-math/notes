\mainChapter{Lineáris algebra}\label{chap-01}

\bgroup
\color{gray!50!black}
\sffamily

A Matematika G2 kurzus első felében lineáris algebrával fogunk foglalkozni.
Ez a matematika azon területe, amely számos tudományágban és gyakorlati
alkalmazásban meghatározó szerepet játszik. Célunk a vektorterekkel, mátrixokkal
és lineáris leképezésekkel kapcsolatos alapvető ismeretek átadása, valamint a
lineáris egyenletrendszerek megoldhatóságának és megoldásának vizsgálata.
A vektorterek és a mátrixok a matematika és a mérnöki tudományok számos
területén alapvető szerepet töltenek be, segítenek leírni, megérteni
bonyolultabb rendszereket és struktúrákat. A lineáris egyenletrendszerek
megoldása szoros kapcsolatban áll a vektorterek és a mátrixok tulajdonságaival.
A mátrixok lehetővé teszik a lineáris transzformációk hatékony leírását.

A jegyzet ezen része a lineáris algebra alapjait igyekszik bemutatni: néhány
korábban tanult definíció felelevenítését követően, a vektorterek definíciója és
azok tulajdonságai, a mátrixműveletek elmélete, majd a lineáris
egyenletrendszerek különböző megoldási módszerei következnek, végül a lineáris
leképezések áttekintésével zárul. A célunk, hogy a témák megértéséhez szükséges
elméleti ismeretek mellett gyakorlati példákon keresztül is bemutassuk a
lineáris algebra széleskörű alkalmazási lehetőségeit.

Ez a jegyzet segít abban, hogy az Olvasó képet kapjon a lineáris algebra
fontosságáról és alkalmazásairól.

\chaptertoc
\egroup

\clearpage