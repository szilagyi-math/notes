\clearpage

\section{Alapfogalmak}\label{sec-01-01}

\begin{definition}[Csoport]
  Legyen G nemüres halmaz, és $\circ$ egy művelet. Ekkor a $(G; \circ)$ csoport,
  ha teljesülnek az alábbiak:
  \begin{enumerate}
    \item $\forall a; b; c \in G: (a \circ b) \circ c = a \circ (b \circ c)$,
          \hfill (\textbf{asszociativitás})

    \item $\exists e \in G: \forall a \in G: e \circ a = a \circ e = a$,
          \hfill(\textbf{egységelem})

    \item $\forall a \in G: \exists a^{-1} \in G:
            a \circ a^{-1} = a^{-1} \circ a = e$.
          \hfill (\textbf{inverzelem})
  \end{enumerate}
\end{definition}

\begin{note}
  Ha a $\circ$ művelet kommutatív, azaz $\forall a, b \in G: a \circ b = b \circ
    a$, akkor a csoportot \textbf{Abel-csoport}nak nevezzük.
\end{note}

\begin{example}
  A $(\mathbb R; \cdot)$, $(\mathbb Q; +)$, $\mathbb C; +$ mindegyike
  Abel-csoport.

  Nem csoport $(\mathbb N; +)$, hiszen nincs inverz elem.

  $(\mathbb Q^*; +)$ sem csoport, mert nem létezik egységelem.
\end{example}

\begin{definition}[Gyűrű]
  Legyen $R$ nemüres halmaz, és $\circ, +$ két művelet. Ekkor a $(R; +, \circ)$
  gyűrű, ha teljesülnek az alábbiak:
  \begin{enumerate}
    \item $(R; +)$ \textbf{Abel-csoport},

    \item $\forall a; b; c \in R: (a \circ b) \circ c = a \circ (b \circ c)$
          \hfill (\textbf{asszociativitás})

    \item teljesül a disztributivitás:
          \begin{itemize}
            \item $\forall a; b; c \in R:
                    a \circ (b + c) = a \circ b + a \circ c$,
                  \hfill (\textbf{$+$ disztributív $\circ$-ra})

            \item $\forall a; b; c \in R:
                    (a + b) \circ c = a \circ c + b \circ c$.
                  \hfill (\textbf{$\circ$ disztributív $+$-ra})
          \end{itemize}
  \end{enumerate}
\end{definition}

\begin{definition}[Test]
  Legyen $T$ nemüres halmaz, és $\circ, +$ két művelet. Ekkor a $(T; +, \circ)$
  test, ha teljesülnek az alábbiak:
  \begin{enumerate}
    \item $(T; +)$ \textbf{Abel-csoport},

    \item $\forall a; b; c \in T: (a \circ b) \circ c = a \circ (b \circ c)$,
          \hfill (\textbf{asszociativitás})

    \item $\exists e \in T: \forall a \in F: e \circ a = a \circ e = a$,
          \hfill (\textbf{egységelem})

    \item $\forall a \in T \exists a^{-1} \in T:
            a \circ a^{-1} = a^{-1} \circ a = e$,
          \hfill (\textbf{inverzelem})

    \item teljesül a \textbf{disztributivitás}.
          % \begin{itemize}
          %   \item $\forall a; b; c \in F: a \circ (b + c) = a \circ b + a \circ c$,
          %         \hfill ($+$ disztributív $\circ$-ra)

          %   \item $\forall a; b; c \in F: (a + b) \circ c = a \circ c + b \circ c$.
          %         \hfill ($\circ$ disztributív $+$-ra)
          % \end{itemize}
  \end{enumerate}
\end{definition}

\begin{example}
  A $(\mathbb R; +, \cdot)$, $(\mathbb Q; +, \cdot)$, $(\mathbb C; +, \cdot)$
  mindegyike test.
\end{example}

\begin{definition}[Vektortér]
  Legyen $V$ nemüres halmaz, és $\circ, +$ két művelet, $T$ test.
  A $(V; +, \circ)$ a $T$ test feletti vektortér, ha teljesülnek az alábbiak:
  \begin{enumerate}
    \item $(V; +)$ Abel-csoport,

    \item $\forall \lambda; \mu \in T \; \land \; \forall \rvec x \in V:
            (\lambda \circ \mu) \circ \rvec x
            = \lambda \circ (\mu \circ \rvec x)$,

    \item ha $\varepsilon$ a $T$-beli egységelem, akkor
          $\forall \rvec x \in V: \varepsilon \circ \rvec x = \rvec x$,

    \item teljesül a disztributivitás:
          \begin{itemize}
            \item $\forall \lambda; \mu \in T \; \land \; \forall \rvec x \in V:
                    \lambda \circ (\rvec x + \rvec y)
                    = \lambda \circ \rvec x + \lambda \circ \rvec y$,

            \item $\forall \lambda; \mu \in T \; \land \; \forall \rvec x \in V:
                    (\lambda + \mu) \circ \rvec x
                    = \lambda \circ \rvec x + \mu \circ \rvec x$.
          \end{itemize}
  \end{enumerate}
\end{definition}

\begin{example}
  A legfeljebb $n$-edfokú polinomok a skalárral való szorzásra és az összeadásra
  vektorteret alkotnak.

  A függvények az összeadásra és a skalárral való szorzásra vektorteret alkotnak.
\end{example}

\begin{definition}[Vektor]
  A vektortér elemeit vektoroknak nevezzük.
  Jelölés: $\rvec x$, vagy $\underbar x$.
\end{definition}

\begin{statement}
  \textbf{A zéruselem és ellentett elem létezése egyértelmű.}

  \begin{proof}
    \begin{enumerate}
      \item {\sffamily A zéruselem létezése egyértelmű}

            Tegyük fel, hogy $\nvec$ és $\hat \nvec$ különböző
            zéruselemek, vagyis $\nvec \neq \hat \nvec$. Ebben az esetben
            $$
              \nvec = \nvec + \hat \nvec = \hat \nvec
              \text.
            $$
            Ez ellentmondás, tehát a zéruselem egyértelmű.

            \bigskip

      \item {\sffamily Az ellentett elem létezése egyértelmű}

            Tegyük fel, hogy $-\rvec v$ és $-\hat{\rvec v}$ egyaránt
            $\rvec v$ ellentettjei, valamint $-\rvec v \neq -\hat{\rvec v}$.
            Ebben az esetben
            $$
              -\hat{\rvec v}
              = (-\rvec v + \rvec v) + (-\hat{\rvec v})
              = (-\rvec v) + (\rvec v + (-\hat{\rvec v}))
              = -\rvec v
              \text.
            $$
            Ez ellentmondás, tehát az ellentett elem egyértelmű.
    \end{enumerate}
  \end{proof}
\end{statement}

\begin{statement}
  0-val való szorzás:
  $\forall \rvec v \in V: 0 \cdot \rvec v = \nvec$.

  \begin{proof}
    \vspace{3em}
  \end{proof}
\end{statement}

\begin{statement}
  Nullvektorral való szorzás:
  $\forall \lambda \in T: \lambda \cdot \nvec = \nvec$.

  \begin{proof}
    \vspace{3em}
  \end{proof}
\end{statement}

\begin{statement}
  $\lambda \cdot \rvec v = \nvec \quad \Longleftrightarrow \quad
    \lambda = 0 \; \lor \; \rvec v = \nvec$

  \begin{proof}
    \vspace{3em}
  \end{proof}
\end{statement}

\begin{definition}[Lineáris függetlenség]

  A $(V; +; \lambda)$ vektortér $\rvec v_1, \rvec v_2, \ldots, \rvec v_n$
  vekrorait lineárisan függetlennek mondjuk, ha a
  $$
    \lambda_1 \rvec v_1
    + \lambda_2 \rvec v_2
    + \ldots
    + \lambda_n \rvec v_n
    = \nvec
  $$
  vektoregyenletnek \textbf{csak a triviális megoldása} létezik, azaz
  $\lambda_1 = \lambda_2 = \ldots = \lambda_n = 0$.

  Ha az egyenletnek nem csak a triviális megoldása létezik, akkor a vektorok
  lineárisan függők.
\end{definition}

\begin{definition}[Altér]
  Legyen $V; +; \lambda$ $\mathbb R$ feletti vektortér, valamint
  $\emptyset \neq L \subset V$. $L$-t altérnek nevezzük a $V$-ben, ha
  $(L; +; \lambda)$ ugyancsak vektortér.
\end{definition}

\begin{example}
  A polinomok vektorterének alterte a legfeljebb $n$-edfokú polinomok
  vektortere.
\end{example}

\begin{statement}
  Alterek metszete ugyancsak altér. Alterek uniója azonban általában nem altér.
\end{statement}

\begin{definition}[Generátorrendszer]
  Legyen $V$ vektortér, valamint $\emptyset \neq G \subset V$. $G$ által
  generált altérnek nevezzük azt a legszűkebb alteret, amely tartalmazza $G$-t.
  Jele: $\mathcal L(G)$.

  $G$ generátorrendszere $V$-nek, ha $\mathcal L(G) = V$.
\end{definition}

\begin{note}
  Ha $G$ véges generátorrendszere $V$-nek, akkor $G$-t végesen generált
  vektorrendszernek nevezzük.
\end{note}

\begin{definition}[Bázis]
  A $V$ vektorrendszer egy lineárisan független generátorrendszerét a $V$
  bázisának nevezzük.
\end{definition}

\begin{statement}
  Végesen generált vektortérben bármely két bázis azonos tagszámú.
\end{statement}

\begin{definition}[Vektortér dimenziója]
  Végesen generált vektortér dimenzióján a bázisainak közös tagszámát értjük.
\end{definition}

\begin{statement}
  Legyen $\{ \rvec b_1; \rvec b_2; \dots; \rvec b_n \}$ a $V$ vektortér egy
  bázisa. Ekkor tetszőleges $V$-beli vektor egyértelműen eéőállítható a
  bázisvektorok lineáris kombinációjaként.

  Azaz $\forall \rvec v \in V: \exists! (\lambda_1; \lambda_2; \dots; \lambda_n)$,
  hogy
  $$
    \rvec v
    = \lambda_1 \rvec b_1
    + \lambda_2 \rvec b_2
    + \ldots
    + \lambda_n \rvec b_n
    \text.
  $$
  A ($\lambda_1; \lambda_2; \dots; \lambda_n$) szám $n$-est az $\rvec v$ vektor
  $\{ \rvec b_1; \rvec b_2; \dots; \rvec b_n \}$ bázisaira vonatkozó
  koordinátáinak nevezzük.

  %   Bizonyítás: ( egzisztencia )
  % {b1,b2...bn}lineárisan függetlenek, mert bázis. Ezért {v,b1,b2...bn}már lineárisan
  % függő, így a
  % λv + α1b1 + α2b2 +···+ αnbn = 0
  % vektoregyenletnekléteziktriviálistólkülönbözőmegoldása,azaznemlehet(λ,α1,α2,...,αn)
  % minden eleme egyszerre 0.
  % Tehát λ̸= 0, mert ellenkező esetben α1 = α2 = ···= αn = 0 állna fent, így oszthatjuk
  % az egyenletet λ-val:
  % v =−
  % α1
  % λ
  % :=ξ1
  % b1 +−
  % α2
  % λ
  % :=ξ2
  % b2 +···+−
  % αn
  % λ
  % :=ξn
  % bn.
  % Bizonyítás: ( unicitás )
  % Tegyük fel hogy (ξ1,ξ2 ...ξn) és (η1,η2 ...ηn) egyaránt v koordinátái a {b1,b2...bn}
  % bázisban, azaz
  % n
  % i=1
  % n
  % i=1
  % ξibi
  % ηibi
  % v =
  % v =
  % Vonjuk ki egymásból a két egyenletet:
  % 0 = (ξ1−η1)
  % b1 + (ξ2−η2)
  % b2 +···+ (ξn−ηn)
  % bn.
  % 0
  % 0
  % 0
  % Ezzel ellentmondásra jutunk, mivel {b1,b2...bn}bázis, ezért a nullvektornak csak trivi-
  % ális előállítása létezik, ami az együtthatók 0 voltát vonná maga után, az pedig a megfelelő
  % koordináták egyenlőségével ekvivalens. Tehát nem igaz a feltevés.
  \begin{proof}[Egzisztencia]
    $\{ \rvec b_1; \rvec b_2; \dots; \rvec b_n \}$ lineárisan
    függetlenek, mert bázis. Ezért $\{ \rvec v, \rvec b_1; \rvec b_2;
      \ldots; \rvec b_n \}$ már lineárisan függő, így a
    $
      \mu \rvec v + \xi_1 \rvec b_1 + \xi_2 \rvec b_2 + \ldots
      + \xi_n \rvec b_n = \nvec
    $
    vektoregyenletnek létezik triviálistól különböző megoldása, azaz nem
    lehet $(\mu; \xi_1; \xi_2; \ldots; \xi_n)$ minden eleme
    egyszerre 0.

    Tehát $\mu \neq 0$, mert ellenkező esetben $\xi_1 = \xi_2
      = \ldots = \xi_n = 0$ állna fent, így oszthatjuk az egyenletet
    $\mu$-vel:
    $$
      \rvec v
      = \underbrace{\left(-\frac{\xi_1}{\lambda}\right)}_{:= \lambda_1} \rvec b_1
      + \underbrace{\left(-\frac{\xi_2}{\lambda}\right)}_{:= \lambda_2} \rvec b_2
      + \dots
      + \underbrace{\left(-\frac{\xi_n}{\lambda}\right)}_{:= \lambda_n} \rvec b_n
      \text.
    $$
  \end{proof}

  \begin{proof}[Unicitás]
    Tegyük fel, hogy a $(\lambda_1; \lambda_2; \ldots; \lambda_n)$ és a
    $(\mu_1; \mu_2; \ldots; \mu_n)$ is a $\rvec v$
    koordinátái a $\{ \rvec b_1; \rvec b_2; \ldots; \rvec b_n \}$
    bázisban, azaz
    $$
      \rvec v = \sum_{i=1}^n \lambda_i \rvec b_i
      \text{ és }
      \rvec v = \sum_{i=1}^n \mu_i \rvec b_i
      \text.
    $$
    Vonjuk ki egymásból a két egyenletet:
    $$
      \nvec
      = \underbrace{(\lambda_1 - \mu_1)}_{0} \rvec b_1
      + \underbrace{(\lambda_2 - \mu_2)}_{0} \rvec b_2
      + \ldots
      + \underbrace{(\lambda_n - \mu_n)}_{0} \rvec b_n
      \text.
    $$
    Ezzel ellentmondásra jutunk, mivel $\{ \rvec b_1; \rvec b_2;
      \ldots; \rvec b_n \}$ bázis, ezért a nullvektornak csak triviális
    előállítása létezik, ami az együtthatók 0 voltát vonná maga után,
    az pedig a megfelelő koordináták egyenlőségével ekvivalens. A
    feltevés tehát hamis.
  \end{proof}
\end{statement}