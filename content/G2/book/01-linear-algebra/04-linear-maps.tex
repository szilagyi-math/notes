\clearpage
\section{Lineáris leképezések}\label{sec-01-04}

\begin{definition}[Lineáris leképezés]
  Legyenek $V_1$ és $V_2$ ugyanazon $T$ test feletti vektorterek. Legyen
  $\varphi: V_1 \rightarrow V_2$ leképezés, melyet lineáris leképezésnek
  nevezünk, ha tetszőleges két $V_1$-beli vektor ($\forall \rvec a; \rvec b \in
    V_1$) és $T$-beli skalár ($\lambda \in T$) esetén teljesülnek az alábbiak:

  \def\arraystretch{1.5}
  \begin{tabular}{>{\bullet\;}l>{$\quad \sim \quad$}l}
    $\varphi(\rvec a + \rvec b) = \varphi(\rvec a) + \varphi(\rvec b)$
     & additív (összegre tagonként hat), \\
    $\varphi(\lambda \rvec a) = \lambda \varphi(\rvec a)$
     & homogén (skalár kiemelhető).
  \end{tabular}
\end{definition}

\begin{note}
  $\varphi(\nvec) = \nvec$ minden lineáris leképezés esetén.

  A linearitás miatt $\varphi(-\rvec a) = - \varphi(\rvec a)$.
\end{note}

\begin{definition}[Homomorfizmus]
  $$
    \operatorname{Hom} (V_1; V_2) := \big\{\;
    \varphi: V_1 \to V_2 \; \big| \; \varphi \text{ lineáris}
    \;\big\}
  $$
\end{definition}

\begin{definition}[Endomorfizmus]
  $$
    \operatorname{End} (V) := \operatorname{Hom} (V; V)
  $$
\end{definition}

\begin{statement}
  $(\operatorname{Hom} (V_1; V_2), +, \lambda)$  vektortér $\Reals$ vagy
  $\mathbb C$ felett.
\end{statement}

\begin{definition}[Izomorfizmus]
  A $\varphi: V_1 \to V_2$ leképezés izomorfizmus, ha lineáris és bijektív.
\end{definition}

\begin{statement}
  Véges dimenziójú vektorterek esetén az egymással izomorf vektorterek dimenziója
  azonos.
\end{statement}

\begin{statement}
  Legyenek $V_1$ és $V_2$ ugyanazon test feletti vektorterek, és
  $\dim V_1 = \dim V_2$. Ekkor $V_1 \simeq V_2$

  \begin{proof}
    \vspace{7em}
  \end{proof}
\end{statement}

\begin{theorem}[Lineáris leképezések alaptétele]
  Legyenek $V_1$ és $V_2$ ugyanazon test feletti vektorterek, és legyen
  $\{ \rvec b_1; \rvec b_2; \ldots; \rvec b_n \}$ bázis $V_1$-ben, és
  $\{ \rvec a_1; \rvec a_2; \ldots; \rvec a_n \}$ tetszőleges vektorrendszer
  $V_2$-ben. Ekkor egyetlen lineáris leképezés létezik, melyre
  $\varphi(\rvec b_i) = \rvec a_i$, ahol $i \in \{1, 2, \ldots, n\}$.

  \begin{proof}[Unicitás, indirekt módon]
    Legyenek $\varphi \neq \psi$ lineáris leképezések, melyekre
    $\varphi(\rvec b_i) = \rvec a_i$ és $\psi(\rvec b_i) = \rvec a_i$,
    $i \in \{1, 2, \ldots, n\}$. Legyen $\rvec x \in V_1$ tetszőleges,
    $\rvec x = \sum_{i=1}^{n} \xi_i \rvec b_i$. Hattassuk $\varphi$-t
    $\rvec x$-re:
    \begin{align*}
      \varphi(\rvec x)
       & = \varphi\left(
      \sum_{i = 1}^{n} \xi_i \rvec b_i
      \right)
      = \sum_{i = 1}^{n} \xi_i \varphi(\rvec b_i)
      = \xi_1 \varphi(\rvec b_1) + \ldots + \xi_n \varphi(\rvec b_n) \\
      =
      \\
       & = \xi_1 \rvec a_1 + \ldots + \xi_n \rvec a_n
      = \xi_1 \psi(\rvec b_1) + \ldots + \xi_n \psi(\rvec b_n)
      = \psi\left(
      \sum_{i = 1}^{n} \xi_i \rvec b_i
      \right)
      = \psi(\rvec x)
    \end{align*}
    Ezzel $\varphi = \psi$, ami ellentmond a feltételnek, tehát a feltevés nem
    igaz.
  \end{proof}

  \begin{proof}[Egzisztencia, konstruktív bizonyítás]
    Legyen $\varphi: V_1 \rightarrow V_2$ leképezés, és
    $\rvec x \mapsto \varphi(\rvec x)$. Ha $(\xi_1; \xi_2; \ldots; \xi_n)$
    az $\rvec x$ koordinátái, akkor $\varphi(\rvec x) = \xi_1 \rvec a_1 +
      \ldots + \xi_n \rvec a_n$. Hasonló módon legyen $(\eta_1; \eta_2; \ldots;
      \eta_n)$ az $\rvec y$ koordinátái. Ekkor:
    \begin{align*}
      \varphi(\rvec x + \rvec y)
       & = (\xi_1 + \eta_1) \rvec a_1 + \ldots + (\xi_n + \eta_n) \rvec a_n
      \\
       & = \xi_1 \rvec a_1 + \ldots + \xi_n \rvec a_n
      + \eta_1 \rvec a_1 + \ldots + \eta_n \rvec a_n
      = \varphi(\rvec x) + \varphi(\rvec y)
      \text,
      \\[3mm]
      \varphi(\lambda \rvec x)
       & = \lambda \xi_1 \rvec a_1 + \ldots + \lambda \xi_n \rvec a_n
      = \lambda (\xi_1 \rvec a_1 + \ldots + \xi_n \rvec a_n)
      = \lambda \varphi(\rvec x)
      \text.
    \end{align*}
  \end{proof}
\end{theorem}

\begin{blueBox}
  \sftitle{Lineáris leképezések mátrixreprezentációja:}

  Legyenek $V_1$ és $V_2$ ugyanazon test feletti vektorterek, és $\dim V_1 = n$,
  valamint $\dim V_2 = k$. Legyen $\{ \rvec a_1; \rvec a_2; \ldots; \rvec a_n \}$
  bázis $V_1$-ben, és $\{ \rvec b_1; \rvec b_2; \ldots; \rvec b_n \}$ bázis
  $V_2$-ben. Legyen $\varphi: V_1 \rightarrow V_2$ lineáris leképezés, ekkor
  $$
    \varphi(\rvec a_i)
    = \alpha_{1i} \rvec b_1
    + \alpha_{2i} \rvec b_2
    + \ldots
    + \alpha_{ki} \rvec b_k
    = \sum_{j=1}^{k} \alpha_{ji} \rvec b_j
    \quad\Rightarrow\quad
    \rmat A := \begin{bmatrix}
      \alpha_{11} & \alpha_{2i} & \cdots & \alpha_{1n} \\
      \alpha_{21} & \alpha_{2i} & \cdots & \alpha_{2n} \\
      \vdots      & \vdots      & \ddots & \vdots      \\
      \alpha_{k1} & \alpha_{ki} & \cdots & \alpha_{kn}
    \end{bmatrix}_{k \times n}
    \text.
  $$
  Az $\rmat A$ mátrixot $\varphi$ leképezést reprezentáló mátrixnak hívjuk,
  segítségével tetszőleges $\rvec x \in V_1$ képét meghatározhatjuk. Legyenek
  $(\xi_1, \xi_2, \ldots, \xi_n)$ az $\rvec x$ koordinátái, ekkor a képét az
  alábbi módon számíthatjuk:
  $$
    \varphi(\rvec x)
    = \varphi\left( \sum_{i=1}^{n} \xi_i \rvec a_i \right)
    = \sum_{i=1}^{n} \xi_i \varphi(\rvec a_i)
    = \begin{bmatrix}
      \alpha_{11} & \alpha_{2i} & \cdots & \alpha_{1n} \\
      \alpha_{21} & \alpha_{2i} & \cdots & \alpha_{2n} \\
      \vdots      & \vdots      & \ddots & \vdots      \\
      \alpha_{k1} & \alpha_{ki} & \cdots & \alpha_{kn}
    \end{bmatrix} \begin{bmatrix}
      \xi_1  \\
      \xi_2  \\
      \vdots \\
      \xi_n
    \end{bmatrix}
    \text.
  $$
\end{blueBox}

\begin{statement}
  $\varphi: \operatorname{Hom} (V_1; V_2) \rightarrow \mathcal M_{k \times n}$
  izomorfizmus, ahol $\dim V_1 = k$ és $\dim V_2 = n$.

  {\sffamily\bfseries Következmény:}
  $\dim(\operatorname{Hom} (V_1; V_2)) = n \cdot k = \dim V_1 \cdot \dim V_2$.
\end{statement}

\begin{statement}
  Legyenek $V_1$, $V_2$ és $V_3$ vektorterek, $\dim V_1 = k$, $\dim V_2 = m$ és
  $\dim V_3 = n$. Legyenek $\varphi: V_1 \rightarrow V_2$ és $\psi: V_2
    \rightarrow V_3$ lineáris leképezések, ekkor a $V_1$-ből $V_3$-ra való
  leképezés ($\psi \circ \varphi: V_1 \rightarrow V_3$) olyan, hogy ha
  $\varphi \leftrightarrow \rmat A \in \mathcal M_{m \times k}$ és
  $\psi \leftrightarrow \rmat B \in \mathcal M_{n \times m}$, akkor
  $\psi \circ \varphi \leftrightarrow \rmat C \in \mathcal M_{n \times k}$, ahol
  $\rmat C = \rmat B \cdot \rmat A$.

  Speciálisan, ha $V_1 = V_2 = V_3 = V$, $\dim V = n$, akkor
  $\rmat A; \rmat B; \rmat C \in \mathcal M_{n \times n}$.

  {\sffamily\bfseries Következmény:}
  Invertálható lineáris leképezés mátrixa invertálható.
\end{statement}

\begin{definition}[Leképezés magtere]
  Legyen $\varphi: V_1 \rightarrow V_2$ lineáris leképezés, ekkor a
  $$
    \ker \varphi := \{
    \rvec v \; | \; \rvec v \in V_1 \land \varphi(\rvec v) = \nvec
    \}
  $$
  halmazt a leképezés magterének nevezzük.
\end{definition}

\begin{statement}
  $\ker \varphi$ altér $V_1$-ben.

  \begin{proof}
    \vspace{6em}
  \end{proof}
\end{statement}

\begin{definition}[Leképezés defektusa]
  A magtér dimenzióját defektusnak nevezzük, és $\operatorname{def} \varphi$-vel
  jelöljük.
\end{definition}

\begin{note}
  \begin{itemize}
    \item Nem létezik olyan vektortér, melynek magtere az üreshalmaz (a
          nullvektor mindig benne van, mert a nullvektor képe mindig nullvektor).

    \item Invertálható lineáris leképezés magtere a nullvektor.
  \end{itemize}
\end{note}

\begin{statement}
  A $\varphi$ leképezés injektív, akkor és csak akkor, ha
  $\ker \varphi = \{ \nvec \} \Leftrightarrow \operatorname{def} \varphi = 0$.

  \begin{proof}
    \vspace{6em}
  \end{proof}
\end{statement}

\begin{definition}[Lineáris leképezés rangja]
  Egy lineáris leképezés rangjának nevezzük a képtér dimenzióját.
  $\rg \varphi = \dim \varphi(V_1)$.
\end{definition}

\begin{theorem}[Rang-nullitás tétele]
  Legyen $V_1$ véges dimenziós vektortér, $\varphi: V_1 \rightarrow V_2$
  lineáris leképezés, ekkor
  $$
    \rg \varphi + \operatorname{def} \varphi = \dim V_1
    \text.
  $$

  \begin{proof}
    \vspace{6em}
  \end{proof}
\end{theorem}

\begin{statement}
  Tetszőleges lineáris leképezés rangja megegyezik bármely bázisra vonatkozó
  mátrixreprezentációjának rangjával. $\varphi: V_1 \rightarrow V_2$, $\dim V_1
    = m$, $\dim V_2 = n \Rightarrow \varphi \leftrightarrow \rmat A$, $\rmat A
    \in \mathcal M_{n \times m}$, $\rg \varphi = \rg \rmat A$.

  \begin{proof}
    \vspace{6em}
  \end{proof}
\end{statement}

\begin{theorem}[Másik bázisra való áttérés mátrixa]
  Legyen $\varphi: V \rightarrow V$ lineáris leképezés,
  $\{ \rvec b_1; \rvec b_2; \ldots; \rvec b_n \}$ és $\{ \hat{\rvec b}_1;
    \hat{\rvec b}_2; \ldots; \hat{\rvec b}_n \}$ bázisok $V$-ben. A
  $\varphi \{ \rvec b_1; \rvec b_2; \ldots \rvec b_n \}$ bázisra vonatkozó
  mátrixa $\rmat A$, a $\varphi \{ \hat{\rvec b}_1; \hat{\rvec b}_2; \ldots;
    \hat{\rvec b}_n \}$ bázisra vonatkozó mátrixa $\hat{\rmat A}$. Jelölje
  $\rmat S$ a $\{ \rvec b_1; \rvec b_2; \ldots; \rvec b_n \}$ bázisról a
  $\{ \hat{\rvec b}_1; \hat{\rvec b}_2; \ldots; \hat{\rvec b}_n \}$ bázisra
  való áttérés mátrixát, ekkor
  $$
    \hat{\rmat A} = \rmat S^{-1} \rmat A \rmat S
    \text.
  $$
  \begin{proof}
    \vspace{10em}
  \end{proof}
\end{theorem}

\begin{note}
  \begin{itemize}
    \item  A fenti tételben szereplő $\rmat A$ és $\hat{\rmat A}$ mátrixok
          hasonlóak.

    \item Hasonló mátrixok determinánsa megegyezik.

    \item Hasonló mátrixok rangja egyenlő.
  \end{itemize}
\end{note}

\begin{definition}[Sajátértékek és sajátvektorok]
  Legyen $V$ a $T$ test feletti vektortér, $\rvec v \in V$, $\rvec v \neq
    \nvec$. $\rvec v$-t a $\varphi: V \rightarrow V$ lineáris leképezés
  sajátvektorának mondjuk, ha önmaga skalárszorosába megy át a leképezés
  során, azaz $\varphi(\rvec v) = \lambda \rvec v$,  $\lambda \in T$.
  $\lambda$-t a $\rvec v$ sajátvektorhoz tartozó sajátértéknek mondjuk.
\end{definition}

\begin{note}
  Ha a $\rvec v$ sajátvektora a $\varphi$-nek, akkor annak skalárszorosa is.
\end{note}

\begin{theorem}[Sajátértékek számítása]
  Az $\rmat A \in \mathcal M_{n \times n}$ mátrix sajátértékeit a
  $$
    \det(\rmat A - \lambda \imat) = 0
  $$
  karakterisztikus egyenlet gyökei.

  \begin{proof}
    Legyen $\rvec v$ az $\rmat A$ sajátvektora. Ekkor teljesül az
    $\rmat A \rvec v = \lambda \rvec v$ egyenlet. Ezt átalakítva:
    $$
      \rmat A \rvec v - \lambda \rvec v = \nvec
      \quad \Rightarrow \quad
      \rmat A \rvec v - \lambda \imat \rvec v = \nvec
      \quad \Rightarrow \quad
      (\rmat A - \lambda \imat) \rvec v = \nvec
      \text.
    $$
    Így egy olyan homogén lineáris egyenletrendszert kapunk, amelynek létezik
    a triviálistól eltérő ($\rvec v \neq \nvec$) megoldása, tehát
    $\det(\rmat A - \lambda \imat) = 0$.
  \end{proof}
\end{theorem}

\begin{note}
  A $\det(\rmat A - \lambda \imat) = 0$ egyenletet \textbf{karakterisztikus
    egyenlet}nek nevezzük.

  A $\det(\rmat A - \lambda \imat)$ polinomot \textbf{karakterisztikus
    polinom}nak nevezzük.
\end{note}

\begin{statement}
  Különböző sajátértékekhez tartozó sajátvektorok lineárisan függetlenek.

  \begin{proof}
    \vspace{10em}
  \end{proof}
\end{statement}

\begin{statement}
  Szimmetrikus mátrix sajátértékei valósak.

  \begin{proof}
    \vspace{10em}
  \end{proof}
\end{statement}

\begin{statement}
  Az $n$-edrendű szimmetrikus mátrixnak van $n$ darab, páronként egymásra
  merőleges sajátvektora.

  \begin{proof}
    \vspace{10em}
  \end{proof}
\end{statement}

\begin{example}
  Határozzuk meg az $\rmat A$ mátrix sajátértékeit és sajátvektorait!

  $$
    \rmat A = \begin{bmatrix}
      2  & -1 \\
      -1 & 2
    \end{bmatrix}
    \qquad
    \rmat A - \lambda \imat = \begin{bmatrix}
      2  & -1 \\
      -1 & 2
    \end{bmatrix} - \lambda \begin{bmatrix}
      1 & 0 \\
      0 & 1
    \end{bmatrix} = \begin{bmatrix}
      2 - \lambda & -1          \\
      -1          & 2 - \lambda
    \end{bmatrix}
  $$

  A karakterisztikus egyenlet, és ennek alapján a sajátértékek:
  $$
    \det(\rmat A - \lambda \imat) = (2 - \lambda)^2 - 1 = 0
    \quad \Rightarrow \quad
    \lambda_1 = 1 \text, \quad \lambda_2 = 3 \text.
  $$

  A sajátvektorokat az $(\rmat A - \lambda_i \imat) \rvec v_i = 0$
  egyenlet segítségével számíthatjuk ki:

  \begin{enumerate}
    \item A $\lambda_1 = 1$ sajátértékhez tartozó sajátvektor:
          $$
            \begin{bmatrix}
              1  & -1 \\
              -1 & 1
            \end{bmatrix} \begin{bmatrix}
              x \\
              y
            \end{bmatrix} = \begin{bmatrix}
              0 \\
              0
            \end{bmatrix}
            \quad \Rightarrow \quad
            x = y
            \quad \Rightarrow \quad
            \rvec v_1 = t_1 \begin{bmatrix}
              1 \\
              1
            \end{bmatrix}
          $$

    \item A $\lambda_2 = 3$ sajátértékhez tartozó sajátvektor:
          $$
            \begin{bmatrix}
              -1 & -1 \\
              -1 & -1
            \end{bmatrix} \begin{bmatrix}
              x \\
              y
            \end{bmatrix} = \begin{bmatrix}
              0 \\
              0
            \end{bmatrix}
            \quad \Rightarrow \quad
            x = -y
            \quad \Rightarrow \quad
            \rvec v_2 = t_2 \begin{bmatrix}
              1 \\
              -1
            \end{bmatrix}
          $$
  \end{enumerate}
\end{example}

\begin{definition}[Skaláris szorzat]
  Legyen $V$ egy $\Reals$ feletti vektortér, és $\langle \phantom{x} \rangle: V
    \times V \rightarrow \Reals$ függvény, melyet skaláris szorzatnak nevezünk,
  ha teljesülnek az alábbiak:
  \begin{enumerate}
    \item $\langle \rvec x; \rvec y \rangle = \langle \rvec y;
            \rvec x \rangle$ minden $\rvec x; \rvec y \in V$ esetén,
          \hfill (\textbf{szimmetrikus})

    \item $\langle \lambda \rvec x; \rvec y \rangle = \lambda \langle \rvec x;
            \rvec y \rangle$ minden $\rvec x; \rvec y \in V$ és $\lambda \in
            \Reals$ esetén,
          \hfill (\textbf{homogén})

    \item $\langle \rvec x_1 + \rvec x_2; \rvec y \rangle = \langle \rvec x_1;
            \rvec y \rangle + \langle \rvec x_2; \rvec y \rangle$ minden $\rvec
            x_1; \rvec x_2; \rvec y \in V$ esetén,
          \hfill (\textbf{additív})

    \item $\langle \rvec x; \rvec x \rangle \geq 0$,
          egyenlőség akkor és csak akkor, ha $\rvec x = \nvec$.
          \hfill (\textbf{nemnegatív})
  \end{enumerate}
\end{definition}

\begin{definition}[Euklideszi tér]
  Legyen $\{ \rvec e_1; \rvec e_2; \ldots; \rvec e_n \}$ kanonikus bázis,
  melyben
  $$
    \rvec x = \begin{bmatrix}
      \xi_1  \\
      \xi_2  \\
      \vdots \\
      \xi_n
    \end{bmatrix}
    \quad\text{és}\quad
    \rvec y = \begin{bmatrix}
      \eta_1 \\
      \eta_2 \\
      \vdots \\
      \eta_n
    \end{bmatrix}
    \text,\quad\text{ekkor}\quad
    \langle \rvec x; \rvec y \rangle := \sum_{i=1}^{n} \xi_i \eta_i
    \text.
  $$
  Az így előállított $(V, \langle \phantom{x} \rangle)$ valós euklideszi tér.

  Jelölése: $\mathbb E^n$: $n$ dimenziós euklideszi tér.

  A valós euklideszi térben értelmezhetjük a vektorok hosszát:
  $\|\rvec x\| = \sqrt{\langle \rvec x; \rvec x \rangle}$.

  \vspace{.5em}
  Valamint értelmezhetjük $\rvec x$ és $\rvec y$ vektorok szögét:
  $\displaystyle
    \cos \angle(\rvec x; \rvec y)
    = \frac{\langle \rvec x; \rvec y \rangle}{\|\rvec x\| \|\rvec y\|}
    \in [-1;1]
  $.
\end{definition}

\begin{note}
  Valós euklideszi térekben érvényesek a
  Cauchy-Bunyakovszkij-Schwartz-egyen\-lőt\-len\-ség:
  $$
    \langle \rvec x; \rvec y \rangle^2
    \leq \|\rvec x\|^2 \cdot \|\rvec y\|^2
    = \langle \rvec x; \rvec x \rangle \cdot \langle \rvec y; \rvec y \rangle
    \text.
  $$

  Ebből következik, hogy a háromszög egyenlőtlenség is teljesül:
  $$
    \|\rvec x + \rvec y\| \leq \|\rvec x\| + \|\rvec y\|
    \text.
  $$
\end{note}

\begin{definition}[Ortonormált bázis]
  A $(V, \langle \phantom{x} \rangle)$ $n$ dimenziós euklideszi tér
  $\{ \rvec e_1; \rvec e_2; \ldots; \rvec e_n \}$ bázisát
  ortonormáltnak mondjuk, ha $\langle \rvec e_i; \rvec e_j \rangle =
    \delta_{ij}$, ahol
  $$
    \delta_{ij} = \begin{cases}
      1, & \text{ha } i = j    \text, \\
      0, & \text{ha } i \neq j \text,
    \end{cases}
  $$
  az úgynevezett Kronecker-delta.
\end{definition}

\begin{definition}[Ortogonális transzformáció]
  Az $n$ dimenziós  euklideszi tér $\mathcal A: V \rightarrow V$ lineáris
  transzformációját ortogonálisnak mondjuk, ha $\langle \mathcal A \rvec x;
    \mathcal A \rvec y \rangle = \langle \rvec x; \rvec y \rangle$, minden $\rvec
    x; \rvec y \in V$ esetén.
\end{definition}

\begin{note}
  \begin{itemize}
    \item Ortogonális transzformáció normatartó.

    \item Ortogonális transzformáció szögtartó.

    \item Ortogonális transzformáció ortonormált bázist ortonormált bázisba visz
          át.
  \end{itemize}
\end{note}

\begin{definition}[Bázistranszformáció]
  Legyenek $\{ \rvec b_1; \rvec b_2; \ldots; \rvec b_n \}$ és
  $\{ \hat{\rvec b}_1; \hat{\rvec b}_2; \ldots; \hat{\rvec b}_n \}$ bázisok
  $V$-ben. Ekkor a $\{ \rvec b_1; \rvec b_2; \ldots; \rvec b_n \} \rightarrow
    \{ \hat{\rvec b}_1; \hat{\rvec b}_2; \ldots; \hat{\rvec b}_n \}$
  bázistranszformáció $\rmat S$ mátrixa a következőképpen írható fel:
  $$
    \left.\begin{array}{rl}
      \hat{\rvec b}_1 & = s_{11} \rvec b_1 + s_{21} \rvec b_2 + \ldots + s_{n1} \rvec b_n \\
      \hat{\rvec b}_2 & = s_{12} \rvec b_1 + s_{22} \rvec b_2 + \ldots + s_{n2} \rvec b_n \\
                      & \vdots                                                            \\
      \hat{\rvec b}_j & = s_{1j} \rvec b_1 + s_{2j} \rvec b_2 + \ldots + s_{nj} \rvec b_n \\
                      & \vdots                                                            \\
      \hat{\rvec b}_n & = s_{1n} \rvec b_1 + s_{2n} \rvec b_2 + \ldots + s_{nn} \rvec b_n
    \end{array}\right\}
    \quad\Rightarrow\quad
    \rmat S = \begin{bmatrix}
      s_{11} & s_{12} & \cdots & s_{1n} \\
      s_{21} & s_{22} & \cdots & s_{2n} \\
      \vdots & \vdots & \ddots & \vdots \\
      s_{n1} & s_{n2} & \cdots & s_{nn}
    \end{bmatrix}
  $$
\end{definition}

\begin{statement}
  A bázistranszformáció mátrixa mindig invertálható.
\end{statement}