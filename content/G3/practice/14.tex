\documentclass{szb-practice}

\title{TODO}
\area{Differenciálegyenletek}
\subject{Matematika G3}
\subjectCode{BMETE94BG03}
\date{Utoljára frissítve: \today}
\docno{14}

\begin{document}
\maketitle

\subsection{Elméleti áttekintő}

\clearpage
\subsection{Feladatok}

\begin{enumerate}
  \item Határozza meg az $\dot{\rvec x} = \rmat A \rvec x$ differenciálegyenlet
        $\rvec x(0; 0) = (0; 1)$ kezdeti érték feltételt kielégítő megoldását,
        ha
        $$
          \rmat A = \begin{bmatrix}
            2 & 1 \\
            3 & 4
          \end{bmatrix}
          \text.
        $$

  \item Adja meg a következő differenciálegyenlet-rendszer általános megoldását!
        $$
          \left\{
          \begin{array}{rcl}
            \dot x & = & 2x + y \\
            \dot y & = & y      \\
            \dot z & = & 3z
          \end{array}
          \right.
        $$

  \item Határozza meg az alábbi differenciálegyenlet-rendszer általános
        megoldását!
        $$
          \left\{
          \begin{array}{rcl}
            \dot x & = & -x - 5y \\
            \dot y & = & x + y
          \end{array}
          \right.
        $$

  \item Adja meg a következő differenciálegyenlet-rendszer általános megoldását!
        $$
          \dot{\rvec x} = \begin{bmatrix}
            2 & 2 & 1 \\
            1 & 3 & 1 \\
            1 & 2 & 2 \\
          \end{bmatrix} \rvec x
        $$
\end{enumerate}

\clearpage
\subsection{Megoldások}
\begin{enumerate}
  \item Alma
\end{enumerate}

\end{document}