\documentclass{szb-practice}

\title{Felületi integrálok}
\area{Vektoranalízis}
\subject{Matematika G3}
\subjectCode{BMETE94BG03}
\date{Utoljára frissítve: \today}
\docno{5}

\begin{document}
\allowdisplaybreaks

\maketitle

\vspace{-1em}
\subsection{Elméleti áttekintő}
\vspace{1em}

\begin{definition}[Reguláris felület]
  Legyen $\surfimage \subseteq \Reals^3$. Azt mondjuk, hogy az $\surfsign$
  reguláris felület, ha $\forall \rvec p \in \surfimage$ ponthoz megadható
  olyan $\rvec p$-t tartalmazó $V \subset \Reals^3$ nyílt halmaz és
  $\surfsign : \surfdomain \subseteq \Reals^2 \rightarrow \surfimage \cap V$
  leképezés, melyre teljesülnek az alábbiak:
  \begin{itemize}
    \item $\surfsign$ differenciálható homeomorfizmus,

    \item $\surfsign$ immerzió (derivált leképezése injektív).
  \end{itemize}
  Ha ezek teljesülnek, akkor $\surfsign$-t parametrációnak,
  $V \cap \surfimage$-t koordinátakörnyezetnek nevezzük.
\end{definition}

\vfill

\begin{definition}[Elemi felület]
  A $\surfsign: \surfdomain \subseteq \Reals^2 \rightarrow \surfimage \subseteq
    \Reals^3$ elemi felület, ha $\surfsign$ legalább egyszer differenciálható és
  injektív.
\end{definition}

\vfill

\begin{definition}[Skalármező skalárérékű felületmenti integrálja]
  Legyen $\varphi: D \subseteq \Reals^3 \rightarrow \Reals$ skalármező,
  $\surfsign: \surfdomain \subseteq \Reals^2 \rightarrow \surfimage
    \subset D$, $t \mapsto \surfsign(t)$ pedig a felület parametrizált
  egyenlete. Ekkor a $\varphi$ skalármező $\surfimage$ felület menti
  skalárértékű integrálja:
  $$
    \iint_{\surfimage} \varphi(\coordvec) \dd \surfvec =
    \iint_{\surfdomain}
    \varphi(\surfsign(s; t))
    \norma{
      \pdv{\surfsign}{s}
      \times
      \pdv{\surfsign}{t}
    }
    \dd s \dd t
    \text.
  $$
  Amennyiben a fekület $z = \Phi(x; y)$ alakban van megadva, akkor:
  $$
    \iint_{\surfimage} \varphi(\coordvec) \dd \surfvec =
    \iint_{\surfdomain}
    \varphi(x; y; \Phi(x; y))
    \sqrt{1 + (\partial_x \varPhi)^2 + (\partial_y \varPhi)^2}
    \dd x \dd y
    \text.
  $$
\end{definition}

\vfill

\begin{definition}[Vektormező skalár- és vektorértékű felületmenti integrálja][nobreak]
  Legyen $\rvec v: D \subseteq \Reals^3 \rightarrow \Reals^3$ vektormező,
  $\surfsign: \surfdomain \subseteq \Reals^2 \rightarrow \surfimage
    \subset D$, $t \mapsto \surfsign(t)$ pedig a felület parametrizált
  egyenlete. Ekkor a $\rvec v$ vektormező $\surfimage$ felület menti
  \begin{itemize}
    \item skalárértékű integrálja:
          $\displaystyle
            \iint_{\surfimage} \scalar{\rvec v(\coordvec)}{\dd \surfvec} =
            \iint_{\surfdomain}
            \scalar{\rvec v(\surfsign(s; t))}{
              \pdv{\surfsign}{s}
              \times
              \pdv{\surfsign}{t}
            }
            \dd s \dd t
            \text,
          $

    \item vektorértékű integrálja:
          $\displaystyle
            \iint_{\surfimage} \rvec v(\coordvec) \times \dd \surfvec =
            \iint_{\surfdomain}
            \rvec v(\surfsign(s; t)) \times
            \left(
            \pdv{\surfsign}{s}
            \times
            \pdv{\surfsign}{t}
            \right)
            \dd s \dd t
            \text.
          $
  \end{itemize}
\end{definition}

\clearpage
\subsection{Feladatok}

\begin{enumerate}
  \item
        % \item Számítsuk ki a megadott felületek felszínét!
        %       \begin{enumerate}
        %         \item $z = x^2 + y^2$ forgásparaboloid $z = 1$ és $z = 4$ síkok közé
        %               eső része,

        %         \item $\surfvec(s;t) = \ijk{e^s \cos t}{e^s \sin t}{s}$,
        %               $s \in (-\infty; 0]$, $t \in [0; 2\pi]$.
        %       \end{enumerate}

        % \item Integrálja a skalármezőket a megadott felületeken!
        %       \begin{enumerate}
        %         \item $f(\coordvec) = x^2 + y^2$, az egységgömb $z > 0$ részén,

        %         \item $g(\coordvec) = x + y + z, $ az $x + 2y + 4z = 4$ sík
        %               első térnyolcadba tartozó részén.
        %       \end{enumerate}

        % \item Integrálja a vektormezőket a megadott felületeken!
        %       A normális kifele mutató legyen!
        %       \begin{itemize}
        %         \item $\rvec u(\coordvec) = \ijk{x + y}{x - y}{z^2}$,\\
        %               $\surfvec (s;t) = \ijk{s + t}{s - t}{s^2 - t^2}$,
        %               $(s;t) \in [0;1]^2$,

        %         \item $\rvec v(\coordvec) = \ijk{y^2 + z^2}{x^2 + z^2}{x^2 + y^2}$,\\
        %               $r = 2$ sugarú, $x = 2$ síkon lévő körön.
        %       \end{itemize}
\end{enumerate}

\clearpage
\subsection{Segédlet}

\subsubsection{Felületek paraméterezése}

\bgroup
\def\tskip{15mm}
\begin{tabular}{
  >{\bullet\;}p{3.25cm}
  p{5cm}
  m{2.5cm}
  m{3.5cm}
  }
  \textbf{Körlap:} \newline \phantom{1} ($xy$ sík)
   & $\surfsign (s;t) = \begin{bmatrix} s \cos t \\ s \sin t \\ 0 \end{bmatrix}$
   & $s \in [0;r]$ \newline $t \in [0, 2\pi]$
   & \begin{tikzpicture}[
         3d view={110}{20},
         baseline,
       ]
       % Origin coordinate
       \coordinate (O) at (0,0,0);

       % Circle
       \draw[fill=red-base!50!white, fill opacity=.75] (O) circle (1);

       % Coordinate system
       \draw[-to] (O) -- ++(1.75,0,0) node[anchor=west] {$x$};
       \draw[-to] (O) -- ++(0,1.75,0) node[anchor=south east] {$y$};
       \draw[-to] (O) -- ++(0,0,1.25) node[anchor=north east] {$z$};

       % Radius
       \draw[-to, thick, draw=blue-base]
       (O) -- (0.6*1.75,0.8*1.75,0) -- (0.6,0.8,0)
       node[midway, anchor=north east, inner sep=.5mm, font=\scriptsize] {$r$};
     \end{tikzpicture}
  \\[\tskip]
  \textbf{Ellipszislap:} \newline \phantom{1} ($xy$ sík)
   & $\surfsign (s;t) = \begin{bmatrix} a \, s \cos t \\ b \, s \sin t \\ 0 \end{bmatrix}$
   & $s \in [0;1]$ \newline $t \in [0, 2\pi]$
   & \begin{tikzpicture}[
         3d view={110}{20},
         baseline,
       ]
       % Origin coordinate
       \coordinate (O) at (0,0,0);

       % Ellipsis
       \draw[fill=red-base!50!white, fill opacity=.75] (O) ellipse (1.4 and .8);

       % Coordinate system
       \draw[-to] (O) -- ++(2.20,0,0) node[anchor=west] {$x$};
       \draw[-to] (O) -- ++(0,1.50,0) node[anchor=south east] {$y$};
       \draw[-to] (O) -- ++(0,0,1.25) node[anchor=north east] {$z$};

       % Half axes
       \begin{scope}[font=\scriptsize]
      \node at (0.7,0,0) [anchor=south east, inner sep=.5mm] {$a$};
      \node at (0,0.4,0) [anchor=south, inner sep=.5mm] {$b$};
    \end{scope}

       \draw[to-to, thick, draw=blue-base] (O) -- (1.4,0,0);
       \draw[to-to, thick, draw=blue-base] (O) -- (0,0.8,0);
     \end{tikzpicture}
  \\[\tskip]
  \textbf{Hengerfelület:}
   & $\surfsign (s;t) = \rvec r_0(s) + t \rvec n$
   & $s \in \mathcal D_{\rvec r_0}$ \newline $t \in [0, T]$
   & \begin{tikzpicture}[
         baseline,
       ]
       % Random potato coordinates
       \coordinate (A) at (0,-.15);
       \coordinate (B) at (.6,-.2);
       \coordinate (C) at (.5,.5);
       \coordinate (D) at (0,.35);
       \coordinate (E) at (-.66,.45);
       \coordinate (F) at (-.75,-.35);

       % + is in the foreground, - is in the background
       \foreach \c in {A,B,C,D,E,F} {
           \coordinate (\c-) at ($(\c) + (.5,.5)$);
           \coordinate (\c+) at ($(\c) - (.375,.375)$);

           % Connect + and - with a line
           \draw[opacity=0.25,red-base] (\c-) -- (\c+) coordinate[pos=.25] (\c75);
         }


       % Connect the dots
       \draw[smooth cycle, thick, red-base, fill=white] plot coordinates {
           (A+) (B+) (C+) (D+) (E+) (F+)
         };
       \draw[smooth cycle, thick, yellow-base, opacity=.25] plot[xshift=1cm] coordinates {
           (A-) (B-) (C-) (D-) (E-) (F-)
         };

       % Draw the normal vector
       \draw[-to, draw=blue-base, thick] (E+) -- (E75) node[below left, xshift=-3mm] {\scriptsize$\rvec n$};

       % Label the curve
       \node[above left=-1mm] at (A+) {\scriptsize$\rvec r_0(s)$};
     \end{tikzpicture}
  \\[\tskip]
  \textbf{Forgásfelület:} \newline \phantom{1} ($z$ tengely körül) \newline \phantom{1} ($z = f(x)$)
   & $\surfsign (s;t) = \begin{bmatrix} s \cos t \\ s \sin t \\ f(s) \end{bmatrix}$
   & $s \in [0;2\pi]$ \newline $t \in \mathcal D_f$
   & \begin{tikzpicture}[
         3d view={110}{20},
         baseline,
       ]
       % Origin coordinate
       \coordinate (O) at (0,0,0);

       % Helper coordinates
       \foreach \a in {0,30,...,330} {
           \foreach \loc in {0,2,...,10} {
               \coordinate (\a-\loc) at
               ({\loc*cos(\a)/10},{\loc*sin(\a)/10},1.4*\loc*\loc/100 - 0.6*\loc*\loc*\loc*\loc/10000);
             }
         }

       % Draw surface in bg
       \foreach \x\y in {150/180,180/210,210/240,240/270,270/300,300/330} {
           \foreach \i/\j in {0/2,2/4,4/6,6/8,8/10} {
               \fill[thick, draw=red-base, fill=red-base!20, rounded corners=.1pt] (\x-\i) -- (\y-\i) -- (\y-\j) -- (\x-\j) -- cycle;
             }
         }

       % Coordinate system
       \draw[-to] (O) -- ++(1.75,0,0) node[anchor=south east] {$x$};
       \draw[-to] (O) -- ++(0,1.75,0) node[anchor=south east] {$y$};
       \draw[-to] (O) -- ++(0,0,1.75) node[anchor=north east] {$z$};

       % Draw surface in fg
       \foreach \x/\y in {120/150,90/120,330/0,0/30,30/60,60/90} {
           \foreach \i/\j in {0/2,2/4,4/6,6/8,8/10} {
               \fill[thick, draw=red-base, fill=red-base!20, rounded corners=.1pt] (\x-\i) -- (\y-\i) -- (\y-\j) -- (\x-\j) -- cycle;
             }
         }
     \end{tikzpicture}
  \\[\tskip]
  \textbf{Gömbfelület:}
   & $\surfsign (s;t) = \begin{bmatrix} R \sin s \cos t \\ R \sin s \sin t \\ R \cos s \end{bmatrix}$
   & $s \in [0;\pi]$ \newline $t \in [0, 2\pi]$
   & \begin{tikzpicture}[
         3d view={135}{35.26},
         % 3d view={110}{20},
         baseline,
       ]
       % Origin coordinate
       \coordinate (O) at (0,0,0);

       % Helper coordinates
       \foreach \s in {0,30,...,360} {
           \foreach \t in {0,30,...,330} {
               \coordinate (\s-\t) at
               (
               {sin(\s/2)*cos(\t)},
               {sin(\s/2)*sin(\t)},
               {cos(\s/2)}
               );
             }
         }

       % Draw surface in bg
       \foreach \x/\y in {330/360,300/330,270/300,240/270,210/240,180/210,150/180,120/150,90/120,60/90,30/60,0/30} {
           \foreach \i/\j in {150/180,180/210,210/240,240/270,270/300,300/330,330/0,120/150,90/120}{
               \fill[thick, draw=red-base, fill=red-base!20, rounded corners=.1pt] (\x-\i) -- (\y-\i) -- (\y-\j) -- (\x-\j) -- cycle;
             }
         }

       % Coordinate system
       \draw[-to] (O) -- ++(1.75,0,0) node[anchor=south east] {$x$};
       \draw[-to] (O) -- ++(0,1.75,0) node[anchor=south west] {$y$};
       \draw[-to] (O) -- ++(0,0,1.75) node[anchor=north east] {$z$};

       % Draw surface in fg
       \foreach \x/\y in {330/360,300/330,270/300,240/270,210/240,180/210,150/180,120/150,90/120,60/90,30/60,0/30} {
           \foreach \i/\j in {0/30,30/60,60/90} {
               \fill[thick, draw=red-base, fill=red-base!20, rounded corners=.1pt] (\x-\i) -- (\y-\i) -- (\y-\j) -- (\x-\j) -- cycle;
             }
         }
     \end{tikzpicture}
  \\[\tskip]
  \textbf{Ellipszoid:}
   & $\surfsign (s;t) = \begin{bmatrix} a \sin s \cos t \\ b \sin s \sin t \\ c \cos s \end{bmatrix}$
   & $s \in [0;\pi]$ \newline $t \in [0, 2\pi]$
   & \begin{tikzpicture}[
         3d view={110}{20},
         baseline,
       ]
       % Origin coordinate
       \coordinate (O) at (0,0,0);

       % Helper coordinates
       \foreach \s in {0,30,...,360} {
           \foreach \t in {0,30,...,330} {
               \coordinate (\s-\t) at
               (
               {0.8*sin(\s/2)*cos(\t)},
               {1.4*sin(\s/2)*sin(\t)},
               {0.6*cos(\s/2)}
               );
             }
         }

       % Draw surface in bg
       \foreach \x/\y in {330/360,300/330,270/300,240/270,210/240,180/210,150/180,120/150,90/120,60/90,30/60,0/30} {
           \foreach \i/\j in {150/180,180/210,210/240,240/270,270/300,300/330,330/0,120/150,90/120}{
               \fill[thick, draw=red-base, fill=red-base!20, rounded corners=.1pt] (\x-\i) -- (\y-\i) -- (\y-\j) -- (\x-\j) -- cycle;
             }
         }

       % Coordinate system
       \draw[-to] (O) -- ++(2.25,0,0) node[anchor=east] {$x$};
       \draw[-to] (O) -- ++(0,2.00,0) node[anchor=south] {$y$};
       \draw[-to] (O) -- ++(0,0,1.25) node[anchor=north east] {$z$};

       % Draw surface in fg
       \foreach \x/\y in {330/360,300/330,270/300,240/270,210/240,180/210,150/180,120/150,90/120,60/90,30/60,0/30} {
           \foreach \i/\j in {0/30,30/60,60/90} {
               \fill[thick, draw=red-base, fill=red-base!20, rounded corners=.1pt] (\x-\i) -- (\y-\i) -- (\y-\j) -- (\x-\j) -- cycle;
             }
         }
     \end{tikzpicture}
  \\[\tskip]
  \textbf{Тórusz:}
   & $\surfsign (s;t) = \begin{bmatrix} (R + r \cos s) \cos t \\ (R + r \cos s) \sin t \\ r \sin s \end{bmatrix}$
   & $s \in [0;2\pi]$ \newline $t \in [0, 2\pi]$
   & \begin{tikzpicture}
       \begin{axis}[
        axis equal image,
        hide axis,
        % z buffer = sort,
        view={110}{20},
        xmax=2.5,
        ymax=2.5,
        zmax=2.25,
        scale=.75,
      ]

      \draw[-to] (axis cs:0,0,0) -- (axis cs:1.25,0,0);
      \draw[-to] (axis cs:0,0,0) -- (axis cs:0,1.25,0);

      \addplot3[
        surf,
        faceted color=red-base,
        fill=red-base!20,
        ultra thin,
        samples=15,
        samples y=48,
        domain=0:2*pi,
        domain y=0:2*pi,
        z buffer=sort,
      ](
      {(1.25+0.25*sin(deg(\x)))*cos(deg(\y))},
      {(1.25+0.25*sin(deg(\x)))*sin(deg(\y))},
      {0.25*cos(deg(\x))}
      );

      \draw[-to] (axis cs:0,0,0) -- (axis cs:0,0,2) node[anchor=north east] {$z$};

      \draw[-to] (axis cs:1.5,0,0) -- (axis cs:2.25,0,0) node[anchor=north] {$x$};
      \draw[-to] (axis cs:0,1.5,0) -- (axis cs:0,1.75,0) node[anchor=north] {$y$};

      \coordinate (C) at (axis cs:0,1.25,1.25);
      \coordinate (T) at (axis cs:0,1.25,1.65);
      \coordinate (T+) at (axis cs:0,1.25,1.75);
      \coordinate (Z) at (axis cs:0,0,1.65);
    \end{axis}

       \begin{scope}[3d view={110}{20},canvas is yz plane at x=0]
      \draw[red-base, fill=red-base!20] (C) circle (0.33);
      \draw[red-base, fill=red-base, ultra thick] (C) circle (0.02);

      \draw[gray] (C) -- (T+);
      \draw[to-to, thick, draw=blue-base]
      (Z) -- (T)
      node[midway, anchor=north] {\scriptsize$R$};
      \draw[-to, thick, draw=blue-base]
      (C) -- ++(0.75*0.8,0.75*0.6) -- ($(C)+(0.33*0.8,0.33*0.6)$)
      node[midway, anchor=south, inner sep=.5mm, font=\scriptsize] {$r$};
    \end{scope}
     \end{tikzpicture}
  \\[\tskip]
  \textbf{Kúp:}
   & $\surfsign (s;t) = \begin{bmatrix} s \cos t \\ s \sin t \\ s \end{bmatrix}$
   & $s \in [0;U]$ \newline $t \in [0, 2\pi]$
   & \begin{tikzpicture}[
         3d view={110}{20},
         baseline,
       ]
       % Origin coordinate
       \coordinate (O) at (0,0,0);

       % Helper coordinates
       \foreach \a in {0,15,...,345} {
           \foreach \loc in {0,10} {
               \coordinate (\a-\loc) at
               ({\loc*cos(\a)/10},{\loc*sin(\a)/10},1.4*\loc*\loc/100 - 0.6*\loc*\loc*\loc*\loc/10000);
             }
         }

       % Draw surface in bg
       \foreach \x\y in {120/135,135/150,150/165,165/180,180/195,195/210,210/225,225/240,240/255,255/270,270/285,285/300,300/315,315/330} {
           \foreach \i/\j in {0/10} {
               \fill[thick, draw=red-base, fill=red-base!20, rounded corners=.1pt] (\x-\i) -- (\y-\i) -- (\y-\j) -- (\x-\j) -- cycle;
             }
         }

       % Coordinate system
       \draw[-to] (O) -- ++(1.75,0,0) node[anchor=south east] {$x$};
       \draw[-to] (O) -- ++(0,1.75,0) node[anchor=south east] {$y$};
       \draw[-to] (O) -- ++(0,0,1.75) node[anchor=north east] {$z$};

       % Draw surface in fg
       \foreach \x/\y in {330/345,345/0,0/15,15/30,75/90,90/105,105/120,45/60,30/45,60/75} {
           \foreach \i/\j in {0/10} {
               \fill[thick, draw=red-base, fill=red-base!20, rounded corners=.1pt] (\x-\i) -- (\y-\i) -- (\y-\j) -- (\x-\j) -- cycle;
             }
         }
     \end{tikzpicture}
\end{tabular}
\egroup

\end{document}