\documentclass{szb-practice}

\usepackage{siunitx}
\sisetup{locale = DE} 

\title{Integrálátalakító tételek, mérnöki példák}
\area{Vektoranalízis}
\subject{Matematika G3}
\subjectCode{BMETE94BG03}
\date{Utoljára frissítve: \today}
\docno{5}

\begin{document}
\allowdisplaybreaks

\maketitle

\subsection{Elméleti áttekintő}

% ~~~~~~~~~~~~~~~~~~~~~~~~~~~~~~~~~~~~~~~~~~~~~~~~~~~~~~~~~~~~~~~~~~~~~~~~~~~~~~
% ~~~~~~~~~~~~~~~~~~~~ Gradient-theorem ~~~~~~~~~~~~~~~~~~~~~~~~~~~~~~~~~~~~~~~~
% ~~~~~~~~~~~~~~~~~~~~~~~~~~~~~~~~~~~~~~~~~~~~~~~~~~~~~~~~~~~~~~~~~~~~~~~~~~~~~~
\begin{theorem}[Gradiens-tétel]
  Legyen $\varphi : U \subseteq \Reals^3 \rightarrow \Reals$ differenciálható
  skalármező, $\curvesign : [a;b] \rightarrow \curveimage \subseteq U$,
  $t \mapsto \curvesign(t)$ folytonos görbe, $\curvesign(a) = \rvec p$,
  $\curvesign(b) = \rvec q$ pedig a görbe kezdő és végpontja. Ekkor:
  $$
    \int_{\curveimage} \scalar{\grad \varphi(\coordv)}{\dd \curvevec}
    =
    \varphi(\rvec q) - \varphi(\rvec p)
    \text.
  $$
  Vagyis, ha egy vektormező valamely skalármező gradiense, akkor annak bármely
  folytonos görbe mentén vett integrálja csak a kezdő- és végpontoktól függ.
\end{theorem}

\begin{blueBox}
  \sftitle{Körintegrál jelölése:}

  Ha $\curvesign$ zárt görbe, akkor a $\varphi(\coordv)$ skalármező egy
  $\curvesign$ görbe mentén vett körintegrálja a következőképpen jelölhető:
  $$
    \oint_{\curveimage} \, \varphi(\coordv) \dd \curvescalar
    \text.
  $$
\end{blueBox}

\begin{note}
  A Gradiens-tételből következik, hogy skalárpotenciálos vektormező zárt görbe
  mentén vett körintegrálja zérus.
\end{note}

\begin{example}
  Integrálja a $\rvec v(\coordv) = \ijk{y + z}{x + z}{x + y}$ vektormezőt
  a $z = 0$ síkon lévő, origó középpontú, $r = 3$ sugárú kör mentén!

  Vizsgáljuk meg, hogy a vektormező skalárpotenciálos-e:
  $$
    \rot \rvec v
    =
    \begin{bmatrix}
      \partial_x \\ \partial_y \\ \partial_z
    \end{bmatrix}
    \times
    \begin{bmatrix}
      y + z \\ x + z \\ x + y
    \end{bmatrix}
    =
    \begin{bmatrix}
      \partial_x (x + z) - \partial_y (y + z) \\
      \partial_y (x + y) - \partial_z (x + z) \\
      \partial_z (y + z) - \partial_x (x + y)
    \end{bmatrix}
    =
    \begin{bmatrix}
      1 - 1 \\ 1 - 1 \\ 1 - 1
    \end{bmatrix}
    =
    \nvec
    \text.
  $$
  Mivel a vektormező skalárpotenciálos, ezért létezik olyan skalármező,
  melynek gradiense maga a $\rvec v$ vektormező. Az integrál értéke tehát
  csak a kezdő- és végpontoktól függ, melyek jelen esetben megegyeznek,
  vagyis az integrál értéke zérus:
  $$
    \oint_{\curveimage} \scalar{\rvec v}{\dd \curvevec} = 0
    \text.
  $$
\end{example}

% ~~~~~~~~~~~~~~~~~~~~~~~~~~~~~~~~~~~~~~~~~~~~~~~~~~~~~~~~~~~~~~~~~~~~~~~~~~~~~~
% ~~~~~~~~~~~~~~~~~~~~ Stokes' theorem ~~~~~~~~~~~~~~~~~~~~~~~~~~~~~~~~~~~~~~~~~
% ~~~~~~~~~~~~~~~~~~~~~~~~~~~~~~~~~~~~~~~~~~~~~~~~~~~~~~~~~~~~~~~~~~~~~~~~~~~~~~
\begin{theorem}[Stokes-tétel]
  Legyen $\surfsign: \surfdomain\subset\Reals^2 \to \surfimage \subset \Reals^3$
  irányított, parametrizált, elemi felület. Legyen továbbá $\rvec v : \Reals^3
    \to \Reals^3$ legalább egyszer folytonosan differenciálható vektormező.
  Jelölje az $\curvesign: \curvedomain\subset\Reals \to \partial\surfimage =
    \curveimage$ a $\surfsign$ peremét indukált, jobbézszabály szerinti
  irányítással. Ekkor:
  $$
    \iint_{\surfimage} \scalar{\rot \rvec v}{\dd \surfvec}
    =
    \oint_{\partial \surfimage} \scalar{\rvec v}{\dd \curvevec}
    \text.
  $$
\end{theorem}

\begin{note}
  Ha $\rvec v$ skalárpotenciálos, akkor az integrál értéke zérus, hiszen
  $\rot \rvec v = \rot \grad \varphi = \rvec 0$.
\end{note}

\begin{example}
  Integrálja a $\rvec v(\coordv) = \ijk{y}{x}{0}$ vektormezőt a $P_1(0;1;0)$,
  $P_2(2;0;0)$ és $P_3(0;0;0)$ által meghatározott háromszög mentén!

  Határozzuk meg a $\rvec v$ vektormező rotációját:
  $$
    \rot \rvec v
    =
    \begin{bmatrix}
      \partial_x \\ \partial_y \\ \partial_z
    \end{bmatrix}
    \times
    \begin{bmatrix}
      y \\ x \\ 0
    \end{bmatrix}
    =
    \begin{bmatrix}
      0 \\ 0 \\ 1 - 1
    \end{bmatrix}
    =
    \nvec
    \text.
  $$
  A Stokes-tétel alapján:
  $$
    \oint_{\partial \surfimage} \scalar{\rvec v}{\dd \curvevec}
    = \int_{\surfimage} \scalar{\rot \rvec v}{\dd \surfvec}
    = \int_{\surfimage} \scalar{\nvec}{\dd \surfvec}
    = 0
    \text.
  $$
\end{example}

\begin{learnMore}[Stokes-tétel Maxwell III. és IV. egyenletében]
  A Stokes-tétel a Maxwell-egyenletekben is fontos szerepet játszik. A harmadik
  és negyedik egyenlet a mágneses tér és az elektromos tér közötti
  kapcsolatot írja le:
  $$
    \begin{aligned}
      (III) & \quad \Rightarrow \quad \rot \rvec E = -\dot{\rvec B}
            &
            & \quad \Rightarrow \quad \text{elektromos tér -- mágneses tér változása,}
      \\
      (IV)  & \quad \Rightarrow \quad \rot \rvec B = \mu_0 \rvec j + \mu_0
      \varepsilon_0 \dot{\rvec E}
            &
            & \quad \Rightarrow \quad \text{mágneses tér -- elektromos tér változása,}
    \end{aligned}
  $$
  ahol $\rvec E$ az elektromos tér, $\rvec B$ a mágneses tér, $\rvec j$ az áram
  sűrűség, $\mu_0$ a mágneses permeabilitás és $\varepsilon_0$ az elektromos
  permittivitás.

  Az egyenletek közötti kapcsolatot a Stokes-tétel segítségével:
  $$
    \begin{aligned}
      (III) & \quad \Rightarrow \quad
      \oint_{\partial \surfimage} \scalar{\rvec E}{\dd \curvevec}
      = -\iint_{\surfimage} \scalar{\dot{\rvec B}}{\dd \surfvec}
      \text,
      \\
      (IV)  & \quad \Rightarrow \quad
      \oint_{\partial \surfimage} \scalar{\rvec B}{\dd \curvevec}
      = \iint_{\surfimage} \scalar{\mu_0 \rvec j + \mu_0 \varepsilon_0 \dot{\rvec E}}{\dd \surfvec}
      \text.
    \end{aligned}
  $$

  A III. egyenlet azt mondja ki, hogy változó mágneses tér maga körül
  balkézszabály szerint elektormos teret indukál, míg a IV. egyenlet azt
  jelenti, hogy az elektromos tér változása jobbkézszabály szerint
  mágneses teret indukál.
\end{learnMore}

\clearpage
\subsection{Feladatok}

\begin{enumerate}
  \item Adott egy $\rvec F(x; y) = \ijz{2xy}{x^2 + 2y}$ erőtér. Vizsgálja meg,
        hogy az $\rvec F$ erőtér konzervatív-e! Amennyiben igen, adja meg egy
        olyan potenciálfüggvényt, melyre $\varphi(0;0) = 0$. Számítsa ki a
        $P_1(0; 0)$ és $P_2(1; 1)$ pontok közötti egyenes szakaszon végzett
        munkát!

  \item Egy $Q = \num{8.85}\pi\,\si{\milli\coulomb}$ nagyságú ponttöltés
        közelében az elektrosztatikus térerősséget az
        $$
          \rvec E(\coordvec) = \frac{Q}{4\pi \varepsilon_0}
          \frac{\coordvec}{|\coordvec|^3}
          \qquad
          \coordvec \neq \nvec
          \qquad \varepsilon_0 = \SI[per-mode=symbol]{8,85e-12}{\farad\per\meter}
        $$
        vektormező írja le. Mutassa meg, hogy az $\rvec E$ vektormező
        konzervatív, és vezesse le a potenciálfüggvényt $\varphi(\infty) = 0$
        határfeltétel mellett! Számítsa ki a $q = \SI{1}{\micro\coulomb}$
        próbatöltés által a $P_1(1; 0; 0)$ és $P_2(2; 0; 0)$ pontok között
        végzett munkát, ha $\rvec F = q \rvec E$.

  \item Egy nagyon hosszú, áramjárta vezető belsejében a mágneses indukció
        jó közelítéssel lineárisan változik a keresztmetszetben:
        $$
          \rvec B(\coordv) = \ijk{ky}{-kx}{0}
        $$
        Igazolja, hogy a $\rvec B$ vektormező forrásmentes, majd adja meg
        a $\rvec B$ vektormező vektorpotenciálját $\rvec A = (A_x; A_y; 0)$
        alakban, melyre $\rvec A(\nvec) = \nvec$ teljesül. Mi $k$ mértékegysége?

  \item Legyen $\rvec v(\coordvec) = \ijk{y \sin x}{z^2 \cos y - \cos x}{v_3}$.
        Határozza meg $v_3$-at, ha tudjuk, hogy $\rvec v$ tetszőleges zárt
        görbén vett vonalintegrálja zérus!

  \item Egy $R = 1$ sugarú, kör keresztmetszetű, $z$ tengellyel egybeeső
        szimmetriavonalú hengerben áramló folyadék sebességét a
        $$
          \rvec v(\coordv) = \ijk{2xy + z}{x^2 + z}{y - x}
        $$
        vektormező írja le. Adja meg a $z = 1$ síkban lévő keresztmetszet
        menti cirkulációt!
        (A cirkuláció a vektormező zárt görbe menti integrálja.)

  \item Jelölje $\surfimage$ az $x^2 + y^2 - z^2 = 1$ egyenletű
        forgáshiperboloid $z = -1$ és $z = 1$ síkok közötti részét.
        Határozza meg a $\rvec v(\coordvec) = \ijk{x^2}{y^3}{z^4}$
        vektormező $\surfimage$ peremén vett integrálját!

  \item Integrálja a $\rvec v(\coordvec) = \ijk{y^2}{z^2}{x^2}$
        vektormezőt az $A(1;0;0)$, $B(0;1;0)$ és $C(0;0;1)$ csúcsokkal
        meghatározott háromszögvonal mentén!
\end{enumerate}

% \vfill
% \subsection{Megoldások}

% \begin{enumerate}
%   \item $W = \SI{-17}{J}$
%   \item \begin{enumerate}
%           \item $b_3 = 2 z \sin y + C$
%           \item $b_3 = \dfrac{z^3 \sin y}{3} - y z \cos x + C$
%         \end{enumerate}
%   \item $\text{cirkuláció} = 0$
%   \item $0$
%   \item $Q = 8\pi\varepsilon_0\sqrt{95} \approx \SI{2,17}{nC}$
%   \item $q_\text{hő} = 8\pi\lambda\alpha R^3 = \SI{5,23}{W}$
% \end{enumerate}

\end{document}