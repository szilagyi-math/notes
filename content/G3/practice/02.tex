
\documentclass[a4paper, 12pt]{scrartcl}

\usepackage{math-practice}

\area{Vektoranalízis}
\title{Differenciáloperátorok II}
\subject{Matematika G3}
\subjectCode{BMETE94BG03}
\date{Utoljára frissítve: \today}
\docno{2}

\begin{document}
\maketitle

\subsection{Elméleti Áttekintő}

\clearpage
\subsection{Feladatok}

\begin{enumerate}
  \item Adja meg a $\rvec v(\rvec r) = \rvec r \cdot |\rvec r|^3$ vektorfüggvény
        divergenciáját és rotációját! Hattassa a Laplace-operátort
        $\operatorname{div} \rvec v$-re!

  \item Határozza meg a $\rvec v(\rvec r)$ vektormező divergenciáját, illetve
        rotációját!
        $$
          \rvec v(\rvec r) = \begin{bmatrix}
            \ln(\sfrac{xy}{z}) \\
            \ln(\sfrac{yz}{x}) \\
            \ln(\sfrac{zx}{y}) \\
          \end{bmatrix}
        $$

  \item Igazolja, hogy tetszőleges Joung-tételt kielégítő vektormező
        rotációjának divergenciája a nullvektor, vagyis
        $$
          \nabla \cdot (\nabla \times \rvec v) \equiv \nvec
          \text.
        $$

  \item Igazolja az alábbi azonosságot, amennyiben $f$ tetszőleges skalár-,
        $\rvec v$ pedig tetszőleges vektormező:
        $$
          \nabla \cdot (f \cdot \rvec v)
          = f \cdot (\nabla \cdot \rvec v)
          + \nabla f \cdot \rvec v
          \text.
        $$
\end{enumerate}
\end{document}