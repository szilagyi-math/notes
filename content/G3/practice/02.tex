\documentclass{szb-practice}

\area{Vektoranalízis}
\title{Operátorok, potenciálosság}
\subject{Matematika G3}
\subjectCode{BMETE94BG03}
\date{Utoljára frissítve: \today}
\docno{2}

\begin{document}
\maketitle

\vspace{-1em}
\subsection{Elméleti Áttekintő}
\vspace{1em}

\begin{definition}[Nabla-operátor]
  Az $\Reals^n$-beli Descartes-koordinátarendszerben, ahol $\rvec x = (x_1;
    x_2; \dots; x_n)$ egy tetszőleges pont koordinátái, a standard bázis pedig
  $\{ \uvec e_1; \uvec e_2; \dots; \uvec e_n \}$ a Nabla egy olyan formális
  differenciáloperátor, melynek koordinátái a parciális derivált operátorok,
  vagyis:
  $$
    \nabla = \sum_{i = 1}^n \uvec e_i \pdv{}{x_i}
    =
    \begin{pmatrix}
      \displaystyle\pdv{}{x_1} &
      \displaystyle\pdv{}{x_2} &
      \hdots                   &
      \displaystyle\pdv{}{x_n}
    \end{pmatrix}^\T
    \text,
  $$
\end{definition}

\begin{blueBox}
  \sftitle{Differenciáloperátorok:}

  Legyen $\rvec v(\coordvec) : \Reals^3 \to \Reals^3$ vektormező,
  $\varphi(\coordvec): \Reals^3 \to \Reals$ skalármező, ahol
  $\coordvec$ az $\Reals^3$-beli Descartes koordináta-rendszerben
  $\coordvec = (x; y; z)$.
  \begin{center}
    \def\arraystretch{1.5}
    \newenvironment{bm}{\bgroup\renewcommand*{\arraystretch}{1.1}\begin{bmatrix}}{\end{bmatrix}\egroup}
    \newcommand{\dspl}[3]{\begin{bm}#1\\#2\\#3\end{bm}}
    \newcommand\nablavec{\dspl{\partial_x}{\partial_y}{\partial_z}}

    \begin{tabular}{*{3}{>{\centering\arraybackslash}p{3.5cm}}}
      \def\arraystretch{1}
      % &
      \bfseries Rotáció
       & \bfseries Divergencia
       & \bfseries Gradiens
      \\
      \hline
      % Jelölés & 
      $\rot \rvec v$
       & $\Div \rvec v$
       & $\grad \varphi$
      \\
      % Operátor & 
      $\nabla \times \rvec v$
       & $\scalar{\nabla}{\rvec v}$
       & $\nabla \cdot \varphi$
      \\
      % Számítás &
      $\nablavec \times \dspl{v_x}{v_y}{v_z}$
       & $\scalar{\nablavec}{\dspl{v_x}{v_y}{v_z}}$
       & $\dspl{\partial_x \varphi}{\partial_y \varphi}{\partial_z \varphi}$
      \\
      % Ért. tart. & 
      $\mathcal D_{\rvec v} = \Reals^3$
       & $\mathcal D_{\rvec v} = \Reals^3$
       & $\mathcal D_{\varphi} = \Reals^3$
      \\
      % Ért. készl. &
      $\mathcal R_{\rvec v} = \Reals^3$
       & $\mathcal R_{\rvec v} = \Reals^3$
       & $\mathcal R_{\varphi} = \Reals$
      \\
      % Ért. készl. &
      $\mathcal R_{\rot \rvec v} = \Reals^3$
       & $\mathcal R_{\Div \rvec v} = \Reals$
       & $\mathcal R_{\grad \varphi} = \Reals^3$
      \\
      % \hline
    \end{tabular}
  \end{center}

  Speciális esetek:
  \begin{itemize}
    \item ha $\Div \rvec v = 0$, akkor a vektromező forrásmentes,
    \item ha $\rot \rvec v = \nvec$, akkor a vektromező örvénymentes.
  \end{itemize}
\end{blueBox}

\begin{definition}[Laplace-operátor][nobreak]
  A Laplace-operátor egy másodrendű differenciáloperátor az $n$ dimenziós
  térben. Megadja egy skalármező gradiensének divergenciáját, azaz:
  $$
    \Delta \varphi
    = \scalar{\nabla}{\nabla} \varphi
    = \Div \grad \varphi
    \text.
  $$
\end{definition}

\begin{blueBox}[][nobreak]
  $! \varPhi; \varPsi : \Reals^3 \rightarrow \Reals$ skalármezők,
  $\rvec u; \rvec v; \rvec w : \Reals^3 \rightarrow \Reals^3$
  vektormezők, $\lambda; \mu \in \Reals$ pedig skalárok.
  \begin{itemize}
    \item Teljesül a linearitás:
          \vspace{-.5em}
          \begin{alignat*}{4}
            \grad & (\lambda \, \varPhi && + \mu \, \varPsi) && = \lambda \, \grad \varPhi && + \mu \, \grad \varPsi
            \\
            \rot  & (\lambda \, \rvec v && + \mu \, \rvec w) && = \lambda \, \rot \rvec v  && + \mu \, \rot \rvec w
            \\
            \Div  & (\lambda \, \rvec v && + \mu \, \rvec w) && = \lambda \, \Div \rvec v  && + \mu \, \Div \rvec w
          \end{alignat*}

    \item Zérusság:
          \vspace{-.5em}
          \begin{alignat*}{1}
            \rot \grad \varPhi & \equiv \nvec
            \\
            \Div \rot \rvec v  & \equiv 0
          \end{alignat*}

    \item Deriválási szabályokhoz hasonló:
          \vspace{-.5em}
          \begin{align*}
            \grad \left( \varPhi \, \varPsi \right)
             & = \varPhi \, \grad \varPsi
            + \varPsi \, \grad \varPhi
            \\
            \Div \left( \varPhi \, \rvec v \right)
             & = \varPhi \, \Div \rvec v \,
            + \scalar{\rvec v}{\grad \varPhi}
            \\
            \rot \left( \varPhi \, \rvec v \right)
             & = \varPhi \, \rot \rvec v
            - \rvec v \times \grad \varPhi
          \end{align*}

    \item Egyéb szabályok:
          \vspace{-.5em}
          \begin{align*}
            \rot \rot \rvec v
             & =\grad \Div \rvec v
            - \Delta \rvec v
            \\
            \rot \left( \rvec u \times \rvec v \right)
             & = \rvec u \, \Div \rvec v
            - \rvec v \, \Div \rvec u
            + (\DD \rvec u) \rvec v
            - (\DD \rvec v) \rvec u
            \\
            \Div \left( \rvec u \times \rvec v \right)
             & = \; \scalar{\rvec v}{\rot \rvec u}
            - \scalar{\rvec u}{\rot \rvec v}
            \\
            \grad \left( \scalar{\rvec u}{\rvec v} \right)
             & = (\DD \rvec u) \rvec v
            + (\DD \rvec v) \rvec u
            + \rvec v \times \rot \rvec u
            + \rvec u \times \rot \rvec v
          \end{align*}
  \end{itemize}
\end{blueBox}

\begin{definition}[Skalárpotenciálosság]
  Egy $\rvec v: V \rightarrow V$ vektormező skalárpotenciálos, ha létezik olyan
  $\varphi: V \rightarrow \mathbb R$ skalármező, hogy $\rvec v = \grad \varphi$.
\end{definition}

\begin{definition}[Vektorpotenciálosság]
  Egy $\rvec v: V \rightarrow V$ vektormező vektorpotenciálos, ha létezik olyan
  $\rvec u: V \rightarrow V$ vektormező, hogy $\rvec v = \rot \rvec u$.
\end{definition}

\begin{theorem}[Örvény- és forrásmentesség]
  Legyen $\rvec v: V \rightarrow V$ mindenhol értelmezett, legalább egyszer
  differenciálható vektormező. Ekkor:
  \begin{itemize}
    \item $\rvec v$ skalárpotenciálos
          $\;\Leftrightarrow\;$
          $\rot \rvec v = \nvec$,
          hiszen $\rot \grad \varphi \equiv \nvec$,
          \hfill (\textbf{örvénymentes})
    \item $\rvec v$ vektorpotenciálos
          $\;\Leftrightarrow\;$
          $\Div \rvec v = 0$,
          hiszen $\Div \rot \rvec u \equiv 0$.
          \hfill (\textbf{forrásmentes})
  \end{itemize}
\end{theorem}

\clearpage
\begin{blueBox}[][nobreak]
  \sftitle{Potenciálfüggvények számítása:}

  Legyen $\varphi$ skalármező $\rvec v$ vektormező skalárpotenciálja. Ebben
  az esetben tudjuk, hogy $\rvec v = \grad \varphi$, vagyis
  $$
    \rvec v = \left(
    \pdv{\varphi}{x_1};
    \pdv{\varphi}{x_2};
    \dots;
    \pdv{\varphi}{x_n}
    \right)^\T
    \text.
  $$
  Ilyen esetben a potenciálfüggvény az alábbi módon számítható:
  $$
    V(\coordvec)
    = \int_0^{x_1} v_1(\xi; x_2; \dots; x_n) \dd \xi
    + \int_0^{x_2} v_2(0; \xi; \dots; x_n) \dd \xi
    + \dots
    + \int_0^{x_n} v_n(0; 0; \dots; \xi) \dd \xi
    \text.
  $$
  Legyen $\rvec u$ vektormező $\rvec v$ vektormező vektorpotenciálja.
  A potenciál számtalan alakban előállhat, ezért keressük ezt az alábbi alakban:
  $$
    \rvec u = \left( u_x; u_y; 0 \right)^\T
  $$
  A potenciál komponensei az alábbi módon számíthatóak:
  $$
    u_x = \int_0^z v_y(x; y; \zeta) \dd \zeta
    \text,
    \qquad
    u_y = \int_0^x v_z(\xi; y; 0) \dd \xi
    - \int_0^z v_x(x; y; \zeta) \dd \zeta
    \text.
  $$
\end{blueBox}

\begin{example}[][nobreak]
  Határozzuk meg a $\rvec v(\coordvec) = \ijk{yz}{zx}{xy}$ vektormező skalár- és
  vektorpotenciálját!

  \boxrule%

  A vektormező rotáciája $\rot \rvec v = \nvec$, vagyis
  $\exists V(\coordvec): \rvec v = \grad V$,
  ahol $V$ a vektormező skalárpotenciálja.
  \begin{align*}
    V(\coordvec)
     & = \int_0^x v_x(\xi; y; z) \dd \xi
    + \int_0^y v_y(0; \xi; z) \dd \xi
    + \int_0^z v_z(0; 0; \xi) \dd \xi
    \\
     & = \int_0^x y z \dd \xi
    + \int_0^y 0 \cdot z \dd \xi
    + \int_0^z 0 \cdot 0 \dd \xi
    = x y z
    \text.
  \end{align*}

  A vektormező divergenciája $\Div \rvec v = 0$, vagyis
  $\exists \rvec u(\coordvec): \rvec v = \rot \rvec u$,
  ahol $\rvec u$ a vektormező vektorpotenciálja.

  Keressük a potenciált $\rvec u = \ijk{u_x}{u_y}{0}$ alakban! Ekkor:
  \begin{align*}
    u_x & = \int_0^z v_y(x; y; \zeta) \dd \zeta
    = \int_0^z x \zeta \dd \zeta
    = \frac{1}{2} x z^2
    \text,
    \\
    u_y & = \int_0^x v_z(\xi; y; 0) \dd \xi
    - \int_0^z v_x(x; y; \zeta) \dd \zeta
    = \int_0^x \xi y \dd \xi - \int_0^z y \zeta \dd \zeta
    = \frac{1}{2} x^2 y - \frac{1}{2} y z^2
    \text.
  \end{align*}
  A potenciálok tehát:
  \vspace{-1.5em}
  $$
    V(\coordvec) = x y z
    \text,
    \qquad
    \rvec u(\coordvec) = \frac12\begin{bmatrix}
      x z^2         \\
      x^2 y - y z^2 \\
      0
    \end{bmatrix}
    \text.
  $$
\end{example}



\clearpage
\subsection{Feladatok}

\begin{enumerate}
  \item Számítsa ki az alábbi skalármezők gradiensét! Hattassa a függvényekre a
        Laplace-operátort is!
        \begin{enumerate}
          \item $\varphi(\coordvec) = 6x^y + \sin e^z$
          \item $\psi(\coordvec) = \coordvec^2 / 2$
          \item $\chi(\coordvec) = xy + xz + yz$
          \item $\omega(\coordvec) = 2 x^2 y + x z^2 + 6 y$
        \end{enumerate}

  \item Számítsa ki az alábbi vektormezők divergenciáját és rotációját!
        Hol lesznek forrás- mentesek, illetve örvénymentesek?
        \begin{enumerate}
          \item $\rvec v(\coordvec) = \coordvec$
          \item $\rvec w(\coordvec) = \ijk{3xy + z^2}{6 e^z}{-5 x^y}$
          \item $\rvec u(\coordvec) = \ijk{\ln(xy / z)}{\ln(yz / x)}{\ln(zx / y)}$
          \item $\rvec s(\coordvec) = \rvec a \| \coordvec \| + \| \rvec a \| \coordvec$
                \hfill ($\rvec a \in \Reals^3$)
        \end{enumerate}

  \item Bizonyítsa be a következő azonosságokat, amennyiben $\varphi, \psi$
        skalármezők, $\rvec v, \rvec w$ pedig vektormezők!
        \begin{enumerate}
          \item $\rot \grad \varPhi \equiv \nvec$
          \item $\Div \rot \rvec v \equiv 0$
          \item $\grad (\varPhi \varPsi) = \varPhi \grad \varPsi + \varPsi \grad \varPhi$
          \item $\Delta (\varPhi \varPsi) = (\Delta \varPhi) \varPsi + 2 \scalar{\grad \varPhi}{\grad \varPsi} + \varPsi (\Delta \varPhi)$
          \item $\Div (\varPhi \rvec v) = \scalar{\grad \varPhi}{\rvec v} + \varPhi \Div \rvec v$
          \item $\Div (\rvec v \times \rvec w) = \scalar{\rot \rvec v}{\rvec w} - \scalar{\rvec v}{\rot \rvec w}$
        \end{enumerate}

  \item Vizsgálja meg, hogy az alábbi vektormezők skalár- illetve
        vektorpotenciálisak-e! Ha igen, adja meg a potenciálfüggvényeket!
        A valós konsztansokat legyenek zérusak, valamint a vektorpotenciált --
        amennyiben létezik -- olyan módon adja meg, hogy a harmadik komponense
        zérus legyen.
        \begin{enumerate}
          \item $\rvec v(\coordvec) = \ijk{y + z}{x + z}{x + y}$
          \item $\rvec w(\coordvec) = \ijk{e^{x + \sin y}}{e^{x + \sin y} \cos y}{0}$
          \item $\rvec u(\coordvec) = \ijk{2zx^3}{3z}{-3x^2z^2}$
        \end{enumerate}
\end{enumerate}

\end{document}