\documentclass{szb-practice}

\area{Vektoranalízis}
\title{Operátorok}
\subject{Matematika G3}
\subjectCode{BMETE94BG03}
\date{Utoljára frissítve: \today}
\docno{2}

\begin{document}
\maketitle

\vspace{-1em}
\subsection{Elméleti Áttekintő}
\vspace{1em}

\begin{definition}[Nabla-operátor]
  Az $\Reals^n$-beli Descartes-koordinátarendszerben, ahol $\rvec x = (x_1;
    x_2; \dots; x_n)$ egy tetszőleges pont koordinátái, a standard bázis pedig
  $\{ \uvec e_1; \uvec e_2; \dots; \uvec e_n \}$ a Nabla egy olyan formális
  differenciáloperátor, melynek koordinátái a parciális derivált operátorok,
  vagyis:
  $$
    \nabla = \sum_{i = 1}^n \uvec e_i \pdv{}{x_i}
    =
    \begin{pmatrix}
      \displaystyle\pdv{}{x_1} &
      \displaystyle\pdv{}{x_2} &
      \hdots                   &
      \displaystyle\pdv{}{x_n}
    \end{pmatrix}^\T
    \text,
  $$
\end{definition}

\begin{blueBox}
  \sftitle{Differenciáloperátorok:}

  Legyen $\rvec v(\coordvec) : \Reals^3 \to \Reals^3$ vektormező,
  $\varphi(\coordvec): \Reals^3 \to \Reals$ skalármező, ahol
  $\coordvec$ az $\Reals^3$-beli Descartes koordináta-rendszerben
  $\coordvec = (x; y; z)$.
  \begin{center}
    \def\arraystretch{1.5}
    \newenvironment{bm}{\bgroup\renewcommand*{\arraystretch}{1.1}\begin{bmatrix}}{\end{bmatrix}\egroup}
    \newcommand{\dspl}[3]{\begin{bm}#1\\#2\\#3\end{bm}}
    \newcommand\nablavec{\dspl{\partial_x}{\partial_y}{\partial_z}}

    \begin{tabular}{*{3}{>{\centering\arraybackslash}p{3.5cm}}}
      \def\arraystretch{1}
      % &
      \bfseries Rotáció
       & \bfseries Divergencia
       & \bfseries Gradiens
      \\
      \hline
      % Jelölés & 
      $\rot \rvec v$
       & $\Div \rvec v$
       & $\grad \varphi$
      \\
      % Operátor & 
      $\nabla \times \rvec v$
       & $\scalar{\nabla}{\rvec v}$
       & $\nabla \cdot \varphi$
      \\
      % Számítás &
      $\nablavec \times \dspl{v_x}{v_y}{v_z}$
       & $\scalar{\nablavec}{\dspl{v_x}{v_y}{v_z}}$
       & $\dspl{\partial_x \varphi}{\partial_y \varphi}{\partial_z \varphi}$
      \\
      % Ért. tart. & 
      $\mathcal D_{\rvec v} = \Reals^3$
       & $\mathcal D_{\rvec v} = \Reals^3$
       & $\mathcal D_{\varphi} = \Reals^3$
      \\
      % Ért. készl. &
      $\mathcal R_{\rvec v} = \Reals^3$
       & $\mathcal R_{\rvec v} = \Reals^3$
       & $\mathcal R_{\varphi} = \Reals$
      \\
      % Ért. készl. &
      $\mathcal R_{\rot \rvec v} = \Reals^3$
       & $\mathcal R_{\Div \rvec v} = \Reals$
       & $\mathcal R_{\grad \varphi} = \Reals^3$
      \\
      % \hline
    \end{tabular}
  \end{center}

  Speciális esetek:
  \begin{itemize}
    \item ha $\Div \rvec v = 0$, akkor a vektromező forrásmentes,
    \item ha $\rot \rvec v = \nvec$, akkor a vektromező örvénymentes.
  \end{itemize}
\end{blueBox}

\begin{definition}[Laplace-operátor][nobreak]
  A Laplace-operátor egy másodrendű differenciáloperátor az $n$ dimenziós
  térben. Megadja egy skalármező gradiensének divergenciáját, azaz:
  $$
    \Delta \varphi
    = \scalar{\nabla}{\nabla} \varphi
    = \Div \grad \varphi
    \text.
  $$
\end{definition}

% \begin{blueBox}[][nobreak]
%   $! \varPhi; \varPsi : \Reals^3 \rightarrow \Reals$ skalármezők,
%   $\rvec u; \rvec v; \rvec w : \Reals^3 \rightarrow \Reals^3$
%   vektormezők, $\lambda; \mu \in \Reals$ pedig skalárok.
%   \begin{itemize}
%     \item Teljesül a linearitás:
%           \vspace{-.5em}
%           \begin{alignat*}{4}
%             \grad & (\lambda \, \varPhi && + \mu \, \varPsi) && = \lambda \, \grad \varPhi && + \mu \, \grad \varPsi
%             \\
%             \rot  & (\lambda \, \rvec v && + \mu \, \rvec w) && = \lambda \, \rot \rvec v  && + \mu \, \rot \rvec w
%             \\
%             \Div  & (\lambda \, \rvec v && + \mu \, \rvec w) && = \lambda \, \Div \rvec v  && + \mu \, \Div \rvec w
%           \end{alignat*}

%     \item Zérusság:
%           \vspace{-.5em}
%           \begin{alignat*}{1}
%             \rot \grad \varPhi & \equiv \nvec
%             \\
%             \Div \rot \rvec v  & \equiv 0
%           \end{alignat*}

%     \item Deriválási szabályokhoz hasonló:
%           \vspace{-.5em}
%           \begin{align*}
%             \grad \left( \varPhi \, \varPsi \right)
%              & = \varPhi \, \grad \varPsi
%             + \varPsi \, \grad \varPhi
%             \\
%             \Div \left( \varPhi \, \rvec v \right)
%              & = \varPhi \, \Div \rvec v \,
%             + \scalar{\rvec v}{\grad \varPhi}
%             \\
%             \rot \left( \varPhi \, \rvec v \right)
%              & = \varPhi \, \rot \rvec v
%             - \rvec v \times \grad \varPhi
%           \end{align*}

%     \item Egyéb szabályok:
%           \vspace{-.5em}
%           \begin{align*}
%             \rot \rot \rvec v
%              & =\grad \Div \rvec v
%             - \Delta \rvec v
%             \\
%             \rot \left( \rvec u \times \rvec v \right)
%              & = \rvec u \, \Div \rvec v
%             - \rvec v \, \Div \rvec u
%             + (\DD \rvec u) \rvec v
%             - (\DD \rvec v) \rvec u
%             \\
%             \Div \left( \rvec u \times \rvec v \right)
%              & = \; \scalar{\rvec v}{\rot \rvec u}
%             - \scalar{\rvec u}{\rot \rvec v}
%             \\
%             \grad \left( \scalar{\rvec u}{\rvec v} \right)
%              & = (\DD \rvec u) \rvec v
%             + (\DD \rvec v) \rvec u
%             + \rvec v \times \rot \rvec u
%             + \rvec u \times \rot \rvec v
%           \end{align*}
%   \end{itemize}
% \end{blueBox}

\clearpage
\subsection{Feladatok}

\begin{enumerate}
  \item
\end{enumerate}

% \begin{enumerate}
%   \item Adja meg a $\rvec v(\coordvec) = \coordvec \cdot |\coordvec|^3$ vektorfüggvény
%         divergenciáját és rotációját! Hattassa a Laplace-operátort
%         $\operatorname{div} \rvec v$-re!

%   \item Határozza meg a $\rvec v(\coordvec)$ vektormező divergenciáját, illetve
%         rotációját!
%         $$
%           \rvec v(\coordvec) = \begin{bmatrix}
%             \ln(\sfrac{xy}{z}) \\
%             \ln(\sfrac{yz}{x}) \\
%             \ln(\sfrac{zx}{y}) \\
%           \end{bmatrix}
%         $$

%   \item Igazolja, hogy tetszőleges Joung-tételt kielégítő vektormező
%         rotációjának divergenciája a nullvektor, vagyis
%         $$
%           \nabla \cdot (\nabla \times \rvec v) \equiv \nvec
%           \text.
%         $$

%   \item Igazolja az alábbi azonosságot, amennyiben $f$ tetszőleges skalár-,
%         $\rvec v$ pedig tetszőleges vektormező:
%         $$
%           \nabla \cdot (f \cdot \rvec v)
%           = f \cdot (\nabla \cdot \rvec v)
%           + \nabla f \cdot \rvec v
%           \text.
%         $$
% \end{enumerate}
\end{document}