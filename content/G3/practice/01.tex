\documentclass{szb-practice}

\title{Ismétlés, operátorok}
\area{Vektoranalízis}
\subject{Matematika G3}
\subjectCode{BMETE94BG03}
\date{Utoljára frissítve: \today}
\docno{1}

\begin{document}
\maketitle

\vspace{-1em}
\subsection{Elméleti Áttekintő}
\vspace{1em}

\begin{definition}[Vektortér]
  Legyen $V$ nemüres halmaz, és $\circ, +$ két művelet, $T$ test.
  A $(V; +, \circ)$ a $T$ test feletti vektortér, ha teljesülnek az alábbiak:
  \begin{enumerate}[itemsep=-.33em]
    \item $(V; +)$ Abel-csoport,
          \begin{itemize}[itemsep=-.33em]
            \item $\forall \rvec x; \rvec y; \rvec z \in V:
                    (\rvec x + \rvec y) + \rvec z
                    = \rvec x + (\rvec y + \rvec z)$,
                  \hfill (asszociativitás)

            \item $\exists \nvec \in V: \forall \rvec x \in V:
                    \nvec + \rvec x = \rvec x + \nvec = \rvec x$,
                  \hfill (zéruselem)

            \item $\forall \rvec x \in V: \exists -\rvec x \in V:
                    \rvec x + (-\rvec x) = (-\rvec x) + \rvec x = \nvec$,
                  \hfill (inverzelem)

            \item $\forall \rvec x; \rvec y \in V:
                    \rvec x + \rvec y = \rvec y + \rvec x$.
                  \hfill (kommutativitás)
          \end{itemize}

    \item $\forall \lambda; \mu \in T \; \land \; \forall \rvec x \in V:
            (\lambda \circ \mu) \circ \rvec x
            = \lambda \circ (\mu \circ \rvec x)$,

    \item ha $\varepsilon$ a $T$-beli egységelem, akkor
          $\forall \rvec x \in V: \varepsilon \circ \rvec x = \rvec x$,

    \item teljesül a disztributivitás:
          \begin{itemize}[itemsep=-.33em]
            \item $\forall \lambda \in T \; \land \; \forall \rvec x; \rvec y \in V:
                    \lambda \circ (\rvec x + \rvec y)
                    = \lambda \circ \rvec x + \lambda \circ \rvec y$,

            \item $\forall \lambda; \mu \in T \; \land \; \forall \rvec x \in V:
                    (\lambda + \mu) \circ \rvec x
                    = \lambda \circ \rvec x + \mu \circ \rvec x$.
          \end{itemize}
  \end{enumerate}
\end{definition}

\begin{example}
  A \textbf{legfeljebb másodfokú polinomok} halmaza az összeadásra és a
  skalárral való szorzásra nézve vektortér.
\end{example}

\begin{definition}[Lineáris függetlenség]
  A $(V; +; \circ)$ vektortér $\rvec v_1, \rvec v_2, \ldots, \rvec v_n$
  vektorait lineárisan függetlennek mondjuk, ha a
  $$
    \lambda_1 \rvec v_1
    + \lambda_2 \rvec v_2
    + \ldots
    + \lambda_n \rvec v_n
    = \nvec
  $$
  vektoregyenletnek \textbf{csak a triviális megoldása} létezik, azaz
  $\lambda_1 = \lambda_2 = \cdots = \lambda_n = 0$.

  Ha az egyenletnek nem csak a triviális megoldása létezik, akkor a vektorok
  lineárisan függőek.
\end{definition}

\begin{example}[][nobreak]
  A legfeljebb másodfokú polinomok vektorterében a
  $$
    \Big\{\; x^2 - x - 2 ;\; x + 1 ;\; x^2 + x \;\Big\}
  $$
  vektorhármas lineárisan összefüggő, hiszen:
  $$
    (1) \cdot (x^2 - x - 2)
    + (2) \cdot (x + 1)
    + (-1) \cdot (x^2 + x)
    = 0
    \text.
  $$
\end{example}

\begin{definition}[Altér]
  Legyen $(V; +; \circ)$ $\Reals$ feletti vektortér, valamint
  $\emptyset \neq L \subset V$. $L$-t altérnek nevezzük a $V$-ben, ha
  $(L; +; \circ)$ ugyancsak vektortér.
\end{definition}

\begin{example}
  A legfeljebb másodfokú polinomok vektorterében az olyan polinomok halmaza,
  melyekben a $x$-es tag együtthatója nulla, altér, hiszen a szokásos
  műveletekre zárt.
  \begin{enumerate}
    \item Összeadásra zárt:
          $$
            (a_2 x^2 + \underline{0 x} + a_0) + (b_2 x^2 + \underline{0 x} + b_0)
            = (a_2 + b_2) x^2 + \underline{0 x} + (a_0 + b_0)
            \text.
          $$

    \item Skalárral való szorzásra zárt:
          $$
            \lambda \cdot (a_2 x^2 + \underline{0 x} + a_0)
            = (\lambda a_2) x^2 + \underline{0 x} + (\lambda a_0)
            \text.
          $$
  \end{enumerate}
\end{example}

\begin{definition}[Generátorrendszer]
  Legyen $V$ vektortér, valamint $\emptyset \neq G \subset V$. $G$ által
  generált altérnek nevezzük azt a legszűkebb alteret, amely tartalmazza $G$-t.
  Jele: $\mathcal L(G)$.

  $G$ generátorrendszere $V$-nek, ha $\mathcal L(G) = V$.
\end{definition}

\begin{example}
  A legfeljebb másodfokú polinomok egy lehetséges \textbf{generátorrendszere}:
  $$
    \big\{
    \;1;
    \;1 + x;
    \;x + x^2;
    \;x^2
    \;\big\}
    \text.
  $$
\end{example}

\begin{definition}[Bázis]
  A $V$ vektortér egy lineárisan független generátorrendszerét a $V$
  bázisának nevezzük.
\end{definition}

\begin{example}
  A legfeljebb másodfokú polinomok egy lehetséges \textbf{bázisa}:
  $$
    \big\{
    \;1;
    \;x;
    \;x^2
    \;\big\}
    \text.
  $$
\end{example}

\begin{definition}[Vektortér dimenziója]
  Végesen generált vektortér dimenzióján a bázisainak közös tagszámát értjük.
\end{definition}

\begin{example}[][nobreak]
  A legfeljebb másodfokú polinomok vektortere \textbf{dimenziója 3}, hiszen
  tetszőleges bázisának három eleme van. Egy másik lehetséges bázis:
  $$
    \big\{
    \;1 + x + x^2;
    \;x + x^2;
    \;x^2
    \;\big\}
    \text.
  $$
\end{example}

\begin{definition}[Lineáris leképezés]
  Legyenek $V_1$ és $V_2$ ugyanazon $T$ test feletti vektorterek. Legyen
  $\varphi: V_1 \rightarrow V_2$ leképezés, melyet lineáris leképezésnek
  nevezünk, ha tetszőleges két $V_1$-beli vektor ($\forall \rvec a; \rvec b \in
    V_1$) és $T$-beli skalár ($\lambda \in T$) esetén teljesülnek az alábbiak:

  \def\arraystretch{1.5}
  \begin{tabular}{>{\bullet\;}l>{$\quad \sim \quad$}l}
    $\varphi(\rvec a + \rvec b) = \varphi(\rvec a) + \varphi(\rvec b)$
     & additív (összegre tagonként hat), \\
    $\varphi(\lambda \rvec a) = \lambda \varphi(\rvec a)$
     & homogén (skalár kiemelhető).
  \end{tabular}
\end{definition}

\begin{definition}[Lineáris leképezés rangja]
  Egy lineáris leképezés rangjának nevezzük a képtér dimenzióját.
  $\rg \varphi = \dim \varphi(V_1)$.
\end{definition}

\begin{definition}[Leképezés magtere]
  Legyen $\varphi: V_1 \rightarrow V_2$ lineáris leképezés, ekkor a
  $$
    \ker \varphi := \{
    \rvec v \; | \; \rvec v \in V_1 \land \varphi(\rvec v) = \nvec
    \}
  $$
  halmazt a leképezés magterének nevezzük.
\end{definition}

\begin{definition}[Leképezés defektusa]
  A magtér dimenzióját defektusnak nevezzük, és $\operatorname{def} \varphi$-vel
  jelöljük.
\end{definition}

\begin{theorem}[Rang-nullitás tétele]
  Legyen $V_1$ véges dimenziós vektortér, $\varphi: V_1 \rightarrow V_2$
  lineáris leképezés, ekkor
  $$
    \rg \varphi + \operatorname{def} \varphi = \dim V_1
    \text.
  $$
\end{theorem}

\begin{blueBox}[Lineáris leképezés mátrixa][nobreak]
  Legyenek $V_1$ és $V_2$ ugyanazon test feletti vektorterek, és $\dim V_1 = n$,
  valamint $\dim V_2 = k$. Legyen $\{ \rvec a_1; \rvec a_2; \ldots; \rvec a_n \}$
  bázis $V_1$-ben, és $\{ \rvec b_1; \rvec b_2; \ldots; \rvec b_k \}$ bázis
  $V_2$-ben. Legyen $\varphi: V_1 \rightarrow V_2$ lineáris leképezés, ekkor
  $$
    \varphi(\rvec a_i)
    = \sum_{j=1}^{k} \alpha_{ji} \rvec b_j
    \quad\Rightarrow\quad
    \rmat A := \begin{bmatrix}
      \alpha_{11} & \alpha_{12} & \cdots & \alpha_{1n} \\
      \alpha_{21} & \alpha_{22} & \cdots & \alpha_{2n} \\
      \vdots      & \vdots      & \ddots & \vdots      \\
      \alpha_{k1} & \alpha_{k2} & \cdots & \alpha_{kn}
    \end{bmatrix}_{k \times n}
    \text.
  $$
  Az $\rmat A$ mátrixot $\varphi$ leképezést reprezentáló mátrixnak hívjuk,
  segítségével tetszőleges $\rvec x \in V_1$ képét meghatározhatjuk. Legyenek
  $(\xi_1, \xi_2, \ldots, \xi_n)$ az $\rvec x$ koordinátái, ekkor a képe:
  $$
    \varphi(\rvec x)
    = \sum_{i=1}^{n} \xi_i \,\varphi(\rvec a_i)
    =
    \begin{bmatrix}
      \alpha_{11} & \alpha_{12} & \cdots & \alpha_{1n} \\
      \alpha_{21} & \alpha_{22} & \cdots & \alpha_{2n} \\
      \vdots      & \vdots      & \ddots & \vdots      \\
      \alpha_{k1} & \alpha_{k2} & \cdots & \alpha_{kn}
    \end{bmatrix}
    \begin{bmatrix}
      \xi_1  \\
      \xi_2  \\
      \vdots \\
      \xi_n
    \end{bmatrix}
    \text.
  $$
\end{blueBox}

\begin{statement}
  Tetszőleges lineáris leképezés rangja megegyezik bármely bázisra vonatkozó
  mátrixreprezentációjának rangjával. $\varphi: V_1 \rightarrow V_2$, $\dim V_1
    = m$, $\dim V_2 = n \Rightarrow \varphi \leftrightarrow \rmat A$, $\rmat A
    \in \mathcal M_{n \times m}$, $\rg \varphi = \rg \rmat A$.
\end{statement}

\begin{example}
  Tekintsük a $\varphi: \mathbb R^2 \rightarrow \mathbb R^2$,
  $(x; y) \mapsto (y + x; 2x)$ leképezést.

  A leképezés lineáris, hiszen a \textbf{összegre tagonként hat}:
  $$
    \varphi \begin{bmatrix}
      x_1 + x_2 \\ y_1 + y_2
    \end{bmatrix} = \begin{bmatrix}
      (y_1 + y_2) + (x_1 + x_2) \\ 2 (x_1 + x_2)
    \end{bmatrix} = \begin{bmatrix}
      y_1 + x_1 \\ 2x_1
    \end{bmatrix} + \begin{bmatrix}
      y_2 + x_2 \\ 2x_2
    \end{bmatrix} = \varphi \begin{bmatrix}
      x_1 \\ y_1
    \end{bmatrix} + \varphi \begin{bmatrix}
      x_2 \\ y_2
    \end{bmatrix}
    \text,
  $$
  valamint a skalárral való szorzás esetén a \textbf{skalár kiemelhető}:
  $$
    \varphi \begin{bmatrix}
      \alpha x \\ \alpha y
    \end{bmatrix} = \begin{bmatrix}
      \alpha y + \alpha x \\ 2\alpha x
    \end{bmatrix} = \begin{bmatrix}
      \alpha(y + x) \\ \alpha(2x)
    \end{bmatrix} = \alpha \begin{bmatrix}
      y + x \\ 2x
    \end{bmatrix} = \alpha \varphi \begin{bmatrix}
      x \\ y
    \end{bmatrix}
    \text.
  $$
  A leképezés mátrixa a standard bázisra vonatkozóan:
  $$
    \varphi \begin{bmatrix} 1 \\ 0 \end{bmatrix}
    = \begin{bmatrix} 1 + 0 \\ 2 \cdot 1 \end{bmatrix}
    = \begin{bmatrix} 1 \\ 2 \end{bmatrix}
    \quad \text{és} \quad
    \varphi \begin{bmatrix} 0 \\ 1 \end{bmatrix}
    = \begin{bmatrix} 1 \\ 0 \end{bmatrix}
    = \begin{bmatrix} 0 + 1 \\ 2 \cdot 0 \end{bmatrix}
    \quad \Rightarrow \quad
    \rmat A = \begin{bmatrix}
      1 & 1 \\
      2 & 0
    \end{bmatrix}
    \text.
  $$
  A \textbf{leképezés rangja} a mátrix rangjával egyezik meg:
  $$
    \rg \varphi = \rg \rmat A
    = \rg \begin{bmatrix}
      1 & 1 \\
      2 & 0
    \end{bmatrix}
    = \rg \begin{bmatrix}
      1 & 1 \\
      0 & 2
    \end{bmatrix}
    = \rg \begin{bmatrix}
      1 & 0 \\
      0 & 1
    \end{bmatrix}
    = 2
    \text.
  $$
  A leképezés \textbf{defektusa} a rang-nullitás tétele alapján:
  $$
    \operatorname{def} \varphi
    = \dim \mathbb R^2 - \rg \varphi
    = 2 - 2
    = 0
    \text.
  $$
  A leképezés \textbf{inverzének} mátrixa:
  $$
    \rmat A^{-1}
    = \frac{1}{\det \rmat A} \begin{bmatrix}
      0  & -1 \\
      -2 & 1
    \end{bmatrix}
    = -\frac{1}{2} \begin{bmatrix}
      0  & -1 \\
      -2 & 1
    \end{bmatrix}
    = \begin{bmatrix}
      0 & 1/2  \\
      1 & -1/2
    \end{bmatrix}
    \text.
  $$
  A $(4;2)$ vektor ősképe:
  \vspace{-1em}
  $$
    \varphi^{-1} \begin{bmatrix} 4 \\ 2 \end{bmatrix}
    = \begin{bmatrix}
      0 & 1/2  \\
      1 & -1/2
    \end{bmatrix} \begin{bmatrix}
      4 \\ 2
    \end{bmatrix}
    = \begin{bmatrix}
      1 \\ 3
    \end{bmatrix}
    \text.
  $$
\end{example}

\begin{definition}[Bázistranszformáció][nobreak]
  Legyenek $\mathcal B = \{ \rvec b_1; \rvec b_2; \ldots; \rvec b_n \}$ és
  $\hat{\mathcal B} = \{ \hat{\rvec b}_1; \hat{\rvec b}_2; \ldots;
    \hat{\rvec b}_n \}$ bázisok $V$-ben. Ekkor a $\mathcal B \rightarrow
    \hat{\mathcal B}$ bázistranszformáció $\rmat T$ mátrixa a következőképpen
  írható fel:
  $$
    \left.\begin{array}{rl}
      \hat{\rvec b}_1 & = t_{11} \rvec b_1 + t_{21} \rvec b_2 + \ldots + t_{n1} \rvec b_n \\
      \hat{\rvec b}_2 & = t_{12} \rvec b_1 + t_{22} \rvec b_2 + \ldots + t_{n2} \rvec b_n \\
                      & \vdots                                                            \\
      \hat{\rvec b}_j & = t_{1j} \rvec b_1 + t_{2j} \rvec b_2 + \ldots + t_{nj} \rvec b_n \\
                      & \vdots                                                            \\
      \hat{\rvec b}_n & = t_{1n} \rvec b_1 + t_{2n} \rvec b_2 + \ldots + t_{nn} \rvec b_n
    \end{array}\right\}
    \quad\Rightarrow\quad
    \rmat T = \begin{bmatrix}
      t_{11} & t_{12} & \cdots & t_{1n} \\
      t_{21} & t_{22} & \cdots & t_{2n} \\
      \vdots & \vdots & \ddots & \vdots \\
      t_{n1} & t_{n2} & \cdots & t_{nn}
    \end{bmatrix}
  $$
\end{definition}

\begin{note}
  Jelölje egy adott vektor koordinátáit a $\mathcal B$ bázisban $\rvec x$, a
  $\hat{\mathcal B}$ bázisban pedig $\rvec x'$. Legyen továbbá $\rmat T$ a
  $\mathcal B \rightarrow \hat{\mathcal B}$ bázistranszformációs mátrixa.

  Ekkor a két koordinátarendszer közötti kapcsolatot a következő egyenletek
  írják le:
  $$
    \rvec x = \rmat T \rvec x'
    \quad\text{és}\quad
    \rvec x' = \rmat T^{-1} \rvec x
    \text.
  $$

  A $\rmat T$ mátrix oszlopai a $\hat{\mathcal B}$ bázisvektorok
  $\mathcal B$ bázisra vonatkozó koordinátái.
\end{note}

\begin{example}
  Egy vektor standard normális bázisban felírt alakja: $\rvec x(2; 1)$. Adjuk
  meg a $\hat{\rvec b}_1(1; 0)$ és $\hat{\rvec b}_2(1; 1)$ bázisra való áttérés
  mátrixát, valamint a vektor koordinátáit ebbern a bázisban!

  A transzformációs mátrix és ennek inverze:
  $$
    \rmat T = \begin{bmatrix}
      1 & 1 \\
      0 & 1
    \end{bmatrix}
    \qquad \text{és} \qquad
    \rmat T^{-1} = \begin{bmatrix}
      1 & -1 \\
      0 & 1
    \end{bmatrix}
    \text.
  $$

  A vektor koordinátái az új bázisban:
  $$
    \rvec x' = \rmat T^{-1} \rvec x = \begin{bmatrix}
      1 & -1 \\
      0 & 1
    \end{bmatrix} \begin{bmatrix}
      2 \\ 1
    \end{bmatrix} = \begin{bmatrix}
      1 \\ 1
    \end{bmatrix}
    \text.
  $$

  Ellenőrzés:
  $$
    \rvec x = \rmat T \rvec x' = \begin{bmatrix}
      1 & 1 \\
      0 & 1
    \end{bmatrix} \begin{bmatrix}
      1 \\ 1
    \end{bmatrix} = \begin{bmatrix}
      2 \\ 1
    \end{bmatrix}
    \text.
  $$
\end{example}

\begin{theorem}[Lineáris leképezés mátrixa új bázisban]
  Legyen $\varphi: V \rightarrow V$ lineáris leképezés,
  $\mathcal B = \{ \rvec b_1; \rvec b_2; \ldots; \rvec b_n \}$ és
  $\hat{\mathcal B} = \{ \hat{\rvec b}_1; \hat{\rvec b}_2; \ldots;
    \hat{\rvec b}_n \}$ bázisok $V$-ben.
  A $\varphi$ leképezés $\mathcal B$ bázisra vonatkozó mátrixa $\rmat A$,
  $\hat{\mathcal B}$ bázisra vonatkozó mátrixa pedig $\hat{\rmat A}$. Jelölje
  $\rmat T$ a $\mathcal B$ bázisról a $\hat{\mathcal B}$ bázisra való áttérés
  mátrixát, ekkor
  $$
    \hat{\rmat A} = \rmat T^{-1} \rmat A \rmat T
    \text.
  $$
\end{theorem}

% \begin{note}
%   A $\rmat A$ és $\hat{\rmat A}$ mátrix hasonló.
% \end{note}

\begin{note}[][nobreak]
  Jelölje egy adott vektor koordinátáit a $\mathcal B$ bázisban $\rvec x$, a
  $\hat{\mathcal B}$ bázisban pedig $\rvec x'$.

  Legyen $\varphi$ leképezés $\mathcal B$-re vonatkoztatott mátrixa
  $\rmat A$, $\hat{\mathcal B}$-re vonatkoztatott mátrixa $\hat{\rmat A}$.

  A $\mathcal B \rightarrow \hat{\mathcal B}$ áttérés mátrixa $\rmat T$. A vektor
  képei az adott bázisokban: $\rvec y = \varphi(\rvec x)$,
  $\rvec y' = \varphi(\rvec x')$.

  Ekkor az alábbi gondolatmenet alapján könnyebben megérthetjük, hogy melyik
  oldalról milyen mátrixszal kell szoroznunk különböző transzformációk során:
  \vspace{-2.5em}
  \begin{multicols}{2}
    \begin{align*}
      \rvec x & = \rmat T \rvec x'
      \\
      \rvec y & = \rmat T \rvec y'
      \\
      \rvec y & = \rmat T \hat{\rmat A} \rvec x'
      \\
      \rvec y & = \underbrace{\rmat T \hat{\rmat A} \rmat T^{-1}}_{\rmat A} \rvec x
    \end{align*}

    \begin{align*}
      \rvec x' & = \rmat T^{-1} \rvec x
      \\
      \rvec y' & = \rmat T^{-1} \rvec y
      \\
      \rvec y' & = \rmat T^{-1} \rmat A \rvec x
      \\
      \rvec y' & = \underbrace{\rmat T^{-1} \rmat A\rmat T}_{\hat{\rmat A}} \rvec x'
    \end{align*}
  \end{multicols}
\end{note}

\begin{example}
  Tekintsük a $\varphi: \mathbb R^2 \rightarrow \mathbb R^2$,
  $(x; y) \mapsto (y + x; 2x)$ leképezést. Adjuk meg a leképezés a standard
  normális, illetve a $\hat{\rvec b}_1(1; 0)$ és $\hat{\rvec b}_2(1; 1)$ bázisra
  vonatkoztatott mátrixát!

  Korábban már minden szükséges mátrixot meghatároztunk:
  $$
    \rmat A = \begin{bmatrix}
      1 & 1 \\
      2 & 0
    \end{bmatrix}
    \text,\qquad
    \rmat T = \begin{bmatrix}
      1 & 1 \\
      0 & 1
    \end{bmatrix}
    \text,\qquad
    \rmat T^{-1} = \begin{bmatrix}
      1 & -1 \\
      0 & 1
    \end{bmatrix}
    \text.
  $$
  A leképezés mátrixa a $\{\hat{\rvec b}_1, \hat{\rvec b}_2\}$ bázisra
  vonatkoztatva:
  $$
    \hat{\rmat A} = \rmat T^{-1} \rmat A \rmat T = \begin{bmatrix}
      1 & -1 \\
      0 & 1
    \end{bmatrix} \begin{bmatrix}
      1 & 1 \\
      2 & 0
    \end{bmatrix} \begin{bmatrix}
      1 & 1 \\
      0 & 1
    \end{bmatrix} = \begin{bmatrix}
      -1 & 0 \\
      2  & 2
    \end{bmatrix}
    \text.
  $$
\end{example}

\begin{definition}[Sajátértékek és sajátvektorok]
  Legyen $V$ a $T$ test feletti vektortér, $\rvec v \in V$, $\rvec v \neq
    \nvec$. $\rvec v$-t a $\varphi: V \rightarrow V$ lineáris leképezés
  sajátvektorának mondjuk, ha önmaga skalárszorosába megy át a leképezés
  során, azaz $\varphi(\rvec v) = \lambda \rvec v$,  $\lambda \in T$.
  $\lambda$-t a $\rvec v$ sajátvektorhoz tartozó sajátértéknek mondjuk.
\end{definition}

\begin{note}
  Ha a $\rvec v$ sajátvektora a $\varphi$-nek, akkor annak skalárszorosa is az.
\end{note}

\begin{theorem}[Sajátértékek számítása]
  Az $\rmat A \in \mathcal M_{n \times n}$ mátrix sajátértékei a
  karakterisztikus egyenlet gyökei:
  $$
    \det(\rmat A - \lambda \imat) = 0
    \text.
  $$
\end{theorem}

\begin{example}[][nobreak]
  Határozzuk meg az $\rmat A$ mátrix sajátértékeit és sajátvektorait!

  $$
    \rmat A = \begin{bmatrix}
      2  & -1 \\
      -1 & 2
    \end{bmatrix}
    \qquad
    \rmat A - \lambda \imat = \begin{bmatrix}
      2  & -1 \\
      -1 & 2
    \end{bmatrix} - \lambda \begin{bmatrix}
      1 & 0 \\
      0 & 1
    \end{bmatrix} = \begin{bmatrix}
      2 - \lambda & -1          \\
      -1          & 2 - \lambda
    \end{bmatrix}
  $$

  A karakterisztikus egyenlet, és ennek alapján a sajátértékek:
  $$
    \det(\rmat A - \lambda \imat) = (2 - \lambda)^2 - 1 = 0
    \quad \Rightarrow \quad
    \lambda_1 = 1 \text, \quad \lambda_2 = 3 \text.
  $$

  A sajátvektorokat az $(\rmat A - \lambda_i \imat) \rvec v_i = 0$
  egyenlet segítségével számíthatjuk ki:

  \begin{enumerate}
    \item A $\lambda_1 = 1$ sajátértékhez tartozó sajátvektor:
          $$
            \begin{bmatrix}
              1  & -1 \\
              -1 & 1
            \end{bmatrix} \begin{bmatrix}
              x \\
              y
            \end{bmatrix} = \begin{bmatrix}
              0 \\
              0
            \end{bmatrix}
            \quad \Rightarrow \quad
            x = y
            \quad \Rightarrow \quad
            \rvec v_1 = t_1 \begin{bmatrix}
              1 \\
              1
            \end{bmatrix}
          $$

    \item A $\lambda_2 = 3$ sajátértékhez tartozó sajátvektor:
          $$
            \begin{bmatrix}
              -1 & -1 \\
              -1 & -1
            \end{bmatrix} \begin{bmatrix}
              x \\
              y
            \end{bmatrix} = \begin{bmatrix}
              0 \\
              0
            \end{bmatrix}
            \quad \Rightarrow \quad
            x = -y
            \quad \Rightarrow \quad
            \rvec v_2 = t_2 \begin{bmatrix}
              1 \\
              -1
            \end{bmatrix}
          $$
  \end{enumerate}
\end{example}

\begin{statement}
  Ha $\rmat A$ háromszögmátrix, akkor a sajátértékek a főátlóbeli elemek.
\end{statement}

\begin{definition}[Diagonalizálhatóság]
  Az $n \times n$-es $\rmat A$ mátrix diagonalizálható, ha hasonló egy
  diagonális mátrixhoz, azaz ha létezik olyan $\rmat Λ$ diagonális mátrix és egy
  $\rmat T$ invertálható mátrix, hogy
  $$
    \rmat Λ = \rmat T^{-1} \rmat A \rmat T
    \text.
  $$
\end{definition}

\begin{theorem}[Diagonizálhatóság szükséges és elégséges feltétele]
  Legyen $\rmat A$ egy $n \times n$-es mátrix. Az $\rmat A$ mátrix akkor és
  csak akkor diagonalizálható, ha létezik $n$ darab lineárisan független
  sajátvektora. Ekkor a diagonális mátrix az $\rmat A$ sajátértékeiből, míg
  a $\rmat T$ transzformációs mátrix $\rmat A$ sajátvektoraiból áll:
  $$
    \rmat Λ
    = \rmat T^{-1} \rmat A \rmat T
    = \begin{bmatrix}
      \lambda_1 & 0         & \cdots & 0         \\
      0         & \lambda_2 & \cdots & 0         \\
      \vdots    & \vdots    & \ddots & \vdots    \\
      0         & 0         & \cdots & \lambda_n
    \end{bmatrix}
    \quad \text{és} \quad
    \rmat T = \begin{bmatrix}
      \rvec v_1 & \rvec v_2 & \cdots & \rvec v_n
    \end{bmatrix}
    \text.
  $$
\end{theorem}

\begin{blueBox}
  \sftitle{Invariáns mennyiségek:}

  Legyen $\rmat A$ egy $3 \times 3$-as mátrix, amelynek sajátértékei $\lambda_1$,
  $\lambda_2$ és $\lambda_3$. Ekkor az alábbi mennyiségek bármely $\rmat A$-hoz
  hasonló mátrix esetén invariánsak:
  \begin{itemize}[itemsep=-.33em]
    \item $
            I_1
            = \operatorname{tr} \rmat A
            = \lambda_1 + \lambda_2 + \lambda_3
          $,

    \item $
            I_2
            = \frac{1}{2} \left(
            (\operatorname{tr} \rmat A)^2 - \operatorname{tr} (\rmat A^2)
            \right)
            = \lambda_1 \lambda_2 + \lambda_2 \lambda_3 + \lambda_3 \lambda_1
          $,

    \item $
            I_3
            = \det \rmat A
            = \lambda_1 \lambda_2 \lambda_3
          $.
  \end{itemize}
\end{blueBox}

\begin{note}
  A karakterisztikus polinom a skalárinvariánsok segítségével:
  $$
    p(\lambda) = \lambda^3 - I_1 \lambda^2 + I_2 \lambda - I_3
    \text.
  $$
\end{note}

\begin{example}[][nobreak]
  Diagonizáljuk az $\rmat A = \begin{bmatrix}
      1 & 1 & 0 \\
      0 & 2 & 1 \\
      0 & 0 & 3
    \end{bmatrix}$ mátrixot és adjuk a skalárinvariánsait!

  Mivel a mátrix felső háromszögmátrix, sajátértékei a főátló elemei:
  $\lambda_1 = 1$, $\lambda_2 = 2$ és $\lambda_3 = 3$. A skalárinvariánsok:
  \begin{align*}
    I_1 & = \lambda_1 + \lambda_2 + \lambda_3 = 1 + 2 + 3 = 6
    \text,                                                                  \\
    I_2 & = \lambda_1 \lambda_2 + \lambda_2 \lambda_3 + \lambda_3 \lambda_1
    = 1 \cdot 2 + 2 \cdot 3 + 3 \cdot 1 = 11
    \text,                                                                  \\
    I_3 & = \lambda_1 \lambda_2 \lambda_3 = 1 \cdot 2 \cdot 3 = 6
    \text.
  \end{align*}
\end{example}

\begin{definition}[Szimmetrikus mátrix]
  Egy $\rmat A \in \mathcal M_{n \times n}$ mátrix szimmetrikus, ha
  $\rmat A = \rmat A^\T$.
\end{definition}

\begin{definition}[Antiszimmetrikus mátrix]
  Egy $\rmat A \in \mathcal M_{n \times n}$ mátrix antiszimmetrikus, ha
  $\rmat A = -\rmat A^\T$.
\end{definition}

\begin{blueBox}
  \sftitle{Kvadratikus mátrix felbontása szimmetrikus és antiszimmetrikus részekre:}
  $$
    \rmat A =
    \underbrace{\frac{1}{2}(\rmat A + \rmat A^\T)}_{\text{Szimmetrikus}}
    + \underbrace{\frac{1}{2}(\rmat A - \rmat A^\T)}_{\text{Antiszimmetrikus}}
  $$
\end{blueBox}

\begin{blueBox}
  \sftitle{Függvénytípusok:}

  Ha $\varphi : \Reals^n \to \Reals$, akkor $\varphi$ skalármező.

  Ha $\rvec v : \Reals^n \to \Reals^k$, akkor $\rvec v$ vektormező.
\end{blueBox}

\begin{blueBox}
  \sftitle{Koordinátavektor jelölése:}

  Jelölje a $\coordvec$ koordinátavektor az $\Reals^n$-beli koordináták
  rendezett $n$-esét!
\end{blueBox}

\begin{definition}[Gradiens]
  Legyen $\varphi: \Reals^n \to \Reals$ differenciálható skalármező. Ekkor a
  $\varphi$ gradiensének nevezzük a következő vektormezőt:
  $$
    \grad \varphi(\coordvec) = \begin{bmatrix}
      \partial_1 \varphi(\coordvec) \\
      \partial_2 \varphi(\coordvec) \\
      \vdots                        \\
      \partial_n \varphi(\coordvec)
    \end{bmatrix}
    \text.
  $$
\end{definition}

\begin{note}
  A gradiens a legnagyobb növekedés irányát mutatja meg, nagysága pedig a
  növekedés mértékét. Mindig merőleges a szintfelületekre.
\end{note}

\begin{example}
  Adjuk meg a $\varphi(\coordvec) = xyz$ skalármező gradiensét!

  $$
    \grad \varphi(\coordvec)
    = \begin{bmatrix}
      \partial_1 \varphi(\coordvec) \\
      \partial_2 \varphi(\coordvec) \\
      \partial_3 \varphi(\coordvec)
    \end{bmatrix}
    = \begin{bmatrix}
      yz \\
      xz \\
      xy
    \end{bmatrix}
    \text.
  $$
\end{example}

\begin{definition}[Jacobi-mátrix]
  Legyen $\rvec v: \Reals^n \to \Reals^k$ vektormező. Ekkor a $\rvec v$
  Jacobi-mátrixának nevezzük a következő $k \times n$-es mátrixot:
  \def\arraystretch{1.5}%
  $$
    \rmat J = \DD \rvec v = \begin{bmatrix}
      \displaystyle\pdv{v_1(\coordvec)}{x_1} & \displaystyle\pdv{v_1(\coordvec)}{x_2} & \cdots & \displaystyle\pdv{v_1(\coordvec)}{x_n} \\
      \displaystyle\pdv{v_2(\coordvec)}{x_1} & \displaystyle\pdv{v_2(\coordvec)}{x_2} & \cdots & \displaystyle\pdv{v_2(\coordvec)}{x_n} \\
      \vdots                                 & \vdots                                 & \ddots & \vdots                                 \\
      \displaystyle\pdv{v_k(\coordvec)}{x_1} & \displaystyle\pdv{v_k(\coordvec)}{x_2} & \cdots & \displaystyle\pdv{v_k(\coordvec)}{x_n}
    \end{bmatrix} = \begin{bmatrix}
      \grad^\T v_1(\coordvec) \vphantom{\displaystyle\pdv{v_1}{x_1}} \\
      \grad^\T v_2(\coordvec) \vphantom{\displaystyle\pdv{v_1}{x_1}} \\
      \vdots                                                         \\
      \grad^\T v_k(\coordvec) \vphantom{\displaystyle\pdv{v_1}{x_1}}
    \end{bmatrix} \in \mathcal M_{k \times n}
  $$
\end{definition}

\begin{definition}[Divergencia]
  Legyen $\rvec v: \Reals^n \to \Reals^n$ vektormező, amelynek Jacobi-mátrixa
  $\rmat J = \DD \rvec v(\coordvec)$. Ekkor a $\rvec v$ divergenciájának
  nevezzük a következő skalármezőt:
  $$
    \Div \rvec v(\coordvec)
    = \trace \rmat J
    = J_{11} + J_{22} + \ldots + J_{nn}
    % = \pdv{v_1(\coordvec)}{x_1}
    % + \pdv{v_2(\coordvec)}{x_2}
    % + \ldots
    % + \pdv{v_n(\coordvec)}{x_n}
    % \text.
  $$
\end{definition}

\begin{note}
  A divergencia tehát a vektormező Jacobi-mátrixának nyoma.
\end{note}

\begin{note}
  Ahol a divergencia zérus, ott a mező forrásmentes.

  Ha a divergencia pozitív, akkor a mező forrás jellegű.

  Ha a divergencia negatív, akkor a mező nyelő jellegű.
\end{note}

\begin{definition}[Vektorinvariáns]
  Legyen $\rmat S \in \mathcal M_{3 \times 3}$ ferdeszimmetrikus mátrix, azaz
  $\rmat S = -\rmat S^\T$. Ekkor létezik egy egyértelmű $\rvec a \in \Reals^3$
  vektor, úgy, hogy
  $$
    \rmat S \rvec x = \rvec a \times \rvec x
    \text.
  $$
  Ekkor az $\rvec a$ vektort a $\rmat S$ mátrix vektorinvariánsának nevezzük.
  Gyakori jelölés:
  $$
    \rmat S = [\rvec a]_{\times} = \begin{bmatrix}
      0    & -a_3 & a_2  \\
      a_3  & 0    & -a_1 \\
      -a_2 & a_1  & 0
    \end{bmatrix}
    \qquad \Leftrightarrow \qquad
    \operatorname{axl}(\rmat S) = \rvec a = \begin{bmatrix}
      S_{32} \\
      S_{13} \\
      S_{21}
    \end{bmatrix}
    \text.
  $$
\end{definition}

\begin{example}[][nobreak]
  Határozzuk meg az $\rvec a(1;2;3)$ és $\rvec x(x;y;z)$ vektorok vektoriális
  szorzatát mátrixszorzás segítségével!
  $$
    \rvec a \times \rvec x
    = [\rvec a]_{\times} \rvec x
    = \begin{bmatrix}
      0  & -3 & 2  \\
      3  & 0  & -1 \\
      -2 & 1  & 0
    \end{bmatrix} \begin{bmatrix}
      x \\ y \\ z
    \end{bmatrix} = \begin{bmatrix}
      -3y + 2z \\
      3x - z   \\
      y - 2x
    \end{bmatrix}
    \text.
  $$
\end{example}

\begin{definition}[Rotáció]
  Legyen $\rvec v: \Reals^3 \to \Reals^3$ vektormező, melynek Jacobi-mátrixa
  $\rmat J = \DD \rvec v(\coordvec)$. Ekkor a $\rvec v$ rotációjának nevezzük
  a következő vektormezőt:
  $$
    \rot \rvec v(\coordvec)
    = \operatorname{axl}(\rmat J - \rmat J^\T)
    = \operatorname{axl}\left(\begin{bmatrix}
        0               & J_{12} - J_{21} & J_{13} - J_{31} \\
        J_{21} - J_{12} & 0               & J_{23} - J_{32} \\
        J_{31} - J_{13} & J_{32} - J_{23} & 0
      \end{bmatrix}\right) = \begin{bmatrix}
      J_{32} - J_{23} \\
      J_{13} - J_{31} \\
      J_{21} - J_{12}
    \end{bmatrix}
    % = \operatorname{axl}\left(\begin{bmatrix}
    %     0
    %      & \partial_2 v_1 - \partial_1 v_2
    %      & \partial_3 v_1 - \partial_1 v_3
    %     \\
    %     \partial_1 v_2 - \partial_2 v_1
    %      & 0
    %      & \partial_3 v_2 - \partial_2 v_3
    %     \\
    %     \partial_1 v_3 - \partial_3 v_1
    %      & \partial_2 v_3 - \partial_3 v_2
    %      & 0
    %   \end{bmatrix}\right) = \begin{bmatrix}
    %   \partial_2 v_3 - \partial_3 v_2 \\
    %   \partial_3 v_1 - \partial_1 v_3 \\
    %   \partial_1 v_2 - \partial_2 v_1
    % \end{bmatrix}
    \text.
  $$
\end{definition}

\begin{note}
  A rotáció tehát a vektormező Jacobi-mátrixának ferdeszimmetrikus részéből
  származtatható.
\end{note}

\begin{note}
  Ahol a rotáció zérus, ott a mező örvénymentes.
\end{note}

\begin{example}
  Adjuk meg a $\rvec v(\coordvec) = \ijk{yz}{xz}{xy}$ vektormező
  divergenciáját és rotációját a Jacobi-mátrix segítségével!

  A Jacobi-mátrix:
  $$
    \rmat J = \DD \rvec v(\coordvec)
    = \begin{bmatrix}
      \partial_x (yz) & \partial_y (yz) & \partial_z (yz) \\
      \partial_x (xz) & \partial_y (xz) & \partial_z (xz) \\
      \partial_x (xy) & \partial_y (xy) & \partial_z (xy)
    \end{bmatrix}
    = \begin{bmatrix}
      0 & z & y \\
      z & 0 & x \\
      y & x & 0
    \end{bmatrix}
    \text.
  $$
  A divergencia:
  $$
    \Div \rvec v(\coordvec)
    = \trace \rmat J
    = J_{11} + J_{22} + J_{33}
    = 0 + 0 + 0
    = 0
    \text.
  $$
  A rotáció:
  $$
    \rot \rvec v(\coordvec)
    = \operatorname{axl}(\rmat J - \rmat J^\T)
    = \begin{bmatrix}
      J_{32} - J_{23} \\
      J_{13} - J_{31} \\
      J_{21} - J_{12}
    \end{bmatrix}
    = \begin{bmatrix}
      z - z \\
      y - y \\
      x - x
    \end{bmatrix}
    = \begin{bmatrix}
      0 \\ 0 \\ 0
    \end{bmatrix}
    \text.
  $$
\end{example}

\begin{definition}[Nabla-operátor]
  Az $\Reals^n$-beli Descartes-koordinátarendszerben, ahol $\rvec x = (x_1;
    x_2; \dots; x_n)$ egy tetszőleges pont koordinátái, a standard bázis pedig
  $\{ \uvec e_1; \uvec e_2; \dots; \uvec e_n \}$ a Nabla egy olyan formális
  differenciáloperátor, melynek koordinátái a parciális derivált operátorok,
  vagyis:
  $$
    \nabla = \sum_{i = 1}^n \uvec e_i \pdv{}{x_i}
    =
    \begin{pmatrix}
      \displaystyle\pdv{}{x_1} &
      \displaystyle\pdv{}{x_2} &
      \hdots                   &
      \displaystyle\pdv{}{x_n}
    \end{pmatrix}^\T
    \text,
  $$
\end{definition}

\begin{note}[][nobreak]
  A Nabla-operátor segítségével a gradiens, divergencia és rotáció műveletek
  egyszerűbben felírhatók:
  \begin{itemize}
    \item $\grad \varphi = \nabla \varphi$,
    \item $\Div \rvec v = \scalar{\nabla}{\rvec v}$,
    \item $\rot \rvec v = \nabla \times \rvec v$.
  \end{itemize}
\end{note}

\begin{example}
  Adjuk meg a $\rvec v(\coordvec) = \ijk{yz}{xz}{xy}$ vektormező
  divergenciáját és rotációját a Nabla-operátor segítségével!

  A divergencia:
  $$
    \Div \rvec v
    = \scalar{\nabla}{\rvec v}
    = \scalar{\begin{bmatrix}
        \partial_x \\ \partial_y \\ \partial_z
      \end{bmatrix}}{\begin{bmatrix}
        yz \\ xz \\ xy
      \end{bmatrix}}
    = \pdv{yz}{x} + \pdv{xz}{y} + \pdv{xy}{z}
    = 0 + 0 + 0
    = 0
    \text.
  $$
  A rotáció:
  $$
    \rot \rvec v
    = \nabla \times \rvec v
    = \begin{bmatrix}
      \partial_x \\ \partial_y \\ \partial_z
    \end{bmatrix} \times \begin{bmatrix}
      yz \\ xz \\ xy
    \end{bmatrix}
    = \begin{bmatrix}
      \partial_y (xy) - \partial_z (xz) \\
      \partial_z (yz) - \partial_x (xy) \\
      \partial_x (xz) - \partial_y (yz)
    \end{bmatrix} = \begin{bmatrix}
      x - x \\ y - y \\ z - z
    \end{bmatrix} = \begin{bmatrix}
      0 \\ 0 \\ 0
    \end{bmatrix}
    \text.
  $$
\end{example}

\begin{definition}[Laplace-operátor]
  A Laplace-operátor egy másodrendű differenciáloperátor az $n$ dimenziós
  térben. Megadja egy skalármező gradiensének divergenciáját, azaz:
  $$
    \Delta \varphi
    = \scalar{\nabla}{\nabla} \varphi
    = \Div \grad \varphi
    \text.
  $$
\end{definition}

\begin{example}
  Hattassuk a Laplace-operátort a $\varphi(\coordvec) = xyz$ skalármezőre!
  $$
    \Delta \varphi
    = \scalar{\nabla}{\nabla} \varphi
    = \Div \grad \varphi
    = \Div \begin{bmatrix}
      yz \\ xz \\ xy
    \end{bmatrix}
    = \pdv{yz}{x} + \pdv{xz}{y} + \pdv{xy}{z}
    = 0 + 0 + 0
    = 0
    \text.
  $$
\end{example}

\clearpage
\subsection{További feladatok}

\begin{enumerate}
  \item Adja meg a $\varphi : \Reals^3 \to \Reals^3$ leképezés mátrixát a
        standard normális, illetve az $\rvec b_1(1; 0; 0)$, $\rvec b_2(1; 1; 0)$
        és $\rvec b_3(1; 1; 1)$ vektorok alkotta bázisban. Adja meg a leképezés
        magterének és képterének dimenzióját is!
        $$
          \varphi : \Reals^3 \to \Reals^3
          \qquad
          \begin{bmatrix}
            x \\ y \\ z
          \end{bmatrix}
          \mapsto
          \begin{bmatrix}
            2x - y \\ y + z \\ y - z
          \end{bmatrix}
        $$

  \item Bontsa fel az $\rmat A$ mátrixot szimmetrikus és antiszimmetrikus
        komponensekre!
        $$
          \rmat A = \begin{bmatrix}
            1 & 1 & 0 \\
            2 & 2 & 1 \\
            1 & 4 & 3
          \end{bmatrix}
        $$

  \item Adja meg a $\varphi$ és $\psi$ leképezések Jacobi-mátrixát!
        $$
          \varphi : \Reals^3 \to \Reals^2
          \quad
          \begin{bmatrix}
            x \\ y \\ z
          \end{bmatrix}
          \mapsto
          \begin{bmatrix}
            x^2 + z \\ y z^2
          \end{bmatrix}
          \qquad\qquad
          \psi : \Reals^2 \to \Reals^2
          \quad
          \begin{bmatrix}
            x \\ y
          \end{bmatrix}
          \mapsto
          \begin{bmatrix}
            \sin \ln (x y^2) \\ \sqrt{e^{xy} + \tan y}
          \end{bmatrix}
        $$

  \item Adja meg az alábbi leképezések gradienseit!
        ($C \in \Reals, \rvec a \in \Reals^3$)
        % \begin{multicols}{3}
        \begin{enumerate}
          \item $f (\coordvec) = C \cdot \coordvec^2$
          \item $g (\coordvec) = |\coordvec|$
          \item $h (\coordvec) = \scalar{\rvec a}{\coordvec}$
        \end{enumerate}
        % \end{multicols}

  \item Adja meg a $\rvec v (\coordvec)$ vektormező divergenciáját és rotációját!
        Mely halmazokon forrás-, illetve örvénymentes a mező?
        $$
          \rvec v(\coordvec) = \ijk{x^2 - y^2}{y^2 - z^2}{z^2 - x^2}
        $$

  \item Adja meg az alábbi vektormezők divergenciáját és rotációját!
        ($C \in \Reals$, $\rvec a \in \Reals^3$)
        % \begin{multicols}{3}
        \begin{enumerate}
          \item $\rvec u(\coordvec) = C \cdot \coordvec$
          \item $\rvec v(\coordvec) = \grad |\coordvec|$
          \item $\rvec w(\coordvec) = \rvec a \cdot \ln |\coordvec|$
        \end{enumerate}
        % \end{multicols}
\end{enumerate}
\end{document}