\documentclass{szb-practice}

\title{Potenciálosság}
\area{Vektoranalízis}
\subject{Matematika G3}
\subjectCode{BMETE94BG03}
\date{Utoljára frissítve: \today}
\docno{3}

\begin{document}
\allowdisplaybreaks

\maketitle

\vspace{-1em}

\subsection{Elméleti áttekinő}

\begin{definition}[Skalárpotenciálosság]
  Egy $\rvec v: V \rightarrow V$ vektormező skalárpotenciálos, ha létezik olyan
  $\varphi: V \rightarrow \mathbb R$ skalármező, hogy $\rvec v = \grad \varphi$.
\end{definition}

\begin{definition}[Vektorpotenciálosság]
  Egy $\rvec v: V \rightarrow V$ vektormező vektorpotenciálos, ha létezik olyan
  $\rvec u: V \rightarrow V$ vektormező, hogy $\rvec v = \rot \rvec u$.
\end{definition}

\begin{theorem}
  Legyen $\rvec v: V \rightarrow V$ mindenhol értelmezett, legalább egyszer
  differenciálható vektormező. Ekkor:
  \begin{itemize}
    \item $\rvec v$ skalárpotenciálos
          $\;\Leftrightarrow\;$
          $\rot \rvec v = \nvec$,
          hiszen $\rot \grad \varphi \equiv \nvec$,
          \hfill (\textbf{örvénymentes})
    \item $\rvec v$ vektorpotenciálos
          $\;\Leftrightarrow\;$
          $\Div \rvec v = 0$,
          hiszen $\Div \rot \rvec u \equiv 0$.
          \hfill (\textbf{forrásmentes})
  \end{itemize}
\end{theorem}

\begin{blueBox}[][nobreak]
  \sftitle{Potenciálfüggvények számítása:}

  Legyen $\varphi$ skalármező $\rvec v$ vektormező skalárpotenciálja. Ebben
  az esetben tudjuk, hogy $\rvec v = \grad \varphi$, vagyis
  \[
    \rvec v = \left(
    \pdv{\varphi}{x_1};
    \pdv{\varphi}{x_2};
    \dots;
    \pdv{\varphi}{x_n}
    \right)^\T
    \text.
  \]
  Ilyen esetben a potenciálfüggvény az alábbi módon számítható:
  \[
    \varphi (\rvec r)
    = \int_0^{x_1} \rvec v(\xi; x_2; \dots; x_n) \dd \xi
    + \int_0^{x_2} \rvec v(0; \xi; \dots; x_n) \dd \xi
    + \dots
    + \int_0^{x_n} \rvec v(0; 0; \dots; \xi) \dd \xi
    \text.
  \]
  Legyen $\rvec u$ vektormező $\rvec v$ vektormező vektorpotenciálja.
  A potenciál számtalan alakban előállhat, ezért keressük ezt az alábbi alakban:
  \[
    \rvec u = \left( u_x; u_y; 0 \right)^\T
  \]
  A potenciál komponensei az alábbi módon számíthatóak:
  \[
    u_x = \int_0^z v_y(x; y; \zeta) \dd \zeta
    \text,
    \qquad
    u_y = \int_0^x v_z(\xi; y; 0) \dd \xi
    - \int_0^z v_x(x; y; \zeta) \dd \zeta
    \text.
  \]
\end{blueBox}

\clearpage
\subsection{Feladatok}

\begin{enumerate}
  \item

        % \item Vizsgálja meg, hogy az alábbi vektormezők skalárpotenciálisak-e!
        %       Amennyiben igen, adja meg a potenciálfüggvényüket!
        %       \begin{enumerate}
        %         \item $\rvec v(\rvec r) = \ijz{3x^2y - y^3}{x^3 - 3xy^2}$
        %         \item $\rvec w(\rvec r) = \ijk{yz}{zx}{xy}$
        %         \item $\rvec u(\rvec r) = \ijk{2xy}{x^2 + z^2}{2yz}$
        %       \end{enumerate}

        % \item Vizsgálja meg, hogy az alábbi vektormezők vektorpotenciálosak-e!
        %       Amennyiben igen, adja meg a potenciálfüggvényüket!
        %       \begin{enumerate}
        %         \item $\rvec v(\rvec r) = \ijk{yz}{zx}{xy}$
        %       \end{enumerate}
\end{enumerate}

\end{document}

