\documentclass{szb-practice}

\title{Görbék, görbementi integrál}
\area{Vektoranalízis}
\subject{Matematika G3}
\subjectCode{BMETE94BG03}
\date{Utoljára frissítve: \today}
\docno{3}

\begin{document}
\allowdisplaybreaks

\maketitle

\vspace{-1em}
\subsection{Elméleti áttekintő}
\vspace{1em}

\begin{definition}[Reguláris görbe]
  Legyen $\curvedomain \subset \Reals$ nem feltétlenül korlátos intervallum.
  Ekkor az $\curvesign : \curvedomain \rightarrow \curveimage \subset \Reals^3$
  immerziót reguláris görbének nevezzük.
\end{definition}

\vfill

\begin{definition}[Pályasebesség, Ívhossz]
  A $v: \curvedomain \subset \Reals \rightarrow \Reals$,
  $t \mapsto \|\dot{\curvesign}(t)\|$ függvényt pályasebességnek hívjuk.

  A pályasebesség $\curvedomain$ feletti integrálját a görbe ívhosszának
  nevezzük:
  $$
    L(\curvesign)
    = \int_{\curvedomain} \|\dot{\curvesign}(t)\| \dd t
    = \int_{\curvedomain} \dd \curvescalar
    \text.
  $$
\end{definition}

\vfill

\begin{definition}[Irányított görbe]
  Egy $\curvesign: [a;b] \rightarrow \Reals^3$ görbe irányított, ha adott egy
  rendezés ($\leq$) a paraméterértékeken. Ekkor $t_1 < t_2$ esetén
  $\curvesign(t_1)$ a görbe korábbi pontja, $\curvesign(t_2)$-höz képest. Ha
  $\curvesign(a) = \curvesign(b)$, akkor a görbe zárt.
\end{definition}

\vfill

\begin{definition}[Skalármező görbe menti skalárértékű integrálja]
  Legyen $\varphi: U \subseteq \Reals^3 \rightarrow \Reals$ skalármező,
  $\curvesign: \curvedomain \rightarrow \curveimage \subset U$,
  $t \mapsto \curvesign(t)$ pedig a görbe parametrizált egyenlete. Ekkor a
  $\varphi$ skalármező $\curveimage$ görbe menti skalárértékű integrálja:
  $$
    \int_{\curveimage} \varphi(\coordvec) \dd \curvescalar =
    \int_{\curvedomain} \varphi(\curvesign(t)) \norma{\dot{\curvesign}(t)} \dd t
    \text.
  $$
\end{definition}

\vfill

\begin{definition}[Vektormező görbe menti skalár- és vektorértékű integrálja][nobreak]
  Legyen $\rvec v: U \subseteq \Reals^3 \rightarrow \Reals^3$ vektormező,
  $\curvesign: \curvedomain \rightarrow \curveimage \subset U$,
  $t \mapsto \curvesign(t)$ pedig a görbe parametrizált egyenlete. Ekkor az
  $\rvec v$ vektormező $\curveimage$ görbe menti
  \begin{itemize}
    \item skalárértékű integrálja:
          $\displaystyle
            \int_{\curveimage} \scalar{\rvec v(\coordvec)}{\dd \curvescalar} =
            \int_{\curvedomain} \scalar{\rvec v(\curvesign(t))}{\dot{\curvesign}(t)} \dd t
            \text,
          $
    \item vektorértékű integrálja:
          $\displaystyle
            \int_{\curveimage} \rvec v(\coordvec) \times \dd \curvescalar =
            \int_{\curvedomain} \rvec v(\curvesign(t)) \times \dot{\curvesign}(t) \dd t
            \text.
          $
  \end{itemize}
\end{definition}

\clearpage
\subsection{Feladatok}

\begin{enumerate}
  \item
\end{enumerate}

\clearpage
\subsection{Segédlet}

\subsubsection{Görbék paraméterezése}

\begin{tabular}{
  >{\bullet\;\bfseries}p{2.5cm}<{:}
  p{5cm}
  p{2.75cm}
  >{\centering\arraybackslash}m{4cm}
  }
  Egyenes
   & $\curvesign = \rvec r_0 + t \rvec v$
   & $t \in \Reals$
   & \begin{tikzpicture}[
         3d view={110}{20},
         baseline,
       ]
       % Origin coordinate
       \coordinate (O) at (0,0,0);

       % Coordinate system
       \draw[-to] (O) -- ++(1.75,0,0) node[anchor=south] {$x$};
       \draw[-to] (O) -- ++(0,1.75,0) node[anchor=north east] {$y$};
       \draw[-to] (O) -- ++(0,0,1.75) node[anchor=north west] {$z$};

       % Line
       \draw[to-to, thick, primaryColor]
       (2,-0.75,1.25) -- ++(-1,3,0.75)
       coordinate[pos=.2] (A)
       coordinate[pos=.8] (B)
       ;

       % v vector
       \draw[-to, draw=ternaryColor, ultra thick]
       (A) -- (B)
       node[anchor=north east] {$\rvec v$}
       ;

       % r_0 vector
       \draw[-to, draw=secondaryColor, ultra thick]
       (O) -- (A)
       node[anchor=south] {$\rvec r_0$}
       ;
     \end{tikzpicture}
  \\[14mm]
  Szakasz
   & $\curvesign = \rvec r_0 + t (\rvec r_1 - \rvec r_0)$
   & $t \in [0;1]$
   & \begin{tikzpicture}[
         3d view={110}{20},
         baseline,
       ]
       % Origin coordinate
       \coordinate (O) at (0,0,0);

       % Coordinate system
       \draw[-to] (O) -- ++(1.75,0,0) node[anchor=south] {$x$};
       \draw[-to] (O) -- ++(0,1.75,0) node[anchor=north east] {$y$};
       \draw[-to] (O) -- ++(0,0,1.75) node[anchor=north west] {$z$};

       % Line segment
       \draw[|-|, thick, red-base]
       (2,-0.75,1.25) coordinate (A) -- ++(-1,3,0.75) coordinate (B)
       ;

       % r_0 vector
       \draw[-to, draw=yellow-base, ultra thick]
       (O) -- (A)
       node[anchor=south west] {$\rvec r_0$}
       ;

       % r_1 vector
       \draw[-to, draw=blue-base, ultra thick]
       (O) -- (B)
       node[anchor=south east] {$\rvec r_1$}
       ;
     \end{tikzpicture}

  \\[14mm]
  Körvonal
   & $\curvesign = \begin{bmatrix} r \cos t \\ r \sin t \\ 0 \end{bmatrix}$
   & $t \in [0;2\pi]$
   & \begin{tikzpicture}[
         3d view={110}{20},
         baseline,
       ]
       % Origin coordinate
       \coordinate (O) at (0,0,0);

       % Circle
       \draw[draw=red-base, thick] (O) circle (1);

       % Coordinate system
       \draw[-to] (O) -- ++(1.75,0,0) node[anchor=west] {$x$};
       \draw[-to] (O) -- ++(0,1.75,0) node[anchor=south east] {$y$};
       \draw[-to] (O) -- ++(0,0,1.25) node[anchor=north east] {$z$};

       % Radius
       \draw[-to, thick, draw=blue-base]
       (O) -- (0.6*1.75,0.8*1.75,0) -- (0.6,0.8,0)
       node[midway, anchor=north east, inner sep=.5mm, font=\scriptsize] {$r$};
     \end{tikzpicture}
  \\[14mm]
  Ellipszis
   & $\curvesign = \begin{bmatrix} a \cos t \\ b \sin t \\ 0 \end{bmatrix}$
   & $t \in [0;2\pi]$
   & \begin{tikzpicture}[
         3d view={110}{20},
         baseline,
       ]
       % Origin coordinate
       \coordinate (O) at (0,0,0);

       % Ellipsis
       \draw[draw=red-base, thick] (O) ellipse (1.4 and .8);

       % Coordinate system
       \draw[-to] (O) -- ++(2.20,0,0) node[anchor=west] {$x$};
       \draw[-to] (O) -- ++(0,1.50,0) node[anchor=south east] {$y$};
       \draw[-to] (O) -- ++(0,0,1.25) node[anchor=north east] {$z$};

       % Half axes
       \begin{scope}[font=\scriptsize]
      \node at (0.7,0,0) [anchor=south east, inner sep=.5mm] {$a$};
      \node at (0,0.4,0) [anchor=south, inner sep=.5mm] {$b$};
    \end{scope}

       \draw[to-to, thick, draw=blue-base] (O) -- (1.4,0,0);
       \draw[to-to, thick, draw=blue-base] (O) -- (0,0.8,0);
     \end{tikzpicture}
  \\[14mm]
  Spirál
   & $\curvesign = \begin{bmatrix} a \cos t \\ a \sin t \\ bt \end{bmatrix}$
   & $t \in \Reals$
   & \begin{tikzpicture}[
         3d view={110}{20},
         baseline,
       ]
       % Origin coordinate
       \coordinate (O) at (0,0,0);

       % Coordinate system
       \draw[-to] (O) -- ++(1.75,0,0) node[anchor=west] {$x$};
       \draw[-to] (O) -- ++(0,1.75,0) node[anchor=south east] {$y$};
       \draw[-to] (O) -- ++(0,0,1.75) node[anchor=north east] {$z$};

       \draw[%
         domain=0:866,% 2.5 revolutions
         samples=100, % More samples -> smoother curve
         smooth,      % Nicer curve
         variable=\t, % Use \t because of convention
         thick,
         red-base,
       ] plot (
       {cos(\t)},
       {sin(\t)},
       {\t/900}
       );

       \begin{scope}[font=\scriptsize]
      \draw[thick, draw=blue-base, inner sep=1mm]
      (O) -- (1,0,0)
      node[anchor=west, pos=.6] {$a$};

      \draw[thick, draw=blue-base, inner sep=0.5mm]
      (1,0,0) -- ++(0,0,1/2.5)
      node[anchor=east, pos=.5] {$b$};
    \end{scope}
     \end{tikzpicture}
\end{tabular}

\subsubsection{Koordináta-transzformációk}

\def\arraystretch{1.1}
\begin{tabular}{
  >{\bullet\;}
  m{2.15cm}
  m{3.15cm}
  m{2.25cm}
  m{2.25cm}
  >{\centering\arraybackslash}m{4cm}
  }
  \textbf{Polár:}
   & $x = r \cos \varphi$ \newline
  $y = r \sin \varphi$
   & $r \in [0; R]$ \newline
  $\varphi \in [0; 2\pi)$
   & $|\rmat J| = r$
   & \begin{tikzpicture}
           % Coordinate system
           \draw[-to] (-1.2,0) -- (1.5,0) node[anchor=north east] {$x$};
           \draw[-to] (0,-0.5) -- (0,1.4) node[anchor=north west] {$y$};

           % Radius
           \draw[-to, thick, draw=red-base]
           (0,0) -- (125:1.5) coordinate(c)
           node[anchor=north east, midway, inner sep=2pt] {$r$}
           ;

           % Radius
           % \draw [decorate, decoration={brace}, draw=blue-base, thick]
           % (-.75mm,-.75mm) -- ++(c)
           % node[below left, midway] {$r$};

           % Angle helpers
           \coordinate (a) at (1,0);
           \coordinate (b) at (0,0);

           % Angle
           \draw pic[
                 "$\varphi$",
                 draw=yellow-base,
                 angle eccentricity=.5,
                 angle radius=7mm,
                 thick,
                 ->,
               ] {angle=a--b--c};
         \end{tikzpicture}
  \\[12mm]
    \textbf{Henger:}
   & $x = r \cos \varphi$ \newline
  $y = r \sin \varphi$ \newline
  $z = z$
   & $r \in [0; R]$ \newline
  $\varphi \in [0; 2\pi]$ \newline
  $z \in \Reals$
   & $|\rmat J| = r$
   & \tdplotsetmaincoords{70}{110}
    \begin{tikzpicture}[
        % 3d view={110}{20},
        baseline,
        tdplot_main_coords,
      ]
      % Origin coordinate
      \coordinate (O) at (0,0,0);

      % Coordinate system
      \draw[-to] (O) -- ++(1.75,0,0) node[anchor=south] {$x$};
      \draw[-to] (O) -- ++(0,1.75,0) node[anchor=north east] {$y$};
      \draw[-to] (O) -- ++(0,0,1.75) node[anchor=north west] {$z$};

      % Helper coordinates
      \coordinate (T) at (0.8,1.2,0);

      % r and z
      \draw[-to, thick, draw=blue-base]
      (T) -- ++(0,0,1)
      node[anchor=north west] {$z$};
      \draw[-to, thick, draw=red-base]
      (O) -- (T)
      node[anchor=south west, pos=.4, fill=white, inner sep=2pt] {$r$};

      \tdplotdrawarc[yellow-base, thick, -to]
      {(O)}{1.2}{0}{56}
      {anchor=south east,font=\scriptsize, inner sep=2pt, black}{$\varphi$};
    \end{tikzpicture}
  \\[12mm]
    \textbf{Gömb:}
   & $x = r \sin \varphi \cos \vartheta $ \newline
  $y = r \sin \varphi \sin \vartheta $ \newline
  $z = r \cos \varphi$
   & $r \in [0; R]$ \newline
  $\varphi \in [0; \pi]$ \newline
  $\vartheta \in [0; 2\pi]$
   & $|\rmat J| = r^2 \sin \varphi$
   & \tdplotsetmaincoords{55}{110}
  \begin{tikzpicture}[
      % 3d view={130}{35.26},
      baseline,
      tdplot_main_coords,
    ]
    % Origin coordinate
    \coordinate (O) at (0,0,0);
    \def\s{1.25}

    % Coordinate system
    \draw[-to] (O) -- ++(1.75,0,0) node[anchor=south] {$x$};
    \draw[-to] (O) -- ++(0,1.75,0) node[anchor=south] {$y$};
    \draw[-to] (O) -- ++(0,0,1.75) node[anchor=north west] {$z$};

    % Helper square
    \draw[gray,dashed]
    (\s,\s,0) coordinate(A) --
    (0,\s,0) coordinate(B) --
    (0,\s,\s) coordinate(C) --
    (\s,\s,\s) coordinate(c) --
    cycle
    ;

    % Helper lines
    \draw[gray]
    (O) -- (C)
    (O) -- (A)
    ;

    % r
    \draw[thick,draw=red-base,-to]
    (O) -- ++(\s,\s,\s)
    node[pos=.5, anchor=south west, font=\scriptsize, inner sep=2pt]{$r$};

    % phi
    \tdplotgetpolarcoords{0.001}{1}{1}
    \tdplotsetthetaplanecoords{\tdplotresphi}

    \tdplotdrawarc[tdplot_rotated_coords, blue-base, thick, -to]
    {(O)}{1}{0}{45}
    {anchor=north east,font=\scriptsize, inner sep=2pt, black}{$\varphi$}

    % theta
    \tdplotdrawarc[yellow-base, thick, -to]
    {(O)}{1}{0}{45}
    {anchor=south,font=\scriptsize, inner sep=2pt, black}{$\vartheta$};
  \end{tikzpicture}
  \\
\end{tabular}

\end{document}