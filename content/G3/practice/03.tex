\documentclass{szb-practice}

\title{Görbék, görbementi integrál}
\area{Vektoranalízis}
\subject{Matematika G3}
\subjectCode{BMETE94BG03}
\date{Utoljára frissítve: \today}
\docno{3}

\begin{document}
\allowdisplaybreaks

\maketitle

\vspace{-1em}
\subsection{Elméleti áttekintő}
\vspace{1em}

\begin{blueBox}[\underline{Görbék paraméterezése:}]
  \begin{tabular}{
    >{\bullet\;\bfseries}p{2.5cm}<{:}
    p{4.35cm}
    p{2.65cm}
    >{\centering\arraybackslash}m{4cm}
    }
    Egyenes
     & $\curvesign(t) = \rvec r_0 + t \rvec v$
     & $t \in \Reals$
     & \begin{tikzpicture}[
           3d view={110}{20},
           baseline,
         ]
         % Origin coordinate
         \coordinate (O) at (0,0,0);

         % Coordinate system
         \draw[-to] (O) -- ++(1.75,0,0) node[anchor=south] {$x$};
         \draw[-to] (O) -- ++(0,1.75,0) node[anchor=north east] {$y$};
         \draw[-to] (O) -- ++(0,0,1.75) node[anchor=north west] {$z$};

         % Line
         \draw[to-to, thick, primaryColor]
         (2,-0.75,1.25) -- ++(-1,3,0.75)
         coordinate[pos=.2] (A)
         coordinate[pos=.8] (B)
         ;

         % v vector
         \draw[-to, draw=ternaryColor, ultra thick]
         (A) -- (B)
         node[anchor=north east] {$\rvec v$}
         ;

         % r_0 vector
         \draw[-to, draw=secondaryColor, ultra thick]
         (O) -- (A)
         node[anchor=south] {$\rvec r_0$}
         ;
       \end{tikzpicture}
    \\[14mm]
    Szakasz
     & $\curvesign(t) = \rvec r_0 + t (\rvec r_1 - \rvec r_0)$
     & $t \in [0;1]$
     & \begin{tikzpicture}[
           3d view={110}{20},
           baseline,
         ]
         % Origin coordinate
         \coordinate (O) at (0,0,0);

         % Coordinate system
         \draw[-to] (O) -- ++(1.75,0,0) node[anchor=south] {$x$};
         \draw[-to] (O) -- ++(0,1.75,0) node[anchor=north east] {$y$};
         \draw[-to] (O) -- ++(0,0,1.75) node[anchor=north west] {$z$};

         % Line segment
         \draw[|-|, thick, red-base]
         (2,-0.75,1.25) coordinate (A) -- ++(-1,3,0.75) coordinate (B)
         ;

         % r_0 vector
         \draw[-to, draw=yellow-base, ultra thick]
         (O) -- (A)
         node[anchor=south west] {$\rvec r_0$}
         ;

         % r_1 vector
         \draw[-to, draw=blue-base, ultra thick]
         (O) -- (B)
         node[anchor=south east] {$\rvec r_1$}
         ;
       \end{tikzpicture}

    \\[14mm]
    Körvonal
     & $\curvesign(t) = \begin{bmatrix} r \cos t \\ r \sin t \\ 0 \end{bmatrix}$
     & $t \in [0;2\pi]$
     & \begin{tikzpicture}[
           3d view={110}{20},
           baseline,
         ]
         % Origin coordinate
         \coordinate (O) at (0,0,0);

         % Circle
         \draw[draw=red-base, thick] (O) circle (1);

         % Coordinate system
         \draw[-to] (O) -- ++(1.75,0,0) node[anchor=west] {$x$};
         \draw[-to] (O) -- ++(0,1.75,0) node[anchor=south east] {$y$};
         \draw[-to] (O) -- ++(0,0,1.25) node[anchor=north east] {$z$};

         % Radius
         \draw[-to, thick, draw=blue-base]
         (O) -- (0.6*1.75,0.8*1.75,0) -- (0.6,0.8,0)
         node[midway, anchor=north east, inner sep=.5mm, font=\scriptsize] {$r$};
       \end{tikzpicture}
    \\[14mm]
    Ellipszis
     & $\curvesign(t) = \begin{bmatrix} a \cos t \\ b \sin t \\ 0 \end{bmatrix}$
     & $t \in [0;2\pi]$
     & \begin{tikzpicture}[
           3d view={110}{20},
           baseline,
         ]
         % Origin coordinate
         \coordinate (O) at (0,0,0);

         % Ellipsis
         \draw[draw=red-base, thick] (O) ellipse (1.4 and .8);

         % Coordinate system
         \draw[-to] (O) -- ++(2.20,0,0) node[anchor=west] {$x$};
         \draw[-to] (O) -- ++(0,1.50,0) node[anchor=south east] {$y$};
         \draw[-to] (O) -- ++(0,0,1.25) node[anchor=north east] {$z$};

         % Half axes
         \begin{scope}[font=\scriptsize]
        \node at (0.7,0,0) [anchor=south east, inner sep=.5mm] {$a$};
        \node at (0,0.4,0) [anchor=south, inner sep=.5mm] {$b$};
      \end{scope}

         \draw[to-to, thick, draw=blue-base] (O) -- (1.4,0,0);
         \draw[to-to, thick, draw=blue-base] (O) -- (0,0.8,0);
       \end{tikzpicture}
    \\[14mm]
    Spirál
     & $\curvesign(t) = \begin{bmatrix} a \cos t \\ a \sin t \\ bt \end{bmatrix}$
     & $t \in \Reals$
     & \begin{tikzpicture}[
           3d view={110}{20},
           baseline,
         ]
         % Origin coordinate
         \coordinate (O) at (0,0,0);

         % Coordinate system
         \draw[-to] (O) -- ++(1.75,0,0) node[anchor=west] {$x$};
         \draw[-to] (O) -- ++(0,1.75,0) node[anchor=south east] {$y$};
         \draw[-to] (O) -- ++(0,0,1.75) node[anchor=north east] {$z$};

         \draw[%
           domain=0:866,% 2.5 revolutions
           samples=100, % More samples -> smoother curve
           smooth,      % Nicer curve
           variable=\t, % Use \t because of convention
           thick,
           red-base,
         ] plot (
         {cos(\t)},
         {sin(\t)},
         {\t/900}
         );

         \begin{scope}[font=\scriptsize]
        \draw[thick, draw=blue-base, inner sep=1mm]
        (O) -- (1,0,0)
        node[anchor=west, pos=.6] {$a$};

        \draw[thick, draw=blue-base, inner sep=0.5mm]
        (1,0,0) -- ++(0,0,1/2.5)
        node[anchor=east, pos=.5] {$b$};
      \end{scope}
       \end{tikzpicture}
  \end{tabular}
\end{blueBox}

\vfill

\begin{definition}[Reguláris görbe]
  Legyen $\curvedomain \subset \Reals$ nem feltétlenül korlátos intervallum.
  Ekkor az $\curvesign : \curvedomain \rightarrow \curveimage \subset \Reals^3$
  immerziót reguláris görbének nevezzük.
\end{definition}

\vfill

\begin{definition}[Sebesség- és gyorsulásvektor][nobreak]
  Legyen $\curvesign : \curvedomain \rightarrow \curveimage \subset \Reals^3$
  reguláris görbe. Ekkor a $\dot{\curvesign}(t)$ és $\ddot{\curvesign}(t)$
  vektorokat a görbe sebesség- és gyorsulásvektorának nevezzük.
\end{definition}

\clearpage

\begin{definition}[Pályasebesség, Ívhossz][nobreak]
  A $v: \curvedomain \subset \Reals \rightarrow \Reals$,
  $t \mapsto \|\dot{\curvesign}(t)\|$ függvényt pályasebességnek hívjuk.

  A pályasebesség $\curvedomain$ feletti integrálját a görbe ívhosszának
  nevezzük:
  $$
    L(t)
    = \int_{\curvedomain} \|\dot{\curvesign}(t)\| \dd t
    = \int_{\curvedomain} \dd \curvescalar
    \text.
  $$
\end{definition}

\vfill

\begin{blueBox}[\underline{Görbe ívhossz szerinti paraméteretése}]
  Legyen $\curvesign : \curvedomain \rightarrow \curveimage$ reguláris görbe,
  $L(t)$ pedig a görbe ívhossza. Ekkor a görbe ívhossz szerinti paraméterezése
  $$
    \curvesign_L : [0; L(\max(\curvedomain))] \rightarrow \curveimage
    \text{, ahol }
    \curvesign_L(s) = \curvesign(L^{-1}(s))
    \text.
  $$
  Az ívhossz szerinti paraméterezésű görbe sebességvektora egységvektor:
  $\|\dot{\curvesign}_L(s)\| = 1$.
\end{blueBox}

\vfill

\begin{definition}[Skalármező görbe menti skalárértékű integrálja]
  Legyen $\varphi(\coordvec): \Reals^3 \to \Reals$ skalármező,
  $\curvesign : \curvedomain \rightarrow \curveimage$ paraméterezett görbe, ahol
  $t \in \curvedomain$ a görbe paraméterezése,
  $\curvesign(\curvedomain) = \curveimage$ a görbe képe,
  $\dd \curvescalar = \norma{\dot{\curvesign}(t)} \dd t$. Ekkor a $\varphi$
  skalármező görbe menti skalárértékű integrálja
  $$
    \int_{\curveimage} \varphi(\coordvec) \dd \curvescalar =
    \int_{\curvedomain} \varphi(\curvesign(t)) \norma{\dot{\curvesign}(t)} \dd t
    \text.
  $$
\end{definition}

\vfill


\begin{definition}[Vektormező görbe menti skalár- és vektorértékű integrálja][nobreak]
  Legyen $\rvec v(\coordvec) : \Reals^3 \to \Reals^3$ vektormező,
  $\curvesign : \curvedomain \rightarrow \curveimage$ paraméterezett görbe, ahol
  $t \in \curvedomain$ a görbe paraméterezése,
  $\curvesign(\curvedomain) = \curveimage$ a görbe képe,
  $\dd \curvevec = \dot{\curvesign}(t) \dd t$. Ekkor a $\rvec v$ vektormező
  görbe menti
  \begin{itemize}
    \item skalárértékű integrálja:
          $\displaystyle
            \int_{\curveimage} \scalar{\rvec v(\coordvec)}{\dd \curvescalar} =
            \int_{\curvedomain} \scalar{\rvec v(\curvesign(t))}{\dot{\curvesign}(t)} \dd t
            \text,
          $
    \item vektorértékű integrálja:
          $\displaystyle
            \int_{\curveimage} \rvec v(\coordvec) \times \dd \curvescalar =
            \int_{\curvedomain} \rvec v(\curvesign(t)) \times \dot{\curvesign}(t) \dd t
            \text.
          $
  \end{itemize}
\end{definition}

\vfill

\begin{theorem}[Gradiens-tétel][nobreak]
  Legyen $\varphi : U \subseteq \Reals^3 \rightarrow \Reals$ differenciálható
  skalármező, $\curvesign : [a;b] \rightarrow \curveimage \subseteq U$,
  $t \mapsto \curvesign(t)$ folytonos görbe, $\curvesign(a) = \rvec p$,
  $\curvesign(b) = \rvec q$ pedig a görbe kezdő és végpontja. Ekkor:
  $$
    \int_{\curveimage} \scalar{\grad \varphi(\coordv)}{\dd \curvevec}
    =
    \varphi(\rvec q) - \varphi(\rvec p)
    \text.
  $$
  Vagyis, ha egy vektormező valamely skalármező gradiense, akkor annak bármely
  folytonos görbe mentén vett integrálja csak a kezdő- és végpontoktól függ.
\end{theorem}

\clearpage
\subsection{Feladatok}

\begin{enumerate}
  \item Számítsa ki a megadott görbék ívhosszát az adott intervallumon!
        \begin{enumerate}
          \item $\curvesign_1(t) = \ijk{t}{\sqrt{6} t^2 / 2}{t^3}, \quad t \in [0; 2]$
          \item $\curvesign_2(t) = \ijz{t \cos t}{t \sin t}, \quad t \in [0; 1]$
          \item $\curvesign_3(t) = \ijk{e^t \cos t}{e^t \sin t}{e^t}, \quad t \in [0; 2\pi]$
          \item $\curvesign_4(t) = \ijz{t - \sin t}{1 - \cos t}, \quad t \in [0; 2\pi]$
        \end{enumerate}

  \item Adja meg a $\curvesign(t) = \ijk{1}{t}{\sqrt{2t}}$ görbe egységsebességű
        paraméterezését!

  \item Integrálja a saklármezőket a megadott görbék mentén!
        \begin{enumerate}
          \item $\varphi(\coordvec) = \sqrt{1 + 4x^2 + 16yz}, \quad \curvesign(t) = \ijk{t}{t^2}{t^4}, \quad t \in [0; 1]$
          \item $\psi(\coordvec) = 2x, \quad$ a $(3;0)$ és $(0;4)$ pontokat összekötő szakasz mentén
          \item $\chi(\coordvec) = x^2 + y^2, \quad$ első síknegyedbeli egységköríven, pozitív irányítással
          \item $\omega(\coordvec) = x^2 + y^2, \quad$ $r = 2$ sugarú, origó középpontú, pozitív irányítású körön
        \end{enumerate}

  \item Számítsa ki az alábbi vektormezők skalárértékű integrálját az adott görbék mentén!
        \begin{enumerate}
          \item $\rvec u(\coordvec) = \ijk{y + z}{x + z}{x + y}, \;\; \curvesign(t) = \ijk{t}{t^2}{t^3}, \;\; t \in [0; 1]$
          \item $\rvec v(\coordvec) = \ijk{-yz}{xz}{x^2 + y^2}, \;\; \curvesign(t) = \ijk{\cos t}{\sin t}{2t}, \;\; t \in [0; 2]$
          \item $\rvec w(\coordvec) = \ijz{y}{x}, \;\;$ az $(1;0)$ pontból a $(0;2)$ pontba
                \begin{itemize}
                  \item egy egyenes szakasz mentén
                  \item origó középpontú ellipszis mentén
                \end{itemize}
        \end{enumerate}

  \item Adja meg a $\rvec v(\coordvec) = \ijk{y^2 - x^2}{2 y z}{-x^2}$
        vektormező $\curvesign(t) = \ijk{t}{t^2}{t^3}$, $t \in [0; 1]$
        görbe menti vektorértékű integrálját!
\end{enumerate}

\end{document}