\documentclass[fleqn]{szb-practice}

\title{Összefoglalás}
\area{Többváltozós analízis}
\subject{Matematika G3}
\subjectCode{BMETE94BG03}
\date{Utoljára frissítve: \today}
\docno{7}

\begin{document}
\maketitle

\vspace{-1em}
\subsection{Elméleti áttekintő}

\begin{blueBox}[\underline{Differenciáloperátorok}]
  Legyen $\rvec v(\coordvec) : \Reals^3 \to \Reals^3$ vektormező,
  $\varphi(\coordvec): \Reals^3 \to \Reals$ skalármező, ahol $\coordvec$ az
  $\Reals^3$-beli Descartes koordináta-rendszerben $\coordvec = (x; y; z)$.
  \begin{center}
    \def\arraystretch{1.25}
    \newenvironment{bm}{\bgroup\renewcommand*{\arraystretch}{1.1}\begin{bmatrix}}{\end{bmatrix}\egroup}
    \newcommand{\dspl}[3]{\begin{bm}#1\\#2\\#3\end{bm}}
    \newcommand\nablavec{\dspl{\partial_x}{\partial_y}{\partial_z}}
    \begin{tabular}{*{3}{>{\centering\arraybackslash}p{3.5cm}}}
      \def\arraystretch{1}
      % &
      \bfseries Rotáció
       & \bfseries Divergencia
       & \bfseries Gradiens
      \\
      \hline
      % Jelölés & 
      $\rot \rvec v$
       & $\Div \rvec v$
       & $\grad \varphi$
      \\
      % Operátor & 
      $\nabla \times \rvec v$
       & $\scalar{\nabla}{\rvec v}$
       & $\nabla \varphi$
      \\
      % Számítás &
      $\nablavec \times \dspl{v_x}{v_y}{v_z}$
       & $\scalar{\nablavec}{\dspl{v_x}{v_y}{v_z}}$
       & $\dspl{\partial_x \varphi}{\partial_y \varphi}{\partial_z \varphi}$
      \\
      % Ért. tart. & 
      $\Domain_{\rvec v} = \Reals^3$
       & $\Domain_{\rvec v} = \Reals^3$
       & $\Domain_{\varphi} = \Reals^3$
      \\
      % Ért. készl. &
      $\Range_{\rvec v} = \Reals^3$
       & $\Range_{\rvec v} = \Reals^3$
       & $\Range_{\varphi} = \Reals$
      \\
      % Ért. készl. &
      $\Range_{\rot \rvec v} = \Reals^3$
       & $\Range_{\Div \rvec v} = \Reals$
       & $\Range_{\grad \varphi} = \Reals^3$
      \\
      % \hline
    \end{tabular}
  \end{center}
\end{blueBox}

\begin{blueBox}[\underline{Azonosságok}][nobreak]
  \begin{itemize}
    \item Teljesül a linearitás:
          \vspace{-.5em}
          \begin{alignat*}{4}
            \grad & (\lambda \, \varPhi && + \mu \, \varPsi) && = \lambda \, \grad \varPhi && + \mu \, \grad \varPsi
            \\
            \rot  & (\lambda \, \rvec v && + \mu \, \rvec w) && = \lambda \, \rot \rvec v  && + \mu \, \rot \rvec w
            \\
            \Div  & (\lambda \, \rvec v && + \mu \, \rvec w) && = \lambda \, \Div \rvec v  && + \mu \, \Div \rvec w
          \end{alignat*}
          \vspace{-2.5em}

    \item Zérusság:
          \vspace{-.5em}
          \begin{align*}
            \rot \grad \varPhi & \equiv \nvec
            \\
            \Div \rot \rvec v  & \equiv 0
          \end{align*}
          % \vspace{-2.5em}

          % \item Deriválási szabályokhoz hasonló:
          %       \vspace{-.5em}
          %       \begin{align*}
          %         \grad \left( \varPhi \, \varPsi \right)
          %          & = \varPhi \, \grad \varPsi
          %         + \varPsi \, \grad \varPhi
          %         \\
          %         \Div \left( \varPhi \, \rvec v \right)
          %          & = \varPhi \, \Div \rvec v \,
          %         + \scalar{\rvec v}{\grad \varPhi}
          %         \\
          %         \rot \left( \varPhi \, \rvec v \right)
          %          & = \varPhi \, \rot \rvec v
          %         - \rvec v \times \grad \varPhi
          %       \end{align*}
          %       \vspace{-2.5em}

          % \item Egyéb szabályok:
          %       \vspace{-.5em}
          %       \begin{align*}
          %         \rot \rot \rvec v
          %          & =\grad \Div \rvec v
          %         - \Delta \rvec v
          %         \\
          %         \rot \left( \rvec u \times \rvec v \right)
          %          & = \rvec u \, \Div \rvec v
          %         - \rvec v \, \Div \rvec u
          %         + (\DD \rvec u) \rvec v
          %         - (\DD \rvec v) \rvec u
          %         \\
          %         \Div \left( \rvec u \times \rvec v \right)
          %          & = \; \scalar{\rvec v}{\rot \rvec u}
          %         - \scalar{\rvec u}{\rot \rvec v}
          %         \\
          %         \grad \left( \scalar{\rvec u}{\rvec v} \right)
          %          & = (\DD \rvec u) \rvec v
          %         + (\DD \rvec v) \rvec u
          %         + \rvec v \times \rot \rvec u
          %         + \rvec u \times \rot \rvec v
          %       \end{align*}
  \end{itemize}
\end{blueBox}

\begin{blueBox}[\underline{Potenciálosság}][nobreak]
  Egy $\rvec v: V \rightarrow V$ vektormező skalárpotenciálos, ha létezik olyan
  $\varphi: V \rightarrow \Reals$ skalármező, hogy $\rvec v = \grad \varphi$.
  Ekkor $\rot \rvec v = \rot \grad \varphi = \nvec$.
  \begin{itemize}
    \item $\rvec v$ skalárpotenciálos
          $\;\Leftrightarrow\;$
          $\rot \rvec v = \nvec$,
          hiszen $\rot \grad \varphi \equiv \nvec$
          \hfill (\textbf{örvénymentes})
  \end{itemize}
  Egy $\rvec v: V \rightarrow V$ vektormező vektorpotenciálos, ha létezik olyan
  $\rvec u: V \rightarrow V$ vektormező, hogy $\rvec v = \rot \rvec u$.
  Ekkor $\Div \rvec v = \Div \rot \rvec u = 0$.
  \begin{itemize}
    \item $\rvec v$ vektorpotenciálos
          $\;\Leftrightarrow\;$
          $\Div \rvec v = 0$,
          hiszen $\Div \rot \rvec u \equiv 0$
          \hfill (\textbf{forrásmentes})
  \end{itemize}
\end{blueBox}

\begin{blueBox}[\underline{Skalárpotenciál}][nobreak]
  Legyen $\varphi$ skalármező $\rvec v$ vektormező skalárpotenciálja. Ebben
  az esetben tudjuk, hogy $\rvec v = \grad \varphi$. Ekkor $\rvec v$
  vektormező skalárpotenciálja:
  $$
    \varphi(\coordvec)
    = \int_0^{x_1} v_1(\xi; x_2; \dots; x_n) \dd \xi
    + \int_0^{x_2} v_2(0; \xi; \dots; x_n) \dd \xi
    + \dots
    + \int_0^{x_n} v_n(0; 0; \dots; \xi) \dd \xi
    \text.
  $$
\end{blueBox}

\begin{blueBox}[\underline{Vektorpotenciál}][nobreak]
  Legyen $\rvec u(u_x; u_y; 0)$ vektormező $\rvec v$ vektormező
  vektorpotenciálja. Ebben az esetben tudjuk, hogy $\rvec v = \rot \rvec u$.
  Ekkor $\rvec u$ komponensei az alábbi módon számíthatóak:
  $$
    u_x = \int_0^z v_y(x; y; \zeta) \dd \zeta
    \text,
    \qquad
    u_y = \int_0^x v_z(\xi; y; 0) \dd \xi
    - \int_0^z v_x(x; y; \zeta) \dd \zeta
    \text.
  $$
\end{blueBox}

\begin{blueBox}[\underline{Vonalmenti integrálok}][nobreak]
  Legyen $\rvec v(\coordvec) : \Reals^3 \to \Reals^3$ vektormező,
  $\varphi(\coordvec): \Reals^3 \to \Reals$ skalármező,
  $\curvesign : \curvedomain \rightarrow \curveimage$ paraméterezett görbe, ahol
  $t \in \curvedomain$ a görbe paraméterezése,
  $\curvesign(\curvedomain) = \curveimage$ a görbe képe,
  $\dd \curvescalar = \norma{\dot{\curvesign}(t)} \dd t$,
  $\dd \curvevec = \dot{\curvesign}(t) \dd t$. Ekkor:
  \begin{itemize}
    % \item a görbe ívhossza:
    %       \begin{equation*}
    %         L = \int_{\curveimage} \dd \curvescalar
    %         = \int_{\curvedomain} \norma{\dot{\curvesign}(t)} \dd t
    %         \text,
    %       \end{equation*}
    %       \vspace{-1.5em}

    \item skalármező görbe menti skalárértékű integrálja:
          \begin{equation*}
            \int_{\curveimage} \varphi(\coordvec) \dd \curvescalar
            = \int_{\curvedomain} \varphi(\curvesign(t)) \norma{\dot{\curvesign}(t)} \dd t
            \text,
          \end{equation*}
          \vspace{-1.5em}

    \item vektormező görbe menti skalárértékű integrálja:
          \begin{equation*}
            \int_{\curveimage} \scalar{\rvec v(\coordvec)}{\dd \curvevec}
            = \int_{\curvedomain} \scalar{\rvec v(\curvesign(t))}{\dot{\curvesign}(t)} \dd t
            \text,
          \end{equation*}
          \vspace{-1.5em}

          % \item vektormező görbe menti vektorértékű integrálja:
          %       \begin{equation*}
          %         \int_{\curveimage} \rvec v(\coordvec) \times \dd \curvevec
          %         = \int_{\curvedomain} \rvec v(\curvesign(t)) \times \dot{\curvesign}(t) \dd t
          %         \text.
          %       \end{equation*}
  \end{itemize}
\end{blueBox}

\begin{blueBox}[\underline{Felületi integrálok}][nobreak]
  Legyen $\rvec v(\coordvec) : \Reals^3 \to \Reals^3$ vektormező,
  $\varphi(\coordvec): \Reals^3 \to \Reals$ skalármező,
  $\surfsign: \surfdomain \rightarrow \surfimage$ paraméterezett felület, ahol
  $s; t \in \surfdomain$ a felület paraméterezése,
  $\surfsign(\surfdomain) = \surfimage$ a felület képe,
  $\dd \surfscalar = \norma{\partial_s \surfsign \times \partial_t \surfsign} \dd s \dd t$,
  $\dd \surfvec = \uvec n \dd \surfscalar = \partial_s \surfsign \times \partial_t \surfsign \dd s \dd t$,
  $\uvec n = (\partial_s \surfsign \times \partial_t \surfsign) / \norma{\partial_s \surfsign \times \partial_t \surfsign}$.
  Ekkor:
  \begin{itemize}
    % \item a felület felszíne:
    %       \begin{equation*}
    %         A = \iint_{\surfimage} \dd \surfscalar
    %         = \iint_{\surfdomain} \norma{\pdv{\surfsign}{s} \times \pdv{\surfsign}{t}} \dd s \dd t
    %         \text,
    %       \end{equation*}
    %       \vspace{-1.5em}

    \item skalármező skalárértékű felületi integrálja:
          \begin{equation*}
            \int_{\surfimage} \varphi(\coordvec) \dd \surfscalar
            = \int_{\surfdomain} \varphi(\surfsign(s; t)) \norma{\pdv{\surfsign}{s} \times \pdv{\surfsign}{t}} \dd s \dd t
            \text,
          \end{equation*}
          \vspace{-1.5em}

    \item vektormező skalárértékű felületi integrálja:
          \begin{equation*}
            \int_{\surfimage} \scalar{\rvec v(\coordvec)}{\dd \surfvec}
            = \int_{\surfdomain} \scalar{\rvec v(\surfsign(s; t))}{\left(\pdv{\surfsign}{s} \times \pdv{\surfsign}{t}\right)} \dd s \dd t
            \text,
          \end{equation*}
          \vspace{-1.5em}

          % \item vektormező vektorértékű felületi integrálja:
          %       \begin{equation*}
          %         \int_{\surfimage} \rvec v(\coordvec) \times \dd \surfvec
          %         = \int_{\surfdomain} \rvec v(\surfsign(s; t)) \times \left(\pdv{\surfsign}{s} \times \pdv{\surfsign}{t}\right) \dd s \dd t
          %         \text.
          %       \end{equation*}
  \end{itemize}
\end{blueBox}

\begin{blueBox}[\underline{Térfogati integrál}][nobreak]
  Legyen $\varphi(\coordvec): \Reals^3 \to \Reals$ skalármező,
  $\volsign: \voldomain \rightarrow \volimage$ paraméterezett tértartomány, ahol
  $r; s; t \in \voldomain$ a tértartomány paraméterezése,
  $\volsign(\voldomain) = \volimage$ a tértartomány képe,
  $\dd \volscalar = \det(\DD \volsign(r; s; t)) \dd r \dd s \dd t$. Ekkor:
  \begin{itemize}
    % \item a tértartomány térfogata:
    %       \begin{equation*}
    %         V = \iiint_{\volimage} \dd \volscalar
    %         = \iiint_{\voldomain} \det \left( \DD \volsign(r; s; t) \right) \dd r \dd s \dd t
    %         \text,
    %       \end{equation*}
    %       \vspace{-1.5em}

    \item skalármező térfogati integrálja:
          \begin{equation*}
            \iiint_{\volimage} \varphi(\coordvec) \dd \volscalar
            = \iiint_{\voldomain} \varphi \left( \volsign(r; s; t) \right)
            \det \left( \DD \volsign(r; s; t) \right)
            \dd r \dd s \dd t
            \text.
          \end{equation*}
  \end{itemize}
\end{blueBox}

\begin{blueBox}[\underline{Integrálási tételek}][nobreak]
  \begin{itemize}
    \item \textbf{Gradiens-tétel}:
          \begin{equation*}
            \int_{\curveimage} \scalar{\grad \varphi(\coordv)}{\dd \curvevec}
            = \varphi(\curvesign(b)) - \varphi(\curvesign(a))
            \text.
          \end{equation*}
          Vagyis ha egy vektormező előáll egy skalármező gradienseként, akkor
          annak bármely zárt görbe mentén vett integrálja csak a kezdő- és
          végpontoktól függ.

    \item \textbf{Stokes-tétel}:
          \begin{equation*}
            \iint_{\surfimage} \scalar{\rot \rvec v}{\dd \surfvec}
            = \oint_{\partial \surfimage} \scalar{\rvec v}{\dd \curvevec}
            \text.
          \end{equation*}
          A tételből következik, hogy skalárpotenciálos vektormező bármely zárt
          görbén vett integrálja zérus.

    \item \textbf{Gauss-Osztogradszkij-tétel}:
          \begin{equation*}
            \iiint_{\volimage} \Div \rvec v \dd \volscalar
            = \oiint_{\partial \volimage} \scalar{\rvec v}{\dd \surfvec}
            \text.
          \end{equation*}
          A tételből következik, hogy vektorpotenciálos vektormező bármely zárt
          felületen vett integrálja zérus.

          % \item \textbf{Green-tétel asszimetrikus alakja}:
          %       \begin{equation*}
          %         \iiint_{\volimage}
          %         \psi \, \Delta \varphi +
          %         \scalar{\grad \psi}{\grad \varphi}
          %         \dd \volscalar
          %         =
          %         \oiint_{\partial \volimage} \scalar{\psi \grad \varphi}{\dd \surfvec}
          %         \text.
          %       \end{equation*}

          % \item \textbf{Green-tétel szimmetrikus alakja}:
          %       \begin{equation*}
          %         \iiint_{\volimage}
          %         \psi \, \Delta \varphi + \varphi \, \Delta \psi
          %         \dd \volscalar
          %         =
          %         \oiint_{\partial \volimage}
          %         \scalar{\psi \grad \varphi - \varphi \grad \psi}{\dd \surfvec}
          %         \text.
          %       \end{equation*}
  \end{itemize}
\end{blueBox}

\begin{blueBox}[\underline{Integrálásos összefüggések}][nobreak]
  \begin{itemize}
    % \item \textbf{Parciális integrálásás}:
    %       \begin{equation*}
    %         \int f'(x) \, g(x) \dd x
    %         = f(x) \, g(x) - \int f(x) \, g'(x) \dd x
    %         \text.
    %       \end{equation*}

    \item \textbf{Belső függvény megjelenik szorzótényezőként}:
          \begin{equation*}
            \int (f \circ g)(x) \, g'(x) \dd x
            = (F \circ g)(x) + C
            \text{, ahol } F(x) \text{ az } f(x) \text{ primitív függvénye.}
          \end{equation*}

    \item \textbf{Hatványfüggvény integrálása}:
          \begin{equation*}
            \int f^{\alpha} (x) \, f'(x) \dd x = \begin{cases}
              \dfrac{f^{\alpha + 1} (x)}{\alpha + 1} + C, & \alpha \neq -1
              \\[2mm]
              \ln |f(x)| + C,                             & \alpha = -1
            \end{cases}
          \end{equation*}
  \end{itemize}
\end{blueBox}

\clearpage
\subsection{Feladatok}

\begin{enumerate}
  \item Adja meg a $\varphi(\coordvec) = {2 x^2 y} + {x y^2 z} + {3 x z^2}$
        skalármező gradiensét a $P(-3; -2; 1)$ pontban!

  \item Adja meg a $\rvec v(\coordvec) = \ijk{x y^2 - z}{y z}{x y + 2 z}$
        vektormező divergenciáját és rotációját a $P(-1; 2; -1)$ pontban!

  \item Adja meg a $\rvec v(\coordvec) = \ijk{y^2}{2xy + e^{3z}}{3ye^{3z}}$
        egy olyan $\varphi$ skalárpotenciálját, melyre $\varphi(\nvec) = 0$.

  \item Legyen $\curvesign(t) = x(t) \cdot \hat{\uvec i} + y(t) \cdot
          \hat{\uvec j}$ a $(2; 1) \to (6; 4)$ egyenes szakasz paraméterezése.
        Adja meg $x(t)$ és $y(t)$ függvényeket, ha a paramétertartomány
        $t \in [0; 1]$. Számítsa ki a $\varphi(x; y) = 3x - 4y$ skalármező
        $\curvesign$ görbe menti integrálját!

  \item Adott az $\rvec F(x; y) = x^2 \cdot \hat{\uvec i} - xy \cdot
          \hat{\uvec j}$ erőmező. Számítsa ki az erőmező munkáját, az origó
        középpontú, $r = 1$ sugarú első síknegyedben lévő körív mentén, ha a
        bejárás az óramutató járásával ellentétes irányú! Mit mondhatunk el,
        ha a bejárási irányt megfordítjuk?

  \item Számítsa ki a $\varphi(x; y; z) = 2x$ skalármező egységgömbömbön
        vett felületi integrálját! Segítség:
        $\surfsign(s; t) = \ijk{\sin s \cos t}{\sin s \sin t}{\cos s}$,
        $\dd \surfscalar = \sin s \dd s \dd t$.

  \item Számítsa ki a $\rvec v$ vektormező $\surfsign$ felületen vett
        fluxusát, ha
        $$
          \rvec v(x; y; z) = \begin{bmatrix}
            x y    \\
            2x + y \\
            z
          \end{bmatrix}
          \text,
          \qquad
          \surfsign(s; t) = \begin{bmatrix}
            s + 2t \\
            -t     \\
            s^2 + 3t
          \end{bmatrix}
          \text,
          \qquad
          \begin{array}{l}
            s \in [0; 3] \\
            t \in [0; 1]
          \end{array}
          \text.
        $$

  \item Számítsa ki a $\rvec v(\coordvec) = \ijk{y^2}{2xy + e^{3z}}{3ye^{3z}}$
        vektormező $(0; 1; 1) \to (0; -1; 1)$ szakaszon vett vonalmenti
        integrálját!

  \item Adja meg a $\rvec v(\coordvec) = \ijk{z - y}{x - z}{y - x}$ vektormezőt
        az alábbi zárt görbén:
        \begin{enumerate}
          \item Az origóból először egy egyenes szakasz mentén eljutunk az
                $(1;0;0)$ pontba.

          \item Ezután egy origó középpontú körív mentén az $(-1;0;0)$ pontba
                jutunk. (A körív síkja legyen az $x y$ sík, és a bejárás
                az óramutató járásával ellentétes irányú.)

          \item Végül egy egyenes szakasz mentén visszatérünk az origóba.
        \end{enumerate}

  \item Határozza meg a $\rvec v(\coordvec) = \ijk{x}{y}{z}$ vektormező
        azon zárt felületen vett felületi integrálját, melyet az
        $x = y^2 + z^2$ forgásparaboloid $z > 0$ része, a $z = 0$ és az
        $x = 4$ síkok határolnak.
\end{enumerate}

\end{document}