\documentclass[fleqn]{szb-practice}

\title{Összefoglalás}
\area{Többváltozós analízis}
\subject{Matematika G3}
\subjectCode{BMETE94BG03}
\date{Utoljára frissítve: \today}
\docno{7}

\begin{document}
\maketitle

\vspace{-1em}
\subsection{Elméleti áttekintő}

\begin{blueBox}[\underline{Differenciáloperátorok}]
  Legyen $\rvec v(\coordvec) : \Reals^3 \to \Reals^3$ vektormező,
  $\varphi(\coordvec): \Reals^3 \to \Reals$ skalármező, ahol $\coordvec$ az
  $\Reals^3$-beli Descartes koordináta-rendszerben $\coordvec = (x; y; z)$.
  \begin{center}
    \def\arraystretch{1.25}
    \newenvironment{bm}{\bgroup\renewcommand*{\arraystretch}{1.1}\begin{bmatrix}}{\end{bmatrix}\egroup}
    \newcommand{\dspl}[3]{\begin{bm}#1\\#2\\#3\end{bm}}
    \newcommand\nablavec{\dspl{\partial_x}{\partial_y}{\partial_z}}
    \begin{tabular}{*{3}{>{\centering\arraybackslash}p{3.5cm}}}
      \def\arraystretch{1}
      % &
      \bfseries Rotáció
       & \bfseries Divergencia
       & \bfseries Gradiens
      \\
      \hline
      % Jelölés & 
      $\rot \rvec v$
       & $\Div \rvec v$
       & $\grad \varphi$
      \\
      % Operátor & 
      $\nabla \times \rvec v$
       & $\scalar{\nabla}{\rvec v}$
       & $\nabla \cdot \varphi$
      \\
      % Számítás &
      $\nablavec \times \dspl{v_x}{v_y}{v_z}$
       & $\scalar{\nablavec}{\dspl{v_x}{v_y}{v_z}}$
       & $\dspl{\partial_x \varphi}{\partial_y \varphi}{\partial_z \varphi}$
      \\
      % Ért. tart. & 
      $\Domain_{\rvec v} = \Reals^3$
       & $\Domain_{\rvec v} = \Reals^3$
       & $\Domain_{\varphi} = \Reals^3$
      \\
      % Ért. készl. &
      $\Range_{\rvec v} = \Reals^3$
       & $\Range_{\rvec v} = \Reals^3$
       & $\Range_{\varphi} = \Reals$
      \\
      % Ért. készl. &
      $\Range_{\rot \rvec v} = \Reals^3$
       & $\Range_{\Div \rvec v} = \Reals$
       & $\Range_{\grad \varphi} = \Reals^3$
      \\
      % \hline
    \end{tabular}
  \end{center}
\end{blueBox}

\begin{blueBox}[\underline{Azonosságok}][nobreak]
  \begin{itemize}
    \item Teljesül a linearitás:
          \vspace{-.5em}
          \begin{alignat*}{4}
            \grad & (\lambda \, \varPhi && + \mu \, \varPsi) && = \lambda \, \grad \varPhi && + \mu \, \grad \varPsi
            \\
            \rot  & (\lambda \, \rvec v && + \mu \, \rvec w) && = \lambda \, \rot \rvec v  && + \mu \, \rot \rvec w
            \\
            \Div  & (\lambda \, \rvec v && + \mu \, \rvec w) && = \lambda \, \Div \rvec v  && + \mu \, \Div \rvec w
          \end{alignat*}
          \vspace{-2.5em}

    \item Zérusság:
          \vspace{-.5em}
          \begin{align*}
            \rot \grad \varPhi & \equiv \nvec
            \\
            \Div \rot \rvec v  & \equiv 0
          \end{align*}
          \vspace{-2.5em}

    \item Deriválási szabályokhoz hasonló:
          \vspace{-.5em}
          \begin{align*}
            \grad \left( \varPhi \, \varPsi \right)
             & = \varPhi \, \grad \varPsi
            + \varPsi \, \grad \varPhi
            \\
            \Div \left( \varPhi \, \rvec v \right)
             & = \varPhi \, \Div \rvec v \,
            + \scalar{\rvec v}{\grad \varPhi}
            \\
            \rot \left( \varPhi \, \rvec v \right)
             & = \varPhi \, \rot \rvec v
            - \rvec v \times \grad \varPhi
          \end{align*}
          \vspace{-2.5em}

    \item Egyéb szabályok:
          \vspace{-.5em}
          \begin{align*}
            \rot \rot \rvec v
             & =\grad \Div \rvec v
            - \Delta \rvec v
            \\
            \rot \left( \rvec u \times \rvec v \right)
             & = \rvec u \, \Div \rvec v
            - \rvec v \, \Div \rvec u
            + (\DD \rvec u) \rvec v
            - (\DD \rvec v) \rvec u
            \\
            \Div \left( \rvec u \times \rvec v \right)
             & = \; \scalar{\rvec v}{\rot \rvec u}
            - \scalar{\rvec u}{\rot \rvec v}
            \\
            \grad \left( \scalar{\rvec u}{\rvec v} \right)
             & = (\DD \rvec u) \rvec v
            + (\DD \rvec v) \rvec u
            + \rvec v \times \rot \rvec u
            + \rvec u \times \rot \rvec v
          \end{align*}
  \end{itemize}
\end{blueBox}

\clearpage
\begin{blueBox}[\underline{Potenciálosság}]
  Egy $\rvec v: V \rightarrow V$ vektormező skalárpotenciálos, ha létezik olyan
  $\varphi: V \rightarrow \Reals$ skalármező, hogy $\rvec v = \grad \varphi$.
  Ekkor $\rot \rvec v = \rot \grad \varphi = \nvec$.

  Egy $\rvec v: V \rightarrow V$ vektormező vektorpotenciálos, ha létezik olyan
  $\rvec u: V \rightarrow V$ vektormező, hogy $\rvec v = \rot \rvec u$.
  Ekkor $\div \rvec v = \div \rot \rvec u = 0$.
\end{blueBox}

\begin{blueBox}[\underline{Vonalmenti integrálok}][nobreak]
  Legyen $\rvec v(\coordvec) : \Reals^3 \to \Reals^3$ vektormező,
  $\varphi(\coordvec): \Reals^3 \to \Reals$ skalármező,
  $\curvesign : \curvedomain \rightarrow \curveimage$ paraméterezett görbe, ahol
  $t \in \curvedomain$ a görbe paraméterezése,
  $\curvesign(\curvedomain) = \curveimage$ a görbe képe,
  $\dd \curvescalar = \norma{\dot{\curvesign}(t)} \dd t$,
  $\dd \curvevec = \dot{\curvesign}(t) \dd t$. Ekkor:
  \begin{itemize}
    \item skalármező görbe menti skalárértékű integrálja:
          \begin{equation*}
            \int_{\curveimage} \varphi(\coordvec) \dd \curvescalar
            = \int_{\curvedomain} \varphi(\curvesign(t)) \norma{\dot{\curvesign}(t)} \dd t
            \text,
          \end{equation*}
          \vspace{-1.5em}

    \item vektormező görbe menti skalárértékű integrálja:
          \begin{equation*}
            \int_{\curveimage} \scalar{\rvec v(\coordvec)}{\dd \curvevec}
            = \int_{\curvedomain} \scalar{\rvec v(\curvesign(t))}{\dot{\curvesign}(t)} \dd t
            \text,
          \end{equation*}
          \vspace{-1.5em}

    \item vektormező görbe menti vektorértékű integrálja:
          \begin{equation*}
            \int_{\curveimage} \rvec v(\coordvec) \times \dd \curvevec
            = \int_{\curvedomain} \rvec v(\curvesign(t)) \times \dot{\curvesign}(t) \dd t
            \text.
          \end{equation*}
  \end{itemize}
\end{blueBox}

\begin{blueBox}[\underline{Felületi integrálok}][nobreak]
  Legyen $\rvec v(\coordvec) : \Reals^3 \to \Reals^3$ vektormező,
  $\varphi(\coordvec): \Reals^3 \to \Reals$ skalármező,
  $\surfsign: \surfdomain \rightarrow \surfimage$ paraméterezett felület, ahol
  $s; t \in \surfdomain$ a felület paraméterezése,
  $\surfsign(\surfdomain) = \surfimage$ a felület képe,
  $\dd \surfscalar = \norma{\partial_s \surfsign \times \partial_t \surfsign} \dd s \dd t$,
  $\dd \surfvec = \uvec n \dd \surfscalar = \partial_s \surfsign \times \partial_t \surfsign \dd s \dd t$,
  $\uvec n = (\partial_s \surfsign \times \partial_t \surfsign) / \norma{\partial_s \surfsign \times \partial_t \surfsign}$.
  Ekkor:
  \begin{itemize}
    \item skalármező skalárértékű felületi integrálja:
          \begin{equation*}
            \int_{\surfimage} \varphi(\coordvec) \dd \surfscalar
            = \int_{\surfdomain} \varphi(\surfsign(s; t)) \norma{\pdv{\surfsign}{s} \times \pdv{\surfsign}{t}} \dd s \dd t
            \text,
          \end{equation*}
          \vspace{-1.5em}

    \item vektormező skalárértékű felületi integrálja:
          \begin{equation*}
            \int_{\surfimage} \scalar{\rvec v(\coordvec)}{\dd \surfvec}
            = \int_{\surfdomain} \scalar{\rvec v(\surfsign(s; t))}{\left(\pdv{\surfsign}{s} \times \pdv{\surfsign}{t}\right)} \dd s \dd t
            \text,
          \end{equation*}
          \vspace{-1.5em}

    \item vektormező vektorértékű felületi integrálja:
          \begin{equation*}
            \int_{\surfimage} \rvec v(\coordvec) \times \dd \surfvec
            = \int_{\surfdomain} \rvec v(\surfsign(s; t)) \times \left(\pdv{\surfsign}{s} \times \pdv{\surfsign}{t}\right) \dd s \dd t
            \text.
          \end{equation*}
  \end{itemize}
\end{blueBox}

\begin{blueBox}[\underline{Térfogati integrál}][nobreak]
  Legyen $\varphi(\coordvec): \Reals^3 \to \Reals$ skalármező,
  $\volsign: \voldomain \rightarrow \volimage$ paraméterezett tértartomány, ahol
  $r; s; t \in \voldomain$ a tértartomány paraméterezése,
  $\volsign(\voldomain) = \volimage$ a tértartomány képe,
  $\dd \volscalar = \det(\DD \volsign(r; s; t)) \dd r \dd s \dd t$. Ekkor:
  \begin{itemize}
    \item skalármező térfogati integrálja:
          \begin{equation*}
            \iiint_{\volimage} \varphi(\coordvec) \dd \volscalar
            = \iiint_{\voldomain} \varphi \left( \volsign(r; s; t) \right)
            \det \left( \DD \volsign(r; s; t) \right)
            \dd r \dd s \dd t
            \text.
          \end{equation*}
  \end{itemize}
\end{blueBox}

\begin{blueBox}[\underline{Integrálási tételek}]
  \begin{itemize}
    \item \textbf{Gradiens-tétel}:
          \begin{equation*}
            \int_{\curveimage} \scalar{\grad \varphi(\coordv)}{\dd \curvevec}
            = \varphi(\curvesign(b)) - \varphi(\curvesign(a))
            \text.
          \end{equation*}
          Vagyis ha egy vektormező előáll egy skalármező gradienseként, akkor
          annak bármely zárt görbe mentén vett integrálja csak a kezdő- és
          végpontoktól függ.

    \item \textbf{Stokes-tétel}:
          \begin{equation*}
            \iint_{\surfimage} \scalar{\rot \rvec v}{\dd \surfvec}
            = \oint_{\partial \surfimage} \scalar{\rvec v}{\dd \curvevec}
            \text.
          \end{equation*}
          A tételből következik, hogy skalárpotenciálos vektormező bármely zárt
          görbén vett integrálja zérus.

    \item \textbf{Gauss-Osztogradszkij-tétel}:
          \begin{equation*}
            \iiint_{\volimage} \Div \rvec v \dd \volscalar
            = \oiint_{\partial \volimage} \scalar{\rvec v}{\dd \surfvec}
            \text.
          \end{equation*}
          A tételből következik, hogy vektorpotenciálos vektormező bármely zárt
          felületen vett integrálja zérus.

    \item \textbf{Green-tétel asszimetrikus alakja}:
          \begin{equation*}
            \iiint_{\volimage}
            \psi \, \Delta \varphi +
            \scalar{\grad \psi}{\grad \varphi}
            \dd \volscalar
            =
            \oiint_{\partial \volimage} \scalar{\psi \grad \varphi}{\dd \surfvec}
            \text.
          \end{equation*}

    \item \textbf{Green-tétel szimmetrikus alakja}:
          \begin{equation*}
            \iiint_{\volimage}
            \psi \, \Delta \varphi + \varphi \, \Delta \psi
            \dd \volscalar
            =
            \oiint_{\partial \volimage}
            \scalar{\psi \grad \varphi - \varphi \grad \psi}{\dd \surfvec}
            \text.
          \end{equation*}
  \end{itemize}
\end{blueBox}

\clearpage
\subsection{Feladatok}

\begin{enumerate}
  \item
\end{enumerate}

\end{document}