\documentclass{szb-practice}

\title{Bevezetés}
\area{Differenciálegyenletek}
\subject{Matematika G3}
\subjectCode{BMETE94BG03}
\date{Utoljára frissítve: \today}
\docno{8}

\usepackage{siunitx}
\sisetup{locale = DE}

\begin{document}
\maketitle

\subsection{Elméleti áttekintő}

\begin{definition}[Differenciálegyenlet]
  Legyen $y : \Reals \rightarrow \Reals$ $n$-szer folytonosan differenciálható
  függvény, vagyis $y^{(0)} = y$, $y^{(1)} = y'$, $y^{(2)} = y''$, $\dots$,
  $y^{(n)} = y^{(n)}$ folytonos függvények, $x$ független változó. Ekkor az
  $F \left( x ; y ; y' ; \dots ; y^{(n)} \right) = 0$ egyenletet $y$-ra
  vonatkoztatott, $n$-edrendű, közönséges differenciálegyenletnek nevezzük.
\end{definition}

\begin{note}
  A \textbf{közönséges} arra utal, hogy egy független változót tartalmaz az
  egyenlet. Ha nem közönséges, akkor \textbf{parciális} (többváltozós) a
  differenciálegyenlet.

  A \textbf{rend} a legmagasabb fokú deriváltra utal.

  A differenciálegyenletek megadási módja alapján:
  \begin{itemize}
    \item \textbf{implicit} megadás:
          $F \left( x ; y ; y' ; \dots ; y^{(n)} \right) = 0$,

    \item \textbf{explicit} megadás:
          $y^{(n)} = f(x; y; y'; \dots; y^{(n-1)})$.
  \end{itemize}
\end{note}

\begin{definition}[Lineáris differenciálegyenlet]
  Azt a differenciálegyenletet, amelyben az ismeretlen függvény, és annak
  deriváltjai csak elsőfokon, szorzatuk pedig egyáltalán nem fordul elő,
  lineáris differenciálegyenletnek mondjuk. Ellenkező esetben nemlineáris.
\end{definition}

\begin{example}
  A következő egyenlet $y$-ra vonatkoztatott, lineáris, másodrendű közönséges
  differenciálegyenlet, a független változó $x$:
  $$
    y'' + 2y' + y = 4e^x.
  $$
\end{example}

\begin{blueBox}[\underline{Newton II. törvénye}][nobreak]
  Az $F = ma$ egyenlet a testre ható erő és a test gyorsulásának kapcsolatát
  írja le. Mivel a gyorsulás az elmozdulás idő szerinti második deriváltja
  ($a = \ddot x$), ezért a mozgásegyenlet egy másodrendű differenciálegyenlet
  lesz:
  $$
    F = m a = m \ddot x
    \text.
  $$
\end{blueBox}

\begin{blueBox}[\underline{Szabadesés során megtett út}]
  Szabadesés során a testre ható erő a test tömegével és a gravitációs
  gyorsulással arányos:
  $$
    F = mg = m \ddot x
    \quad \implies \quad
    \ddot x = g
    \text.
  $$
  Ha kétszer integráljuk az egyenletet, akkor megkapjuk a megtett utat:
  \begin{align*}
    \dot x & = \int g \dd t = gt + v_0 \text,                                \\
    x      & = \int (gt + v_0) \dd t = \frac{1}{2} gt^2 + v_0 t + x_0 \text.
  \end{align*}
\end{blueBox}

\begin{blueBox}[\underline{Ferde hajítás}]
  Egy testet $\rvec v(v_x; v_y)$ kezdeti sebességgel elhajítunk. A testre ható
  erő a gravitációs erő $-y$ irányú. $x$ irányban a testre nem hat erő.
  A mozgásegyenletek:
  \vspace{-1em}
  \begin{center}
    \def\arraystretch{1.5}
    \begin{tabular}{r l c l}
      $(x)$ & $m \ddot x = 0$   & $\implies$ & $\ddot x = 0$,
      \\
      $(y)$ & $m \ddot y = -mg$ & $\implies$ & $\ddot y = -g$.
    \end{tabular}
  \end{center}
  \vspace{-1em}
  Ha ezeket az egyenleteket kétszer integráljuk, akkor megkapjuk a vízszintes
  és függőleges elmozdulást:
  \begin{align*}
    x(t) & = v_x t + x_0 \text,                     \\
    y(t) & = -\frac{1}{2} gt^2 + v_y t + y_0 \text.
  \end{align*}
  Az időt kifejezve az $x(t)$ egyenletből, majd ebből a pályák egyenlete:
  $$
    t = \frac{x - x_0}{v_x}
    \quad \implies \quad
    y(x) = -\frac{g}{2 v_x^2} (x - x_0)^2 + \frac{v_y}{v_x} (x - x_0) + y_0
    \text.
  $$
  A pályák lefele nyíló parabolák.
\end{blueBox}

% \begin{blueBox}[\underline{Rugalmas szál differenciálegyenlete}][nobreak]
%   Egy egyik végén befogott, másik végén $F$ erővel terhelt rúd
%   lehajásfüggvénye:
%   $$
%     v''(x) = \frac{-M_h(x)}{IE}
%     \text{,\quad ahol \quad}
%     M_h(x) = F (L - x) = M_0 - Fx
%     \text.
%   $$
%   Integráljuk az egyenletet kétszer:
%   \begin{align*}
%     v'(x) & = \frac{-M_0 x}{IE} + \frac{F x^2}{2IE} + C_1
%     \text,                                                           \\
%     v(x)  & = \frac{-M_0 x^2}{2IE} + \frac{F x^3}{6IE} + C_1 x + C_2
%     \text.
%   \end{align*}
% \end{blueBox}

\begin{definition}[Megoldásgörbe]
  A közönséges differenciálegyenlet megoldásfüggvényének görbéje a
  differenciálegyenlet integrálgörbéje/megoldásgörbéje.
\end{definition}

\begin{note}
  Végtelen sok megoldásgörbe létezik. Egy $n$-edrendű differenciálegyenlet
  esetén $n$ darab tetszőleges konstans jelenik meg a megoldásban.
\end{note}

\begin{definition}[Differenciálegyenlet megoldása][nobreak]
  Az $F \left( x ; y ; y' ; \dots ; y^{(n)} \right) = 0$ differenciálegyenlet
  megoldása a $\varphi : \Reals \rightarrow \Reals$ függvény, ha
  $\forall x \in \Reals$-re
  $$
    F \left( x ; \varphi(x) ; \varphi'(x) ; \dots ; \varphi^{(n)}(x) \right) = 0
    \text.
  $$
\end{definition}

\begin{example}[][nobreak]
  Az $y'' + 2y' + y = 4e^x$ differenciálegyenlet megoldása a
  $\varphi(x) = e^x + C_1 e^{-x} + C_2 x e^{-x}$ függvény:
  \begin{align*}
    \varphi'(x)  & = e^x - C_1 e^{-x} + C_2 e^{-x} - C_2 x e^{-x},  \\
    \varphi''(x) & = e^x + C_1 e^{-x} - 2C_2 e^{-x} + C_2 x e^{-x}.
  \end{align*}
  Behelyettesítve a $\varphi$, $\varphi'$, és $\varphi''$ kifejezéseket:
  \begin{align*}
    \varphi''(x) + 2\varphi'(x) + \varphi(x)
     & = e^x + C_1 e^{-x} - 2C_2 e^{-x} + C_2 x e^{-x}
    \\
     & \quad + 2 \left( e^x - C_1 e^{-x} + C_2 e^{-x} - C_2 x e^{-x} \right)
    \\
     & \quad + e^x + C_1 e^{-x} + C_2 x e^{-x}
    \\
     & = 4e^x.
  \end{align*}
\end{example}

\begin{definition}[Cauchy-feladat][nobreak]
  Ha az $n$-ed rendű differenciálegyenlet olyan megoldását keressük, hogy
  $$
    y(x_0) = y_0, \quad
    y'(x_0) = y_0', \quad
    y''(x_0) = y_0'', \quad
    \dots, \quad
    y^{(n)}(x_0) = y_0^{(n)}
  $$
  feltételeket kiegyenlíti, akkor Cauchy-feladatról / kezdeti érték feladatról
  beszélünk.
\end{definition}

\begin{example}
  Határozzuk meg az $y'' + 2y' + y = 4e^x$ differenciálegyenlet azon megoldását,
  amely kielégíti az $y(0) = 2$ és $y'(0) = 1$ kezdeti feltételeket!

  Az általános megoldás és ennek deriváltja:
  $$
    \begin{aligned}
      y(x)  & = e^x + C_1 e^{-x} + C_2 x e^{-x} \text,
      \\
      y'(x) & = e^x - C_1 e^{-x} + C_2 e^{-x} - C_2 x e^{-x} \text.
    \end{aligned}
  $$
  Behelyettesítve a kezdeti feltételeket:
  $$
    \begin{aligned}
      y(0)  & = 1 + C_1 + 0 = 2
            & \qquad \Rightarrow \qquad C_1 = 1 \text,
      \\
      y'(0) & = 1 + (C_2 - 1) - 0 = 1
            & \qquad \Rightarrow \qquad C_2 = 1 \text.
    \end{aligned}
  $$
  Vagyis a keresett partikuláris megoldás:
  $$
    y(x) = e^x + e^{-x} + x e^{-x}
    \text.
  $$
\end{example}

\begin{blueBox}[\underline{Izoklinák és iránymező}][nobreak]
  Az $y' = f(x; y)$ differenciálegyenlet izoklinái azok a görbék, amelyek
  mentén a megoldásgörbék érintőinek meredeksége állandó. Az izoklinák
  egyenlete:
  $$
    f(x; y) = c
    \text.
  $$
  Az iránymező a megoldásgörbék érintőinek irányát mutatja meg egy tetszőleges
  pontban.
\end{blueBox}

\begin{blueBox}[\underline{Görbesereg differenciálegyenlete}]
  Minden $n$-edrendű differenciálegyenlethez tartozik egy $n$ paraméteres $g$
  görbesereg:
  $$
    g(x; y; c_1; \dots; c_n) = 0
    \text.
  $$
  Hogyan kapható meg a egy adott görbesereghez tartozó differenciálegyenlet?
  \begin{itemize}
    \item Deriváljuk az egyenletet $n$-szer.
    \item Fejezzük ki a $c_1$, $c_2$, $\dots$, $c_n$ paramétereket az
          egyenletekből.
    \item Helyettesítsük vissza ezeket az eredeti egyenletbe.
  \end{itemize}
\end{blueBox}

\begin{example}
  Adja meg az alábbi görbesereg differenciálegyenletét:
  $$
    y = c_1 e^x + c_2 x e^{x} = e^x (c_1 + c_2 x)
    \text.
  $$
  Deriváljuk az egyenletet kétszer:
  \begin{align*}
    y'  & = c_1 e^x + c_2 e^x + c_2 x e^x = e^x (c_1 + c_2 + c_2 x) \text,     \\
    y'' & = c_1 e^x + 2 c_2 e^x + c_2 x e^x = e^x (c_1 + 2 c_2 + c_2 x) \text.
  \end{align*}
  Számítsuk ki $y' - y$ különbséget:
  $$
    y' - y
    = e^x (c_1 + c_2 + c_2 x) - e^x (c_1 + c_2 x)
    = c_2 e^x
    \quad \implies \quad
    c_2 = e^{-x} (y' - y)
    \text.
  $$
  Az első konstans az eredeti egyenletből:
  $$
    c_1
    = ye^{-x} - c_2x
    = e^{-x}(y - x(y' - y))
    = e^{-x} (y - xy' + xy)
    \text.
  $$
  A konstansokat helyettesítsük be az $y''$-s egyenletbe:
  \begin{align*}
    y''
     & = e^x (c_1 + 2 c_2 + c_2 x)              \\
     & = (y - xy' + xy) + 2(y' - y) + x(y' - y) \\
     & = 2y' - y
    \text.
  \end{align*}
  A keresett differenciálegyenlet tehát:
  $$
    y'' - 2y' + y = 0
    \text.
  $$
\end{example}

\clearpage
\subsection{Feladatok}

\begin{enumerate}
  \item Osztályozza az alábbi differenciálegyenleteket:
        \begin{multicols}{2}
          \begin{enumerate}
            \item $y' = \cosh x - 3xy$,
            \item $y'' = y'^2 \, \cos x$,
            \item $\left( 1 + y^{(\mathrm{IV})} \right)^2 - y'' = x^3 y''' + xy$,
            \item $y'' = e^y \, \ln x$.
          \end{enumerate}
        \end{multicols}

        %   \item Newton második törvénye ($F = ma$) segítségével vezesse le a
        %         szabadesés egyenletét, ha a légellenállást elhanyagoljuk!

        %   \item Egy testet $\rvec v(v_x; v_y)$ kezdeti sebességgel elhajítunk. Adja
        %         meg a test vízszintes és függőleges elmozdulását leíró
        %         differenciálegyenleteit!

        %   \item Vezesse le egy $L$ hosszúságú, egyik végénél befogott, másik oldalon
        %         $F$ erővel terhelt rúd lehajlásfüggvényét!

  \item Mik lesznek az izoklinák és a vonalelemek?
        \begin{enumerate}
          \item $y' = y / x \qquad x > 0$
          \item $y' = -x / y \qquad y < 0$
        \end{enumerate}

  \item Az $(y')^4 + y^2 = -1$ differenciálegyenlet megoldása-e az
        $y = x^2 - 1$ függvény?

  \item Megoldása-e az $y = C_1 \sin 2x + C_2 \cos 2x$ függvény az
        $y'' + 4y = 0$ differenciálegyenletnek?

  \item Megoldása-e az $y = \ln x$ függvény az $xy'' + y' = 0$
        differenciálegyenletnek?

  \item Adja meg az $y'' + 4y = 0$ differenciálegyenlet $y(0) = 0$ és
        $y'(0) = 1$ kezdeti feltételek melletti megoldását!

  \item Adja meg az $y'' + 4y = 0$ differenciálegyenlet $y(0) = 1$ és
        $y'(\pi/4) = 2$ kezdeti feltételek melletti megoldását!

  \item Adja meg az alábbi görbeserekek differenciálegyenleteit:
        \begin{enumerate}[a)]
          \item $y = c x^2$,
          \item $x^2 + y^2 = cx$,
          \item $y = c_1 e^x + c_2 e^{2x}$.
        \end{enumerate}

  \item Adja meg az olyan $xy$ síkban elhelyezkedő körök differenciálegyenletét,
        amelyek az $x$-tengelyt az origóban érintik!

        %   \item Adja meg az $y' = 3y^{2/3}$ differenciálegyenlet $y(0) = 0$ kezdeti
        %         feltétel melletti megoldását! Vizsgálja meg a megoldás egyértelműségét!

        %   \item Adja meg azt a tértartományt, ahol az $y' = x^2 + y^2$
        %         differenciálegyenlet megoldása egyértelmű!

        %   \item Adja meg azt a tértartományt, ahol az $y'' = y + 3 \sqrt{y} + e^{y'}$
        %         differenciálegyenlet megoldása egyértelmű!

        %   \item Számítsa ki az $y' = y$ differenciálegyenlet szukcesszív
        %         approximációjával kapott első négy közelítő függvényt, ha a kezdeti
        %         feltétel $y(0) = 1$!

        %   \item Számítsa ki az $y' = xy$ differenciálegyenlet szukcesszív
        %         approximációjával kapott első négy közelítő függvényt, ha a kezdeti
        %         feltétel $y(0) = 1$!

        %   \item Oldja meg a következő szétválasztható differenciálegyenleteket!
        %         \begin{alignat*}{9}
        %           a & ) \quad (2x + 1) y' - 3y = 0
        %           \text,                                                            \\
        %           b & ) \quad \sqrt{1 + x^2} \, y' - \sqrt{1 - y^2} = 0
        %           \text,                                                            \\
        %           c & ) \quad y' = \frac{1 - x - y}{2x - 2y - 3}
        %           \text,                                                            \\
        %           d & ) \quad xy' = y(1 + \ln x - \ln y)
        %           \text,                                                            \\
        %           e & ) \quad 2xyy' = x^2 + y^2
        %           \text,                                                            \\
        %           f & ) \quad y' = \frac{3x^2 + 4x + 2}{2y - 2}
        %           \text,
        %             & \qquad                                            & y(0) = -1
        %           \text,                                                            \\
        %           g & ) \quad xy' = x \cdot e^{\sfrac{y}{x}} + y
        %           \text,
        %             & \qquad                                            & y(1) = 0
        %           \text.
        %         \end{alignat*}

        %   \item Adja meg azon görbét, amelynek bármely pontjában az érintő a
        %         a koordináta-ten\-ge\-lyek közé eső részét az adott pontban felezi!

        %   \item Newton-törvénye értelmében ismert, hogy egy test hőmérsékletének
        %         változása a környezet hőmérsékletével való különbséggel arányos.
        %         Egy kenyeret a $t = 0$ időpillanatban kiveszünk a
        %         $\SI{200}{\degreeCelsius}$-os sütőből, majd hűlni hagyjuk. 20 perc
        %         után $\SI{60}{\degreeCelsius}$-ra hűl le. Mennyi idő múlva éri el a
        %         kenyér hőmérséklete a $\SI{30}{\degreeCelsius}$-ot, ha a környezet
        %         hőmérséklete $\SI{20}{\degreeCelsius}$?

        %   \item $\SI{100}{\kilogram}$ $\SI{10}{\percent}$-os sóoldatot tartalmazó
        %         edénybe másodpercenként $\SI{10}{\liter}$ tiszta víz áramlik be.
        %         Mikor lesz a sóoldat koncentrációja $\SI{5}{\percent}$-os, ha a
        %         keveredés azonnal megtörténik, és ugyanilyen sebességgel folyik ki az
        %         edényből a keverék?

        %         % OLD 9
        %   \item Oldja meg a következő elsőrendű differenciálegyenleteket!
        %         \begin{align*}
        %           a) \quad y' & - \frac{y}{x} = x \, e^x
        %           \text,                                      \\
        %           b) \quad y' & + 2xy + x \, e^{-x^2} = 0
        %           \text,                                      \\
        %           c) \quad y' & + \frac{1 - 2x}{x^2} \, y = 1
        %           \text,                                      \\
        %           d) \quad y' & + y \cos x = \cos x \sin x
        %           \text,                                      \\[2mm]
        %           e) \quad y' & (1 + x^2) + 2xy = \tan x
        %           \text.
        %         \end{align*}

        %   \item $\SI{10}{\liter}$ vizet tartalmazó edénybe literenként
        %         $\SI{0,3}{\kilogram}$ sót tartalmazó oldat folyik be
        %         $\SI[per-mode = symbol]{2}{\liter\per\minute}$ sebességgel. Az edényen
        %         a folyadék azonnal elkeveredik, majd ugyanilyen sebességgel kifolyik.
        %         Mennyi só lesz $\SI{5}{\minute}$ múlva az edényben?

        %   \item Egy testet függőlegesen hajítunk lefelé $v_0$ kezdeti sebességgel.
        %         Határozza meg a mozgás sebességének változását, amennyiben a testre csak
        %         a nehézségi erő hat, valamint a levegő fékezőerelye a sebességgel
        %         egyenesen arányos!

        %   \item Oldja meg a következő másodrendű, hiányos differenciálegyenleteket!
        %         \begin{align*}
        %           a) \quad y'' & = 6x + \sin x
        %           \text,                                                     \\[2mm]
        %           b) \quad y'' & x^2 + \ln x = 1
        %           \text,                                                     \\[2mm]
        %           c) \quad y'' & - \frac{x}{x^2 - 1} \, y' = 0 \quad (x > 1)
        %           \text,                                                     \\[2mm]
        %           d) \quad y'' & (1 + y^2) = y \, y'^2
        %           \text,                                                     \\[2mm]
        %           e) \quad y'' & {}^2 - y' = 0
        %           \text,                                                     \\[2mm]
        %           e) \quad y'' & y = 1 + y'^2
        %           \text.                                                     \\[2mm]
        %         \end{align*}

        %         % OLD 10
        %   \item Egy soros RL-körre konstans $u_0$ feszültséget kapcsolunk. Adja meg
        %         az áram időfüggvényét!

        %         \begin{minipage}{.25\textwidth}
        %           \centering
        %           \begin{tikzpicture}[european resistors]
        %             \draw
        %             % (1.5,0) to[short,o-]
        %             (0,0)
        %             to (0,1.5)
        %             to[R=$R$] (2,1.5)
        %             to[L=$L$] (4,1.5)
        %             to (4,0)
        %             % to[short,-o] (2.5,0)
        %             to[battery2] (0,0)
        %             -- cycle
        %             ;
        %           \end{tikzpicture}
        %         \end{minipage}\begin{minipage}{.4\textwidth}
        %           \begin{align*}
        %             u_R & = R \cdot i
        %             \\
        %             u_L & = L \cdot \odv{i}{t}
        %           \end{align*}
        %         \end{minipage}

        %   \item Határozza meg, hogy milyen alakot vesz fel saját súlyának hatására a $C$
        %         és $D$ végein felfüggesztett, homogén, hajlékony lánc!

        %   \item Oldja meg az alábbi egzakt, illetve egyakttá tehető
        %         differenciálegyenleteket!
        %         \begin{enumerate}
        %           \item $\displaystyle
        %                   (2x + 2 \sin y) \dd x
        %                   + (2x \cos y - \sin y) \dd y
        %                   = 0
        %                 $,
        %                 \vspace{2mm}

        %           \item $\displaystyle
        %                   (1 - xy) \dd x
        %                   + (xy - x^2) \dd y
        %                   = 0
        %                 $,

        %           \item $\displaystyle
        %                   \left( \ln (y^2 + 1) \right) \dd x
        %                   + \frac{2y(x - 1)}{y^2 + 1} \dd y
        %                   = 0
        %                 $,

        %           \item $\displaystyle
        %                   \left( e^{-y} \right) \dd x
        %                   + \left( x \, e^{-y} - 2y \, e^{-2y} \right) \dd y
        %                   = 0
        %                 $.
        %         \end{enumerate}

        %   \item Oldja meg az alábbi Bernoulli-féle differenciálegyenletet!
        %         $$
        %           y' - \frac{3y}{x} = x^4 y^{1/3}
        %         $$

        %   \item Oldja meg az alábbi Ricatti féle differenciálegyenletet!
        %         $$
        %           y' = xy^2 - (4 - 2x^2)y + x^3 + 4x + 1
        %         $$

        %         % OLD 11
        %   \item Határozza meg, hogy milyen halmazon lineárisan függetlenek az alábbi
        %         függvények!
        %         $$
        %           H_1 = \Big\{\;
        %           \;e^x;
        %           \;xe^x;
        %           \;x^2e^x
        %           \;\Big\}
        %           \hspace{2cm}
        %           H_2 = \Big\{\;
        %           \;1;
        %           \;\cos x;
        %           \;\sin x;
        %           \;\Big\}
        %         $$

        %   \item Hol lesz a $\{ \; e^x; \; -(x + 1) \; \}$ függvényrendszer az
        %         $xy'' - (x + 1)y' + y = 0$ differenciálegyenlet alaphalmaza?

        %   \item Hol lesz a $\{ \; 1; \; x; \; \ln x \; \}$ függvényrendszer az
        %         $xy''' + 2y'' = 0$ differenciálegyenlet alaphalmaza?

        %   \item Adja meg az $y'' - 5y' + 6y = 0$ differenciálegyenlet $y(0) = 1$ és
        %         $y'(0) = 1$ kezdeti feltételek melletti megoldását!

        %   \item Adja meg a $8y''' + 12y'' + 6y' + y = 0$ differenciálegyenlet
        %         általános megoldását!

        %   \item Adja meg a $y''' - 2y'' + 2y' = 0$ differenciálegyenlet
        %         általános megoldását!

        %   \item Adja meg a $y^{(\mathbf{V})} - 2y^{(\mathbf{IV})} + 8y''' - 16y'' + 16y'
        %           -32y = 0$ differenciálegyenlet általános megoldását!

        %   \item Adja meg azt a legkisebb rendű differenciálegyenletet, amelynek a
        %         $\{\; 6x^2; \; 5e^{2x} \;\}$ függvényrendszer az alaphalmaza!

        %   \item Adja meg azt a legkisebb rendű differenciálegyenletet, amelynek a
        %         $\{\; 7x; \; \sin 5x \;\}$ függvényrendszer az alaphalmaza!

        %   \item Adja meg azt a legkisebb rendű differenciálegyenletet, amelynek a
        %         $\{\; x e^x; \; e^{2x} \cos x \;\}$ függvényrendszer az alaphalmaza!

        %   \item Adja meg az $y'' - 5y' + 6y = 2 \sin 2x$ differenciálegyenlet
        %         általános megoldását!

        %   \item Adja meg az $y'' - 5y' + 6y = 2x e^x + e^{2x}$ differenciálegyenlet
        %         általános megoldását!

        %   \item Adja meg az $y'' + 4y = x^2 \sin 2x$ differenciálegyenlet
        %         általános megoldását!

        %         % OLD 12
        %   \item Egy $\SI{6}{\meter}$ hosszú lánc súrlódásmentesen csúszik az asztalon.
        %         Ha a csúszás akkor következik be, amikor $\SI{1}{\meter}$-nyi lánc
        %         lóg lefelé, akkor mennyi idő múlva esik le a lánc?

        %   \item REZGŐRENDSZER TODO

        %   \item Adja meg az alábbi Euler-féle differenciálegyenlet megoldását!
        %         $$
        %           y'' - \frac{3}{x} \, y' + 20 \frac{y}{x^2} = 0
        %           \quad
        %           y(1) = 2
        %           \quad
        %           y'(1) = 0
        %         $$

        %   \item Az $(1 + x^2) y'' + xy' - y = 0$ differenciálegyenleteit
        %         $y_1(x) = x$ megoldás ismeri. Adja meg az általános megoldást!

        %   \item Az $xy'' - (1 + x) y' + y = 0$ differenciálegyenleteit
        %         $y_1(x) = e^x$ megoldás ismeri. Adja meg az általános megoldást!

        %   \item Oldja meg az alábbi differenciálegyenlet-rendszert!
        %         $$
        %           \left\{
        %           \begin{array}{l}
        %             \dot x = 2x + y \\
        %             \dot y = 3x + 4y
        %           \end{array}
        %           \right.
        %           \qquad
        %           \begin{array}{l}
        %             x(0) = 0 \\
        %             y(0) = 1
        %           \end{array}
        %         $$

        %   \item Adja meg az alábbi differenciálegyenlet-rendszer általános megoldását!
        %         $$
        %           \left\{
        %           \begin{array}{l}
        %             \dot x = 2x + y \\
        %             \dot y = y
        %             \dot z = 3z
        %           \end{array}
        %           \right.
        %         $$

        %   \item Adja meg az alábbi differenciálegyenlet-rendszer általános megoldását!
        %         $$
        %           \left\{
        %           \begin{array}{l}
        %             \dot x = -x - 5y \\
        %             \dot y = x + y
        %           \end{array}
        %           \right.
        %         $$

        %         % OLD 13
        %   \item Adja meg az alábbi differenciálegyenlet-rendszer általános megoldását!
        %         $$
        %           \dot{\rvec x} = \begin{bmatrix}
        %             2 & 2 & 1 \\
        %             1 & 3 & 1 \\
        %             1 & 2 & 2 \\
        %           \end{bmatrix} \rvec x
        %         $$

        %   \item Adja meg az alábbi differenciálegyenlet-rendszer általános megoldását!
        %         $$
        %           \dot{\rvec x} = \begin{bmatrix}
        %             0 & 1 & 1 \\
        %             0 & 0 & 1 \\
        %             0 & 0 & 0 \\
        %           \end{bmatrix} \rvec x
        %         $$

        %   \item Oldja meg az $y''' + 5y'' + 6y' = 0$ differenciálegyenletet
        %         differenciálegyenlet-rendszer segítségével!

        %   \item Osztályozza stabilitás szempontjából az alábbi differenciálegyenleteket!
        %         \begin{multicols}{2}
        %           \begin{enumerate}
        %             \item $
        %                     \ddot x + 6 \dot x + 5x = 0
        %                   $

        %             \item $
        %                     \ddot x + \dot x + 5x = 0
        %                   $

        %             \item $
        %                     \ddot x + 9x = 0
        %                   $

        %             \item $
        %                     \dddot x + 8\ddot x + 22\dot x + 20x = 0
        %                   $
        %           \end{enumerate}
        %         \end{multicols}

        %   \item Vizsgálja az $\ddot x + (1 - e^{2 - \mu}) \dot x + x = 0$
        %         differenciálegyenlet stabilitását $\mu$ függvényében!

        %   \item Rajzolja fel az alábbi differenciálegyenletek trajektóriáit!
        %         \begin{multicols}{3}
        %           \begin{enumerate}
        %             \item $
        %                     \left\{
        %                     \begin{array}{l}
        %                       \dot x = 2x \\
        %                       \dot y = y
        %                     \end{array}
        %                     \right.
        %                   $

        %             \item $
        %                     \left\{
        %                     \begin{array}{l}
        %                       \dot x = x - y \\
        %                       \dot y = x + y
        %                     \end{array}
        %                     \right.
        %                   $

        %             \item $
        %                     \left\{
        %                     \begin{array}{l}
        %                       \dot x = 2x - 3y \\
        %                       \dot y = -4y
        %                     \end{array}
        %                     \right.
        %                   $
        %           \end{enumerate}
        %         \end{multicols}
\end{enumerate}
\end{document}