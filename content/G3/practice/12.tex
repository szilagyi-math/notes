\documentclass{szb-practice}

\title{Lineáris, állandó együtthatós DE}
\area{Differenciálegyenletek}
\subject{Matematika G3}
\subjectCode{BMETE94BG03}
\date{Utoljára frissítve: \today}
\docno{12}

\begin{document}
\maketitle

\subsection{Elméleti áttekintő}

\begin{blueBox}[\underline{Lineáris differenciálegyenletek}]
  $$
    y^{(n)}
    + a_{n-1}(x) y^{(n-1)}
    + \dots
    + a_1(x) y'
    + a_0(x) y
    = f(x)
  $$
  Ha $f(x) = 0$, akkor homogén, egyébként inhomogén.

  A homogén általános megoldás alakja:
  $$
    y(x)
    = c_1 y_1(x)
    + c_2 y_2(x)
    + \dots
    + c_n y_n(x)
    \text.
  $$
  Ez az $n$ darab függvény az integrálásra és
  deriválásra vektorteret alkot. Lineáris függetlenségüket
  A Wronsky-determimáns segítségével ellenőrizhetjük:
  $$
    \rmat{W} = \begin{bmatrix}
      y_1         & y_2         & \dots  & y_n         \\
      y_1'        & y_2'        & \dots  & y_n'        \\
      \vdots      & \vdots      & \ddots & \vdots      \\
      y_1^{(n-1)} & y_2^{(n-1)} & \dots  & y_n^{(n-1)} \\
    \end{bmatrix}
  $$
  Ha $\det \rmat{W} \neq 0$, akkor lineárisan függetlenek,
  $\left\{ y_1; y_2; \dots; y_n \right\}$ adják a diffegyenlet
  alaprendszerét. Az alaprendszer valamennyi függvénye, és
  lineáris kombinációjuk is megoldás lesz.
\end{blueBox}

\begin{blueBox}[\underline{Állandó együtthatós, homogén, lineáris differenciálegyenletek}]
  $$
    y^{(n)}
    + a_{n-1} y^{(n-1)}
    + \dots
    + a_1 y'
    + a_0 y
    = 0
  $$
  \emph{\underline{Megoldási módszer:}} \quad Próbafüggvény-módszer

  Legyen $y = e^{\lambda x}$, $y' = \lambda e^{\lambda x}$,
  $\dots$, $y^{(n)} = \lambda^n e^{\lambda x}$. Ekkor az
  alábbi egyenletet kapjuk:
  \[
    \lambda^n
    + a_{n-1} \lambda^{n-1}
    + \dots
    + a_{1} \lambda^{1}
    + a_{0}
    = 0
  \]
  Ez $\lambda$-ra nézve $n$-edfokú polinom, melynek
  $n$ darab megoldása van, melyek között páros számú
  komplex gyök is lehet.
  \begin{itemize}
    \item $\lambda_1 \in \mathbb{R}$, egyszeres

          $y_1(x) = e^{\lambda_1 x}$

    \item $\lambda_1 = \lambda_2 = \dots = \lambda_s \in \mathbb{R}$,
          többszörös, akkor belső rezonancia áll fenn

          $y_1 = e^{\lambda_1 x}$,
          $y_2 = x e^{\lambda_1 x}$,
          $\dots$,
          $y_s = x^{s-1} e^{\lambda_1 x}$

    \item $\lambda_{12} = \alpha \pm i\beta$, egyszeres komplex gyökpár

          $y_1 = e^{\alpha x} \cos (\beta x)$

          $y_2 = e^{\alpha x} \sin (\beta x)$

    \item $\lambda_{2s} \in \mathbb{C}$, $s$-szeres multiplicitású

          $y_1 = e^{\alpha x} \cos (\beta x)$
          \tabto{4cm}
          $y_3 = x y_1$
          \tabto{7cm}
          $y_5 = x^2 y_1$
          \tabto{10cm}
          $\dots$
          \tabto{11.5cm}
          $y_{2s-1} = x^{s-1} y_1$

          $y_2 = e^{\alpha x} \sin (\beta x)$
          \tabto{4cm}
          $y_4 = x y_2$
          \tabto{7cm}
          $y_6 = x^2 y_2$
          \tabto{10cm}
          $\dots$
          \tabto{11.5cm}
          $y_{2s\phantom{-1}} = x^{s-1} y_2$
  \end{itemize}
\end{blueBox}

\begin{blueBox}[\underline{Állandó együtthatós, inhomogén, lineáris differenciálegyenletek}]
  $$
    y^{(n)}
    + a_{n-1} y^{(n-1)}
    + \dots
    + a_1 y'
    + a_0 y
    = f(x)
  $$

  \emph{\underline{Megoldási módszer:}} \quad
  Megoldást a gerjesztésnek megfelelő
  formában kell keresni.
  \begin{itemize}
    \item $f(x) = e^x$
          \tabto{4.6cm} -- \tabto{5.2cm}
          $y_\mathrm{ih} = A e^x$

    \item $f(x) = p(x)$, polinom
          \tabto{4.6cm} -- \tabto{5.2cm}
          $y_\mathrm{ih} = A_0 + A_1 x + A_2 x^2 + \dots$

    \item $f(x)$ trigonometrikus
          \tabto{4.6cm} -- \tabto{5.2cm}
          $y_\mathrm{ih} = A \cos (\beta x) + B \sin (\beta x)$
  \end{itemize}
  \[
    y = y_\mathrm{h} + y_\mathrm{ih}
  \]
\end{blueBox}



\clearpage
\subsection{Feladatok}

\begin{enumerate}
  \item Határozza meg, hogy milyen halmazon lineárisan függetlenek az alábbi
        függvények!
        $$
          H_1 = \Big\{\;
          \;e^x;
          \;xe^x;
          \;x^2e^x
          \;\Big\}
          \hspace{2cm}
          H_2 = \Big\{\;
          \;1;
          \;\cos x;
          \;\sin x;
          \;\Big\}
        $$

  \item Hol lesz a $\{ \; e^x; \; -(x + 1) \; \}$ függvényrendszer az
        $xy'' - (x + 1)y' + y = 0$ differenciálegyenlet alaphalmaza?

  \item Hol lesz a $\{ \; 1; \; x; \; \ln x \; \}$ függvényrendszer az
        $xy''' + 2y'' = 0$ differenciálegyenlet alaphalmaza?

  \item Adja meg az $y'' - 5y' + 6y = 0$ differenciálegyenlet $y(0) = 1$ és
        $y'(0) = 1$ kezdeti feltételek melletti megoldását!

  \item Adja meg a $8y''' + 12y'' + 6y' + y = 0$ differenciálegyenlet
        általános megoldását!

  \item Adja meg a $y''' - 2y'' + 2y' = 0$ differenciálegyenlet
        általános megoldását!

  \item Adja meg a $y^{(\mathbf{V})} - 2y^{(\mathbf{IV})} + 8y''' - 16y'' + 16y'
          -32y = 0$ differenciálegyenlet általános megoldását!

  \item Adja meg azt a legkisebb rendű differenciálegyenletet, amelynek a
        $\{\; 6x^2; \; 5e^{2x} \;\}$ függvényrendszer az alaphalmaza!

  \item Adja meg azt a legkisebb rendű differenciálegyenletet, amelynek a
        $\{\; 7x; \; \sin 5x \;\}$ függvényrendszer az alaphalmaza!

  \item Adja meg azt a legkisebb rendű differenciálegyenletet, amelynek a
        $\{\; x e^x; \; e^{2x} \cos x \;\}$ függvényrendszer az alaphalmaza!

  \item Adja meg az $y'' - 5y' + 6y = 2 \sin 2x$ differenciálegyenlet
        általános megoldását!

  \item Adja meg az $y'' - 5y' + 6y = 2x e^x + e^{2x}$ differenciálegyenlet
        általános megoldását!

  \item Adja meg az $y'' + 4y = x^2 \sin 2x$ differenciálegyenlet
        általános megoldását!
\end{enumerate}

\end{document}