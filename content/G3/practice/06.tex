\documentclass{szb-practice}

\usepackage{siunitx}
\sisetup{locale = DE} 

\title{Integrálátalakító tételek}
\area{Vektoranalízis}
\subject{Matematika G3}
\subjectCode{BMETE94BG03}
\date{Utoljára frissítve: \today}
\docno{6}

\usepackage{lipsum}

\begin{document}
\maketitle

\subsection{Elméleti áttekintő}

% ~~~~~~~~~~~~~~~~~~~~~~~~~~~~~~~~~~~~~~~~~~~~~~~~~~~~~~~~~~~~~~~~~~~~~~~~~~~~~~
% ~~~~~~~~~~~~~~~~~~~~ Gradient-theorem ~~~~~~~~~~~~~~~~~~~~~~~~~~~~~~~~~~~~~~~~
% ~~~~~~~~~~~~~~~~~~~~~~~~~~~~~~~~~~~~~~~~~~~~~~~~~~~~~~~~~~~~~~~~~~~~~~~~~~~~~~
\begin{theorem}[Gradiens-tétel]
  Legyen $\varphi : U \subseteq \Reals^3 \rightarrow \Reals$ differenciálható
  skalármező, $\curvesign : [a;b] \rightarrow \curveimage \subseteq U$,
  $t \mapsto \curvevec(t)$ folytonos görbe, $\curvevec(a) = \rvec p$,
  $\curvevec(b) = \rvec q$ pedig a görbe kezdő és végpontja. Ekkor:
  $$
    \int_{\curveimage} \scalar{\grad \varphi(\coordv)}{\dd \curvevec}
    =
    \varphi(\rvec q) - \varphi(\rvec p)
    \text.
  $$
  Vagyis, ha egy vektormező valamely skalármező gradiense, akkor annak bármely
  folytonos görbe mentén vett integrálja csak a kezdő- és végpontoktól függ.
\end{theorem}

\begin{blueBox}
  \sftitle{Körintegrál jelölése:}

  Ha $\curvesign$ zárt görbe, akkor a $\varphi(\coordv)$ skalármező egy
  $\curvesign$ görbe mentén vett körintegrálja a következőképpen jelölhető:
  $$
    \oint_{\curveimage} \, \varphi(\coordv) \dd \curvescalar
    \text.
  $$
\end{blueBox}

\begin{note}
  A Gradiens-tételből következik, hogy skalárpotenciálos vektormező zárt görbe
  mentén vett körintegrálja zérus.
\end{note}

\begin{example}
  Integrálja a $\rvec v(\coordv) = \ijk{y + z}{x + z}{x + y}$ vektormezőt
  a $z = 0$ síkon lévő, origó középpontú, $r = 3$ sugárú kör mentén!

  Vizsgáljuk meg, hogy a vektormező skalárpotenciálos-e:
  $$
    \rot \rvec v
    =
    \begin{bmatrix}
      \partial_x \\ \partial_y \\ \partial_z
    \end{bmatrix}
    \times
    \begin{bmatrix}
      y + z \\ x + z \\ x + y
    \end{bmatrix}
    =
    \begin{bmatrix}
      \partial_x (x + z) - \partial_y (y + z) \\
      \partial_y (x + y) - \partial_z (x + z) \\
      \partial_z (y + z) - \partial_x (x + y)
    \end{bmatrix}
    =
    \begin{bmatrix}
      1 - 1 \\ 1 - 1 \\ 1 - 1
    \end{bmatrix}
    =
    \nvec
    \text.
  $$
  Mivel a vektormező skalárpotenciálos, ezért létezik olyan skalármező,
  melynek gradiense maga a $\rvec v$ vektormező. Az integrál értéke tehát
  csak a kezdő- és végpontoktól függ, melyek jelen esetben megegyeznek,
  vagyis az integrál értéke zérus:
  $$
    \oint_{\curveimage} \scalar{\rvec v}{\dd \curvevec} = 0
    \text.
  $$
\end{example}

% ~~~~~~~~~~~~~~~~~~~~~~~~~~~~~~~~~~~~~~~~~~~~~~~~~~~~~~~~~~~~~~~~~~~~~~~~~~~~~~
% ~~~~~~~~~~~~~~~~~~~~ Stokes' theorem ~~~~~~~~~~~~~~~~~~~~~~~~~~~~~~~~~~~~~~~~~
% ~~~~~~~~~~~~~~~~~~~~~~~~~~~~~~~~~~~~~~~~~~~~~~~~~~~~~~~~~~~~~~~~~~~~~~~~~~~~~~
\begin{theorem}[Stokes-tétel]
  Legyen $\surfsign: \surfdomain\subset\Reals^2 \to \surfimage \subset \Reals^3$
  irányított, parametrizált, elemi felület. Legyen továbbá $\rvec v : \Reals^3
    \to \Reals^3$ legalább egyszer folytonosan differenciálható vektormező.
  Jelölje az $\curvesign: \curvedomain\subset\Reals \to \partial\surfimage =
    \curveimage$ a $\surfsign$ peremét indukált, jobbézszabály szerinti
  irányítással. Ekkor:
  $$
    \iint_{\surfimage} \scalar{\rot \rvec v}{\dd \surfvec}
    =
    \oint_{\partial \surfimage} \scalar{\rvec v}{\dd \curvevec}
    \text.
  $$
\end{theorem}

\begin{note}
  Ha $\rvec v$ skalárpotenciálos, akkor az integrál értéke zérus, hiszen
  $\rot \rvec v = \rot \grad \varphi = \rvec 0$.
\end{note}

\begin{example}
  Integrálja a $\rvec v(\coordv) = \ijk{y}{x}{0}$ vektormezőt a $P_1(0;1;0)$,
  $P_2(2;0;0)$ és $P_3(0;0;0)$ által meghatározott háromszög mentén!

  Határozzuk meg a $\rvec v$ vektormező rotációját:
  $$
    \rot \rvec v
    =
    \begin{bmatrix}
      \partial_x \\ \partial_y \\ \partial_z
    \end{bmatrix}
    \times
    \begin{bmatrix}
      y \\ x \\ 0
    \end{bmatrix}
    =
    \begin{bmatrix}
      0 \\ 0 \\ 1 - 1
    \end{bmatrix}
    =
    \nvec
    \text.
  $$
  A Stokes-tétel alapján:
  $$
    \oint_{\partial \surfimage} \scalar{\rvec v}{\dd \curvevec}
    = \int_{\surfimage} \scalar{\rot \rvec v}{\dd \surfvec}
    = \int_{\surfimage} \scalar{\nvec}{\dd \surfvec}
    = 0
    \text.
  $$
\end{example}

\begin{learnMore}[Stokes-tétel Maxwell III. és IV. egyenletében]
  A Stokes-tétel a Maxwell-egyenletekben is fontos szerepet játszik. A harmadik
  és negyedik egyenlet a mágneses tér és az elektromos tér közötti
  kapcsolatot írja le:
  $$
    \begin{aligned}
      (III) & \quad \Rightarrow \quad \rot \rvec E = -\dot{\rvec B}
            &
            & \quad \Rightarrow \quad \text{elektromos tér -- mágneses tér változása,}
      \\
      (IV)  & \quad \Rightarrow \quad \rot \rvec B = \mu_0 \rvec j + \mu_0
      \varepsilon_0 \dot{\rvec E}
            &
            & \quad \Rightarrow \quad \text{mágneses tér -- elektromos tér változása,}
    \end{aligned}
  $$
  ahol $\rvec E$ az elektromos tér, $\rvec B$ a mágneses tér, $\rvec j$ az áram
  sűrűség, $\mu_0$ a mágneses permeabilitás és $\varepsilon_0$ az elektromos
  permittivitás.

  Az egyenletek közötti kapcsolatot a Stokes-tétel segítségével:
  $$
    \begin{aligned}
      (III) & \quad \Rightarrow \quad
      \oint_{\partial \surfimage} \scalar{\rvec E}{\dd \curvevec}
      = -\iint_{\surfimage} \scalar{\dot{\rvec B}}{\dd \surfvec}
      \text,
      \\
      (IV)  & \quad \Rightarrow \quad
      \oint_{\partial \surfimage} \scalar{\rvec B}{\dd \curvevec}
      = \iint_{\surfimage} \scalar{\mu_0 \rvec j + \mu_0 \varepsilon_0 \dot{\rvec E}}{\dd \surfvec}
      \text.
    \end{aligned}
  $$

  A III. egyenlet azt mondja ki, hogy változó mágneses tér maga körül
  balkézszabály szerint elektormos teret indukál, míg a IV. egyenlet azt
  jelenti, hogy az elektromos tér változása jobbkézszabály szerint
  mágneses teret indukál.
\end{learnMore}

% ~~~~~~~~~~~~~~~~~~~~~~~~~~~~~~~~~~~~~~~~~~~~~~~~~~~~~~~~~~~~~~~~~~~~~~~~~~~~~~
% ~~~~~~~~~~~~~~~~~~~~ Divergence theorem ~~~~~~~~~~~~~~~~~~~~~~~~~~~~~~~~~~~~~~
% ~~~~~~~~~~~~~~~~~~~~~~~~~~~~~~~~~~~~~~~~~~~~~~~~~~~~~~~~~~~~~~~~~~~~~~~~~~~~~~
\begin{theorem}[Gauss-Osztogradszkij-tétel]
  Legyen $\volsign: \voldomain \subset \Reals^3 \rightarrow \volimage \subset
    \Reals^3$, irányított,parametrizált tértartomány. Legyen továbbá $\rvec v:
    \Reals^3 \rightarrow \Reals^3$ legalább egyszer folytonosan
  differenciálható vektormező. Jelölje a $\partial \volimage =
    \surfimage$ az $\volsign$ peremét indukált irányítással. Ekkor:
  $$
    \iiint_{\volimage} \Div \rvec v \dd \volscalar
    =
    \oiint_{\partial \volimage} \scalar{\rvec v}{\dd \surfvec}
    \text.
  $$
\end{theorem}

\begin{note}
  Ha $\rvec v$ vektorpotenciálos, akkor az integrál értéke zérus, hiszen
  $\Div \rvec v = \Div \rot \rvec u = 0$.
\end{note}

\begin{example}
  Integrálja a $\rvec v(\coordv) = \ijk{x^2 y z}{x y^2 z}{2 x y z^2}$
  vektormezőt az első térnyolcadban lévő egységkocka felületén kifele mutató
  irányítással!

  Határozzuk meg a $\rvec v$ vektormező divergenciáját:
  $$
    \Div \rvec v
    = \pdv{x^2 y z}{x} + \pdv{x y^2 z}{y} + \pdv{2 x y z^2}{z}
    = 8 x y z
    \text.
  $$

  A Gauss-Osztogradszkij-tétel alapján:
  $$
    \oiint_{\partial \volimage} \scalar{\rvec v}{\dd \surfvec}
    = \iiint_{\volimage} \Div \rvec v \dd \volscalar
    = \int_0^1 \int_0^1 \int_0^1 8 x y z \dd z \dd y \dd x
    = 1
    \text.
  $$
\end{example}

\begin{learnMore}[Gauss-Osztogradszkij-tétel Maxwell I. és II. egyenletében]
  A Gauss-Osztogradszkij-tétel a Maxwell-egyenletekben is fontos szerepet
  játszik. Az első két egyenlet az elektromos és mágneses tér forrásosságát
  írja le:
  $$
    \begin{aligned}
      (I)  & \quad \Rightarrow \quad \Div \rvec E = \frac{\rho}{\varepsilon_0}
           &
           & \quad \Rightarrow \quad \text{elektromos tér forrásos,}
      \\
      (II) & \quad \Rightarrow \quad \Div \rvec B = 0
           &
           & \quad \Rightarrow \quad \text{mágneses tér forrásmentes,}
    \end{aligned}
  $$
  ahol $\rvec E$ az elektromos tér, $\rvec B$ a mágneses tér,
  $\rho$ az elektromos töltéssűrűség, $\varepsilon_0$ az elektromos
  permittivitás.

  Az egyenletek közötti kapcsolatot a Gauss-Osztogradszkij-tétel segítségével:
  $$
    \begin{aligned}
      (I)  & \quad \Rightarrow \quad
      \oiint_{\partial \volimage} \scalar{\rvec E}{\dd \surfvec}
      = \iiint_{\volimage} \Div \rvec E \dd \volscalar
      = \iiint_{\volimage} \frac{\rho}{\varepsilon_0} \dd \volscalar
      \text,
      \\
      (II) & \quad \Rightarrow \quad
      \oiint_{\partial \volimage} \scalar{\rvec B}{\dd \surfvec}
      = \iiint_{\volimage} \Div \rvec B \dd \volscalar
      = 0
      \text.
    \end{aligned}
  $$
  Az első egyenlet azt mondja ki, hogy zárt felületen áthaladő elektromos
  tér fluxusa arányos az elektromos töltéssűrűség térfogati integráljával.
  A második egyenlet pedig azt jelenti, hogy a mágneses tér fluxusa zárt
  felületen zérus, a mágneses tér forrásmentes.
\end{learnMore}

% ~~~~~~~~~~~~~~~~~~~~~~~~~~~~~~~~~~~~~~~~~~~~~~~~~~~~~~~~~~~~~~~~~~~~~~~~~~~~~~
% ~~~~~~~~~~~~~~~~~~~~ Green's theorem ~~~~~~~~~~~~~~~~~~~~~~~~~~~~~~~~~~~~~~~~~
% ~~~~~~~~~~~~~~~~~~~~~~~~~~~~~~~~~~~~~~~~~~~~~~~~~~~~~~~~~~~~~~~~~~~~~~~~~~~~~~
\begin{theorem}[Green-tétel asszimetrikus alakja]
  Legyenek $\varphi; \psi: \mathbb R^3 \rightarrow \mathbb R$ kétszeresen
  folytonos skalármezők, $\volimage \subset \mathbb R^3$ parametrizált,
  irányított tértartomány, $\partial \volimage = \surfimage$ perem indukált
  irányítással. Ekkor:
  $$
    \iiint_{\volimage}
    \psi \, \Delta \varphi +
    \scalar{\grad \psi}{\grad \varphi}
    \dd \volscalar
    =
    \oiint_{\partial \volimage} \scalar{\psi \grad \varphi}{\dd \surfvec}
    \text.
  $$
\end{theorem}

\begin{note}
  $\psi = 1$ választásával visszanyerjük a Gauss-Osztogradszkij-tételt:
  $$
    \iiint_{\volimage} \Delta \varphi \dd \volscalar
    = \iiint_{\volimage} \Div \underbrace{\grad \varphi}_{\rvec v} \dd \volscalar
    = \oiint_{\partial \volimage} \scalar{\underbrace{\grad \varphi}_{\rvec v}}{\dd \surfvec}
    \text.
  $$
\end{note}

\begin{example}
  Tekintsünk egy $R = 1\,\text{m}$ sugarú tömör alumínium gömböt, amelynek
  stacionárius hőmér\-sék\-let-eloszlása $\varphi(\coordv) = T_0 (1 - \rvec r^2)$
  függvény írja le, ahol $T_0 = \SI{10}{\kelvin}$ a gömb belső hőmérséklete.
  Határozza meg a gömb felületén kifelé irányuló összes  $\dot Q$ hőáramot,
  ha a hőfluxus sűrűsége $\rvec q = -\lambda \grad \varphi$, ahol
  $\lambda = \SI[per-mode=symbol]{205}{\watt\per\meter\per\kelvin}$ az
  alumínium hővezetési tényezője, és
  $$
    \dot Q = \oiint_{\partial \volimage} \scalar{\rvec q}{\dd \surfvec}
    \text.
  $$
  Használjuk a Green-tétel asszimetrikus alakját $\psi = -\lambda$ állandó
  választással:
  $$
    \dot Q
    = \oiint_{\partial \volimage} \scalar{-\lambda \grad \varphi}{\dd \surfvec}
    = - \iiint_{\volimage} \psi \, \Delta \varphi + \scalar{\underbrace{\grad \lambda}_{= \nvec}}{\grad \varphi} \dd \volscalar
    = -\lambda \iiint_{\volimage} \Delta \varphi \dd \volscalar
    \text.
  $$
  Számítsuk ki a térfogat belsejében a $\Delta \varphi$ értékét:
  $$
    \Delta \varphi
    =\pdv[order=2]{\varphi}{x} + \pdv[order=2]{\varphi}{y} + \pdv[order=2]{\varphi}{z}
    = -6 T_0
    \text.
  $$
  A hőáram összesen:
  $$
    \dot Q = \iiint_{\volimage} \underbrace{-\lambda (-6 T_0)}_{=\text{const}} \dd \volscalar
    = -\lambda (-6 T_0) \frac{4\pi R^3}{3}
    = 8 \pi \lambda T_0 R^3
    \approx \SI{5,15e4}{\watt}
    \text.
  $$
\end{example}

\begin{theorem}[Green-tétel szimmetrikus alakja]
  Legyenek $\varphi; \psi: \mathbb R^3 \rightarrow \mathbb R$ kétszeresen
  folytonos skalármezők, $\volimage \subset \mathbb R^3$ parametrizált,
  irányított tértartomány, $\partial \volimage = \surfimage$ perem indukált
  irányítással. Ekkor:
  $$
    \iiint_{\volimage}
    \psi \, \Delta \varphi + \varphi \, \Delta \psi
    \; \dd \volscalar
    =
    \oiint_{\partial \volimage}
    \scalar{\psi \grad \varphi - \varphi \grad \psi}{\dd \surfvec}
  $$
\end{theorem}

\clearpage
\subsection{Feladatok}

\begin{enumerate}
  \item Egy automata raktárrendszer ferromágneses manipulátora egy előre
        számított mágneses potenciálmezőben mozog. A számított Joule-potenciál a
        munkatérben
        $$
          \varphi(\coordv) = 2x^2 y + 3yz
          \text.
        $$
        A csípőkaron lévő, $Q = \SI{1}{\coulomb}$ ekvivalens töltéssel
        modellezett végfogót a vezérlőnek az
        $$
          A(0;0;0) \rightarrow B(2;1;3)
        $$
        pontok között kell mozgatnia. Mennyi munkát végez a mágneses tér a
        végfogón, és miért nem kell törődnünk az útvonal pontos alakjával?
        ($\rvec F = -Q \grad \varphi$, a munka az erő pálya menti integrálja)

  \item Egy mágneses rendszer vezérlője egy
        $$
          \rvec B(\coordv) = \ijk{y \, \sin x}{z^2 \, \cos y - \cos x}{b_3}
        $$
        vektormezővel modellezett mágneses teret hoz létre. Határozza meg
        $b_3$ értékét, ha
        \begin{enumerate}
          \item $\rvec B$ bármely zárt görbe menti cirkulációja zérus,
          \item $\rvec B$ bármely zárt felület menti fluxusa zérus.
        \end{enumerate}

  \item Egy $R = 1$ sugarú, kör keresztmetszetű, $z$ tengellyel egybeeső
        szimmetriavonalú hengerben áramló folyadék sebességét a
        $$
          \rvec v(\coordv) = \ijk{2xy + z}{x^2 + z}{y - x}
        $$
        vektormező írja le. Adja meg a $z = 1$ síkban lévő keresztmetszet
        menti cirkulációt!
        (A cirkuláció a vektormező zárt görbe menti integrálja.)

  \item Jelölje $\surfimage$ az $x^2 + y^2 - z^2 = 1$ egyenlezű
        forgáshiperboloid $z = -1$ és $z = 1$ síkok közötti részét.
        Határozza meg a $\rvec v(\coordvec) = \ijk{x^2}{y^3}{z^4}$
        vektormező $\surfimage$ peremén vett integrálját!

  \item Egy fotonikus chipeket hordozó wafer-darabot egy ellipszoid alakú
        Faraday-kalitkába rögzítenek. A kalitka belsejében lineáris
        feszültségelosztással ($\rvec E(\coordv) = \coordv$) térerőt állítanak
        elő. Számolja ki a Faraday-kalitka belsejében lévő nettó töltést, ha
        $\varepsilon_0 = \SI[per-mode=symbol]{8,85e-12}{\farad\per\meter}$,
        az ellipszoid egyenlete pedig:
        $$
          \frac{(x - 2)^2}{5} + \frac{(y + 3)^2}{19} + \frac{(z - 1)^2}{4} = 1
          \text.
        $$

  \item Egy drón IMU-modulját teljes egészében kitöltő, hővezető műgyanta gömb
        alakú, sugara $R = 0,02 \, \text{m}$, A vezérlő egység folyamatosan
        hőt disszipál, az állandósult hő\-mér\-sék\-let-mező jó közelítéssel
        $$
          \varphi(\coordv) = T_c - \alpha \rvec r^2
          \text, \quad
          \alpha = \SI[per-mode=symbol]{1,3e5}{\kelvin\per\meter\squared}
          \text.
        $$
        Becsülje meg, mekkora teljes hőáram távozik a burkolaton át, ha
        $$
          q_{\text{hő}}
          = -\lambda \oiint_{\partial \volimage} \scalar{\grad \varphi}{\dd \surfvec}
          \text,\quad
          \lambda = \SI[per-mode=symbol]{0,2}{\watt\per\meter\per\kelvin}
          \text.
        $$
\end{enumerate}

% \vfill
% \subsection{Megoldások}

% \begin{enumerate}
%   \item $W = \SI{-17}{J}$
%   \item \begin{enumerate}
%           \item $b_3 = 2 z \sin y + C$
%           \item $b_3 = \dfrac{z^3 \sin y}{3} - y z \cos x + C$
%         \end{enumerate}
%   \item $\text{cirkuláció} = 0$
%   \item $0$
%   \item $Q = 8\pi\varepsilon_0\sqrt{95} \approx \SI{2,17}{nC}$
%   \item $q_\text{hő} = 8\pi\lambda\alpha R^3 = \SI{5,23}{W}$
% \end{enumerate}

\end{document}