\documentclass[a4paper, 12pt]{scrartcl}

\usepackage{math-practice}

\newcommand\coordv{\rvec r}
\newcommand\arcv{\rvec r}
\newcommand\surfv{\rvec \varrho}

\title{Integrálátalakító tételek}
\area{Többváltozós analízis}
\subject{Matematika G3}
\subjectCode{BMETE94BG03}
\date{Utoljára frissítve: \today}
\docno{6}

\begin{document}
\allowdisplaybreaks

\maketitle

\subsection{Elméleti áttekintő}

% ~~~~~~~~~~~~~~~~~~~~~~~~~~~~~~~~~~~~~~~~~~~~~~~~~~~~~~~~~~~~~~~~~~~~~~~~~~~~~~
% ~~~~~~~~~~~~~~~~~~~~ Gradient-theorem ~~~~~~~~~~~~~~~~~~~~~~~~~~~~~~~~~~~~~~~~
% ~~~~~~~~~~~~~~~~~~~~~~~~~~~~~~~~~~~~~~~~~~~~~~~~~~~~~~~~~~~~~~~~~~~~~~~~~~~~~~
\begin{theorem}[Gradiens-tétel]
  Legyen $\varphi : U \subseteq \mathbb R^3 \rightarrow \mathbb R$
  differenciálható skalármező, $\arcv : [a;b] \rightarrow \gamma \subseteq U$,
  $t \mapsto \arcv(t)$ folytonos görbe, $\arcv(a) = \rvec p$, $\arcv(b) =
    \rvec q$ pedig a görbe kezdő és végpontja. Ekkor:
  $$
    \int_\gamma \scalar{\grad \varphi(\coordv)}{\dd \rvec r}
    =
    \varphi(\rvec q) - \varphi(\rvec p)
    \text.
  $$
  Vagyis, ha egy vektormező valamely skalármező gradiense, akkor annak bármely
  folytonos görbe mentén vett integrálja csak a kezdő- és végpontoktól függ.
\end{theorem}

\begin{blueBox}
  \sftitle{Körintegrál jelölése:}

  Ha $\gamma$ zárt görbe, akkor a $\varphi(\rvec r)$ skalármező $\gamma$ görbe
  mentén vett körintegrálja a következőképpen jelölhető:
  $$
    \oint_\gamma \varphi(\rvec r) \dd s
    \text.
  $$
\end{blueBox}

\begin{note}
  A Gradiens-tételből következik, hogy skalárpotenciálos vektormező zárt görbe
  mentén vett körintegrálja zérus.
\end{note}

\begin{example}
  Integrálja a $\rvec v(\rvec r) = \ijk{y + z}{x + z}{x + y}$ vektormezőt
  a $z = 0$ síkon lévő, origó középpontú, $r = 3$ sugárú kör mentén!

  Vizsgáljuk meg, hogy a vektormező skalárpotenciálos-e:
  $$
    \rot \rvec v
    =
    \begin{bmatrix}
      \partial_x \\ \partial_y \\ \partial_z
    \end{bmatrix}
    \times
    \begin{bmatrix}
      y + z \\ x + z \\ x + y
    \end{bmatrix}
    =
    \begin{bmatrix}
      \partial_x (x + z) - \partial_y (y + z) \\
      \partial_y (x + y) - \partial_z (x + z) \\
      \partial_z (y + z) - \partial_x (x + y)
    \end{bmatrix}
    =
    \begin{bmatrix}
      1 - 1 \\ 1 - 1 \\ 1 - 1
    \end{bmatrix}
    =
    \nvec
    \text.
  $$
  Mivel a vektormező skalárpotenciálos, ezért létezik olyan skalármező,
  melynek gradiense maga a $\rvec v$ vektormező. Az integrál értéke tehát
  csak a kezdő- és végpontoktól függ, melyek jelen esetben megegyeznek,
  vagyis az integrál értéke zérus:
  $$
    \oint_\gamma \scalar{\rvec v}{\dd \arcv} = 0
    \text.
  $$
\end{example}

% ~~~~~~~~~~~~~~~~~~~~~~~~~~~~~~~~~~~~~~~~~~~~~~~~~~~~~~~~~~~~~~~~~~~~~~~~~~~~~~
% ~~~~~~~~~~~~~~~~~~~~ Stokes' theorem ~~~~~~~~~~~~~~~~~~~~~~~~~~~~~~~~~~~~~~~~~
% ~~~~~~~~~~~~~~~~~~~~~~~~~~~~~~~~~~~~~~~~~~~~~~~~~~~~~~~~~~~~~~~~~~~~~~~~~~~~~~
\begin{theorem}[Stokes-tétel]
  Legyen $S$ egy $\mathbb R^3$-beli irányított, parametrizált felület.
  Legyen továbbá $\rvec v : \mathbb R^3 \rightarrow \mathbb R^3$ legalább
  egyszer folytonosan differenciálható vektormező. Jelölje a $\partial S =
    \gamma$ az $S$ peremét indukált irányítással. Ekkor:
  $$
    \int_S \scalar{\rot \rvec v}{\dd \rvec S}
    =
    \oint_{\partial S} \scalar{\rvec v}{\dd \rvec r}
    \text.
  $$
\end{theorem}

\begin{note}
  Ha $\rvec v$ skalárpotenciálos, akkor az integrál értéke zérus, hiszen
  $\rot \rvec v = \rot \grad \varphi = \rvec 0$.
\end{note}

\begin{example}
  Integrálja a $\rvec v(\rvec r) = \ijk{y}{x}{0}$ vektormezőt a $P_1(0;1;0)$,
  $P_2(2;0;0)$ és $P_3(0;0;0)$ által meghatározott háromszög mentén!

  Határozzuk meg a $\rvec v$ vektormező rotációját:
  $$
    \rot \rvec v
    =
    \begin{bmatrix}
      \partial_x \\ \partial_y \\ \partial_z
    \end{bmatrix}
    \times
    \begin{bmatrix}
      y \\ x \\ 0
    \end{bmatrix}
    =
    \begin{bmatrix}
      0 \\ 0 \\ 1 - 1
    \end{bmatrix}
    =
    \nvec
    \text.
  $$
  A Stokes-tétel alapján:
  $$
    \oint_{\partial S} \scalar{\rvec v}{\dd \arcv}
    = \int_S \scalar{\rot \rvec v}{\dd \rvec S}
    = \int_S \scalar{\nvec}{\dd \rvec S}
    = 0
    \text.
  $$
\end{example}

\begin{learnMore}[Stokes-tétel Maxwell III. és IV. egyenletében]
  A Stokes-tétel a Maxwell-egyenletekben is fontos szerepet játszik. A harmadik
  és negyedik egyenlet a mágneses tér és az elektromos tér közötti
  kapcsolatot írja le:
  $$
    \begin{aligned}
      (III) & \quad \Rightarrow \quad \rot \rvec E = -\dot{\rvec B}
            &
            & \quad \Rightarrow \quad \text{elektromos tér -- mágneses tér változása,}
      \\
      (IV)  & \quad \Rightarrow \quad \rot \rvec B = \mu_0 \rvec j + \mu_0
      \varepsilon_0 \dot{\rvec E}
            &
            & \quad \Rightarrow \quad \text{mágneses tér -- elektromos tér változása,}
    \end{aligned}
  $$
  ahol $\rvec E$ az elektromos tér, $\rvec B$ a mágneses tér, $\rvec j$ az áram
  sűrűség, $\mu_0$ a mágneses permeabilitás és $\varepsilon_0$ az elektromos
  permittivitás.

  Az egyenletek közötti kapcsolatot a Stokes-tétel segítségével:
  $$
    \begin{aligned}
      (III) & \quad \Rightarrow \quad
      \oint_{\partial S} \scalar{\rvec E}{\dd \rvec r}
      = -\int_S \scalar{\dot{\rvec B}}{\dd \rvec S}
      \text,
      \\
      (IV)  & \quad \Rightarrow \quad
      \oint_{\partial S} \scalar{\rvec B}{\dd \rvec r}
      = \int_S \scalar{\mu_0 \rvec j + \mu_0 \varepsilon_0 \dot{\rvec E}}{\dd \rvec S}
      \text.
    \end{aligned}
  $$

  A III. egyenlet azt mondja ki, hogy változó mágneses tér maga körül
  balkézszabály szerint elektormos teret indukál, míg a IV. egyenlet azt
  jelenti, hogy az elektromos tér változása jobbkézszabály szerint
  mágneses teret indukál.
\end{learnMore}

% ~~~~~~~~~~~~~~~~~~~~~~~~~~~~~~~~~~~~~~~~~~~~~~~~~~~~~~~~~~~~~~~~~~~~~~~~~~~~~~
% ~~~~~~~~~~~~~~~~~~~~ Divergence theorem ~~~~~~~~~~~~~~~~~~~~~~~~~~~~~~~~~~~~~~
% ~~~~~~~~~~~~~~~~~~~~~~~~~~~~~~~~~~~~~~~~~~~~~~~~~~~~~~~~~~~~~~~~~~~~~~~~~~~~~~
\begin{theorem}[Gauss-Osztogradszkij-tétel]
  Legyen $V$ egy $\mathbb R^3$-beli irányított, parametrizált térfogat. Legyen
  továbbá $\rvec v : \mathbb R^3 \rightarrow \mathbb R^3$ legalább egyszer
  folytonosan differenciálható vektormező. Jelölje $\partial V = S$ a $V$
  peremét indukált irányítással. Ekkor:
  $$
    \int_V \Div \rvec v \, \dd V
    =
    \oint_{\partial V} \scalar{\rvec v}{\dd \rvec S}
    \text.
  $$
\end{theorem}

\begin{note}
  Ha $\rvec v$ vektorpotenciálos, akkor az integrál értéke zérus, hiszen
  $\Div \rvec v = \Div \rot \rvec u = 0$.
\end{note}

\begin{example}
  Integrálja a $\rvec v(\coordv) = \ijk{x^2 y z}{x y^2 z}{2 x y z^2}$
  vektormezőt az első térnyolcadban lévő egységkocka felületén kifele mutató
  irányítással!

  Határozzuk meg a $\rvec v$ vektormező divergenciáját:
  $$
    \Div \rvec v
    = \pdv{x^2 y z}{x} + \pdv{x y^2 z}{y} + \pdv{2 x y z^2}{z}
    = 8 x y z
    \text.
  $$

  A Gauss-Osztogradszkij-tétel alapján:
  $$
    \oint_{\partial V} \scalar{\rvec v}{\dd \rvec S}
    = \int_V \Div \rvec v \dd V
    = \int_0^1 \int_0^1 \int_0^1 8 x y z \dd z \dd y \dd x
    = 1
    \text.
  $$
\end{example}

\begin{learnMore}[Gauss-Osztogradszkij-tétel Maxwell I. és II. egyenletében]
  A Gauss-Osztogradszkij-tétel a Maxwell-egyenletekben is fontos szerepet
  játszik. Az első két egyenlet az elektromos és mágneses tér forrásosságát
  írja le:
  $$
    \begin{aligned}
      (I)  & \quad \Rightarrow \quad \Div \rvec E = \frac{\rho}{\varepsilon_0}
           &
           & \quad \Rightarrow \quad \text{elektromos tér forrásos,}
      \\
      (II) & \quad \Rightarrow \quad \Div \rvec B = 0
           &
           & \quad \Rightarrow \quad \text{mágneses tér forrásmentes,}
    \end{aligned}
  $$
  ahol $\rvec E$ az elektromos tér, $\rvec B$ a mágneses tér,
  $\rho$ az elektromos töltéssűrűség, $\varepsilon_0$ az elektromos
  permittivitás.

  Az egyenletek közötti kapcsolatot a Gauss-Osztogradszkij-tétel segítségével:
  $$
    \begin{aligned}
      (I)  & \quad \Rightarrow \quad
      \oint_{\partial V} \scalar{\rvec E}{\dd \rvec S}
      = \int_V \Div \rvec E \dd V
      = \int_V \frac{\rho}{\varepsilon_0} \dd V
      \text,
      \\
      (II) & \quad \Rightarrow \quad
      \oint_{\partial V} \scalar{\rvec B}{\dd \rvec S}
      = \int_V \Div \rvec B \dd V
      = 0
      \text.
    \end{aligned}
  $$
  Az első egyenlet azt mondja ki, hogy zárt felületen áthaladő elektromos
  tér fluxusa arányos az elektromos töltéssűrűség térfogati integráljával.
  A második egyenlet pedig azt jelenti, hogy a mágneses tér fluxusa zárt
  felületen zérus, a mágneses tér forrásmentes.
\end{learnMore}

% ~~~~~~~~~~~~~~~~~~~~~~~~~~~~~~~~~~~~~~~~~~~~~~~~~~~~~~~~~~~~~~~~~~~~~~~~~~~~~~
% ~~~~~~~~~~~~~~~~~~~~ Green's theorem ~~~~~~~~~~~~~~~~~~~~~~~~~~~~~~~~~~~~~~~~~
% ~~~~~~~~~~~~~~~~~~~~~~~~~~~~~~~~~~~~~~~~~~~~~~~~~~~~~~~~~~~~~~~~~~~~~~~~~~~~~~
\begin{theorem}[Green-tétel asszimetrikus alakja]
  Legyenek $\varphi; \psi: \mathbb R^3 \rightarrow \mathbb R$ kétszeresen
  folytonos skalármezők, $V \subset \mathbb R^3$ parametrizált, irányított
  tértartomány, $\partial V = S$ perem indukált irányítással. Ekkor:
  \[
    \int_V
    \psi \, \Delta \varphi +
    \scalar{\grad \psi}{\grad \varphi}
    \dd V
    =
    \oint_{\partial V} \scalar{\psi \grad \varphi}{\dd \rvec S}
    \text.
  \]
\end{theorem}

\begin{note}
  $\psi = 1$ választásával visszanyerjük a Gauss-Osztogradszkij-tételt:
  $$
    \int_V \Delta \varphi \dd V
    = \int_V \Div \underbrace{\grad \varphi}_{\rvec v} \dd V
    = \oint_{\partial V} \scalar{\underbrace{\grad \varphi}_{\rvec v}}{\dd \rvec S}
    \text.
  $$
\end{note}

\begin{example}
  Tekintsünk egy $R = 1\,\text{m}$ sugarú tömör alumínium gömböt, amelynek
  stacionárius hőmérséklet-eloszlása $\varphi(\coordv) = T_0 (1 - \rvec r^2)$
  függvény írja le, ahol $T_0 = 10\,\text{K}$ a gömb belső hőmérséklete.
  Határozza meg a gömb felületén kifelé irányuló összes  $\dot Q$ hőáramot,
  ha a hőfluxus sűrűsége $\rvec q = -\lambda \grad \varphi$, ahol
  $\lambda = 205\,\text{W/(m·K)}$ az alumínium hővezetési tényezője, és
  $$
    \dot Q = \oint_{\partial V} \scalar{\rvec q}{\dd \rvec S}
    \text.
  $$
  Használjuk a Green-tétel asszimetrikus alakját $\psi = -\lambda$ állandó
  választással:
  $$
    \dot Q
    = \oint_{\partial V} \scalar{-\lambda \grad \varphi}{\dd \rvec S}
    = - \int_V \psi \, \Delta \varphi + \scalar{\underbrace{\grad \lambda}_{= \nvec}}{\grad \varphi} \dd V
    = -\lambda \int_V \Delta \varphi \, \dd V
    \text.
  $$
  Számítsuk ki a térfogat belsejében a $\Delta \varphi$ értékét:
  $$
    \Delta \varphi
    =\pdv[order=2]{\varphi}{x} + \pdv[order=2]{\varphi}{y} + \pdv[order=2]{\varphi}{z}
    = -6 T_0
    \text.
  $$
  A hőáram összesen:
  $$
    \dot Q = \int_V \underbrace{-\lambda (-6 T_0)}_{=\text{const}} \dd V
    = -\lambda (-6 T_0) \frac{4\pi R^3}{3}
    = 8 \pi \lambda T_0 R^3
    \approx 5,15 \times 10^{4}\,\text{W}
    \text.
  $$
\end{example}

\begin{theorem}[Green-tétel szimmetrikus alakja]
  Legyenek $\varphi; \psi: \mathbb R^3 \rightarrow \mathbb R$ kétszeresen
  folytonos skalármezők, $V \subset \mathbb R^3$ parametrizált, irányított
  tértartomány, $\partial V = S$ perem indukált irányítással. Ekkor:
  \[
    \int_V
    \psi \, \Delta \varphi + \varphi \, \Delta \psi
    \; \dd V
    =
    \oint_{\partial V}
    \scalar{\psi \grad \varphi - \varphi \grad \psi}{\dd \rvec S}
  \]
\end{theorem}

\begin{note}
  $\psi = 1$ választásával visszanyerjük a szimmetrikus alakból is
  visszanyerhető a Gauss-Osztogradszkij-tétel.
  ($\Delta 1 = 0$ és $\grad 1 = \nvec$)
\end{note}

\clearpage
\subsection{Feladatok}

\begin{enumerate}
  \item Egy automata raktárrendszer ferromágneses manipulátora egy előre
        számított mágneses potenciálmezőben mozog. A számított Joule-potenciál a
        munkatérben
        $$
          \varphi(\coordv) = 2x^2 y + 3yz
          \text.
        $$
        A csípőkaron lévő, $Q = 1\,\text{C}$ ekvivalens töltéssel modellezett
        végfogót a vezérlőnek az
        $$
          A(0;0;0) \rightarrow B(2;1;3)
        $$
        pontok között kell mozgatnia. Mennyi munkát végez a mágneses tér a
        végfogón, és miért nem kell törődnünk az útvonal pontos alakjával?
        ($\rvec F = -Q \grad \varphi$, a munka az erő pálya menti integrálja)

  \item Egy elektromágneses rendszer mágneses térerősségét a
        $$
          \rvec B(\coordv) = \ijk{y \, \sin x}{z^2 \, \cos y - \cos x}{b_3}
        $$
        vektormező írja le. Határozza meg $b_3$ értékét, ha tudjuk, hogy
        a vektormező bármely zárt görbe mentén vett integrálja zérus!
        (Vagyis mikor lesz $\rot \rvec B = \nvec$)

  \item Egy $R = 1$ sugarú, kör keresztmetszetű, $z$ tengellyel egybeeső
        szimmetriavonalú hengerben áramló folyadék sebességét a
        $$
          \rvec v(\coordv) = \ijk{2xy + z}{x^2 + z}{y - x}
        $$
        vektormező írja le. Adja meg a $z = 1$ síkban lévő keresztmetszet
        menti cirkulációt!
        (A cirkuláció a vektormező zárt görbe menti integrálja.)

  \item Egy fotonikus chipeket hordozó wafer-darabot egy ellipszoid alakú
        Faraday-kalitkába rögzítenek. A kalitka belsejében lineáris
        feszültségelosztással ($\rvec E(\coordv) = \coordv$) térerőt állítanak
        elő. Számolja ki a Faraday-kalitka belsejében lévő nettó töltést, ha
        $\varepsilon_0 = 8,85 \times 10^{-12} \, \text{F/m}$, az ellipszoid
        egyenlete pedig:
        $$
          \frac{(x - 2)^2}{5} + \frac{(y + 3)^2}{19} + \frac{(z - 1)^2}{4} = 1
          \text.
        $$

  \item Egy drón IMU-modulját teljes egészében kitöltő, hővezető műgyanta gömb
        alakú, sugara $R = 0,02 \, \text{m}$, A vezérlő egység folyamatosan
        hőt disszipál, az állandósult hő\-mér\-sék\-let-mező jó közelítéssel
        $$
          \varphi(\coordv) = T_c - \alpha \rvec r^2
          \text, \quad
          \alpha = 1,3 \times 10^5 \,\text{K/m}^2
          \text.
        $$
        Becsülje meg, mekkora teljes hőáram távozik a burkolaton át, ha
        $$
          q_{\text{hő}}
          = -\lambda \oint_{\partial V} \scalar{\grad \varphi}{\dd \rvec S}
          \text,\quad
          \lambda = 0,2 \, \text{W/(m K)}
          \text.
        $$
\end{enumerate}

% \clearpage
% \subsection{Megoldások}

% \begin{enumerate}
%   \item A munka és a potenciál közötti kapcsolat $\rvec F = - Q \grad \varphi$
%         egyenletet felhasználva:
%         $$
%           W = \int_A^B \scalar{\rvec F}{\dd \arcv}
%           = -Q \int_A^B \scalar{\grad \varphi}{\dd \arcv}
%           \text.
%         $$

%         Mivel a $\varphi$ skalármezőből származtatott erőmező örvénymentes
%         ($\rot \grad \varphi = \nvec$), ezért a Gradiens-tétel alapján a megtett
%         munka csak a kezdő- és végpontoktól függ. A potenciálfüggvény
%         értéke a kezdő- és végpontokban:
%         \begin{align*}
%           \varphi(A) & = 2 \cdot 0^2 \cdot 0 + 3 \cdot 0 \cdot 0 = 0
%           \, \text{J/C}
%           \text,
%           \\
%           \varphi(B) & = 2 \cdot 2^2 \cdot 1 + 3 \cdot 1 \cdot 3 = 8 + 9 = 17
%           \, \text{J/C}
%           \text.
%         \end{align*}
%         Az e
%         $$
%           W
%           = \int_A^B \scalar{\rvec F}{\dd \arcv}
%           = -Q \int_A^B \scalar{\grad \varphi}{\dd \arcv}
%           = -Q(\varphi(B) - \varphi(A))
%           = -17 \, \text{J}
%           \text.
%         $$
% \end{enumerate}

\end{document}

