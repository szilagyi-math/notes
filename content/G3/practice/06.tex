\documentclass{szb-practice}

\usepackage{siunitx}
\sisetup{locale = DE}

\title{Integrálátalakító tételek II}
\area{Vektoranalízis}
\subject{Matematika G3}
\subjectCode{BMETE94BG03}
\date{Utoljára frissítve: \today}
\docno{6}

\begin{document}
\allowdisplaybreaks

\maketitle

\vspace{-1em}
\subsection{Elméleti áttekintő}
\vspace{1em}

\begin{definition}[Térfogat]
  Legyen $\volsign: \voldomain \rightarrow \volimage$ paraméterezett
  tértartomány, ahol $r; s; t \in \voldomain$ a tértartomány paraméterezése,
  $\volsign(\voldomain) = \volimage$ a tértartomány képe,
  $\dd \volscalar = \det(\DD \volsign(r; s; t)) \dd r \dd s \dd t$.
  Ekkor a térfogat:
  $$
    \operatorname{Vol} \volimage
    = \iiint_{\volimage} \dd \volscalar
    = \iiint_{\volimage} \left|
    \det \left( \DD \volsign(r; s; t) \right)
    \right|
    \dd r \dd s \dd t
    \text.
  $$
\end{definition}

\begin{definition}[Térfogati integrál]
  Legyen $\varphi(\coordvec): \Reals^3 \to \Reals$ skalármező,
  $\volsign: \voldomain \rightarrow \volimage$ paraméterezett tértartomány, ahol
  $r; s; t \in \voldomain$ a tértartomány paraméterezése,
  $\volsign(\voldomain) = \volimage$ a tértartomány képe,
  $\dd \volscalar = \det(\DD \volsign(r; s; t)) \dd r \dd s \dd t$. Ekkor
  a $\varphi$ skalármező térfogaton vett integrálja:
  \begin{equation*}
    \iiint_{\volimage} \varphi(\coordvec) \dd \volscalar
    = \iiint_{\voldomain} \varphi \left( \volsign(r; s; t) \right)
    \det \left( \DD \volsign(r; s; t) \right)
    \dd r \dd s \dd t
    \text.
  \end{equation*}
\end{definition}

\begin{blueBox}[\underline{Tértartományok paraméterezése}][nobreak]
  \def\tskip{14mm}
  \begin{tabular}{
    >{\hspace{-.5em}\bullet\;}p{2.25cm}
    p{5.5cm}
    m{2.5cm}
    >{\centering\arraybackslash}m{3.25cm}
    }
    \textbf{Gömb:}
     & $\volsign (r; s; t) = \begin{bmatrix} r \sin s \cos t \\ r \sin s \sin t \\ r \cos s \end{bmatrix}$
     & $r \in [0; R]$ \newline $s \in [0;\pi]$ \newline $t \in [0, 2\pi]$
     & \begin{tikzpicture}[
           3d view={135}{35.26},
           % 3d view={110}{20},
           baseline,
           scale=.75,
         ]
         % Origin coordinate
         \coordinate (O) at (0,0,0);

         % Helper coordinates
         \foreach \s in {0,30,...,360} {
             \foreach \t in {0,30,...,330} {
                 \coordinate (\s-\t) at
                 (
                 {sin(\s/2)*cos(\t)},
                 {sin(\s/2)*sin(\t)},
                 {cos(\s/2)}
                 );
               }
           }

         % Draw surface in bg
         \foreach \x/\y in {330/360,300/330,270/300,240/270,210/240,180/210,150/180,120/150,90/120,60/90,30/60,0/30} {
             \foreach \i/\j in {150/180,180/210,210/240,240/270,270/300,300/330,330/0,120/150,90/120}{
                 \fill[thick, draw=red-base, fill=red-base!20, rounded corners=.1pt] (\x-\i) -- (\y-\i) -- (\y-\j) -- (\x-\j) -- cycle;
               }
           }

         % Coordinate system
         \draw[-to] (O) -- ++(1.75,0,0) node[anchor=south east] {$x$};
         \draw[-to] (O) -- ++(0,1.75,0) node[anchor=south west] {$y$};
         \draw[-to] (O) -- ++(0,0,1.75) node[anchor=north east] {$z$};

         % Draw surface in fg
         \foreach \x/\y in {330/360,300/330,270/300,240/270,210/240,180/210,150/180,120/150,90/120,60/90,30/60,0/30} {
             \foreach \i/\j in {0/30,30/60,60/90} {
                 \fill[thick, draw=red-base, fill=red-base!20, rounded corners=.1pt] (\x-\i) -- (\y-\i) -- (\y-\j) -- (\x-\j) -- cycle;
               }
           }
       \end{tikzpicture}
    \\[\tskip]
    \textbf{Ellipszoid:}
     & $\volsign (r;s;t) = \begin{bmatrix} a \, r \sin s \cos t \\ b \, r \sin s \sin t \\ c \, r \cos s \end{bmatrix}$
     & $r \in [0;1]$ \newline $s \in [0;\pi]$ \newline $t \in [0, 2\pi]$
     & \begin{tikzpicture}[
           3d view={110}{20},
           baseline,
           scale=.85
         ]
         % Origin coordinate
         \coordinate (O) at (0,0,0);

         % Helper coordinates
         \foreach \s in {0,30,...,360} {
             \foreach \t in {0,30,...,330} {
                 \coordinate (\s-\t) at
                 (
                 {0.8*sin(\s/2)*cos(\t)},
                 {1.4*sin(\s/2)*sin(\t)},
                 {0.6*cos(\s/2)}
                 );
               }
           }

         % Draw surface in bg
         \foreach \x/\y in {330/360,300/330,270/300,240/270,210/240,180/210,150/180,120/150,90/120,60/90,30/60,0/30} {
             \foreach \i/\j in {150/180,180/210,210/240,240/270,270/300,300/330,330/0,120/150,90/120}{
                 \fill[thick, draw=red-base, fill=red-base!20, rounded corners=.1pt] (\x-\i) -- (\y-\i) -- (\y-\j) -- (\x-\j) -- cycle;
               }
           }

         % Coordinate system
         \draw[-to] (O) -- ++(2.25,0,0) node[anchor=east] {$x$};
         \draw[-to] (O) -- ++(0,2.00,0) node[anchor=south] {$y$};
         \draw[-to] (O) -- ++(0,0,1.25) node[anchor=north east] {$z$};

         % Draw surface in fg
         \foreach \x/\y in {330/360,300/330,270/300,240/270,210/240,180/210,150/180,120/150,90/120,60/90,30/60,0/30} {
             \foreach \i/\j in {0/30,30/60,60/90} {
                 \fill[thick, draw=red-base, fill=red-base!20, rounded corners=.1pt] (\x-\i) -- (\y-\i) -- (\y-\j) -- (\x-\j) -- cycle;
               }
           }
       \end{tikzpicture}
    \\[\tskip]
    \textbf{Тórusz:}
     & $\volsign (r;s;t) = \begin{bmatrix} (R + r \cos s) \cos t \\ (R + r \cos s) \sin t \\ r \sin s \end{bmatrix}$
     & $r \in [0;r_0]$ \newline $s \in [0;2\pi]$ \newline $t \in [0, 2\pi]$
     & \begin{tikzpicture}
         \begin{axis}[
          axis equal image,
          hide axis,
          % z buffer = sort,
          view={110}{20},
          xmax=2.5,
          ymax=2.5,
          zmax=2.25,
          scale=.66,
        ]

        \draw[-to] (axis cs:0,0,0) -- (axis cs:1.25,0,0);
        \draw[-to] (axis cs:0,0,0) -- (axis cs:0,1.25,0);

        \addplot3[
          surf,
          faceted color=red-base,
          fill=red-base!20,
          ultra thin,
          samples=15,
          samples y=48,
          domain=0:2*pi,
          domain y=0:2*pi,
          z buffer=sort,
        ](
        {(1.25+0.25*sin(deg(\x)))*cos(deg(\y))},
        {(1.25+0.25*sin(deg(\x)))*sin(deg(\y))},
        {0.25*cos(deg(\x))}
        );

        \draw[-to] (axis cs:0,0,0) -- (axis cs:0,0,2) node[anchor=north east] {$z$};

        \draw[-to] (axis cs:1.5,0,0) -- (axis cs:2.25,0,0) node[anchor=north] {$x$};
        \draw[-to] (axis cs:0,1.5,0) -- (axis cs:0,1.75,0) node[anchor=north] {$y$};

        \coordinate (C) at (axis cs:0,1.25,1.25);
        \coordinate (T) at (axis cs:0,1.25,1.65);
        \coordinate (T+) at (axis cs:0,1.25,1.75);
        \coordinate (Z) at (axis cs:0,0,1.65);
      \end{axis}

         \begin{scope}[3d view={110}{20},canvas is yz plane at x=0]
        \draw[red-base, fill=red-base!20] (C) circle (0.33);
        \draw[red-base, fill=red-base, ultra thick] (C) circle (0.02);

        \draw[gray] (C) -- (T+);
        \draw[to-to, thick, draw=blue-base]
        (Z) -- (T)
        node[midway, anchor=north] {\scriptsize$R$};
        \draw[-to, thick, draw=blue-base]
        (C) -- ++(0.75*0.8,0.75*0.6) -- ($(C)+(0.33*0.8,0.33*0.6)$)
        node[midway, anchor=south, inner sep=.5mm, font=\scriptsize] {$r_0$};
      \end{scope}
       \end{tikzpicture}
  \end{tabular}
\end{blueBox}

% ~~~~~~~~~~~~~~~~~~~~~~~~~~~~~~~~~~~~~~~~~~~~~~~~~~~~~~~~~~~~~~~~~~~~~~~~~~~~~~
% ~~~~~~~~~~~~~~~~~~~~ Divergence theorem ~~~~~~~~~~~~~~~~~~~~~~~~~~~~~~~~~~~~~~
% ~~~~~~~~~~~~~~~~~~~~~~~~~~~~~~~~~~~~~~~~~~~~~~~~~~~~~~~~~~~~~~~~~~~~~~~~~~~~~~
\begin{theorem}[Gauss-Osztogradszkij-tétel]
  Legyen $\volsign: \voldomain \subset \Reals^3 \rightarrow \volimage \subset
    \Reals^3$, irányított,parametrizált tértartomány. Legyen továbbá $\rvec v:
    \Reals^3 \rightarrow \Reals^3$ legalább egyszer folytonosan
  differenciálható vektormező. Jelölje a $\partial \volimage =
    \surfimage$ az $\volsign$ peremét indukált irányítással. Ekkor:
  $$
    \iiint_{\volimage} \Div \rvec v \dd \volscalar
    =
    \oiint_{\partial \volimage} \scalar{\rvec v}{\dd \surfvec}
    \text.
  $$
\end{theorem}

\begin{note}
  Ha $\rvec v$ vektorpotenciálos, akkor az integrál értéke zérus, hiszen
  $\Div \rvec v = \Div \rot \rvec u = 0$.
\end{note}

\begin{example}
  Integrálja a $\rvec v(\coordv) = \ijk{x^2 y z}{x y^2 z}{2 x y z^2}$
  vektormezőt az első térnyolcadban lévő egységkocka felületén kifele mutató
  irányítással!

  Határozzuk meg a $\rvec v$ vektormező divergenciáját:
  $$
    \Div \rvec v
    = \pdv{x^2 y z}{x} + \pdv{x y^2 z}{y} + \pdv{2 x y z^2}{z}
    = 8 x y z
    \text.
  $$

  A Gauss-Osztogradszkij-tétel alapján:
  $$
    \oiint_{\partial \volimage} \scalar{\rvec v}{\dd \surfvec}
    = \iiint_{\volimage} \Div \rvec v \dd \volscalar
    = \int_0^1 \int_0^1 \int_0^1 8 x y z \dd z \dd y \dd x
    = 1
    \text.
  $$
\end{example}

\begin{learnMore}[Gauss-Osztogradszkij-tétel Maxwell I. és II. egyenletében]
  A Gauss-Osztogradszkij-tétel a Maxwell-egyenletekben is fontos szerepet
  játszik. Az első két egyenlet az elektromos és mágneses tér forrásosságát
  írja le:
  $$
    \begin{aligned}
      (I)  & \quad \Rightarrow \quad \Div \rvec E = \frac{\rho}{\varepsilon_0}
           &
           & \quad \Rightarrow \quad \text{elektromos tér forrásos,}
      \\
      (II) & \quad \Rightarrow \quad \Div \rvec B = 0
           &
           & \quad \Rightarrow \quad \text{mágneses tér forrásmentes,}
    \end{aligned}
  $$
  ahol $\rvec E$ az elektromos tér, $\rvec B$ a mágneses tér,
  $\rho$ az elektromos töltéssűrűség, $\varepsilon_0$ az elektromos
  permittivitás.

  Az egyenletek közötti kapcsolatot a Gauss-Osztogradszkij-tétel segítségével:
  $$
    \begin{aligned}
      (I)  & \quad \Rightarrow \quad
      \oiint_{\partial \volimage} \scalar{\rvec E}{\dd \surfvec}
      = \iiint_{\volimage} \Div \rvec E \dd \volscalar
      = \iiint_{\volimage} \frac{\rho}{\varepsilon_0} \dd \volscalar
      \text,
      \\
      (II) & \quad \Rightarrow \quad
      \oiint_{\partial \volimage} \scalar{\rvec B}{\dd \surfvec}
      = \iiint_{\volimage} \Div \rvec B \dd \volscalar
      = 0
      \text.
    \end{aligned}
  $$
  Az első egyenlet azt mondja ki, hogy zárt felületen áthaladó elektromos
  tér fluxusa arányos az elektromos töltéssűrűség térfogati integráljával.
  A második egyenlet pedig azt jelenti, hogy a mágneses tér fluxusa zárt
  felületen zérus, a mágneses tér forrásmentes.
\end{learnMore}

% ~~~~~~~~~~~~~~~~~~~~~~~~~~~~~~~~~~~~~~~~~~~~~~~~~~~~~~~~~~~~~~~~~~~~~~~~~~~~~~
% ~~~~~~~~~~~~~~~~~~~~ Green's theorem ~~~~~~~~~~~~~~~~~~~~~~~~~~~~~~~~~~~~~~~~~
% ~~~~~~~~~~~~~~~~~~~~~~~~~~~~~~~~~~~~~~~~~~~~~~~~~~~~~~~~~~~~~~~~~~~~~~~~~~~~~~
\begin{theorem}[Green-tétel asszimetrikus alakja]
  Legyenek $\varphi; \psi: \mathbb R^3 \rightarrow \mathbb R$ kétszeresen
  folytonos skalármezők, $\volimage \subset \mathbb R^3$ parametrizált,
  irányított tértartomány, $\partial \volimage = \surfimage$ perem indukált
  irányítással. Ekkor:
  $$
    \iiint_{\volimage}
    \psi \, \Delta \varphi +
    \scalar{\grad \psi}{\grad \varphi}
    \dd \volscalar
    =
    \oiint_{\partial \volimage} \scalar{\psi \grad \varphi}{\dd \surfvec}
    \text.
  $$
\end{theorem}

\begin{note}
  $\psi = 1$ választásával visszanyerjük a Gauss-Osztogradszkij-tételt:
  $$
    \iiint_{\volimage} \Delta \varphi \dd \volscalar
    = \iiint_{\volimage} \Div \underbrace{\grad \varphi}_{\rvec v} \dd \volscalar
    = \oiint_{\partial \volimage} \scalar{\underbrace{\grad \varphi}_{\rvec v}}{\dd \surfvec}
    \text.
  $$
\end{note}

\begin{example}
  Tekintsünk egy $R = 1\,\text{m}$ sugarú tömör alumínium gömböt, amelynek
  stacionárius hőmér\-sék\-let-eloszlása $\varphi(\coordv) = T_0 (1 - \rvec r^2)$
  függvény írja le, ahol $T_0 = \SI{10}{\kelvin}$ a gömb belső hőmérséklete.
  Határozza meg a gömb felületén kifelé irányuló összes  $\dot Q$ hőáramot,
  ha a hőfluxus sűrűsége $\rvec q = -\lambda \grad \varphi$, ahol
  $\lambda = \SI[per-mode=symbol]{205}{\watt\per\meter\per\kelvin}$ az
  alumínium hővezetési tényezője, és
  $$
    \dot Q = \oiint_{\partial \volimage} \scalar{\rvec q}{\dd \surfvec}
    \text.
  $$
  Használjuk a Green-tétel asszimetrikus alakját $\psi = -\lambda$ állandó
  választással:
  $$
    \dot Q
    = \oiint_{\partial \volimage} \scalar{-\lambda \grad \varphi}{\dd \surfvec}
    = - \iiint_{\volimage} \psi \, \Delta \varphi + \scalar{\underbrace{\grad \lambda}_{= \nvec}}{\grad \varphi} \dd \volscalar
    = -\lambda \iiint_{\volimage} \Delta \varphi \dd \volscalar
    \text.
  $$
  Számítsuk ki a térfogat belsejében a $\Delta \varphi$ értékét:
  $$
    \Delta \varphi
    =\pdv[order=2]{\varphi}{x} + \pdv[order=2]{\varphi}{y} + \pdv[order=2]{\varphi}{z}
    = -6 T_0
    \text.
  $$
  A hőáram összesen:
  $$
    \dot Q = \iiint_{\volimage} \underbrace{-\lambda (-6 T_0)}_{=\text{const}} \dd \volscalar
    = -\lambda (-6 T_0) \frac{4\pi R^3}{3}
    = 8 \pi \lambda T_0 R^3
    \approx \SI{5,15e4}{\watt}
    \text.
  $$
\end{example}

\begin{theorem}[Green-tétel szimmetrikus alakja]
  Legyenek $\varphi; \psi: \mathbb R^3 \rightarrow \mathbb R$ kétszeresen
  folytonos skalármezők, $\volimage \subset \mathbb R^3$ parametrizált,
  irányított tértartomány, $\partial \volimage = \surfimage$ perem indukált
  irányítással. Ekkor:
  $$
    \iiint_{\volimage}
    \psi \, \Delta \varphi + \varphi \, \Delta \psi
    \; \dd \volscalar
    =
    \oiint_{\partial \volimage}
    \scalar{\psi \grad \varphi - \varphi \grad \psi}{\dd \surfvec}
  $$
\end{theorem}

\clearpage
\subsection{Feladatok}

\begin{enumerate}
  \item Vezesse le az $R$ sugarú gömb térfogatát térfogati integrál
        segítségével, majd számítsa ki a $\varphi(\coordvec) = 1 / \|\coordvec\|$
        skalármező térfogati integrálját a gömbön! A tértartomány ajánlott
        paraméterezése:
        $$
          \volsign(r; s; t) = \begin{bmatrix}
            r \sin s \cos t \\
            r \sin s \sin t \\
            r \cos s
          \end{bmatrix}
          \qquad
          \begin{array}{l}
            r \in [0; R]   \\
            s \in [0; \pi] \\
            t \in [0; 2\pi]
          \end{array}
        $$

  \item Vezesse le az $R$ sugarú, $h$ magasságú henger térfogatát térfogati
        integrál segítségével, majd számítsa ki a
        $\varphi(\coordvec) = x^2 + y^2$ skalármező térfogati integrálját a
        hengerben! A tértartomány ajánlott paraméterezése:
        $$
          \volsign(r; s; t) = \begin{bmatrix}
            r \cos s \\
            r \sin s \\
            t
          \end{bmatrix}
          \qquad
          \begin{array}{l}
            r \in [0; R]    \\
            s \in [0; 2\pi] \\
            t \in [0; h]
          \end{array}
        $$

  \item Legyen $\rvec v(\coordvec) = \ijk{y \sin x}{z^2 \cos y - \cos x}{v_3}$.
        Határozza meg $v_3$-at, ha tudjuk, hogy $\rvec v$ tetszőleges zárt
        felületen vett felületi integrálja zérus!

  \item Legyen $\rvec v(\coordvec) = \coordvec$. Adja meg a vektormező alábbi
        zárt felületeken vett felületi integráljait:
        \begin{enumerate}
          \item az $R = 2$ sugarú gömb felületén befelé mutató irányítással,
          \item az $x^2 + y^2 = 4$ hengerfelületen, amelyet a $z = -1$ és
                $z = 1$ síkok zárnak le, kifelé mutató irányítással,
          \item a $z = x^2 + y^2$ forgásparaboloid és a $z = 4$ sík által
                határolt test felületén kifelé mutató irányítással.
        \end{enumerate}

  \item Egy fotonikus chipeket hordozó wafer-darabot egy ellipszoid alakú
        Faraday-kalitkába rögzítenek. A kalitka belsejében lineáris
        feszültségelosztással ($\rvec E(\coordv) = \coordv$) térerőt állítanak
        elő. Számolja ki a Faraday-kalitka belsejében lévő nettó töltést, ha
        $\varepsilon_0 = \SI[per-mode=symbol]{8,85e-12}{\farad\per\meter}$,
        az ellipszoid egyenlete pedig:
        $$
          \frac{(x - 2)^2}{5} + \frac{(y + 3)^2}{19} + \frac{(z - 1)^2}{4} = 1
          \text.
        $$

  \item Egy drón IMU-modulját teljes egészében kitöltő, hővezető műgyanta gömb
        alakú, sugara $R = \SI{0.02}{\meter}$, A vezérlő egység folyamatosan
        hőt disszipál, az állandósult hő\-mér\-sék\-let-mező jó közelítéssel
        $$
          \varphi(\coordv) = T_c - \alpha \rvec r^2
          \text, \quad
          \alpha = \SI[per-mode=symbol]{1,3e5}{\kelvin\per\meter\squared}
          \text.
        $$
        Becsülje meg, mekkora teljes hőáram távozik a burkolaton át, ha
        $$
          q_{\text{hő}}
          = - \oiint_{\partial \volimage} \scalar{\lambda \grad \varphi}{\dd \surfvec}
          \text,\quad
          \lambda = \SI[per-mode=symbol]{0,2}{\watt\per\meter\per\kelvin}
          \text.
        $$
\end{enumerate}

\end{document}