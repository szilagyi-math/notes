\documentclass{szb-practice}

\title{Differenciálegyenlet-rendszerek I}
\area{Differenciálegyenletek}
\subject{Matematika G3}
\subjectCode{BMETE94BG03}
\date{Utoljára frissítve: \today}
\docno{13}

\usepackage{siunitx}
\sisetup{locale = DE}

\begin{document}
\maketitle

\subsection{Elméleti áttekintő}

\begin{blueBox}[\underline{Euler-féle differenciálegyenletek}]%
  \textbf{\underline{Áltanos alak}}:
  $$
    a_n y^{(n)}
    + \frac{a_{n-1}}{x} y^{(n-1)}
    + \frac{a_{n-2}}{x^2} y^{(n-2)}
    + \dots
    + \frac{a_1}{x^{n-1}} y'
    + \frac{a_0}{x^n} y
    = f(x)
  $$
  A differenciálegyenlet $x = 0$-ban nincs értelmezve, $x > 0$, vagy $x < 0$
  intervallumon lehet megoldani.

  \textbf{\underline{Megoldási módszer}}: $x = e^t$ helyettesítés, majd konstans együtthatós
  differenciálegyenlet megoldása:
  $$
    x = e^t
    \quad \Rightarrow \quad
    t = \ln x
    \quad \Rightarrow \quad
    y' = \frac{\dot y}{e^t}
    \text, \quad
    y'' = \frac{\ddot y - \dot y}{e^{2t}}
    \text, \quad
    \dots
    \text.
  $$
\end{blueBox}

\begin{blueBox}[\underline{Másodrendű, változó együtthatós, lineáros differenciálegyenletek}]%
  \textbf{\underline{Áltanos alak}}:
  $$
    a_2(x) y'' + a_1(x) y' + a_0(x) y = 0
  $$

  \textbf{\underline{Megoldási módszer}}: Ha ismerjük az egyik megoldást ($y_1$),
  akkor a másik ($y_2$) megolddás konstans variációs módszerrel határozható meg:
  $$
    y_2(x) = v(x) y_1(x)
    \text.
  $$

  $y_1$ és $y_2$ lineárisan függetlenségét a Wronski-determináns
  segítségével ellenőrizni kell.
\end{blueBox}

\begin{blueBox}[\underline{Állandó együtthetós, lineáris, homogén, elsőrendű DE-rendszerek}]%
  \textbf{\underline{Áltanos alak}}:
  $$
    \dot{\rvec x} = \rmat A \, \rvec x
    \text,
  $$
  ahol $\rmat A$ egy kvadratikus mátrix, $\rvec x$ pedig az ismeretlenek
  vektora, $t$ pedig a független változó (általában idő).

  \textbf{\underline{Megoldási módszer}}:
  \begin{enumerate}
    \item Határozzuk meg a mátrix sajátértékeit az
          $\det(\rmat A - \lambda \imat) = 0$ karakterisztikus egyenlet
          segítségével.

    \item Határozzuk meg a sajátértékekhez tartozó sajátvektorokat a
          $(\rmat A - \lambda_i \imat) \rvec v_i = \rvec 0$ egyenlet
          segítségével.

    \item Az alaphalmaz ekkor:
          $$
            \rmat X(t) = \begin{bmatrix}
              \rvec v_1 e^{\lambda_1 t}
               & \rvec v_2 e^{\lambda_2 t}
               & \dots
               & \rvec v_n e^{\lambda_n t}
            \end{bmatrix}
            \text.
          $$

    \item Az általános megoldás:
          $$
            \rvec x(t)
            = \rmat X(t) \, \rvec C
            = C_1 \, \rvec v_1 e^{\lambda_1 t}
            + C_2 \, \rvec v_2 e^{\lambda_2 t}
            + \dots
            + C_n \, \rvec v_n e^{\lambda_n t}
            \text.
          $$
  \end{enumerate}
\end{blueBox}

\clearpage
\subsection{Feladatok}

\begin{enumerate}
  \item Egy $\SI{6}{\meter}$ hosszú lánc súrlódásmentesen csúszik az asztalon.
        Ha a csúszás akkor következik be, amikor $\SI{1}{\meter}$-nyi lánc
        lóg lefelé, akkor mennyi idő múlva esik le a lánc?

  \item Írja fel az alábbi rezgőrendszer mozgásegyenletét, ha a rugóerő az
        $F_r = k x$ Hooke-törvény szerint, a csillapító erő pedig
        $F_c = b v$ alakú, ahol $k$ a rugóállandó, $b$ a csillapítási
        tényező, $x$ a kitérés, $v$ a sebesség. A test tömege $m$.

        \begin{center}
          \tikzstyle{spring}=[thick,decorate,decoration={zigzag,pre length=0.3cm,post
              length=0.3cm,segment length=6}]

          \tikzstyle{damper}=[thick,decoration={markings,
              mark connection node=dmp,
              mark=at position 0.5 with
                {
                  \node (dmp) [thick,inner sep=0pt,transform shape,rotate=-90,minimum
                    width=15pt,minimum height=3pt,draw=none] {};
                  \draw [thick] ($(dmp.north east)+(2pt,0)$) -- (dmp.south east) -- (dmp.south
                  west) -- ($(dmp.north west)+(2pt,0)$);
                  \draw [thick] ($(dmp.north)+(0,-5pt)$) -- ($(dmp.north)+(0,5pt)$);
                }
            }, decorate]

          \begin{tikzpicture}[
              thick,
            ]
            % mass
            \draw[fill=gray!10, rounded corners] (1.75,0)
            rectangle ++(2.5,1.75)
            node[midway] {$m$}
            ;

            % wall
            \draw[ultra thick] (0,2.75) -- (0,0) -- (5,0);
            \fill[pattern=north east lines]
            (0,2.75) -- (0,0) -- (5,0)
            -- (5,-0.15) -- ++(-5.15,0) -- ++(0,2.9) -- cycle;

            % spring
            \draw[spring] (0,1.25) -- (1.75,1.25)
            node[midway, above left] {$k$}
            ;

            % damper
            \draw[damper] (0,0.5) -- (1.75,0.5)
            node[midway, above left] {$b$}
            ;

            % null line
            \draw[dashed] (1.75,1) -- ++(0,1.75);
            \draw[->] (1.75,2.525) -- ++(1.5,0) node[above left] {$x$};
          \end{tikzpicture}
        \end{center}

  \item Adja meg az alábbi Euler-féle differenciálegyenlet megoldását!
        $$
          y'' - \frac{3}{x} \, y' + 20 \frac{y}{x^2} = 0
          \quad
          y(1) = 2
          \quad
          y'(1) = 0
        $$

  \item Az $(1 + x^2) y'' + xy' - y = 0$ differenciálegyenleteit
        $y_1(x) = x$ megoldás ismeri. Adja meg az általános megoldást!

  \item Az $xy'' - (1 + x) y' + y = 0$ differenciálegyenleteit
        $y_1(x) = e^x$ megoldás ismeri. Adja meg az általános megoldást!

  \item Oldja meg az alábbi differenciálegyenlet-rendszert!
        $$
          \left\{
          \begin{array}{l}
            \dot x = 2x + y \\
            \dot y = 3x + 4y
          \end{array}
          \right.
          \qquad
          \begin{array}{l}
            x(0) = 0 \\
            y(0) = 1
          \end{array}
        $$

  \item Adja meg az alábbi differenciálegyenlet-rendszer általános megoldását!
        $$
          \left\{
          \begin{array}{l}
            \dot x = 2x + y \\
            \dot y = y      \\
            \dot z = 3z
          \end{array}
          \right.
        $$

  \item Adja meg az alábbi differenciálegyenlet-rendszer általános megoldását!
        $$
          \left\{
          \begin{array}{l}
            \dot x = -x - 5y \\
            \dot y = x + y
          \end{array}
          \right.
        $$

  \item Adja meg az alábbi differenciálegyenlet-rendszer általános megoldását!
        $$
          \dot{\rvec x} = \begin{bmatrix}
            2 & 2 & 1 \\
            1 & 3 & 1 \\
            1 & 2 & 2 \\
          \end{bmatrix} \rvec x
        $$
\end{enumerate}

\end{document}