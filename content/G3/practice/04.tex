\documentclass{szb-practice}

\title{Felületek, felületi integrál}
\area{Vektoranalízis}
\subject{Matematika G3}
\subjectCode{BMETE94BG03}
\date{Utoljára frissítve: \today}
\docno{4}

\begin{document}
\allowdisplaybreaks

\maketitle

\vspace{-1em}
\subsection{Elméleti áttekintő}
\vspace{1em}

\begin{definition}[Reguláris felület]
  Legyen $\surfimage \subseteq \Reals^3$. Azt mondjuk, hogy az $\surfimage$
  reguláris felület, ha $\forall \rvec p \in \surfimage$ ponthoz megadható
  olyan $\rvec p$-t tartalmazó $V \subset \Reals^3$ nyílt halmaz és
  $\surfsign : \surfdomain \subseteq \Reals^2 \rightarrow \surfimage \cap V$
  leképezés, melyre teljesülnek az alábbiak:
  \vspace{-.33em}
  \begin{itemize}[itemsep=-.33em]
    \item $\surfsign$ differenciálható homeomorfizmus,

    \item $\surfsign$ immerzió (derivált leképezése injektív).
  \end{itemize}
  \vspace{-.33em}
  Ha ezek teljesülnek, akkor $\surfsign$-t parametrációnak,
  $V \cap \surfimage$-t koordinátakörnyezetnek nevezzük.
\end{definition}

\begin{definition}[Elemi felület]
  A $\surfsign: \surfdomain \subseteq \Reals^2 \rightarrow \surfimage \subseteq
    \Reals^3$ elemi felület, ha $\surfsign$ legalább egyszer differenciálható és
  injektív.
\end{definition}

\begin{definition}[Felszín]
  Legyen $\surfsign: \surfdomain \subseteq \Reals^2 \rightarrow \surfimage
    \subseteq \Reals^3$ elemi felület. Ekkor a $\surfimage$ felület felszíne:
  $$
    A = \iint_{\surfdomain}
    \norma{
      \pdv{\surfsign}{s}
      \times
      \pdv{\surfsign}{t}
    }
    \dd s \dd t
    \text.
  $$
\end{definition}

\begin{definition}[Skalármező skalárérékű felületmenti integrálja]
  Legyen $\varphi(\coordvec): \Reals^3 \to \Reals$ skalármező,
  $\surfsign: \surfdomain \rightarrow \surfimage$ paraméterezett felület, ahol
  $s; t \in \surfdomain$ a felület paraméterezése,
  $\surfsign(\surfdomain) = \surfimage$ a felület képe,
  $\dd \surfscalar = \norma{\partial_s \surfsign \times \partial_t \surfsign} \dd s \dd t$,
  Ekkor a $\varphi$ skalármező $\surfimage$ felület menti integrálja:
  $$
    \iint_{\surfimage} \varphi(\coordvec) \dd \surfscalar =
    \iint_{\surfdomain}
    \varphi(\surfsign(s; t))
    \norma{
      \pdv{\surfsign}{s}
      \times
      \pdv{\surfsign}{t}
    }
    \dd s \dd t
    \text.
  $$
  Amennyiben a fekület $z = \Phi(x; y)$ alakban van megadva, akkor:
  $$
    \iint_{\surfimage} \varphi(\coordvec) \dd \surfvec =
    \iint_{\surfdomain}
    \varphi(x; y; \Phi(x; y))
    \sqrt{1 + (\partial_x \varPhi)^2 + (\partial_y \varPhi)^2}
    \dd x \dd y
    \text.
  $$
\end{definition}

\begin{definition}[Vektormező skalár- és vektorértékű felületmenti integrálja][nobreak]
  Legyen $\rvec v(\coordvec) : \Reals^3 \to \Reals^3$ vektormező,
  $\surfsign: \surfdomain \rightarrow \surfimage$ paraméterezett felület, ahol
  $s; t \in \surfdomain$ a felület paraméterezése,
  $\surfsign(\surfdomain) = \surfimage$ a felület képe,
  $\dd \surfvec = \uvec n \dd \surfscalar = \partial_s \surfsign \times \partial_t \surfsign \dd s \dd t$,
  $\uvec n = (\partial_s \surfsign \times \partial_t \surfsign) / \norma{\partial_s \surfsign \times \partial_t \surfsign}$.
  Ekkor a $\rvec v$ vektormező $\surfimage$ felület menti\dots
  \begin{itemize}
    \item skalárértékű integrálja:
          $\displaystyle
            \iint_{\surfimage} \scalar{\rvec v(\coordvec)}{\dd \surfvec} =
            \iint_{\surfdomain}
            \scalar{\rvec v(\surfsign(s; t))}{
              \pdv{\surfsign}{s}
              \times
              \pdv{\surfsign}{t}
            }
            \dd s \dd t
            \text,
          $

    \item vektorértékű integrálja:
          $\displaystyle
            \iint_{\surfimage} \rvec v(\coordvec) \times \dd \surfvec =
            \iint_{\surfdomain}
            \rvec v(\surfsign(s; t)) \times
            \left(
            \pdv{\surfsign}{s}
            \times
            \pdv{\surfsign}{t}
            \right)
            \dd s \dd t
            \text.
          $
  \end{itemize}
\end{definition}

\begin{note}
  Vektormező felületmenti integrálját fluxusnak is nevezzük. Például a
  mágneses fluxus a mágneses indukció vektormezőjének felületmenti
  integrálja:
  $$
    \Phi_B = \iint_{\surfimage} \scalar{\rvec B(\coordvec)}{\dd \surfvec}
    \text.
  $$
\end{note}

\begin{blueBox}[\underline{A felületi normális irányítottsága}]
  Egy $\surfsign: (s; t) \in \surfdomain \rightarrow \surfimage$ paraméterezett
  felület\dots
  \begin{itemize}
    \item kifelé mutató normálisa: $\displaystyle
            \rvec n_{\text{ki}}
            = \pdv{\surfsign}{s} \times \pdv{\surfsign}{t}
            \text,
          $

    \item befelé mutató normálisa: $\displaystyle
            \rvec n_{\text{be}}
            = \pdv{\surfsign}{t} \times \pdv{\surfsign}{s}
            \text.
          $
  \end{itemize}
\end{blueBox}

\begin{note}
  Felület befelé és kifelé mutató normálisa \textbf{azonos nagyságú}, de
  \textbf{ellentétes irányú}.
\end{note}

\begin{blueBox}[\underline{Koordináta-transzformációk}][nobreak]
  \def\arraystretch{1.1}
  \begin{tabular}{
    >{\bullet\;}
    m{2.15cm}
    m{3.15cm}
    m{2.25cm}
    m{2.25cm}
    >{\centering\arraybackslash}m{3.25cm}
    }
    \textbf{Polár:}
     & $x = r \cos \varphi$ \newline
    $y = r \sin \varphi$
     & $r \in [0; R]$ \newline
    $\varphi \in [0; 2\pi)$
     & $|\rmat J| = r$
     & \begin{tikzpicture}
             % Coordinate system
             \draw[-to] (-1.2,0) -- (1.5,0) node[anchor=north east] {$x$};
             \draw[-to] (0,-0.5) -- (0,1.4) node[anchor=north west] {$y$};

             % Radius
             \draw[-to, thick, draw=red-base]
             (0,0) -- (125:1.5) coordinate(c)
             node[anchor=north east, midway, inner sep=2pt] {$r$}
             ;

             % Radius
             % \draw [decorate, decoration={brace}, draw=blue-base, thick]
             % (-.75mm,-.75mm) -- ++(c)
             % node[below left, midway] {$r$};

             % Angle helpers
             \coordinate (a) at (1,0);
             \coordinate (b) at (0,0);

             % Angle
             \draw pic[
                   "$\varphi$",
                   draw=yellow-base,
                   angle eccentricity=.5,
                   angle radius=7mm,
                   thick,
                   ->,
                 ] {angle=a--b--c};
           \end{tikzpicture}
    \\[12mm]
      \textbf{Henger:}
     & $x = r \cos \varphi$ \newline
    $y = r \sin \varphi$ \newline
    $z = z$
     & $r \in [0; R]$ \newline
    $\varphi \in [0; 2\pi]$ \newline
    $z \in \mathbb R$
     & $|\rmat J| = r$
     & \tdplotsetmaincoords{70}{110}
      \begin{tikzpicture}[
          % 3d view={110}{20},
          baseline,
          tdplot_main_coords,
        ]
        % Origin coordinate
        \coordinate (O) at (0,0,0);

        % Coordinate system
        \draw[-to] (O) -- ++(1.75,0,0) node[anchor=south] {$x$};
        \draw[-to] (O) -- ++(0,1.75,0) node[anchor=north east] {$y$};
        \draw[-to] (O) -- ++(0,0,1.75) node[anchor=north west] {$z$};

        % Helper coordinates
        \coordinate (T) at (0.8,1.2,0);

        % r and z
        \draw[-to, thick, draw=blue-base]
        (T) -- ++(0,0,1)
        node[anchor=north west] {$z$};
        \draw[-to, thick, draw=red-base]
        (O) -- (T)
        node[anchor=south west, pos=.4] {$r$};

        \tdplotdrawarc[yellow-base, thick, -to]
        {(O)}{1.2}{0}{56}
        {anchor=south east,font=\scriptsize, inner sep=2pt, black}{$\varphi$};
      \end{tikzpicture}
    \\[12mm]
      \textbf{Gömb:}
     & $x = r \sin \varphi \cos \vartheta $ \newline
    $y = r \sin \varphi \sin \vartheta $ \newline
    $z = r \cos \varphi$
     & $r \in [0; R]$ \newline
    $\varphi \in [0; \pi]$ \newline
    $\vartheta \in [0; 2\pi]$
     & $|\rmat J| = r^2 \sin \varphi$
     & \tdplotsetmaincoords{55}{110}
    \begin{tikzpicture}[
        % 3d view={130}{35.26},
        baseline,
        tdplot_main_coords,
      ]
      % Origin coordinate
      \coordinate (O) at (0,0,0);
      \def\s{1.25}

      % Coordinate system
      \draw[-to] (O) -- ++(1.75,0,0) node[anchor=south] {$x$};
      \draw[-to] (O) -- ++(0,1.75,0) node[anchor=south] {$y$};
      \draw[-to] (O) -- ++(0,0,1.75) node[anchor=north west] {$z$};

      % Helper square
      \draw[gray,dashed]
      (\s,\s,0) coordinate(A) --
      (0,\s,0) coordinate(B) --
      (0,\s,\s) coordinate(C) --
      (\s,\s,\s) coordinate(c) --
      cycle
      ;

      % Helper lines
      \draw[gray]
      (O) -- (C)
      (O) -- (A)
      ;

      % r
      \draw[thick,draw=red-base,-to]
      (O) -- ++(\s,\s,\s)
      node[pos=.5, anchor=south west, font=\scriptsize, inner sep=2pt]{$r$};

      % phi
      \tdplotgetpolarcoords{0.001}{1}{1}
      \tdplotsetthetaplanecoords{\tdplotresphi}

      \tdplotdrawarc[tdplot_rotated_coords, blue-base, thick, -to]
      {(O)}{1}{0}{45}
      {anchor=north east,font=\scriptsize, inner sep=2pt, black}{$\varphi$}

      % theta
      \tdplotdrawarc[yellow-base, thick, -to]
      {(O)}{1}{0}{45}
      {anchor=south,font=\scriptsize, inner sep=2pt, black}{$\vartheta$};
    \end{tikzpicture}
    \\
  \end{tabular}
\end{blueBox}

\begin{blueBox}[\underline{Felületek paraméterezése}][nobreak]
  \def\tskip{14mm}
  \begin{tabular}{
    >{\hspace{-.5em}\bullet\;}p{3.10cm}
    p{5.05cm}
    m{2.1cm}
    >{\centering\arraybackslash}m{3.25cm}
    }
    \textbf{Körlap:} \newline \phantom{1} ($xy$ sík)
     & $\surfsign (s;t) = \begin{bmatrix} s \cos t \\ s \sin t \\ 0 \end{bmatrix}$
     & $s \in [0;r]$ \newline $t \in [0, 2\pi]$
     & \begin{tikzpicture}[
           3d view={110}{20},
           baseline,
         ]
         % Origin coordinate
         \coordinate (O) at (0,0,0);

         % Circle
         \draw[fill=red-base!50!white, fill opacity=.75] (O) circle (1);

         % Coordinate system
         \draw[-to] (O) -- ++(1.75,0,0) node[anchor=west] {$x$};
         \draw[-to] (O) -- ++(0,1.75,0) node[anchor=south east] {$y$};
         \draw[-to] (O) -- ++(0,0,1.25) node[anchor=north east] {$z$};

         % Radius
         \draw[-to, thick, draw=blue-base]
         (O) -- (0.6*1.75,0.8*1.75,0) -- (0.6,0.8,0)
         node[midway, anchor=north east, inner sep=.5mm, font=\scriptsize] {$r$};
       \end{tikzpicture}
    \\[\tskip]
    \textbf{Ellipszislap:} \newline \phantom{1} ($xy$ sík)
     & $\surfsign (s;t) = \begin{bmatrix} a \, s \cos t \\ b \, s \sin t \\ 0 \end{bmatrix}$
     & $s \in [0;1]$ \newline $t \in [0, 2\pi]$
     & \begin{tikzpicture}[
           3d view={110}{20},
           baseline,
         ]
         % Origin coordinate
         \coordinate (O) at (0,0,0);

         % Ellipsis
         \draw[fill=red-base!50!white, fill opacity=.75] (O) ellipse (1.4 and .8);

         % Coordinate system
         \draw[-to] (O) -- ++(2.20,0,0) node[anchor=west] {$x$};
         \draw[-to] (O) -- ++(0,1.50,0) node[anchor=south east] {$y$};
         \draw[-to] (O) -- ++(0,0,1.25) node[anchor=north east] {$z$};

         % Half axes
         \begin{scope}[font=\scriptsize]
        \node at (0.7,0,0) [anchor=south east, inner sep=.5mm] {$a$};
        \node at (0,0.4,0) [anchor=south, inner sep=.5mm] {$b$};
      \end{scope}

         \draw[to-to, thick, draw=blue-base] (O) -- (1.4,0,0);
         \draw[to-to, thick, draw=blue-base] (O) -- (0,0.8,0);
       \end{tikzpicture}
    \\[\tskip]
    \textbf{Hengerfelület:}
     & $\surfsign (s;t) = \rvec r_0(s) + t \rvec n$
     & $s \in \mathcal D_{\rvec r_0}$ \newline $t \in [0, T]$
     & \begin{tikzpicture}[
           baseline,
         ]
         % Random potato coordinates
         \coordinate (A) at (0,-.15);
         \coordinate (B) at (.6,-.2);
         \coordinate (C) at (.5,.5);
         \coordinate (D) at (0,.35);
         \coordinate (E) at (-.66,.45);
         \coordinate (F) at (-.75,-.35);

         % + is in the foreground, - is in the background
         \foreach \c in {A,B,C,D,E,F} {
             \coordinate (\c-) at ($(\c) + (.25,.25)$);
             \coordinate (\c+) at ($(\c) - (.25,.25)$);

             % Connect + and - with a line
             \draw[opacity=0.25,red-base] (\c-) -- (\c+) coordinate[pos=.25] (\c75);
           }


         % Connect the dots
         \draw[smooth cycle, thick, red-base] plot coordinates {
             (A+) (B+) (C+) (D+) (E+) (F+)
           };
         \draw[smooth cycle, thick, yellow-base, opacity=.25] plot[xshift=1cm] coordinates {
             (A-) (B-) (C-) (D-) (E-) (F-)
           };

         % Draw the normal vector
         \draw[-to, draw=blue-base, thick] (E+) -- (E75) node[below left, xshift=-3mm] {\scriptsize$\rvec n$};

         % Label the curve
         \node[above left=-1mm] at (A+) {\scriptsize$\rvec r_0(s)$};
       \end{tikzpicture}
    \\[\tskip]
    \textbf{Forgásfelület:} \newline \phantom{i} ($z$ tengely körül) \newline \phantom{i} ($z = f(x)$)
     & $\surfsign (s;t) = \begin{bmatrix} s \cos t \\ s \sin t \\ f(s) \end{bmatrix}$
     & $s \in [0;2\pi]$ \newline $t \in \mathcal D_f$
     & \begin{tikzpicture}[
           3d view={110}{20},
           baseline,
           scale=.85,
         ]
         % Origin coordinate
         \coordinate (O) at (0,0,0);

         % Helper coordinates
         \foreach \a in {0,30,...,330} {
             \foreach \loc in {0,2,...,10} {
                 \coordinate (\a-\loc) at
                 ({\loc*cos(\a)/10},{\loc*sin(\a)/10},1.4*\loc*\loc/100 - 0.6*\loc*\loc*\loc*\loc/10000);
               }
           }

         % Draw surface in bg
         \foreach \x\y in {150/180,180/210,210/240,240/270,270/300,300/330} {
             \foreach \i/\j in {0/2,2/4,4/6,6/8,8/10} {
                 \fill[thick, draw=red-base, fill=red-base!20, rounded corners=.1pt] (\x-\i) -- (\y-\i) -- (\y-\j) -- (\x-\j) -- cycle;
               }
           }

         % Coordinate system
         \draw[-to] (O) -- ++(1.75,0,0) node[anchor=south east] {$x$};
         \draw[-to] (O) -- ++(0,1.75,0) node[anchor=south east] {$y$};
         \draw[-to] (O) -- ++(0,0,1.75) node[anchor=north east] {$z$};

         % Draw surface in fg
         \foreach \x/\y in {120/150,90/120,330/0,0/30,30/60,60/90} {
             \foreach \i/\j in {0/2,2/4,4/6,6/8,8/10} {
                 \fill[thick, draw=red-base, fill=red-base!20, rounded corners=.1pt] (\x-\i) -- (\y-\i) -- (\y-\j) -- (\x-\j) -- cycle;
               }
           }
       \end{tikzpicture}
    \\[\tskip]
    \textbf{Gömbfelület:}
     & $\surfsign (s;t) = \begin{bmatrix} R \sin s \cos t \\ R \sin s \sin t \\ R \cos s \end{bmatrix}$
     & $s \in [0;\pi]$ \newline $t \in [0, 2\pi]$
     & \begin{tikzpicture}[
           3d view={135}{35.26},
           % 3d view={110}{20},
           baseline,
           scale=.75,
         ]
         % Origin coordinate
         \coordinate (O) at (0,0,0);

         % Helper coordinates
         \foreach \s in {0,30,...,360} {
             \foreach \t in {0,30,...,330} {
                 \coordinate (\s-\t) at
                 (
                 {sin(\s/2)*cos(\t)},
                 {sin(\s/2)*sin(\t)},
                 {cos(\s/2)}
                 );
               }
           }

         % Draw surface in bg
         \foreach \x/\y in {330/360,300/330,270/300,240/270,210/240,180/210,150/180,120/150,90/120,60/90,30/60,0/30} {
             \foreach \i/\j in {150/180,180/210,210/240,240/270,270/300,300/330,330/0,120/150,90/120}{
                 \fill[thick, draw=red-base, fill=red-base!20, rounded corners=.1pt] (\x-\i) -- (\y-\i) -- (\y-\j) -- (\x-\j) -- cycle;
               }
           }

         % Coordinate system
         \draw[-to] (O) -- ++(1.75,0,0) node[anchor=south east] {$x$};
         \draw[-to] (O) -- ++(0,1.75,0) node[anchor=south west] {$y$};
         \draw[-to] (O) -- ++(0,0,1.75) node[anchor=north east] {$z$};

         % Draw surface in fg
         \foreach \x/\y in {330/360,300/330,270/300,240/270,210/240,180/210,150/180,120/150,90/120,60/90,30/60,0/30} {
             \foreach \i/\j in {0/30,30/60,60/90} {
                 \fill[thick, draw=red-base, fill=red-base!20, rounded corners=.1pt] (\x-\i) -- (\y-\i) -- (\y-\j) -- (\x-\j) -- cycle;
               }
           }
       \end{tikzpicture}
    \\[\tskip]
    \textbf{Ellipszoid:}
     & $\surfsign (s;t) = \begin{bmatrix} a \sin s \cos t \\ b \sin s \sin t \\ c \cos s \end{bmatrix}$
     & $s \in [0;\pi]$ \newline $t \in [0, 2\pi]$
     & \begin{tikzpicture}[
           3d view={110}{20},
           baseline,
           scale=.85
         ]
         % Origin coordinate
         \coordinate (O) at (0,0,0);

         % Helper coordinates
         \foreach \s in {0,30,...,360} {
             \foreach \t in {0,30,...,330} {
                 \coordinate (\s-\t) at
                 (
                 {0.8*sin(\s/2)*cos(\t)},
                 {1.4*sin(\s/2)*sin(\t)},
                 {0.6*cos(\s/2)}
                 );
               }
           }

         % Draw surface in bg
         \foreach \x/\y in {330/360,300/330,270/300,240/270,210/240,180/210,150/180,120/150,90/120,60/90,30/60,0/30} {
             \foreach \i/\j in {150/180,180/210,210/240,240/270,270/300,300/330,330/0,120/150,90/120}{
                 \fill[thick, draw=red-base, fill=red-base!20, rounded corners=.1pt] (\x-\i) -- (\y-\i) -- (\y-\j) -- (\x-\j) -- cycle;
               }
           }

         % Coordinate system
         \draw[-to] (O) -- ++(2.25,0,0) node[anchor=east] {$x$};
         \draw[-to] (O) -- ++(0,2.00,0) node[anchor=south] {$y$};
         \draw[-to] (O) -- ++(0,0,1.25) node[anchor=north east] {$z$};

         % Draw surface in fg
         \foreach \x/\y in {330/360,300/330,270/300,240/270,210/240,180/210,150/180,120/150,90/120,60/90,30/60,0/30} {
             \foreach \i/\j in {0/30,30/60,60/90} {
                 \fill[thick, draw=red-base, fill=red-base!20, rounded corners=.1pt] (\x-\i) -- (\y-\i) -- (\y-\j) -- (\x-\j) -- cycle;
               }
           }
       \end{tikzpicture}
    \\[\tskip]
    \textbf{Тórusz:}
     & $\surfsign (s;t) = \begin{bmatrix} (R + r \cos s) \cos t \\ (R + r \cos s) \sin t \\ r \sin s \end{bmatrix}$
     & $s \in [0;2\pi]$ \newline $t \in [0, 2\pi]$
     & \begin{tikzpicture}
         \begin{axis}[
          axis equal image,
          hide axis,
          % z buffer = sort,
          view={110}{20},
          xmax=2.5,
          ymax=2.5,
          zmax=2.25,
          scale=.66,
        ]

        \draw[-to] (axis cs:0,0,0) -- (axis cs:1.25,0,0);
        \draw[-to] (axis cs:0,0,0) -- (axis cs:0,1.25,0);

        \addplot3[
          surf,
          faceted color=red-base,
          fill=red-base!20,
          ultra thin,
          samples=15,
          samples y=48,
          domain=0:2*pi,
          domain y=0:2*pi,
          z buffer=sort,
        ](
        {(1.25+0.25*sin(deg(\x)))*cos(deg(\y))},
        {(1.25+0.25*sin(deg(\x)))*sin(deg(\y))},
        {0.25*cos(deg(\x))}
        );

        \draw[-to] (axis cs:0,0,0) -- (axis cs:0,0,2) node[anchor=north east] {$z$};

        \draw[-to] (axis cs:1.5,0,0) -- (axis cs:2.25,0,0) node[anchor=north] {$x$};
        \draw[-to] (axis cs:0,1.5,0) -- (axis cs:0,1.75,0) node[anchor=north] {$y$};

        \coordinate (C) at (axis cs:0,1.25,1.25);
        \coordinate (T) at (axis cs:0,1.25,1.65);
        \coordinate (T+) at (axis cs:0,1.25,1.75);
        \coordinate (Z) at (axis cs:0,0,1.65);
      \end{axis}

         \begin{scope}[3d view={110}{20},canvas is yz plane at x=0]
        \draw[red-base, fill=red-base!20] (C) circle (0.33);
        \draw[red-base, fill=red-base, ultra thick] (C) circle (0.02);

        \draw[gray] (C) -- (T+);
        \draw[to-to, thick, draw=blue-base]
        (Z) -- (T)
        node[midway, anchor=north] {\scriptsize$R$};
        \draw[-to, thick, draw=blue-base]
        (C) -- ++(0.75*0.8,0.75*0.6) -- ($(C)+(0.33*0.8,0.33*0.6)$)
        node[midway, anchor=south, inner sep=.5mm, font=\scriptsize] {$r$};
      \end{scope}
       \end{tikzpicture}
    \\[\tskip]
    \textbf{Kúp:}
     & $\surfsign (s;t) = \begin{bmatrix} s \cos t \\ s \sin t \\ s \end{bmatrix}$
     & $s \in [0;U]$ \newline $t \in [0, 2\pi]$
     & \begin{tikzpicture}[
           3d view={110}{20},
           baseline,
         ]
         % Origin coordinate
         \coordinate (O) at (0,0,0);

         % Helper coordinates
         \foreach \a in {0,15,...,345} {
             \foreach \loc in {0,10} {
                 \coordinate (\a-\loc) at
                 ({\loc*cos(\a)/10},{\loc*sin(\a)/10},1.4*\loc*\loc/100 - 0.6*\loc*\loc*\loc*\loc/10000);
               }
           }

         % Draw surface in bg
         \foreach \x\y in {120/135,135/150,150/165,165/180,180/195,195/210,210/225,225/240,240/255,255/270,270/285,285/300,300/315,315/330} {
             \foreach \i/\j in {0/10} {
                 \fill[thick, draw=red-base, fill=red-base!20, rounded corners=.1pt] (\x-\i) -- (\y-\i) -- (\y-\j) -- (\x-\j) -- cycle;
               }
           }

         % Coordinate system
         \draw[-to] (O) -- ++(1.25,0,0) node[anchor=south east] {$x$};
         \draw[-to] (O) -- ++(0,1.25,0) node[anchor=south east] {$y$};
         \draw[-to] (O) -- ++(0,0,1.75) node[anchor=north east] {$z$};

         % Draw surface in fg
         \foreach \x/\y in {330/345,345/0,0/15,15/30,75/90,90/105,105/120,45/60,30/45,60/75} {
             \foreach \i/\j in {0/10} {
                 \fill[thick, draw=red-base, fill=red-base!20, rounded corners=.1pt] (\x-\i) -- (\y-\i) -- (\y-\j) -- (\x-\j) -- cycle;
               }
           }
       \end{tikzpicture}
    \\[\tskip]
    \textbf{Möbius-szalag:}
     & $\surfsign(s;t) = \scalebox{.95}{$\begin{bmatrix} (R + s \cos \sfrac{t}{2}) \cos t \\ (R + s \cos \sfrac{t}{2}) \sin t \\ s \, \sin \sfrac{t}{2} \end{bmatrix}$}$
     & $s \in [-S;S]$ \newline $t \in [0, 2\pi]$
     & \begin{tikzpicture}
         \begin{axis}[
          axis equal image,
          hide axis,
          % z buffer = sort,
          view={110}{20},
          xmax=2.5,
          ymax=2.5,
          zmax=2.25,
          scale=.66,
        ]

        \draw[] (axis cs:0,0,0) -- (axis cs:1.75,0,0);
        \draw[] (axis cs:0,0,0) -- (axis cs:0,1.75,0);

        \addplot3[
          surf,
          faceted color=red-base,
          fill=red-base!20,
          ultra thin,
          samples=15,
          samples y=48,
          domain=-.5:.5,
          domain y=0:2*pi,
          z buffer=sort,
        ](
        {(1.25 + x*cos(deg(y)/2))*cos(deg(y))-1},
        {(1.25 + x*cos(deg(y)/2))*sin(deg(y))},
        {x*sin(deg(y)/2)}
        );

        \draw[-to] (axis cs:0,0,0) -- (axis cs:0,0,2) node[anchor=north east] {$z$};
        \draw[-to] (axis cs:1.5,0,0) -- (axis cs:2.25,0,0) node[anchor=north] {$x$};
        \draw[-to] (axis cs:0,1.5,0) -- (axis cs:0,1.75,0) node[anchor=north] {$y$};
      \end{axis}
       \end{tikzpicture}
  \end{tabular}
\end{blueBox}

\clearpage
\subsection{Feladatok}

\begin{enumerate}
  \item Számítsuk ki a megadott felületek felszínét!
        \begin{enumerate}
          \item $z = x^2 + y^2$ forgásparaboloid $z = 1$ és $z = 4$ síkok közé
                eső része,

          \item $\surfsign(s;t) = \ijk{e^s \cos t}{e^s \sin t}{s}$,
                $s \in (-\infty; 0]$, $t \in [0; 2\pi]$.
        \end{enumerate}

  \item Integrálja a skalármezőket a megadott felületeken!
        \begin{enumerate}
          \item $\varphi(\coordvec) = x^2 + y^2$, az egységgömb $z > 0$ részén,

          \item $\psi(\coordvec) = x + y + z, $ az $x + 2y + 4z = 4$ sík
                első térnyolcadba tartozó részén.
        \end{enumerate}

  \item Integrálja a vektormezőket a megadott felületeken!
        A normális kifelé mutató legyen!
        \begin{itemize}
          \item $\rvec u(\coordvec) = \ijk{x + y}{x - y}{z^2}$,\\
                $\surfsign (s;t) = \ijk{s + t}{s - t}{s^2 - t^2}$,
                $(s;t) \in [0;1]^2$,

          \item $\rvec v(\coordvec) = \ijk{y^2 + z^2}{x^2 + z^2}{x^2 + y^2}$,\\
                $r = 2$ sugarú, $x = 2$ síkon lévő körön.
        \end{itemize}

  \item Adja meg a $\rvec v(\coordvec) = \ijk{x}{-y}{z}$ vektormező
        $\surfsign(s; t)= \ijk{s + 2t}{t}{s - t}$, $s \in [0;3]$, $t \in [0;1]$
        felületen vett vektorértékű integrálját! A normális irányítottsága
        legyen kifelé mutató!
\end{enumerate}

\end{document}