\documentclass[]{szb-practice}

\title{Elsőrendű lineáris és másodrendű hiányos DE}
\area{Differenciálegyenletek}
\subject{Matematika G3}
\subjectCode{BMETE94BG03}
\date{Utoljára frissítve: \today}
\docno{10}

\usepackage{siunitx}
\sisetup{locale = DE}

\begin{document}
\maketitle

\subsection{Elméleti áttekintő}

\begin{blueBox}[\underline{Elsőrendű, lineáris differenciálegyenlet}]
  $$
    y' + p(x) y = q(x)
  $$
  Ha $q(x) \equiv 0 $, akkor az egyenlet homogén, egyébként inhomogén.

  Az általános megoldás a homogén és inhomogén megoldás
  összegeként adódik:
  \begin{itemize}
    \item A homogén megoldást meghaphatjuk, ha megoldjuk az
          $y' + p(x)y = 0$ egyenletet a tanult módszerek alapján.

    \item Az inhomogén megoldást konstans variálással
          kaphatjuk meg. Ez azt jelenti, hogy a homogén megoldásban
          lévő konstanst $x$-től függőnek vesszük, majd megoldjuk
          az egyenletet.
  \end{itemize}
\end{blueBox}

\begin{blueBox}[\underline{Hiányos másodrendű egyenletek}]
  Az $y'' = f(x; y; y')$ hiányos, ha $x$, $y$, vagy $y'$
  hiányzik az egyenletből.
  \begin{enumerate}
    \item $y'' = f(x)$
          \tabto{2.65cm} $\rightarrow$ \tabto{3.65cm}
          $y$ és $y'$ hiányos, kétszer integrálunk

    \item $y'' = f(x; y')$
          \tabto{2.65cm} $\rightarrow$ \tabto{3.65cm}
          $p(x) = y'(x)$ helyettesítés,

          \tabto{3.65cm}
          így $p$-re elsőrendű

    \item $y'' = f(y; y')$
          \tabto{2.65cm} $\rightarrow$ \tabto{3.65cm}
          $P(y) = y'$ helyettesítés,

          \tabto{3.65cm}
          ekkor $y'' = \partial_y P \cdot y' = \partial_y P \cdot P$

          \tabto{3.65cm}
          így $P$-re elsőrendű, szeparábilis
  \end{enumerate}
\end{blueBox}

\clearpage
\subsection{Feladatok}

\begin{enumerate}
  \item Oldja meg a következő elsőrendű differenciálegyenleteket!
        \begin{align*}
          a) \quad y' & - \frac{y}{x} = x \, e^x
          \text,                                      \\
          b) \quad y' & + 2xy + x \, e^{-x^2} = 0
          \text,                                      \\
          c) \quad y' & + \frac{1 - 2x}{x^2} \, y = 1
          \text,                                      \\
          d) \quad y' & + y \cos x = \cos x \sin x
          \text,                                      \\[2mm]
          e) \quad y' & (1 + x^2) + 2xy = \tan x
          \text.
        \end{align*}

  \item $\SI{10}{\liter}$ vizet tartalmazó edénybe literenként
        $\SI{0,3}{\kilogram}$ sót tartalmazó oldat folyik be
        $\SI[per-mode = symbol]{2}{\liter\per\minute}$ sebességgel. Az edényben
        a folyadék azonnal elkeveredik, majd ugyanilyen sebességgel kifolyik.
        Mennyi só lesz $\SI{5}{\minute}$ múlva az edényben?

  \item Egy testet függőlegesen hajítunk lefelé $v_0$ kezdeti sebességgel.
        Határozza meg a mozgás sebességének változását, amennyiben a testre csak
        a nehézségi erő hat, valamint a levegő fékezőerelye a sebességgel
        egyenesen arányos!

  \item Oldja meg a következő másodrendű, hiányos differenciálegyenleteket!
        \begin{align*}
          a) \quad y'' & = 6x + \sin x
          \text,                                                     \\[2mm]
          b) \quad y'' & x^2 + \ln x = 1
          \text,                                                     \\[2mm]
          c) \quad y'' & - \frac{x}{x^2 - 1} \, y' = 0 \quad (x > 1)
          \text,                                                     \\[2mm]
          d) \quad y'' & (1 + y^2) = y \, y'^2
          \text,                                                     \\[2mm]
          e) \quad y'' & {}^2 - y' = 0
          \text,                                                     \\[2mm]
          e) \quad y'' & y = 1 + y'^2
          \text.                                                     \\[2mm]
        \end{align*}
\end{enumerate}

\end{document}