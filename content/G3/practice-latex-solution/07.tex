\documentclass[fleqn]{szb-practice}

\title{Összefoglalás}
\area{Többváltozós analízis}
\subject{Matematika G3}
\subjectCode{BMETE94BG03}
\date{Utoljára frissítve: \today}
\docno{7}

\begin{document}
\maketitle

% ~~~~~~~~~~~~~~~~~~~~~~~~~~~~~~~~~~~~~~~~~~~~~~~~~~~~~~~~~~~~~~~~~~~~~~~~~~~~~~
% 111111111111111111111111111111111111111111111111111111111111111111111111111111
% ~~~~~~~~~~~~~~~~~~~~~~~~~~~~~~~~~~~~~~~~~~~~~~~~~~~~~~~~~~~~~~~~~~~~~~~~~~~~~~
\subsection{Gradiens}

Adja meg a $\varphi(\coordvec) = {2 x^2 y} + {x y^2 z} + {3 x z^2}$ skalármező
gradiensét a $P(-3; -2; 1)$ pontban!

A gradiens paraméteresen:
\begin{equation*}
  \grad \varphi = \begin{bmatrix}
    \partial_x \varphi \\
    \partial_y \varphi \\
    \partial_z \varphi
  \end{bmatrix} = \begin{bmatrix}
    4 x y + y^2 z + 3z^2 \\
    2 x^2 + 2 x y z      \\
    x y^2 + 6 x z
  \end{bmatrix}
  \text.
\end{equation*}

A $P(-3; -2; 1)$ pontban:
\begin{equation*}
  \grad \varphi(-3; -2; 1)
  = \begin{bmatrix}
    4 \cdot (-3) \cdot (-2) + (-2)^2 \cdot 1 + 3 \cdot 1^2 \\
    2 \cdot (-3)^2 + 2 \cdot (-3) \cdot (-2) \cdot 1       \\
    (-3) \cdot (-2)^2 + 6 \cdot (-3) \cdot 1
  \end{bmatrix} = \begin{bmatrix}
    24 + 4 + 3 \\
    18 + 12    \\
    -12 - 18
  \end{bmatrix} = \begin{bmatrix}
    31 \\
    30 \\
    -30
  \end{bmatrix}
  \text.
\end{equation*}

% ~~~~~~~~~~~~~~~~~~~~~~~~~~~~~~~~~~~~~~~~~~~~~~~~~~~~~~~~~~~~~~~~~~~~~~~~~~~~~~
% 222222222222222222222222222222222222222222222222222222222222222222222222222222
% ~~~~~~~~~~~~~~~~~~~~~~~~~~~~~~~~~~~~~~~~~~~~~~~~~~~~~~~~~~~~~~~~~~~~~~~~~~~~~~

\subsection{Divergencia, rotáció}

Adja meg a $\rvec v(\coordvec) = \ijk{x y^2 - z}{y z}{x y + 2 z}$ vektormező
divergenciáját és rotációját a $P(-1; 2; -1)$ pontban!

A divergencia paraméteresen:
\begin{equation*}
  \Div \rvec v
  = \pdv{v_x}{x} + \pdv{v_y}{y} + \pdv{v_z}{z}
  = y^2 + z + 2
  \text.
\end{equation*}

A $P(-1; 2; -1)$ pontban:
\begin{equation*}
  \Div \rvec v(-1; 2; -1)
  = (2)^2 + (-1) + 2
  = 4 - 1 + 2
  = 5
  \text.
\end{equation*}

A rotáció paraméteresen:
\begin{equation*}
  \rot \rvec v = \begin{bmatrix}
    \partial_x \\ \partial_y \\ \partial_z
  \end{bmatrix} \times \begin{bmatrix}
    v_x \\ v_y \\ v_z
  \end{bmatrix} = \begin{bmatrix}
    \partial_y v_z - \partial_z v_y \\
    \partial_z v_x - \partial_x v_z \\
    \partial_x v_y - \partial_y v_x
  \end{bmatrix} = \begin{bmatrix}
    x - y  \\
    -1 - y \\
    -2xy
  \end{bmatrix}
  \text.
\end{equation*}

A $P(-1; 2; -1)$ pontban:
\begin{equation*}
  \rot \rvec v(-1; 2; -1)
  = \begin{bmatrix}
    (-1) - 2 \\
    -1 - 2   \\
    -2 \cdot (-1) \cdot 2
  \end{bmatrix} = \begin{bmatrix}
    -3 \\
    -3 \\
    4
  \end{bmatrix}
  \text.
\end{equation*}

% ~~~~~~~~~~~~~~~~~~~~~~~~~~~~~~~~~~~~~~~~~~~~~~~~~~~~~~~~~~~~~~~~~~~~~~~~~~~~~~
% 333333333333333333333333333333333333333333333333333333333333333333333333333333
% ~~~~~~~~~~~~~~~~~~~~~~~~~~~~~~~~~~~~~~~~~~~~~~~~~~~~~~~~~~~~~~~~~~~~~~~~~~~~~~

\subsection{Skalárpotenciál}

Adja meg a $\rvec v(\coordvec) = \ijk{y^2}{2xy + e^{3z}}{3ye^{3z}}$ egy olyan
$\varphi$ skalárpotenciálját, melyre $\varphi(\nvec) = 0$.

\vspace{-2em}
\begin{align*}
  \varphi(\coordvec)
   & = \int_0^x v_x(\xi; y; z) \dd \xi
  + \int_0^y v_y(0; \eta; z) \dd \eta
  + \int_0^z v_z(0; 0; \zeta) \dd \zeta \\
   & = \int_0^x y^2 \dd \xi
  + \int_0^y e^{3z} \dd \eta
  + \int_0^z 0 \dd \zeta
  = x y^2 + y e^{3z} \text.
\end{align*}

% ~~~~~~~~~~~~~~~~~~~~~~~~~~~~~~~~~~~~~~~~~~~~~~~~~~~~~~~~~~~~~~~~~~~~~~~~~~~~~~
% 444444444444444444444444444444444444444444444444444444444444444444444444444444
% ~~~~~~~~~~~~~~~~~~~~~~~~~~~~~~~~~~~~~~~~~~~~~~~~~~~~~~~~~~~~~~~~~~~~~~~~~~~~~~

\subsection{Skalármező vonalintegrálja}

Legyen $\curvesign(t) = x(t) \cdot \hat{\uvec i} + y(t) \cdot \hat{\uvec j}$ a
$(2; 1) \to (6; 4)$ egyenes szakasz paraméterezése. Adja meg $x(t)$ és $y(t)$
függvényeket, ha a paramétertartomány $t \in [0; 1]$. Számítsa ki a
$\varphi(x; y) = 3x - 4y$ skalármező $\curvesign$ görbe menti integrálját!

A paraméterezés:
\begin{equation*}
  \curvesign(t) = \begin{bmatrix}
    x(t) \\
    y(t)
  \end{bmatrix} = \begin{bmatrix}
    2 + 4t \\
    1 + 3t
  \end{bmatrix}
\end{equation*}

A sebességvektor, és ennek normája:
\begin{equation*}
  \dot{\curvesign}(t) = \begin{bmatrix}
    4 \\
    3
  \end{bmatrix}
  \text,
  \qquad
  \|\dot{\curvesign}(t)\| = \sqrt{4^2 + 3^2} = 5
  \text.
\end{equation*}

A skalármező átparaméterezve:
\begin{equation*}
  \varphi(\curvesign(t))
  = 3 (2 + 4t) - 4 (1 + 3t)
  = 6 + 12t - 4 - 12t
  = 2
  \text.
\end{equation*}

A vonalintegrál:
\begin{equation*}
  \int_{\curveimage} \varphi(\coordvec) \dd \curvescalar
  = \int_0^1 \varphi(\curvesign(t)) \|\dot{\curvesign}(t)\| \dd t
  = \int_0^1 2 \cdot 5 \dd t
  = 10
  \text.
\end{equation*}

% ~~~~~~~~~~~~~~~~~~~~~~~~~~~~~~~~~~~~~~~~~~~~~~~~~~~~~~~~~~~~~~~~~~~~~~~~~~~~~~
% 555555555555555555555555555555555555555555555555555555555555555555555555555555
% ~~~~~~~~~~~~~~~~~~~~~~~~~~~~~~~~~~~~~~~~~~~~~~~~~~~~~~~~~~~~~~~~~~~~~~~~~~~~~~

\subsection{Vektormező vonalintegrálja}

Adott az $\rvec F(x; y) = x^2 \cdot \hat{\uvec i} - xy \cdot \hat{\uvec j}$
erőmező. Számítsa ki az erőmező munkáját, az origó középpontú, $r = 1$ sugarú
első síknegyedben lévő körív mentén, ha a bejárás az óramutató járásával
ellentétes irányú! Mit mondhatunk el, ha a bejárási irányt megfordítjuk?

A kör paraméterezése:
\begin{equation*}
  \curvesign(t) = \begin{bmatrix}
    \cos t \\
    \sin t
  \end{bmatrix}
  \text,
  \qquad
  t \in [0; \pi/2]
  \text.
\end{equation*}

A sebességvektor:
\begin{equation*}
  \dot{\curvesign}(t) = \begin{bmatrix}
    -\sin t \\
    \cos t
  \end{bmatrix}
  \text.
\end{equation*}

A vektormező átparaméterezve:
\begin{equation*}
  \rvec F(\curvesign(t))
  = \begin{bmatrix}
    \cos^2 t \\
    -\cos t \sin t
  \end{bmatrix}
  \text.
\end{equation*}

A vonalintegrál:
\begin{align*}
  \int_{\curveimage} \rvec F(\coordvec) \cdot \dd \curvevec
   & = \int_0^{\pi/2} \rvec F(\curvesign(t)) \cdot \dot{\curvesign}(t) \dd t
  = \int_0^{\pi/2} \begin{bmatrix}
                     \cos^2 t \\
                     -\cos t \sin t
                   \end{bmatrix} \cdot \begin{bmatrix}
                                         -\sin t \\
                                         \cos t
                                       \end{bmatrix} \dd t                    \\
   & = \int_0^{\pi/2} \left( -\cos^2 t \sin t - \cos^2 t \sin t \right) \dd t
  = -2 \int_0^{\pi/2} \cos^2 t \sin t \dd t                                   \\
   & = -2 \left[ -\frac{\cos^3 t}{3} \right]_0^{\pi/2}
  = -2 \left( 0 - \left(-\frac{1}{3}\right)\right)
  = -\frac{2}{3}
  \text.
\end{align*}

Ha a bejárási irányt megfordítjuk, akkor az integrál előjele is megfordul,
vagyis az eredmény $2/3$ lesz.

% ~~~~~~~~~~~~~~~~~~~~~~~~~~~~~~~~~~~~~~~~~~~~~~~~~~~~~~~~~~~~~~~~~~~~~~~~~~~~~~
% 666666666666666666666666666666666666666666666666666666666666666666666666666666
% ~~~~~~~~~~~~~~~~~~~~~~~~~~~~~~~~~~~~~~~~~~~~~~~~~~~~~~~~~~~~~~~~~~~~~~~~~~~~~~

\subsection{Skalármező felületi integrálja}

Számítsa ki a $\varphi(x; y; z) = 2x$ skalármező egységgömbömbön vett felületi
integrálját! Segítség:
$\surfsign(s; t) = \ijk{\sin s \cos t}{\sin s \sin t}{\cos s}$,
$\dd \surfscalar = \sin s \dd s \dd t$.

A felület paraméterezése:
\begin{equation*}
  \surfsign(s; t) = \begin{bmatrix}
    \sin s \cos t \\
    \sin s \sin t \\
    \cos s
  \end{bmatrix}
  \text,
  \qquad
  \begin{array}{l}
    s \in [0; \pi] \\
    t \in [0; 2\pi]
  \end{array}
  \text.
\end{equation*}

A skalármező átparaméterezve:
\begin{equation*}
  \varphi(\surfsign(s; t))
  = 2 \sin s \cos t
  \text.
\end{equation*}

A felületi integrál:
\begin{align*}
  \iint_{\surfimage} \varphi(\coordvec) \dd \surfscalar
   & = \int_0^{2\pi} \int_0^{\pi} \varphi(\surfsign(s; t)) \sin s \dd s \dd t
  = \int_0^{2\pi} \int_0^{\pi} 2 \sin s \cos t \cdot \sin s \dd s \dd t       \\
   & = 2 \int_0^{2\pi} \cos t \dd t \int_0^{\pi} \sin^2 s \dd s
  = 2 \cdot 0 \cdot \frac{\pi}{2}
  = 0
  \text.
\end{align*}

% ~~~~~~~~~~~~~~~~~~~~~~~~~~~~~~~~~~~~~~~~~~~~~~~~~~~~~~~~~~~~~~~~~~~~~~~~~~~~~~
% 777777777777777777777777777777777777777777777777777777777777777777777777777777
% ~~~~~~~~~~~~~~~~~~~~~~~~~~~~~~~~~~~~~~~~~~~~~~~~~~~~~~~~~~~~~~~~~~~~~~~~~~~~~~

\subsection{Vektormező felületi integrálja}

Számítsa ki a $\rvec v$ vektormező $\surfsign$ felületen vett fluxusát, ha
\begin{equation*}
  \rvec v(x; y; z) = \begin{bmatrix}
    x y    \\
    2x + y \\
    z
  \end{bmatrix}
  \text,
  \qquad
  \surfsign(s; t) = \begin{bmatrix}
    s + 2t \\
    -t     \\
    s^2 + 3t
  \end{bmatrix}
  \text,
  \qquad
  \begin{array}{l}
    s \in [0; 3] \\
    t \in [0; 1]
  \end{array}
  \text.
\end{equation*}

A felület normálvektora:
\begin{equation*}
  \rvec n = \pdv{\surfsign}{s} \times \pdv{\surfsign}{t}
  = \begin{bmatrix}
    1 \\
    0 \\
    2s
  \end{bmatrix} \times \begin{bmatrix}
    2  \\
    -1 \\
    3
  \end{bmatrix} = \begin{bmatrix}
    2s     \\
    4s - 3 \\
    -1
  \end{bmatrix}
  \text.
\end{equation*}

A vektormező átparaméterezve:
\begin{equation*}
  \rvec v(\surfsign(s; t))
  = \begin{bmatrix}
    (s + 2t)(-t)     \\
    2(s + 2t) + (-t) \\
    s^2 + 3t
  \end{bmatrix} = \begin{bmatrix}
    -s t - 2t^2 \\
    2s + 3t     \\
    s^2 + 3t
  \end{bmatrix}
  \text.
\end{equation*}

A felületi integrál:
\begin{align*}
  \iint_{\surfimage} \scalar{\rvec v}{\dd \surfvec}
   & = \int_0^1 \int_0^3 \rvec v(\surfsign(s; t)) \cdot \rvec n \dd s \dd t
  = \int_0^1 \int_0^3 \begin{bmatrix}
                        -s t - 2t^2 \\
                        2s + 3t     \\
                        s^2 + 3t
                      \end{bmatrix} \cdot \begin{bmatrix}
                                            2s     \\
                                            4s - 3 \\
                                            -1
                                          \end{bmatrix} \dd s \dd t           \\
   & = \int_0^1 \int_0^3 7s^2 - 6s + 12st - 2s^2 t - 4s t^2 - 12t \dd s \dd t
  = \dots
  = 30
  \text.
\end{align*}

% ~~~~~~~~~~~~~~~~~~~~~~~~~~~~~~~~~~~~~~~~~~~~~~~~~~~~~~~~~~~~~~~~~~~~~~~~~~~~~~
% 888888888888888888888888888888888888888888888888888888888888888888888888888888
% ~~~~~~~~~~~~~~~~~~~~~~~~~~~~~~~~~~~~~~~~~~~~~~~~~~~~~~~~~~~~~~~~~~~~~~~~~~~~~~

\subsection{Gradiens tétel}

Számítsa ki a $\rvec v(\coordvec) = \ijk{y^2}{2xy + e^{3z}}{3ye^{3z}}$
vektormező $(0; 1; 1) \to (0; -1; 1)$ szakaszon vett vonalmenti
integrálját!

A potenciálfüggvény:
\begin{equation*}
  \varphi(\coordvec) = x y^2 + y e^{3z}
  \text.
\end{equation*}

A vonalintegrál:
\begin{equation*}
  \int_{\curveimage} \scalar{\rvec v}{\dd \curvevec}
  = \varphi(0; -1; 1) - \varphi(0; 1; 1)
  = \left( 0 \cdot (-1)^2 + (-1) \cdot e^3 \right)
  - \left( 0 \cdot 1^2 + 1 \cdot e^3 \right)
  = -2 e^3
  \text.
\end{equation*}

% ~~~~~~~~~~~~~~~~~~~~~~~~~~~~~~~~~~~~~~~~~~~~~~~~~~~~~~~~~~~~~~~~~~~~~~~~~~~~~~
% 999999999999999999999999999999999999999999999999999999999999999999999999999999
% ~~~~~~~~~~~~~~~~~~~~~~~~~~~~~~~~~~~~~~~~~~~~~~~~~~~~~~~~~~~~~~~~~~~~~~~~~~~~~~

\subsection{Stokes-tétel}

Adja meg a $\rvec v(\coordvec) = \ijk{z - y}{x - z}{y - x}$ vektormezőt az
alábbi zárt görbén:
\begin{enumerate}
  \item Az origóból először egy egyenes szakasz mentén eljutunk az
        $(1;0;0)$ pontba.

  \item Ezután egy origó középpontú körív mentén az $(-1;0;0)$ pontba
        jutunk. (A körív síkja legyen az $x y$ sík, és a bejárás
        az óramutató járásával ellentétes irányú.)

  \item Végül egy egyenes szakasz mentén visszatérünk az origóba.
\end{enumerate}

A görbe egy félkört határol, melynek paraméteretése:
\begin{equation*}
  \surfsign(s; t) = \begin{bmatrix}
    s \cos t \\
    s \sin t \\
    0
  \end{bmatrix}
  \text,
  \qquad
  \begin{array}{l}
    s \in [0; 1] \\
    t \in [0; \pi]
  \end{array}
  \text.
\end{equation*}

A felület normálvektora:
\begin{equation*}
  \rvec n = \pdv{\surfsign}{s} \times \pdv{\surfsign}{t}
  = \begin{bmatrix}
    \cos t \\
    \sin t \\
    0
  \end{bmatrix} \times \begin{bmatrix}
    -s \sin t \\
    s \cos t  \\
    0
  \end{bmatrix} = \begin{bmatrix}
    0 \\
    0 \\
    s \cos^2 t + s \sin^2 t
  \end{bmatrix} = \begin{bmatrix}
    0 \\
    0 \\
    s
  \end{bmatrix}
  \text.
\end{equation*}

A vektormező rotációja:
\begin{equation*}
  \rot \rvec v = \begin{bmatrix}
    \partial_x \\ \partial_y \\ \partial_z
  \end{bmatrix} \times \begin{bmatrix}
    v_x \\ v_y \\ v_z
  \end{bmatrix} = \begin{bmatrix}
    \partial_x \\ \partial_y \\ \partial_z
  \end{bmatrix} \times \begin{bmatrix}
    z - y \\
    x - z \\
    y - x
  \end{bmatrix} = \begin{bmatrix}
    2 \\ 2 \\ 2
  \end{bmatrix}
  \text.
\end{equation*}

Az integrál:
\begin{align*}
  \iint_{\surfimage} \scalar{\rot \rvec v}{\dd \surfvec}
   & = \int_0^{\pi} \int_0^1 \rot \rvec v(\surfsign(s; t)) \cdot \rvec n \dd s \dd t
  = \int_0^{\pi} \int_0^1 \begin{bmatrix}
                            2 \\ 2 \\ 2
                          \end{bmatrix} \cdot \begin{bmatrix}
                                                0 \\ 0 \\ s
                                              \end{bmatrix} \dd s \dd t              \\
   & = \int_0^{\pi} \int_0^1 2s \dd s \dd t
  = \int_0^{\pi} \left[ s^2 \right]_0^1 \dd t
  = \int_0^{\pi} 1 \dd t
  = \left[ t \right]_0^{\pi}
  = \pi
  \text.
\end{align*}

% ~~~~~~~~~~~~~~~~~~~~~~~~~~~~~~~~~~~~~~~~~~~~~~~~~~~~~~~~~~~~~~~~~~~~~~~~~~~~~~
% 101010101010101010101010101010101010101010101010101010101010101010101010101010
% ~~~~~~~~~~~~~~~~~~~~~~~~~~~~~~~~~~~~~~~~~~~~~~~~~~~~~~~~~~~~~~~~~~~~~~~~~~~~~~

\subsection{Gauss-Osztogradszkij-tétel}

Határozza meg a $\rvec v(\coordvec) = \ijk{x}{y}{z}$ vektormező azon zárt
felületen vett felületi integrálját, melyet az $x = y^2 + z^2$ forgásparaboloid
$z > 0$ része, a $z = 0$ és az $x = 4$ síkok határolnak.

A tértartomány paraméterezése:
\begin{equation*}
  \volsign(r; s; t) = \begin{bmatrix}
    s^2           \\
    r \, s \cos t \\
    r \, s \sin t
  \end{bmatrix}
  \text,
  \qquad
  \begin{array}{l}
    r \in [0; 1] \\
    s \in [0; 2] \\
    t \in [0; \pi]
  \end{array}
  \text.
\end{equation*}

A Jacobi-mátrix:
\begin{equation*}
  \DD \volsign = \begin{bmatrix}
    0        & 2s       & 0           \\
    s \cos t & r \cos t & -r s \sin t \\
    s \sin t & r \sin t & r s \cos t
  \end{bmatrix}
\end{equation*}

Ennek determinánsának abszolút értéke:
\begin{equation*}
  \left| \det \DD \volsign \right|
  = 2 r s^3
  \text.
\end{equation*}

A vektormező diverzenciája:
\begin{equation*}
  \Div \rvec v
  = \pdv{v_x}{x} + \pdv{v_y}{y} + \pdv{v_z}{z}
  = 1 + 1 + 1
  = 3
  \text.
\end{equation*}

Az integrál:
\begin{align*}
  \oiint_{\partial \volimage} \scalar{\rvec v}{\dd \surfvec}
   & = \iiint_{\volimage} \Div \rvec v \dd \volscalar
  = \iiint_{\volimage} 3 \dd \volscalar
  = \int_0^{\pi} \int_0^2 \int_0^1 3 \cdot 2 r s^3 \dd r \dd s \dd t \\
   & = 6 \int_0^{\pi} \dd t \int_0^2 s^3 \dd s \int_0^1 r \dd r
  = 6 \cdot \pi \cdot \left[ \frac{s^4}{4} \right]_0^2 \cdot \left[ \frac{r^2}{2} \right]_0^1
  = 6 \cdot \pi \cdot 4 \cdot \frac{1}{2}
  = 12 \pi
  \text.
\end{align*}

\end{document}