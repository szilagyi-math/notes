\documentclass{szb-solution}

\usepackage{siunitx}
\sisetup{locale = DE}

\title{Integrálátalakító tételek II}
\area{Vektoranalízis}
\subject{Matematika G3}
\subjectCode{BMETE94BG03}
\date{Utoljára frissítve: \today}
\docno{6}

\begin{document}
\allowdisplaybreaks

\maketitle

% ~~~~~~~~~~~~~~~~~~~~~~~~~~~~~~~~~~~~~~~~~~~~~~~~~~~~~~~~~~~~~~~~~~~~~~~~~~~~~~
% 111111111111111111111111111111111111111111111111111111111111111111111111111111
% ~~~~~~~~~~~~~~~~~~~~~~~~~~~~~~~~~~~~~~~~~~~~~~~~~~~~~~~~~~~~~~~~~~~~~~~~~~~~~~
\subsection{Gömb térfogata, gömbi integrál}

Vezesse le az $R$ sugarú gömb térfogatát térfogati integrál
segítségével, majd számítsa ki a $\varphi(\coordvec) = 1 / \|\coordvec\|$
skalármező térfogati integrálját a gömbön!

A tértartomány paraméterezése:
$$
  \volsign(r; s; t) = \begin{bmatrix}
    r \sin s \cos t \\
    r \sin s \sin t \\
    r \cos s
  \end{bmatrix}
  \qquad
  \begin{array}{l}
    r \in [0; R]   \\
    s \in [0; \pi] \\
    t \in [0; 2\pi]
  \end{array}
$$

A Jacobi-mátrix:
$$
  \DD \volsign(r; s; t)
  = \begin{bmatrix}
    \sin s \cos t & r \cos s \cos t & -r \sin s \sin t \\
    \sin s \sin t & r \cos s \sin t & r \sin s \cos t  \\
    \cos s        & -r \sin s       & 0
  \end{bmatrix}
  \text.
$$

Ennek determinánsa az utolsó sor alapján kifejtve:
\begin{align*}
  \det
   & \left( \DD \volsign(r; s; t) \right)
  = -\cos s \cdot \begin{vmatrix}
                    r \cos s \cos t & -r \sin s \sin t \\
                    r \cos s \sin t & r \sin s \cos t
                  \end{vmatrix}
  + r \sin s \cdot \begin{vmatrix}
                     \sin s \cos t & -r \sin s \sin t \\
                     \sin s \sin t & r \sin s \cos t
                   \end{vmatrix}
  \\
   & = r^2 \cos s \left(
  \cos^2 t \cos s \sin s + \sin^2 t \cos s \sin s
  \right) +
  + r^2 \sin s \left(
  \sin^2 s \cos^2 t + \sin^2 s \sin^2 t
  \right)
  \\
   & = r^2 \cos s \cdot \cos s \sin s + r^2 \sin s \cdot \sin^2 s
  \\
   & = r^2 \sin s
  \text.
\end{align*}

A gömb térfogata tehát:
\begin{align*}
  V
   & = \iiint_{\volimage} \dd \volscalar
  = \iiint_{\voldomain} \det \left( \DD \volsign(r; s; t) \right)
  \dd r \dd s \dd t
  \\
   & = \int_0^{2\pi} \int_0^{\pi} \int_0^R r^2 \sin s \dd r \dd s \dd t
  = \int_0^{2\pi} \int_0^{\pi} \left[ \frac{r^3}{3} \right]_0^R \sin s \dd s \dd t
  \\
   & = \int_0^{2\pi} \int_0^{\pi} \frac{R^3}{3} \sin s \dd s \dd t
  = \int_0^{2\pi} \left[ -\frac{R^3}{3} \cos s \right]_0^{\pi} \dd t
  \\
   & = \int_0^{2\pi} \frac{2R^3}{3} \dd t
  = \left[ \frac{2R^3}{3} t \right]_0^{2\pi}
  = \frac{4\pi R^3}{3}
  \text.
\end{align*}

A skalármező átparaméterezése:
$$
  \varphi(\volsign(r; s; t)) = \frac{1}{\|\volsign(r; s; t)\|}
  = \frac{1}{\sqrt{r^2 \sin^2 s \cos^2 t + r^2 \sin^2 s \sin^2 t + r^2 \cos^2 s}}
  = \frac{1}{r}
  \text.
$$

A skalármező térfogati integrálja a gömbön:
\begin{align*}
  I
   & = \iiint_{\volimage} \varphi(\coordvec) \dd \volscalar
  = \iiint_{\voldomain} \varphi(\volsign(r; s; t))
  \det \left( \DD \volsign(r; s; t) \right) \dd r \dd s \dd t
  \\
   & = \int_0^{2\pi} \int_0^{\pi} \int_0^R \frac{1}{r} r^2 \sin s \dd r \dd s \dd t
  = \int_0^{2\pi} \int_0^{\pi} \int_0^R r \sin s \dd r \dd s \dd t
  \\
   & = \int_0^{2\pi} \int_0^{\pi} \left[ \frac{r^2}{2} \right]_0^R \sin s \dd s \dd t
  = \int_0^{2\pi} \int_0^{\pi} \frac{R^2}{2} \sin s \dd s \dd t
  \\
   & = \int_0^{2\pi} \left[ -\frac{R^2}{2} \cos s \right]_0^{\pi} \dd t
  = \int_0^{2\pi} R^2 \dd t
  \\
   & = \left[ R^2 t \right]_0^{2\pi}
  = 2\pi R^2
  \text.
\end{align*}

% ~~~~~~~~~~~~~~~~~~~~~~~~~~~~~~~~~~~~~~~~~~~~~~~~~~~~~~~~~~~~~~~~~~~~~~~~~~~~~~
% 222222222222222222222222222222222222222222222222222222222222222222222222222222
% ~~~~~~~~~~~~~~~~~~~~~~~~~~~~~~~~~~~~~~~~~~~~~~~~~~~~~~~~~~~~~~~~~~~~~~~~~~~~~~
\subsection{Henger térfogata}

Vezesse le az $R$ sugarú, $h$ magasságú henger térfogatát térfogati integrál
segítségével, majd számítsa ki a $\varphi(\coordvec) = x^2 + y^2$ skalármező
térfogati integrálját a hengerben!

A tértartomány paraméterezése:
$$
  \volsign(r; s; t) = \begin{bmatrix}
    r \cos t \\
    r \sin t \\
    s
  \end{bmatrix}
  \qquad
  \begin{array}{l}
    r \in [0; R] \\
    s \in [0; h] \\
    t \in [0; 2\pi]
  \end{array}
$$
A Jacobi-mátrix:
$$
  \DD \volsign(r; s; t) = \begin{bmatrix}
    \cos t & 0 & -r \sin t \\
    \sin t & 0 & r \cos t  \\
    0      & 1 & 0
  \end{bmatrix}
  \text.
$$
Ennek determinánsa:
$$
  \det \left( \DD \volsign(r; s; t) \right) = r
  \text.
$$
A henger térfogata tehát:
\begin{align*}
  V
   & = \iiint_{\volimage} \dd \volscalar
  = \iiint_{\voldomain} \det \left( \DD \volsign(r; s; t) \right)
  \dd r \dd s \dd t
  \\
   & = \int_0^{2\pi} \int_0^h \int_0^R r \dd r \dd s \dd t
  = \int_0^{2\pi} \int_0^h \left[ \frac{r^2}{2} \right]_0^R \dd s \dd t
  \\
   & = \int_0^{2\pi} \int_0^h \frac{R^2}{2} \dd s \dd t
  = \int_0^{2\pi} \left[ \frac{R^2}{2} s \right]_0^h \dd t
  \\
   & = \int_0^{2\pi} \frac{R^2 h}{2} \dd t
  = \left[ \frac{R^2 h}{2} t \right]_0^{2\pi}
  = \pi R^2 h
  \text.
\end{align*}

A skalármező átparaméterezése:
$$
  \varphi(\volsign(r; s; t)) = (r \cos t)^2 + (r \sin t)^2 = r^2
  \text.
$$
A skalármező térfogati integrálja a hengerben:
\begin{align*}
  I
   & = \iiint_{\volimage} \varphi(\coordvec) \dd \volscalar
  = \iiint_{\voldomain} \varphi(\volsign(r; s; t))
  \det \left( \DD \volsign(r; s; t) \right) \dd r \dd s \dd t
  \\
   & = \int_0^{2\pi} \int_0^h \int_0^R r^2 \cdot r \dd r \dd s \dd t
  = \int_0^{2\pi} \int_0^h \int_0^R r^3 \dd r \dd s \dd t
  \\
   & = \int_0^{2\pi} \int_0^h \left[ \frac{r^4}{4} \right]_0^R \dd s \dd t
  = \int_0^{2\pi} \int_0^h \frac{R^4}{4} \dd s \dd t
  \\
   & = \int_0^{2\pi} \left[ \frac{R^4}{4} s \right]_0^h \dd t
  = \int_0^{2\pi} \frac{R^4 h}{4} \dd t
  \\
   & = \left[ \frac{R^4 h}{4} t \right]_0^{2\pi}
  = \frac{\pi R^4 h}{2}
  \text.
\end{align*}

% ~~~~~~~~~~~~~~~~~~~~~~~~~~~~~~~~~~~~~~~~~~~~~~~~~~~~~~~~~~~~~~~~~~~~~~~~~~~~~~
% 333333333333333333333333333333333333333333333333333333333333333333333333333333
% ~~~~~~~~~~~~~~~~~~~~~~~~~~~~~~~~~~~~~~~~~~~~~~~~~~~~~~~~~~~~~~~~~~~~~~~~~~~~~~
\subsection{Vektormező ismeretlen komponense}

Legyen $\rvec v(\coordvec) = \ijk{y \sin x}{z^2 \cos y - \cos x}{v_3}$.
Határozza meg $v_3$-at, ha tudjuk, hogy $\rvec v$ tetszőleges zárt felületen
vett felületi integrálja zérus!

A Gauss-Osztogradszkij tétel szerint:
$$
  \oiint_{\partial \volimage} \scalar{\rvec v}{\dd \surfvec}
  =
  \iiint_{\volimage} \Div \rvec v \dd \volscalar
  \text,
$$
vagyis a tetszőleges zárt felületen vett felületi integrál akkor zérus, ha
a vektormező divergenciája zérus. Tehát:
$$
  \Div \rvec v
  = \pdv{}{x} (y \sin x) + \pdv{}{y} (z^2 \cos y - \cos x) + \pdv{v_3}{z}
  = y \cos x - z^2 \sin y + \pdv{v_3}{z}
  = 0
  \text.
$$
Ebből következik, hogy
$$
  \pdv{v_3}{z} = -y \cos x + z^2 \sin y
  \text.
$$
Integráljuk ezt $z$ szerint:
$$
  v_3 = \int \left( -y \cos x + z^2 \sin y \right) \dd z
  = -y z \cos x + \frac{z^3}{3} \sin y + f(x; y)
  \text,
$$
ahol $f(x; y)$ tetszőleges függvény, nem befolyásolja a dirvergenciát.


% ~~~~~~~~~~~~~~~~~~~~~~~~~~~~~~~~~~~~~~~~~~~~~~~~~~~~~~~~~~~~~~~~~~~~~~~~~~~~~~
% 333333333333333333333333333333333333333333333333333333333333333333333333333333
% ~~~~~~~~~~~~~~~~~~~~~~~~~~~~~~~~~~~~~~~~~~~~~~~~~~~~~~~~~~~~~~~~~~~~~~~~~~~~~~
\subsection{Zárt felületi integrálok}

Legyen $\rvec v(\coordvec) = \coordvec$. Adja meg a vektormező alábbi
zárt felületeken vett felületi integráljait:
\begin{enumerate}[a)]
  \item az $R = 2$ sugarú gömb felületén befelé mutató irányítással:
        $$
          I
          = \oiint_{\partial \volimage} \scalar{\rvec v}{\dd \surfvec}
          = - \iiint_{\volimage} \Div \rvec v \dd \volscalar
          = -3 \cdot \operatorname{Vol}(\volimage)
          = -3 \cdot \frac{4\pi R^3}{3}
          = -4 \pi R^3
          = -32 \pi
          \text.
        $$

  \item az $x^2 + y^2 = 4$ hengerfelületen, amelyet a $z = -1$ és
        $z = 1$ síkok zárnak le, kifelé mutató irányítással:
        $$
          I
          = \oiint_{\partial \volimage} \scalar{\rvec v}{\dd \surfvec}
          = \iiint_{\volimage} \Div \rvec v \dd \volscalar
          = 3 \cdot \operatorname{Vol}(\volimage)
          = 3 \cdot (\pi R^2 h)
          = 3 \cdot (\pi \cdot 2^2 \cdot 2)
          = 24 \pi
          \text.
        $$

  \item a $z = x^2 + y^2$ forgásparaboloid és a $z = 4$ sík által
        határolt test felületén kifelé mutató irányítással:

        Itt már nem tudjuk kapásból felírni a tértartomány térfogatát, így
        parametrizálnunk kell a tértartományt:
        $$
          \volsign(r; s; t) = \begin{bmatrix}
            r \, s \cos t \\
            r \, s \sin t \\
            r^2
          \end{bmatrix}
          \qquad
          \begin{array}{l}
            r \in [0; 2] \\
            s \in [0; 1] \\
            t \in [0; 2\pi]
          \end{array}
        $$
        A Jacobi-mátrix:
        $$
          \DD \volsign(r; s; t) = \begin{bmatrix}
            s \cos t & r \cos t & -r s \sin t \\
            s \sin t & r \sin t & r s \cos t  \\
            2 r      & 0        & 0
          \end{bmatrix}
          \text.
        $$
        Ennek determinánsa:
        \begin{align*}
          \det \left( \DD \volsign(r; s; t) \right)
           & = 2 r^3 s \cos^2 t + 2 r^2 s \sin^2 t
          = 2 r^3 s (\cos^2 t + \sin^2 t)
          = 2 r^3 s
          \text.
        \end{align*}
        A felületi integrál:
        \begin{align*}
          I
           & = \oiint_{\partial \volimage} \scalar{\rvec v}{\dd \surfvec}
          = \iiint_{\volimage} \Div \rvec v \dd \volscalar
          = 3 \cdot \iiint_{\volimage} \dd \volscalar
          \\
           & = 3 \cdot \iiint_{\voldomain} \det \left( \DD \volsign(r; s; t) \right) \dd r \dd s \dd t
          = 3 \cdot \int_0^{2\pi} \int_0^1 \int_0^2 2 r^3 s \dd r \dd s \dd t
          \\
           & = 3 \cdot \int_0^{2\pi} \int_0^1 \left[ \frac{r^4}{2} \right]_0^2 s \dd s \dd t
          = 3 \cdot \int_0^{2\pi} \int_0^1 8 s \dd s \dd t
          \\
           & = 3 \cdot \int_0^{2\pi} \left[ 4 s^2 \right]_0^1 \dd t
          = 3 \cdot \int_0^{2\pi} 4 \dd t
          = 12 \cdot \left[ t \right]_0^{2\pi}
          = 24 \pi
          \text.
        \end{align*}

        Akár hengerkordinátákkal is megoldhattuk volna a feladatot:

        A helyettesítés:
        $$
          \begin{cases}
            x = r \cos t \\
            y = r \sin t \\
            z = z
          \end{cases}
          \qquad
          \begin{array}{l}
            r \in [0; 2]    \\
            t \in [0; 2\pi] \\
            z \in [r^2; 4]
          \end{array}
        $$

        A Jacobi-mátrix:
        $$
          \DD \volsign(r; t; z) = \begin{bmatrix}
            \cos t & -r \sin t & 0 \\
            \sin t & r \cos t  & 0 \\
            0      & 0         & 1
          \end{bmatrix}
          \text.
        $$
        Ennek determinánsa:
        $$
          \det \left( \DD \volsign(r; t; z) \right) = r
          \text.
        $$
        A felületi integrál:
        \begin{align*}
          I
           & = \oiint_{\partial \volimage} \scalar{\rvec v}{\dd \surfvec}
          = \iiint_{\volimage} \Div \rvec v \dd \volscalar
          = 3 \cdot \iiint_{\volimage} \dd \volscalar
          \\
           & = 3 \cdot \iiint_{\voldomain} \det \left( \DD \volsign(r; t; z) \right) \dd r \dd t \dd z
          = 3 \cdot \int_0^{2\pi} \int_0^2 \int_{r^2}^4 r \dd z \dd r \dd t
          \\
           & = 3 \cdot \int_0^{2\pi} \int_0^2 r (4 - r^2) \dd r \dd t
          = 3 \cdot \int_0^{2\pi} \left[ 2 r^2 - \frac{r^4}{4} \right]_0^2 \dd t
          \\
           & = 3 \cdot \int_0^{2\pi} \left( 8 - 4 \right) \dd t
          = 12 \cdot \left[ t \right]_0^{2\pi}
          = 24 \pi
          \text.
        \end{align*}
\end{enumerate}

% ~~~~~~~~~~~~~~~~~~~~~~~~~~~~~~~~~~~~~~~~~~~~~~~~~~~~~~~~~~~~~~~~~~~~~~~~~~~~~~
% 444444444444444444444444444444444444444444444444444444444444444444444444444444
% ~~~~~~~~~~~~~~~~~~~~~~~~~~~~~~~~~~~~~~~~~~~~~~~~~~~~~~~~~~~~~~~~~~~~~~~~~~~~~~
\subsection{Faraday-kalitka}

Egy fotonikus chipeket hordozó wafer-darabot egy ellipszoid alakú
Faraday-kalitkába rögzítenek. A kalitka belsejében lineáris
feszültségelosztással ($\rvec E(\coordv) = \coordv$) térerőt állítanak
elő. Számolja ki a Faraday-kalitka belsejében lévő nettó töltést, ha
$\varepsilon_0 = \SI[per-mode=symbol]{8,85e-12}{\farad\per\meter}$,
az ellipszoid egyenlete pedig:
$$
  \frac{(x - 2)^2}{5} + \frac{(y + 3)^2}{19} + \frac{(z - 1)^2}{4} = 1
  \text.
$$

A nettó töltés kiszámításához a Gauss-törvényt alkalmazzuk:
$$
  \oiint_{\partial \volimage} \scalar{\rvec E}{\dd \surfvec}
  = \frac{Q_{\text{net}}}{\varepsilon_0}
  \text,
$$
ahol $\rvec E$ az elektromos térerősség, $\varepsilon_0$ a vákuum
permittivitása, $Q_{\text{net}}$ pedig a kalitka belsejében lévő nettó töltés.

A vektormező divergenciája:
$$
  \Div \rvec E = 3
  \text.
$$

A tértartomány nagysága:
$$
  \operatorname{Vol}(\volimage)
  = \frac{4\pi}{3} \sqrt{5 \cdot 19 \cdot 4}
  = \frac{8\pi \sqrt{95}}{3}
$$

A Gauss-törvény alapján a nettó töltés:
$$
  Q_{\text{net}}
  = \varepsilon_0 \oiint_{\partial \volimage} \scalar{\rvec E}{\dd \surfvec}
  = \varepsilon_0 \iiint_{\volimage} \Div \rvec E \dd \volscalar
  = 3 \, \varepsilon_0 \, \operatorname{Vol}(\volimage)
  = \SI[per-mode=symbol]{8,85e-12}{} \cdot 8 \pi \sqrt{95}
  \approx \SI{2,17}{\nano\coulomb}
  \text.
$$

% ~~~~~~~~~~~~~~~~~~~~~~~~~~~~~~~~~~~~~~~~~~~~~~~~~~~~~~~~~~~~~~~~~~~~~~~~~~~~~~
% 555555555555555555555555555555555555555555555555555555555555555555555555555555
% ~~~~~~~~~~~~~~~~~~~~~~~~~~~~~~~~~~~~~~~~~~~~~~~~~~~~~~~~~~~~~~~~~~~~~~~~~~~~~~
\subsection{Hőáram}

Egy drón IMU-modulját teljes egészében kitöltő, hővezető műgyanta gömb alakú,
sugara $R = \SI{0.02}{\meter}$, A vezérlő egység folyamatosan hőt disszipál, az
állandósult hő\-mér\-sék\-let-mező jó közelítéssel
$$
  \varphi(\coordv) = T_c - \alpha \rvec r^2
  \text, \quad
  \alpha = \SI[per-mode=symbol]{1,3e5}{\kelvin\per\meter\squared}
  \text.
$$
Becsülje meg, mekkora teljes hőáram távozik a burkolaton át, ha
$$
  q_{\text{hő}}
  = -\oiint_{\partial \volimage} \scalar{\lambda \grad \varphi}{\dd \surfvec}
  \text,\quad
  \lambda = \SI[per-mode=symbol]{0,2}{\watt\per\meter\per\kelvin}
  \text.
$$
Használjuk a Green-tétel asszimetrikus alakját $\psi = -\lambda$ állandó
választással:
$$
  q_\text{hő}
  = \oiint_{\partial \volimage} \scalar{-\lambda \grad \varphi}{\dd \surfvec}
  = - \iiint_{\volimage} \psi \, \Delta \varphi + \scalar{\underbrace{\grad \lambda}_{= \nvec}}{\grad \varphi} \dd \volscalar
  = -\lambda \iiint_{\volimage} \Delta \varphi \dd \volscalar
  \text.
$$
A hőmérséklet-mező Laplace-operátora:
$$
  \Delta \varphi
  = \Div \grad \varphi
  = \Div (-2 \alpha \coordvec)
  = -2 \alpha \Div \coordvec
  = -6 \alpha
  \text.
$$
A tértartomány nagysága:
$$
  \operatorname{Vol}(\volimage)
  = \frac{4\pi R^3}{3}
  \text.
$$
A teljes hőáram:
\begin{align*}
  q_\text{hő}
   & = -\lambda \iiint_{\volimage} \Delta \varphi \dd \volscalar
  = 6 \alpha \lambda \operatorname{Vol}(\volimage)
  = \lambda \, 6 \, \alpha \, \frac{4\pi R^3}{3}
  \\
   & = 8 \pi \alpha \lambda R^3
  = 8 \pi \cdot 0,2 \cdot \SI[per-mode=symbol]{1,3e5}{} \cdot (0,02)^3
  \approx \SI{5.23}{\watt}
  \text.
\end{align*}

\end{document}