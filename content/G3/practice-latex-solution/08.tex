\documentclass{szb-solution}

\title{Bevezetés}
\area{Differenciálegyenletek}
\subject{Matematika G3}
\subjectCode{BMETE94BG03}
\date{Utoljára frissítve: \today}
\docno{8}

\usepackage{siunitx}
\sisetup{locale = DE}

\begin{document}
\maketitle

% ~~~~~~~~~~~~~~~~~~~~~~~~~~~~~~~~~~~~~~~~~~~~~~~~~~~~~~~~~~~~~~~~~~~~~~~~~~~~~~
% 111111111111111111111111111111111111111111111111111111111111111111111111111111
% ~~~~~~~~~~~~~~~~~~~~~~~~~~~~~~~~~~~~~~~~~~~~~~~~~~~~~~~~~~~~~~~~~~~~~~~~~~~~~~
\subsection{Differenciálegyenletek osztályozása}

\begin{center}
  \def\arraystretch{1.5}
  \begin{tabular}{ll|c|c|c|}
       & Egyenlet                                                       & Megadási mód & Rend & Lineáris?    \\
    \hline
    a) & $y' = \cosh x - 3xy$                                           & Explicit     & 1    & lineáris     \\
    b) & $y'' = y'^2 \, \cos x$                                         & Explicit     & 2    & nem lineáris \\
    c) & $\left( 1 + y^{(\mathrm{IV})} \right)^2 - y'' = x^3 y''' + xy$ & Implicit     & 4    & nem lineáris \\
    d) & $y'' = e^y \, \ln x$                                           & Explicit     & 2    & nem lineáris \\
    \hline
  \end{tabular}
\end{center}


% ~~~~~~~~~~~~~~~~~~~~~~~~~~~~~~~~~~~~~~~~~~~~~~~~~~~~~~~~~~~~~~~~~~~~~~~~~~~~~~
% 222222222222222222222222222222222222222222222222222222222222222222222222222222
% ~~~~~~~~~~~~~~~~~~~~~~~~~~~~~~~~~~~~~~~~~~~~~~~~~~~~~~~~~~~~~~~~~~~~~~~~~~~~~~
\subsection{Izoklinák és vonalelemek}

\begin{enumerate}
  \item $y' = y / x \qquad x > 0$

        \textbf{Izoklinák}:
        $\sfrac{y}{x} = c \implies y = cx$
        \quad -- \quad origón áthaladó egyenesek

        \textbf{Vonalelemek}:
        $y' = \sfrac{y}{x} = \sfrac{cx}{x} = c$
        \quad -- \quad állandó meredekségű egyenesek

        \textbf{Megoldásgörbék}:
        $y = Cx$
        \quad -- \quad origón áthaladó egyenesek

  \item $y' = -x / y \qquad y < 0$

        \textbf{Izoklinák}:
        $-\sfrac{x}{y} = c \implies y = -\sfrac{x}{c}$
        \quad -- \quad origón áthaladó egyenesek

        \textbf{Vonalelemek}:
        $y' = -\sfrac{x}{y} = c$
        \quad -- \quad állandó meredekségű egyenesek

        \textbf{Megoldásgörbék}:
        $x^2 + y^2 = C$
        \quad -- \quad origó középpontú félkörök
\end{enumerate}


% ~~~~~~~~~~~~~~~~~~~~~~~~~~~~~~~~~~~~~~~~~~~~~~~~~~~~~~~~~~~~~~~~~~~~~~~~~~~~~~
% 333333333333333333333333333333333333333333333333333333333333333333333333333333
% ~~~~~~~~~~~~~~~~~~~~~~~~~~~~~~~~~~~~~~~~~~~~~~~~~~~~~~~~~~~~~~~~~~~~~~~~~~~~~~
\subsection{Megoldás-e I}

Az $(y')^4 + y^2 = -1$ differenciálegyenlet megoldása-e az $y = x^2 - 1$
függvény?

Az első derivált:
$$
  y' = 2x
  \text.
$$

Helyettesítsük be a differenciálegyenletbe:
$$
  (y')^4 + y^2
  = (2x)^4 + (x^2 - 1)^2
  = 16x^4 + (x^4 - 2x^2 + 1)
  = 17x^4 - 2x^2 + 1
  \text.
$$
Mivel ez a kifejezés nem egyenlő $-1$-gyel minden $x$ értékre, ezért a
megadott függvény \textbf{nem} megoldása a differenciálegyenletnek.


% ~~~~~~~~~~~~~~~~~~~~~~~~~~~~~~~~~~~~~~~~~~~~~~~~~~~~~~~~~~~~~~~~~~~~~~~~~~~~~~
% 444444444444444444444444444444444444444444444444444444444444444444444444444444
% ~~~~~~~~~~~~~~~~~~~~~~~~~~~~~~~~~~~~~~~~~~~~~~~~~~~~~~~~~~~~~~~~~~~~~~~~~~~~~~
\subsection{Megoldás-e II}

Megoldása-e az $y = C_1 \sin 2x + C_2 \cos 2x$ függvény az $y'' + 4y = 0$
differenciálegyenletnek?

Az első és második derivált:
\begin{align*}
  y'  & = 2C_1 \cos 2x - 2C_2 \sin 2x \text,  \\
  y'' & = -4C_1 \sin 2x - 4C_2 \cos 2x \text.
\end{align*}

Vegyük észre, hogy $y'' = -4y$, tehát
$$
  y'' + 4y = -4y + 4y = 0
  \text,
$$
tehát a megadott függvény valóban megoldása a differenciálegyenletnek.


% ~~~~~~~~~~~~~~~~~~~~~~~~~~~~~~~~~~~~~~~~~~~~~~~~~~~~~~~~~~~~~~~~~~~~~~~~~~~~~~
% 555555555555555555555555555555555555555555555555555555555555555555555555555555
% ~~~~~~~~~~~~~~~~~~~~~~~~~~~~~~~~~~~~~~~~~~~~~~~~~~~~~~~~~~~~~~~~~~~~~~~~~~~~~~
\subsection{Megoldás-e III}

Megoldása-e az $y = \ln x$ függvény az $xy'' + y' = 0$ differenciálegyenletnek?

Az első és második derivált:
\begin{align*}
  y'  & = \frac{1}{x},    \\
  y'' & = -\frac{1}{x^2}.
\end{align*}

Helyettesítsük be a differenciálegyenletbe:
$$
  xy'' + y' = x \cdot \left( -\frac{1}{x^2} \right) + \frac{1}{x} = -\frac{1}{x} + \frac{1}{x} = 0
  \text,
$$
tehát a megadott függvény valóban megoldása a differenciálegyenletnek.


% ~~~~~~~~~~~~~~~~~~~~~~~~~~~~~~~~~~~~~~~~~~~~~~~~~~~~~~~~~~~~~~~~~~~~~~~~~~~~~~
% 666666666666666666666666666666666666666666666666666666666666666666666666666666
% ~~~~~~~~~~~~~~~~~~~~~~~~~~~~~~~~~~~~~~~~~~~~~~~~~~~~~~~~~~~~~~~~~~~~~~~~~~~~~~
\subsection{Cauchy-feladat I}

Adja meg az $y'' + 4y = 0$ differenciálegyenlet $y(0) = 0$ és $y'(0) = 1$
kezdeti feltételek melletti megoldását!

Az általános megoldás és annak első deriváltja:
\begin{align*}
  y  & = C_1 \sin 2x + C_2 \cos 2x \text,   \\
  y' & = 2C_1 \cos 2x - 2C_2 \sin 2x \text.
\end{align*}

Az első egyenletből:
$$
  y(0)
  = C_1 \underbrace{\sin 0}_{=0} + C_2 \underbrace{\cos 0}_{=1}
  = C_2
  = 0
  \text.
$$

A második egyenletből:
$$
  y'(0)
  = 2C_1 \underbrace{\cos 0}_{=1} - 2 \cdot 0 \cdot \underbrace{\sin 0}_{=0}
  = 2C_1
  = 1
  \implies C_1 = \frac{1}{2}
  \text.
$$

Tehát a keresett megoldás:
$$
  y = \frac{1}{2} \sin 2x
  \text.
$$


% ~~~~~~~~~~~~~~~~~~~~~~~~~~~~~~~~~~~~~~~~~~~~~~~~~~~~~~~~~~~~~~~~~~~~~~~~~~~~~~
% 777777777777777777777777777777777777777777777777777777777777777777777777777777
% ~~~~~~~~~~~~~~~~~~~~~~~~~~~~~~~~~~~~~~~~~~~~~~~~~~~~~~~~~~~~~~~~~~~~~~~~~~~~~~
\subsection{Cauchy-feladat II}

Adja meg az $y'' + 4y = 0$ differenciálegyenlet $y(0) = 1$ és $y'(\pi/4) = 2$
kezdeti feltételek melletti megoldását!

Az általános megoldás és annak első deriváltja:
\begin{align*}
  y  & = C_1 \sin 2x + C_2 \cos 2x \text,   \\
  y' & = 2C_1 \cos 2x - 2C_2 \sin 2x \text.
\end{align*}

Az első egyenletből:
$$
  y(0)
  = C_1 \underbrace{\sin 0}_{=0} + C_2 \underbrace{\cos 0}_{=1}
  = C_2
  = 1
  \text.
$$

A második egyenletből:
$$
  y'(\pi/4)
  = 2C_1 \underbrace{\cos \pi/2}_{=0} - 2 \cdot 1 \cdot \underbrace{\sin \pi/2}_{=1}
  = -2
  \neq 2
  \text.
$$

Ellentmondásra jutottunk, tehát a megadott kezdeti feltételek mellett nincs
megoldás.


% ~~~~~~~~~~~~~~~~~~~~~~~~~~~~~~~~~~~~~~~~~~~~~~~~~~~~~~~~~~~~~~~~~~~~~~~~~~~~~~
% 888888888888888888888888888888888888888888888888888888888888888888888888888888
% ~~~~~~~~~~~~~~~~~~~~~~~~~~~~~~~~~~~~~~~~~~~~~~~~~~~~~~~~~~~~~~~~~~~~~~~~~~~~~~
\subsection{Görbeseregek differenciálegyenletei}

\begin{enumerate}
  \item $y = c x^2$:

        Deriváljuk az egyenletet, és fejezzük ki a $c$ konstansot:
        $$
          y' = 2cx
          \quad \text{és} \quad
          c = \frac{y}{x^2}
          \quad \implies \quad
          y' = 2 \frac{y}{x}
          \text.
        $$


  \item $x^2 + y^2 = cx$:

        Deriváljuk az egyenletet implicit módon:
        $$
          2x + 2yy' = c
          \text.
        $$

        Helyettesítsük be az eredeti egyenletbe $c$-t:
        $$
          x^2 + y^2 = 2x^2 + 2yy'x
          \quad \implies \quad
          2xyy' + x^2 - y^2 = 0
          \text.
        $$

  \item $y = c_1 e^x + c_2 e^{2x}$:

        Számítsuk ki az első deriváltat:
        $$
          y'  = c_1 e^x + 2c_2 e^{2x} \text.
        $$

        Számítsuk ki $y' - y$ különbséget:
        $$
          y' - y
          = \left( c_1 e^x + 2c_2 e^{2x} \right) - \left( c_1 e^x + c_2 e^{2x} \right)
          = c_2 e^{2x}
          \quad \implies \quad
          c_2 = e^{-2x} (y' - y)
          \text.
        $$

        A kapott konstanst helyettesítsük be az eredeti egyenletbe:
        $$
          y
          = c_1 e^x + e^{-2x} (y' - y) e^{2x}
          = c_1 e^x + y' - y
          \quad \implies \quad
          c_1 = e^{-x} (2y - y')
          \text.
        $$

        Most számítsuk ki a második deriváltat is, és helyettesítsük be
        a konstansokat:
        \begin{align*}
          y''
           & = c_1 e^x + 4c_2 e^{2x}                            \\
           & = e^{-x} (2y - y') e^x + 4 e^{-2x} (y' - y) e^{2x} \\
           & = 2y - y' + 4y' - 4y                               \\
           & = -2y + 3y' \text.
        \end{align*}

        Végül rendezzük az egyenletet:
        $$
          y'' + 2y - 3y' = 0
          \text.
        $$
\end{enumerate}


% ~~~~~~~~~~~~~~~~~~~~~~~~~~~~~~~~~~~~~~~~~~~~~~~~~~~~~~~~~~~~~~~~~~~~~~~~~~~~~~
% 999999999999999999999999999999999999999999999999999999999999999999999999999999
% ~~~~~~~~~~~~~~~~~~~~~~~~~~~~~~~~~~~~~~~~~~~~~~~~~~~~~~~~~~~~~~~~~~~~~~~~~~~~~~
\subsection{Körök differenciálegyenlete}

Adja meg az olyan $xy$ síkban elhelyezkedő körök differenciálegyenletét,
amelyek az $x$-ten\-gelyt az origóban érintik!

Az ilyen körök egyenlete:
$$
  x^2 + (y - r)^2 = r^2
  \text.
$$

Deriváljuk az egyenletet implicit módon:
$$
  2x + 2(y - r)y' = 0
  \quad \implies \quad
  2x + 2yy' - 2ry' = 0
  \quad \implies \quad
  r = \frac{x}{y'} + y
$$

Helyettesítsük be az eredeti egyenletbe $r$-t:
$$
  x^2 + \left(
  t - \frac{x}{y'} + y
  \right)^2 = \left(
  \frac{x}{y'} + y
  \right)^2
  \text.
$$


\end{document}