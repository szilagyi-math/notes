\documentclass[fleqn]{szb-solution}

\title{Görbék, görbementi integrál}
\area{Vektoranalízis}
\subject{Matematika G3}
\subjectCode{BMETE94BG03}
\date{Utoljára frissítve: \today}
\docno{3}

\begin{document}

\maketitle


% ~~~~~~~~~~~~~~~~~~~~~~~~~~~~~~~~~~~~~~~~~~~~~~~~~~~~~~~~~~~~~~~~~~~~~~~~~~~~~~
% 111111111111111111111111111111111111111111111111111111111111111111111111111111
% ~~~~~~~~~~~~~~~~~~~~~~~~~~~~~~~~~~~~~~~~~~~~~~~~~~~~~~~~~~~~~~~~~~~~~~~~~~~~~~
\subsection{Görbék ívhossza}

\begin{enumerate}[a)]
  \item $\curvesign_1(t) = \ijk{t}{\sqrt{6} t^2 / 2}{t^3}, \quad t \in [0; 2]$
        \begin{align*}
          L
           & = \int_0^2 \norma{\dot{\curvesign_1}(t)} \dd t
          = \int_0^2 \sqrt{1^2 + (\sqrt 6 t)^2 + (3t)^2} \dd t
          = \int_0^2 \sqrt{1 + 6t^2 + 9t^4} \dd t
          \\
           & = \int_0^2 1 + 3t^2 \dd t
          = \left[ t + t^3 \right]_0^2
          = 10
        \end{align*}

  \item $\curvesign_2(t) = \ijz{t \cos t}{t \sin t}, \quad t \in [0; 1]$
        \begin{align*}
          L
           & = \int_0^{1} \norma{\dot{\curvesign_2}(t)} \dd t
          = \int_0^{1} \sqrt{(\cos t - t \sin t)^2 + (\sin t + t \cos t)^2} \dd t
          \\
           & = \int_0^{1} \sqrt{1 + t^2} \dd t
          = \int_0^{1} \cosh u \sqrt{1 + \sinh^2 u} \dd u
          = \int_0^{1} \cosh^2 u \dd u
          \\
           & = \int_{u_1}^{u_2} \frac{1 + \cosh 2u}{2} \dd u
          = \left[ \frac{u}{2} + \frac{\sinh 2u}{4} \right]_{u_1}^{u_2}
          = \left[ \frac{\arcsinh t}{2} + \frac{t \sqrt{t^2 + 1}}{2} \right]_0^{1}
          \\
           & = \frac{\arcsinh 1 + \sqrt{2}}{2}
          \approx 1,1478
        \end{align*}

  \item $\curvesign_3(t) = \ijk{e^t \cos t}{e^t \sin t}{e^t}, \quad t \in [0; 2\pi]$
        \begin{align*}
          L
           & = \int_0^{2\pi} \norma{\dot{\curvesign_3}(t)} \dd t
          = \int_0^{2\pi} \sqrt{e^{2t} (\cos t - \sin t) + e^{2t} (\sin t + \cos t) + e^{4t}} \dd t
          \\
           & = \int_0^{2\pi} \sqrt{3} e^t \dd t
          = \left[ \sqrt{3} e^t \right]_0^{2\pi}
          = \sqrt{3} (e^{2\pi} - 1)
          \approx 925,7667
        \end{align*}

  \item $\curvesign_4(t) = \ijz{t - \sin t}{1 - \cos t}, \quad t \in [0; 2\pi]$
        \begin{align*}
          L
           & = \int_0^{2\pi} \norma{\dot{\curvesign_4}(t)} \dd t
          = \int_0^{2\pi} \sqrt{(1 - \cos t)^2 + \sin^2 t} \dd t
          \\
           & = \int_0^{2\pi} \sqrt{1 - 2\cos t + \cos^2 t + \sin^2 t} \dd t
          = \int_0^{2\pi} \sqrt{2 - 2\cos t} \dd t
          \\
           & = \int_0^{2\pi} \sqrt{4 \sin^2 \frac{t}{2}} \dd t
          = 2 \int_0^{2\pi} \sin \frac{t}{2} \dd t
          = \left[ -4 \cos \frac{t}{2} \right]_0^{2\pi}
          = 8
        \end{align*}
\end{enumerate}

% ~~~~~~~~~~~~~~~~~~~~~~~~~~~~~~~~~~~~~~~~~~~~~~~~~~~~~~~~~~~~~~~~~~~~~~~~~~~~~~
% 222222222222222222222222222222222222222222222222222222222222222222222222222222
% ~~~~~~~~~~~~~~~~~~~~~~~~~~~~~~~~~~~~~~~~~~~~~~~~~~~~~~~~~~~~~~~~~~~~~~~~~~~~~~
\subsection{Ívhossz szerinti paraméterezés}

Adja meg a $\curvesign(t) = \ijk{t+3}{t^2 / 2}{(2\sqrt{2}/3)t^{\sfrac32}}$ görbe
egységsebességű paraméterezését!

A görbe sebességvektora, és ennek abszolóz értéke:
$$
  \dot{\curvesign}(t) = \begin{bmatrix}
    1 \\
    t \\
    \sqrt{2t}
  \end{bmatrix}
  \quad \Rightarrow \quad
  \norma{\dot{\curvesign}(t)} = \sqrt{1 + 2t + t^2} = |1 + t|
  \text.
$$
Az ívhossz szerinti integrál:
$$
  L(t)
  = \int_0^t |1 + \tau| \dd \tau
  = \int_0^t (1 + \tau) \dd \tau
  = t + \frac{t^2}{2}
  \text.
$$
Ennek inverze:
$$
  t(L) = -1 + \sqrt{1 + 2L}
  \text.
$$
Az egységsebességű paraméterezés:
$$
  \curvesign_s(L) = \begin{bmatrix}
    -1 + \sqrt{1 + 2L} + 3     \\
    (-1 + \sqrt{1 + 2L})^2 / 2 \\
    (2\sqrt{2}/3)(-1 + \sqrt{1 + 2L})^{\sfrac32}
  \end{bmatrix}
$$

% ~~~~~~~~~~~~~~~~~~~~~~~~~~~~~~~~~~~~~~~~~~~~~~~~~~~~~~~~~~~~~~~~~~~~~~~~~~~~~~
% 333333333333333333333333333333333333333333333333333333333333333333333333333333
% ~~~~~~~~~~~~~~~~~~~~~~~~~~~~~~~~~~~~~~~~~~~~~~~~~~~~~~~~~~~~~~~~~~~~~~~~~~~~~~
\subsection{Skalármezők görbe menti skalárértékű integrálja}

\begin{enumerate}[a)]
  \item $\varphi(\coordvec) = \sqrt{1 + 4x^2 + 16yz}, \quad \curvesign(t) = \ijk{t}{t^2}{t^4}, \quad t \in [0; 1]$
        \begin{align*}
          \int_\gamma \varphi(\coordvec) \dd \curvescalar
           & = \int_0^1 \varphi(\curvesign(t)) \norma{\dot{\curvesign}(t)} \dd t
          = \int_0^1 \sqrt{1 + 4t^2 + 16t^6} \sqrt{1^2 + (2t)^2 + (4t^3)^2} \dd t
          \\
           & = \int_0^1 1 + 4t^2 + 16t^6 \dd t
          = \left[ t + \frac{4}{3} t^3 + \frac{16}{7} t^7 \right]_0^1
          = 1 + \frac{4}{3} + \frac{16}{7}
          = \frac{97}{21}
        \end{align*}

  \item $\psi(\coordvec) = 2x, \quad$ a $(3;0)$ és $(0;4)$ pontokat összekötő
        szakasz mentén

        A szakasz paraméteres egyenlete és annak deriváltja:
        $$
          \curvesign (t) = \begin{bmatrix}
            3 \\ 0
          \end{bmatrix} + t \begin{bmatrix}
            0 - 3 \\ 4 - 0
          \end{bmatrix} = \begin{bmatrix}
            3 - 3t \\ 4t
          \end{bmatrix}
          \text, \quad
          t \in [0,1]
          \text, \quad
          \dot{\curvesign} (t) = \begin{bmatrix}
            -3 \\ 4
          \end{bmatrix}
          \text, \quad
          \norma{\dot{\curvesign}(t)}
          = 5
          \text.
        $$
        Végezzük el az integrálást:
        $$
          \int_\gamma \psi(\coordvec) \dd \curvescalar
          = \int_0^1 \psi(\curvesign(t)) \norma{\curvesign} \dd t
          = \int_0^1 2 (3 - 3t) \cdot 5 \dd t
          % = 30 \int_0^1 (1 - t) \dd t
          = 30 \left[ t - \frac{t^2}{2} \right]_0^1
          = 15
          \text.
        $$

  \item $\chi(\coordvec) = x^2 + y^2, \quad$ első síknegyedbeli egységköríven,
        pozitív irányítással

        A görbe paraméteres egyenlete és annak deriváltja:
        $$
          \curvesign(t) = \begin{bmatrix}
            \cos t \\ \sin t
          \end{bmatrix}
          \text, \quad
          t \in [0, \pi/2]
          \text, \quad
          \dot{\curvesign}(t) = \begin{bmatrix}
            -\sin t \\ \cos t
          \end{bmatrix}
          \text, \quad
          \norma{\dot{\curvesign}(t)}
          = 1
          \text.
        $$
        Végezzük el az integrálást:
        $$
          \int_\gamma \chi(\coordvec) \dd \curvescalar
          = \int_0^{\pi/2} \chi(\curvesign (t)) \norma{\dot{\curvesign}(t)} \dd t
          = \int_0^{\pi/2} (\cos^2 t + \sin^2 t) \dd t
          = \int_0^{\pi/2} \dd t
          % = \left[ t \right]_0^{\pi/2}
          = \frac{\pi}{2}
          \text.
        $$

  \item $\omega(\coordvec) = x^2 + y^2, \quad$ $r = 2$ sugarú, origó középpontú,
        pozitív irányítású körön

        A görbe paraméteres egyenlete és annak deriváltja:
        $$
          \curvesign(t) = \begin{bmatrix}
            2 \cos t \\ 2 sin t
          \end{bmatrix}
          \text, \quad
          t \in [0, 2 \pi]
          \text, \quad
          \dot{\curvesign}(t) = \begin{bmatrix}
            -2 \sin t \\ 2 \cos t
          \end{bmatrix}
          \text, \quad
          \norma{\dot{\curvesign}(t)}
          = 2
          \text.
        $$
        Végezzük el az integrálást:
        $$
          \int_\gamma \omega(\coordvec) \dd \curvescalar
          = \int_0^{2\pi} \omega(\curvesign(t)) \norma{\dot{\curvesign}(t)} \dd t
          = \int_0^{2\pi} (4 \cos^2 t + 4 \sin^2 t) \cdot 2 \dd t
          = \int_0^{2\pi} 8 \dd t
          % = \left[ t \right]_0^{\pi/2}
          = 16 \pi
          \text.
        $$
\end{enumerate}


% ~~~~~~~~~~~~~~~~~~~~~~~~~~~~~~~~~~~~~~~~~~~~~~~~~~~~~~~~~~~~~~~~~~~~~~~~~~~~~~
% 444444444444444444444444444444444444444444444444444444444444444444444444444444
% ~~~~~~~~~~~~~~~~~~~~~~~~~~~~~~~~~~~~~~~~~~~~~~~~~~~~~~~~~~~~~~~~~~~~~~~~~~~~~~
\subsection{Vektormezők görbe menti skalárértékű integrálja}

\begin{enumerate}[a)]
  \item $\rvec u(\coordvec) = \ijk{y + z}{x + z}{x + y}, \;\; \curvesign(t) = \ijk{t}{t^2}{t^3}, \;\; t \in [0; 1]$
        \begin{align*}
          \int_\gamma \scalar{\rvec u(\coordvec)}{\dd \curvevec}
           & = \int_0^1 \scalar{\rvec u(\curvesign (t))}{\dot{\curvesign}(t)} \dd t
          = \int_0^1
          \begin{bmatrix}
            t^2 + t^3 \\
            t + t^3   \\
            t + t^2
          \end{bmatrix}^\T
          \begin{bmatrix}
            1  \\
            2t \\
            3t^2
          \end{bmatrix} \dd t
          \\
           & = \int_0^1 (t^2 + t^3 + 2t^2 + 2t^4 + 3t^3 + 3t^4) \dd t
          \\
           & = \int_0^1 (3t^2 + 4t^3 + 5t^4) \dd t
          = \left[ t^3 + t^4 + t^5 \right]_0^1
          = 3
        \end{align*}


  \item $\rvec v(\coordvec) = \ijk{-yz}{xz}{x^2 + y^2}, \;\; \curvesign(t) = \ijk{\cos t}{\sin t}{2t}, \;\; t \in [0; 2]$
        \begin{align*}
          \int_\gamma \scalar{\rvec v(\coordvec)}{\dd \curvevec}
           & = \int_0^2 \scalar{\rvec v(\curvesign (t))}{\dot{\curvesign}(t)} \dd t
          = \int_0^2
          \begin{bmatrix}
            -2t \sin t \\
            2t \cos t  \\
            \cos^2 t + \sin^2 t
          \end{bmatrix}^\T
          \begin{bmatrix}
            - \sin t \\
            \cos t   \\
            2
          \end{bmatrix} \dd t
          \\
           & = \int_0^2 2t ( \sin^2 t + \cos^2 t ) + 2 ( \cos^2 t + \sin^2 t ) \dd t
          \\
           & = \int_0^2 2t + 2 \dd t
          = \left[ t^2 + 2t \right]_0^2
          = 8
        \end{align*}


  \item $\rvec w(\coordvec) = \ijz{y}{x}, \;\;$ az $(1;0)$ pontból a $(0;2)$ pontba
        \begin{itemize}
          \item Az egyenes szakasz mentén:
                $$
                  \curvesign(t) = \begin{bmatrix}
                    1 \\ 0
                  \end{bmatrix} + t \begin{bmatrix}
                    0 - 1 \\ 2 - 0
                  \end{bmatrix} = \begin{bmatrix}
                    1 - t \\ 2t
                  \end{bmatrix}
                  \text, \quad
                  t \in [0;1]
                  \text, \quad
                  \dot{\curvesign}(t) = \begin{bmatrix}
                    -1 \\ 2
                  \end{bmatrix}
                $$
                $$
                  \int_0^1 \begin{bmatrix}
                    2t \\ 1 - t
                  \end{bmatrix}^\T
                  \begin{bmatrix}
                    -1 \\ 2
                  \end{bmatrix}
                  \dd t
                  = \int_0^1 (-2t + 2 - 2t) \dd t
                  = \int_0^1 2 - 4t \dd t
                  = \left[ 2t - 2t^2 \right]_0^1
                  = 0
                $$

          \item Az ellipszis mentén:
                $$
                  \curvesign (t) = \begin{bmatrix}
                    \cos t \\ 2 \sin t
                  \end{bmatrix}
                  \text,\quad
                  t \in [0; \pi/2]
                  \text,\quad
                  \dot{\curvesign r}(t) = \begin{bmatrix}
                    - \sin t \\ 2 \cos t
                  \end{bmatrix}
                $$
                $$
                  \hspace{-2.3cm}
                  \int_0^{\pi / 2} \begin{bmatrix}
                    2 \sin t \\ \cos t
                  \end{bmatrix}^\T
                  \begin{bmatrix}
                    - \sin t \\ 2 \cos t
                  \end{bmatrix}
                  \dd t
                  = 2 \int_0^{\pi / 2} (\cos^2 t - \sin^2 t) \dd t
                  = 2 \int_0^{\pi / 2} \cos 2t \dd t
                  = 2 \left[ \frac{\sin 2t}{2} \right]_0^{\pi / 2}
                  = 0
                $$

          \item A gradiens-tételt felhasználva: a $\rvec w(\coordvec)$
                vektormező potenciálfüggvénye:
                $$
                  \varphi(\coordvec) = xy
                  \text, \quad
                  \grad \varphi(\coordvec) = \rvec w(\coordvec)
                  \text.
                $$
                A gradiens-tétel szerint a görbementi integrál értéke csak a
                kezdő- és végponttól függ, ezért a két pontot összekötő
                tetszőleges görbe mentén az integrál értéke:
                $$
                  \int_\gamma \scalar{\rvec w(\coordvec)}{\dd \curvevec}
                  = \varphi(0;2) - \varphi(1;0)
                  = 0 - 0
                  = 0
                  \text.
                $$
        \end{itemize}



\end{enumerate}

% ~~~~~~~~~~~~~~~~~~~~~~~~~~~~~~~~~~~~~~~~~~~~~~~~~~~~~~~~~~~~~~~~~~~~~~~~~~~~~~
% 555555555555555555555555555555555555555555555555555555555555555555555555555555
% ~~~~~~~~~~~~~~~~~~~~~~~~~~~~~~~~~~~~~~~~~~~~~~~~~~~~~~~~~~~~~~~~~~~~~~~~~~~~~~
\subsection{Vektormező görbementi vektorértékű integrálja}

Adja meg a $\rvec v(\coordvec) = \ijk{y^2 - x^2}{2 y z}{-x^2}$
vektormező $\curvesign(t) = \ijk{t}{t^2}{t^3}$, $t \in [0; 1]$
görbe menti vektorértékű integrálját!
\begin{align*}
  \int_{\curveimage} \rvec v(\coordvec) \times \dd \curvevec
   & = \int_0^1 \rvec v(\curvesign(t)) \times \dot{\curvesign}(t) \dd t
  = \int_0^1
  \begin{bmatrix}
    t^4 - t^2 \\
    2 t^5     \\
    -t^2
  \end{bmatrix}
  \times
  \begin{bmatrix}
    1 \\ 2t \\ 3t^2
  \end{bmatrix} \dd t
  \\
   & =
  \int_0^1
  \begin{bmatrix}
    6t^7 + 2t^3        \\
    -t^2 - 3t^6 + 3t^4 \\
    2t^3
  \end{bmatrix} \dd t
  =
  \dots
  =
  \begin{bmatrix}
    5/4     \\
    -17/105 \\
    -1/2
  \end{bmatrix}
\end{align*}

\end{document}