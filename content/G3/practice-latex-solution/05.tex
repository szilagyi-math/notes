\documentclass{szb-practice}

\usepackage{siunitx}
\sisetup{locale = DE} 

\title{Integrálátalakító tételek, mérnöki példák}
\area{Vektoranalízis}
\subject{Matematika G3}
\subjectCode{BMETE94BG03}
\date{Utoljára frissítve: \today}
\docno{5}

\begin{document}
\allowdisplaybreaks

\maketitle

% ~~~~~~~~~~~~~~~~~~~~~~~~~~~~~~~~~~~~~~~~~~~~~~~~~~~~~~~~~~~~~~~~~~~~~~~~~~~~~~
% 111111111111111111111111111111111111111111111111111111111111111111111111111111
% ~~~~~~~~~~~~~~~~~~~~~~~~~~~~~~~~~~~~~~~~~~~~~~~~~~~~~~~~~~~~~~~~~~~~~~~~~~~~~~
\subsection{Erőtér vizsgálata}

Adott egy $\rvec F(x; y) = \ijz{2xy}{x^2 + 2y}$ erőtér. Vizsgálja meg, hogy az
$\rvec F$ erőtér konzervatív-e! Amennyiben igen, adja meg egy olyan
potenciálfüggvényt, melyre $\varphi(0;0) = 0$. Számítsa ki a $P_1(0; 0)$ és
$P_2(1; 1)$ pontok közötti egyenes szakaszon végzett munkát!

Az $\rvec F$ erőtér konzervatív, hiszen $\partial_y F_x = \partial_x F_y$:
$$
  \pdv{}{y}(2xy) = 2x = \pdv{}{x}(x^2 + 2y)
  \text.
$$

A potenciálfüggvény:
$$
  \varphi(x; y)
  = \int_0^\xi F_x(\xi; y) \dd \xi + \int_0^\eta F_y(0; \eta) \dd \eta
  = \int_0^x 2\xi y \dd \xi + \int_0^y (0^2 + 2\eta) \dd \eta
  = x^2 y + y^2
  \text.
$$

A végzett munka:
$$
  \int_{(0;0) \to (1;1)} \scalar{\rvec F}{\dd \curvevec}
  = \varphi(1; 1) - \varphi(0; 0)
  = 1^2 \cdot 1 + 1^2 - 0
  = 2
  \text.
$$

Ha az erőtér nem lenne konzervatív, akkor először a két pontot összekötő
szakasz paraméterezését kellene megadni:
$$
  \curvesign(t) = \begin{bmatrix}
    t \\
    t
  \end{bmatrix}
  \text, \quad
  t \in [0; 1]
  \text, \quad
  \dot{\curvesign}(t) = \begin{bmatrix}
    1 \\
    1
  \end{bmatrix}
$$

A végzett munka:
$$
  \int_{\curveimage} \scalar{\rvec F}{\dd \curvevec}
  = \int_0^1 \scalar{\rvec F(\curvesign(t))}{\dot{\curvesign}(t)} \dd t
  = \int_0^1 \scalar{\begin{bmatrix}2t^2 \\ t^2 + 2t\end{bmatrix}}{\begin{bmatrix}1 \\ 1\end{bmatrix}} \dd t
  = \int_0^1 (3t^2 + 2t) \dd t
  = 2
  \text.
$$

% ~~~~~~~~~~~~~~~~~~~~~~~~~~~~~~~~~~~~~~~~~~~~~~~~~~~~~~~~~~~~~~~~~~~~~~~~~~~~~~
% 222222222222222222222222222222222222222222222222222222222222222222222222222222
% ~~~~~~~~~~~~~~~~~~~~~~~~~~~~~~~~~~~~~~~~~~~~~~~~~~~~~~~~~~~~~~~~~~~~~~~~~~~~~~
\subsection{Elektrosztatikus tér vizsgálata}

Egy $Q = \num{8.85}\pi\,\si{\milli\coulomb}$ nagyságú ponttöltés
közelében az elektrosztatikus tér:
$$
  \rvec E(\coordvec) = \frac{Q}{4\pi \varepsilon_0}
  \frac{\coordvec}{|\coordvec|^3}
  \qquad
  \coordvec \neq \nvec
  \qquad \varepsilon_0 = \SI[per-mode=symbol]{8,85e-12}{\farad\per\meter}
  \text.
$$

A vektormező konzervatív, hiszen az értelmezési tartományán
($\Reals \setminus \{0\}$) örvénymentes:
$$
  \rot \rvec E(\coordvec) = \begin{bmatrix}
    \partial_x \\ \partial_y \\ \partial_z
  \end{bmatrix} \times \begin{bmatrix}
    \frac{Q}{4\pi \varepsilon_0} \frac{x}{(x^2 + y^2 + z^2)^{3/2}} \\
    \frac{Q}{4\pi \varepsilon_0} \frac{y}{(x^2 + y^2 + z^2)^{3/2}} \\
    \frac{Q}{4\pi \varepsilon_0} \frac{z}{(x^2 + y^2 + z^2)^{3/2}}
  \end{bmatrix} = \begin{bmatrix}
    \frac{Q}{4\pi \varepsilon_0} \frac{-3}{2(x^2 + y^2 + z^2)^{5/2}} (2yz - 2zy) \\
    \frac{Q}{4\pi \varepsilon_0} \frac{-3}{2(x^2 + y^2 + z^2)^{5/2}} (2zx - 2xz) \\
    \frac{Q}{4\pi \varepsilon_0} \frac{-3}{2(x^2 + y^2 + z^2)^{5/2}} (2xy - 2yx)
  \end{bmatrix} = \nvec
  \text.
$$

A gömbszimmetria miatt:
$$
  E(r) = |\rvec E(\coordvec)| = \frac{Q}{4\pi \varepsilon_0 r^2}
  \text.
$$
Ennek $r$ szerinti primitív függvénye:
$$
  \Phi(r) = -\frac{Q}{4\pi \varepsilon_0 r}
  \text.
$$
A potenciálfüggvény tehát:
$$
  \varphi(\coordvec) = \Phi(|\coordvec|) = -\frac{Q}{4\pi \varepsilon_0 |\coordvec|}
  \text.
$$

A mező által végzett munka a $P_1(1; 0; 0)$ és $P_2(2; 0; 0)$ pontok között:
$$
  W_\text{mező}
  = q \int_{(1;0;0) \to (2;0;0)} \scalar{\rvec E}{\dd \curvevec}
  = q (\varphi(2; 0; 0) - \varphi(1; 0; 0))
  = \frac{q \, Q}{4\pi \varepsilon_0} \left(-\frac{1}{2} + 1\right)
  = \frac{q \, Q}{8\pi \varepsilon_0}
  \text.
$$
A töltés által végzett munka ennek ellentettje:
$$
  W_\text{töltés} = -W_\text{mező}
  = -\frac{q \, Q}{8\pi \varepsilon_0}
  = \SI{-125}{\joule}
  \text.
$$

% ~~~~~~~~~~~~~~~~~~~~~~~~~~~~~~~~~~~~~~~~~~~~~~~~~~~~~~~~~~~~~~~~~~~~~~~~~~~~~~
% 333333333333333333333333333333333333333333333333333333333333333333333333333333
% ~~~~~~~~~~~~~~~~~~~~~~~~~~~~~~~~~~~~~~~~~~~~~~~~~~~~~~~~~~~~~~~~~~~~~~~~~~~~~~
\subsection{Áramjárta vezető}

Egy nagyon hosszú, áramjárta vezető belsejében a mágneses indukció jó
közelítéssel lineárisan változik a keresztmetszetben:
$$
  \rvec B(\coordv) = \ijk{ky}{-kx}{0}
$$

A vektormezőnek létezik vektorpotenciálja, hiszen $\Div \rvec B = 0$:
$$
  \Div \rvec B(\coordv)
  = \pdv{}{x}(ky) + \pdv{}{y}(-kx) + \pdv{}{z}(0)
  = 0 + 0 + 0
  = 0
  \text.
$$

Keressük a $\rvec B = \rot \rvec A$ alakú vektorpotenciált, ahol $A_z = 0$.
Ekkor:
\begin{align*}
  A_x & = \int_0^z B_y(x; y; \zeta) \dd \zeta = \int_0^z -kx \dd \zeta = -kxz
  \text,                                                                         \\
  A_y & = -\int_0^z B_x(x; y; \zeta) \dd \zeta + \int_0^x B_z(\xi; y; 0) \dd \xi
  = -\int_0^z ky \dd \zeta + 0 = -kyz
  \text.
\end{align*}

Mivel $\rvec B$ mértékegysége a $\si{\tesla}$, így $k$ mértékegysége
$\si{\tesla\per\meter}$.

% ~~~~~~~~~~~~~~~~~~~~~~~~~~~~~~~~~~~~~~~~~~~~~~~~~~~~~~~~~~~~~~~~~~~~~~~~~~~~~~
% 444444444444444444444444444444444444444444444444444444444444444444444444444444
% ~~~~~~~~~~~~~~~~~~~~~~~~~~~~~~~~~~~~~~~~~~~~~~~~~~~~~~~~~~~~~~~~~~~~~~~~~~~~~~

\subsection{Vektormező ismeretlen komponense}

Legyen $\rvec v(\coordvec) = \ijk{y \sin x}{z^2 \cos y - \cos x}{v_3}$.
Határozza meg $v_3$-at, ha tudjuk, hogy $\rvec v$ tetszőleges zárt
görbén vett vonalintegrálja zérus!

Tetszőleges zárt görbén vett vonalintegrál zérus, ha a vektormező örvénymentes,
vagyis:
$$
  \rot \rvec v = \begin{bmatrix}
    \partial_x \\ \partial_y \\ \partial_z
  \end{bmatrix} \times \begin{bmatrix}
    y \sin x \\ z^2 \cos y - \cos x \\ v_3
    % \end{bmatrix} = \begin{bmatrix}
    %   \partial_y v_3 - \partial_z (z^2 \cos y - \cos x) \\
    %   \partial_z (y \sin x) - \partial_x v_3            \\
    %   \partial_x (z^2 \cos y - \cos x) - \partial_y (y \sin x)
  \end{bmatrix} = \begin{bmatrix}
    \partial_y v_3 - 2z \cos y \\
    0 - \partial_x v_3         \\
    \sin x - \sin x
  \end{bmatrix} = \begin{bmatrix}
    0 \\ 0 \\ 0
  \end{bmatrix}
  \text.
$$

A második koordináta alapján $v_3$ nem függ $x$-től, így $v_3 = v_3(y; z)$. Az
első koordináta alapján:
$$
  v_3 = \int 2z \cos y \dd y = 2z \sin y + f(z)
  \text,
$$
ahol $f(z)$ tetszőleges $z$-függvény, nem befolyásolja a rotációt.

% ~~~~~~~~~~~~~~~~~~~~~~~~~~~~~~~~~~~~~~~~~~~~~~~~~~~~~~~~~~~~~~~~~~~~~~~~~~~~~~
% 555555555555555555555555555555555555555555555555555555555555555555555555555555
% ~~~~~~~~~~~~~~~~~~~~~~~~~~~~~~~~~~~~~~~~~~~~~~~~~~~~~~~~~~~~~~~~~~~~~~~~~~~~~~
\subsection{Áramló folyadék keresztmetszet menti cirkulációja}

Egy $R = 1$ sugarú, kör keresztmetszetű, $z$ tengellyel egybeeső
szimmetriavonalú hengerben áramló folyadék sebességét a
$$
  \rvec v(\coordv) = \ijk{2xy + z}{x^2 + z}{y - x}
$$
vektormező írja le. A $z = 1$ síkban lévő keresztmetszet menti cirkuláció
kiszámításához alkalmazzuk a Stokes-tételt:
$$
  \oint_{\partial \surfimage} \scalar{\rvec v}{\dd \curvevec}
  = \iint_{\surfimage} \scalar{\rot \rvec v}{\dd \surfvec}
  \text.
$$
A sebességmező rotációja:
$$
  \rot \rvec v(\coordv)
  = \begin{bmatrix}
    \partial_x \\ \partial_y \\ \partial_z
  \end{bmatrix} \times \begin{bmatrix}
    2xy + z \\ x^2 + z \\ y - x
  \end{bmatrix} = \begin{bmatrix}
    1 - 1 \\ 1 + 1 \\ 2x - 2x
  \end{bmatrix} = \begin{bmatrix}
    0 \\ 2 \\ 0
  \end{bmatrix}
  \text.
$$
A kör keresztmetszet normálvektora a $z$ tengely irányába mutat, így a
cirkuláció zérus.

Ha ezt nem vettük volna észre, akkor a felület paraméterezése:
$$
  \surfsign(s; t) = \begin{bmatrix}
    s \cos t \\
    s \sin t \\
    1
  \end{bmatrix}
  \text, \quad
  \begin{array}{l}
    s \in [0; 1] \\
    t \in [0; 2\pi]
  \end{array}
  \text, \quad
  \rvec n
  = \pdv{\surfsign}{s} \times \pdv{\surfsign}{t}
  = \begin{bmatrix}
    \cos t \\
    \sin t \\
    0
  \end{bmatrix} \times \begin{bmatrix}
    -s \sin t \\
    s \cos t  \\
    0
  \end{bmatrix}
  = \begin{bmatrix}
    0 \\
    0 \\
    s
  \end{bmatrix}
  \text.
$$

Az integrál értéke:
$$
  \iint_{\surfimage} \scalar{\rot \rvec v}{\dd \surfvec}
  = \int_0^{2\pi} \int_0^1 \scalar{\begin{bmatrix}0 \\ 2 \\ 0\end{bmatrix}}{\begin{bmatrix}0 \\ 0 \\ s\end{bmatrix}} \dd s \dd t
  = \int_0^{2\pi} \int_0^1 0 \dd s \dd t
  = 0
  \text.
$$

Ha a Stokes-tételt sem ismerjük, akkor a kör kerületének
paraméterezése:
$$
  \curvesign(t) = \begin{bmatrix}
    \cos t \\
    \sin t \\
    1
  \end{bmatrix}
  \text, \quad
  t \in [0; 2\pi]
  \text, \quad
  \dot{\curvesign}(t) = \begin{bmatrix}
    -\sin t \\
    \cos t  \\
    0
  \end{bmatrix}
  \text.
$$
A vektormező átparaméterezése:
$$
  \rvec v(\curvesign(t))
  = \begin{bmatrix}
    2 \cos t \sin t + 1 \\
    \cos^2 t + 1        \\
    \sin t - \cos t
  \end{bmatrix}
  \text.
$$
Az integrál értéke:
\begin{align*}
  \oint_{\partial \surfimage} \scalar{\rvec v}{\dd \curvevec}
   & = \int_0^{2\pi} \scalar{\rvec v(\curvesign(t))}{\dot{\curvesign}(t)} \dd t
  \\
   & = \int_0^{2\pi} (2 \cos t \sin t + 1)(-\sin t) + (\cos^2 t + 1) \cos t + (\sin t - \cos t) \cdot 0 \dd t
  \\
   & = \int_0^{2\pi} -2 \cos t \sin^2 t - \sin t + \cos^3 t + \cos t \dd t
  = 0
  \text.
\end{align*}

% ~~~~~~~~~~~~~~~~~~~~~~~~~~~~~~~~~~~~~~~~~~~~~~~~~~~~~~~~~~~~~~~~~~~~~~~~~~~~~~
% 666666666666666666666666666666666666666666666666666666666666666666666666666666
% ~~~~~~~~~~~~~~~~~~~~~~~~~~~~~~~~~~~~~~~~~~~~~~~~~~~~~~~~~~~~~~~~~~~~~~~~~~~~~~
\subsection{Forgásparaboloid peremén vett integrál}

Jelölje $\surfimage$ az $x^2 + y^2 - z^2 = 1$ egyenletű forgáshiperboloid
$z = -1$ és $z = 1$ síkok közötti részét. Határozza meg a
$\rvec v(\coordvec) = \ijk{x^2}{y^3}{z^4}$ vektormező $\surfimage$ peremén vett
integrálját!

Használjuk a Stokes-tételt. A vektormező rotációja:
$$
  \rot \rvec v(\coordvec)
  = \begin{bmatrix}
    \partial_x \\ \partial_y \\ \partial_z
  \end{bmatrix} \times \begin{bmatrix}
    x^2 \\ y^3 \\ z^4
  \end{bmatrix} = \begin{bmatrix}
    0 - 0 \\ 0 - 0 \\ 0 - 0
  \end{bmatrix} = \begin{bmatrix}
    0 \\ 0 \\ 0
  \end{bmatrix} = \nvec
  \text.
$$
A Stokes-tétel alapján:
$$
  \oint_{\partial \surfimage} \scalar{\rvec v}{\dd \curvevec}
  = \int_{\surfimage} \scalar{\rot \rvec v}{\dd \surfvec}
  = \int_{\surfimage} \scalar{\nvec}{\dd \surfvec}
  = 0
  \text.
$$

Amennyiben a Stokes-tételt nem ismerjük, akkor a peremkörök paraméterezése:
$$
  \curvesign_{1,2}(t) = \begin{bmatrix}
    \cos t \\
    \sin t \\
    \pm 1
  \end{bmatrix}
  \text, \quad
  t \in [0; 2\pi]
  \text, \quad
  \dot{\curvesign}_{1,2}(t) = \begin{bmatrix}
    -\sin t \\
    \cos t  \\
    0
  \end{bmatrix}
  \text.
$$

A vektormező átparaméterezése:
$$
  \rvec v(\curvesign_{1,2}(t))
  = \begin{bmatrix}
    \cos^2 t \\
    \sin^3 t \\
    (\pm 1)^4
  \end{bmatrix} = \begin{bmatrix}
    \cos^2 t \\
    \sin^3 t \\
    1
  \end{bmatrix}
  \text.
$$

A $z = 1$ síkon lévő kör menti integrál:
\begin{align*}
  \oint_{\curveimage_1} \scalar{\rvec v}{\dd \curvevec}
   & = \int_0^{2\pi} \scalar{\rvec v(\curvesign_1(t))}{\dot{\curvesign}_1(t)} \dd t
  \\
   & = \int_0^{2\pi} \scalar{\begin{bmatrix}\cos^2 t \\ \sin^3 t \\ 1\end{bmatrix}}{\begin{bmatrix}-\sin t \\ \cos t \\ 0\end{bmatrix}} \dd t
  \\
   & = \int_0^{2\pi} -\cos^2 t \sin t + \sin^3 t \cos t + 0 \dd t
  = 0
  \text.
\end{align*}

A $z = -1$ síkon lévő kör menti integrál:
\begin{align*}
  \oint_{\curveimage_2} \scalar{\rvec v}{\dd \curvevec}
   & = \int_0^{2\pi} \scalar{\rvec v(\curvesign_2(t))}{\dot{\curvesign}_2(t)} \dd t
  \\
   & = \int_0^{2\pi} \scalar{\begin{bmatrix}\cos^2 t \\ \sin^3 t \\ 1\end{bmatrix}}{\begin{bmatrix}-\sin t \\ \cos t \\ 0\end{bmatrix}} \dd t
  \\
   & = \int_0^{2\pi} -\cos^2 t \sin t + \sin^3 t \cos t + 0 \dd t
  = 0
  \text.
\end{align*}

A két integrál összege 0, ami megegyezik a Stokes-tétel segítségével kapott
eredménnyel.

% ~~~~~~~~~~~~~~~~~~~~~~~~~~~~~~~~~~~~~~~~~~~~~~~~~~~~~~~~~~~~~~~~~~~~~~~~~~~~~~
% 777777777777777777777777777777777777777777777777777777777777777777777777777777
% ~~~~~~~~~~~~~~~~~~~~~~~~~~~~~~~~~~~~~~~~~~~~~~~~~~~~~~~~~~~~~~~~~~~~~~~~~~~~~~
\subsection{Háromszögvonal menti integrál}

Integrálja a $\rvec v(\coordvec) = \ijk{y^2}{z^2}{x^2}$ vektormezőt az
$A(1;0;0)$, $B(0;1;0)$ és $C(0;0;1)$ csúcsokkal meghatározott háromszögvonal
mentén!

Alkalmazzuk a Stokes-tételt. A vektormező rotációja:
$$
  \rot \rvec v(\coordvec)
  = \begin{bmatrix}
    \partial_x \\ \partial_y \\ \partial_z
  \end{bmatrix} \times \begin{bmatrix}
    y^2 \\ z^2 \\ x^2
  \end{bmatrix} = \begin{bmatrix}
    -2z \\ -2x \\ -2y
  \end{bmatrix}
  \text.
$$

A háromszög által meghatározott tértartomány paraméterezése:
$$
  \surfsign(s; t) = \begin{bmatrix}
    s \\
    t \\
    1 - s - t
  \end{bmatrix}
  \text, \quad
  \begin{array}{l}
    s \in [0; 1] \\
    t \in [0; 1 - s]
  \end{array}
  \text, \quad
  \rvec n
  = \pdv{\surfsign}{s} \times \pdv{\surfsign}{t}
  = \begin{bmatrix}
    1 \\
    0 \\
    -1
  \end{bmatrix} \times \begin{bmatrix}
    0 \\
    1 \\
    -1
  \end{bmatrix} = \begin{bmatrix}
    1 \\
    1 \\
    1
  \end{bmatrix}
  \text.
$$

Az integrál értéke:
\begin{align*}
  \iint_{\surfimage} \scalar{\rot \rvec v}{\dd \surfvec}
   & = \int_0^1 \int_0^{1 - s} \scalar{\begin{bmatrix}-2(1 - s - t) \\ -2s \\ -2t\end{bmatrix}}{\begin{bmatrix}1 \\ 1 \\ 1\end{bmatrix}} \dd t \dd s
  \\
   & = \int_0^1 \int_0^{1 - s} -2 + 2s + 2t - 2s - 2t \dd t \dd s
  \\
   & = \int_0^1 \int_0^{1 - s} -2 \dd t \dd s
  = \int_0^1 -2(1 - s) \dd s
  = -2 + 1 = -1
  \text.
\end{align*}

Ha a Stokes-tételt nem ismerjük, akkor a háromszög éleinek paraméterezése:
\begin{align*}
  \curvesign_{AB}(t) & = \begin{bmatrix}
                           1 - t \\
                           t     \\
                           0
                         \end{bmatrix}
  \text, \quad
  t \in [0; 1]
  \text, \quad
  \dot{\curvesign}_{AB}(t) = \begin{bmatrix}
                               -1 \\
                               1  \\
                               0
                             \end{bmatrix}
  \text, \quad
  \rvec v(\curvesign_{AB}(t)) = \begin{bmatrix}
                                  t^2 \\
                                  0   \\
                                  (1 - t)^2
                                \end{bmatrix}
  \text,                                \\
  \curvesign_{BC}(t) & = \begin{bmatrix}
                           0     \\
                           1 - t \\
                           t
                         \end{bmatrix}
  \text, \quad
  t \in [0; 1]
  \text, \quad
  \dot{\curvesign}_{BC}(t) = \begin{bmatrix}
                               0  \\
                               -1 \\
                               1
                             \end{bmatrix}
  \text, \quad
  \rvec v(\curvesign_{BC}(t)) = \begin{bmatrix}
                                  (1 - t)^2 \\
                                  t^2       \\
                                  0
                                \end{bmatrix}
  \text,                                \\
  \curvesign_{CA}(t) & = \begin{bmatrix}
                           t \\
                           0 \\
                           1 - t
                         \end{bmatrix}
  \text, \quad
  t \in [0; 1]
  \text, \quad
  \dot{\curvesign}_{CA}(t) = \begin{bmatrix}
                               1 \\
                               0 \\
                               -1
                             \end{bmatrix}
  \text, \quad
  \rvec v(\curvesign_{CA}(t)) = \begin{bmatrix}
                                  0         \\
                                  (1 - t)^2 \\
                                  t^2
                                \end{bmatrix}
  \text.
\end{align*}

Az egyes élek menti integrálok:
\begin{align*}
  \int_{\curveimage_{AB}} \scalar{\rvec v}{\dd \curvevec}
   & = \int_0^1 \scalar{\rvec v(\curvesign_{AB}(t))}{\dot{\curvesign}_{AB}(t)} \dd t
  = \int_0^1 \scalar{\begin{bmatrix}t^2 \\ 0 \\ (1 - t)^2\end{bmatrix}}{\begin{bmatrix}-1 \\ 1 \\ 0\end{bmatrix}} \dd t
  = \int_0^1 -t^2 \dd t
  = -\frac{1}{3}
  \text,                                                                             \\
  \int_{\curveimage_{BC}} \scalar{\rvec v}{\dd \curvevec}
   & = \int_0^1 \scalar{\rvec v(\curvesign_{BC}(t))}{\dot{\curvesign}_{BC}(t)} \dd t
  = \int_0^1 \scalar{\begin{bmatrix}(1 - t)^2 \\ t^2 \\ 0\end{bmatrix}}{\begin{bmatrix}0 \\ -1 \\ 1\end{bmatrix}} \dd t
  = \int_0^1 -t^2 \dd t
  = -\frac{1}{3}
  \text,                                                                             \\
  \int_{\curveimage_{CA}} \scalar{\rvec v}{\dd \curvevec}
   & = \int_0^1 \scalar{\rvec v(\curvesign_{CA}(t))}{\dot{\curvesign}_{CA}(t)} \dd t
  = \int_0^1 \scalar{\begin{bmatrix}0 \\ (1 - t)^2 \\ t^2\end{bmatrix}}{\begin{bmatrix}1 \\ 0 \\ -1\end{bmatrix}} \dd t
  = \int_0^1 -t^2 \dd t
  = -\frac{1}{3}
  \text.
\end{align*}

\end{document}