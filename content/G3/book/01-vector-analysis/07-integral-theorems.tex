\clearpage
\section{Integrálátalakító tételek}

\begin{theorem}[Stokes-tétel]
  Legyen $\surfsign: \surfdomain\subset\Reals^2 \to \surfimage \subset \Reals^3$
  irányított, parametrizált, elemi felület. Legyen továbbá $\rvec v : \Reals^3
    \to \Reals^3$ legalább egyszer folytonosan differenciálható vektormező.
  Jelölje az $\curvesign: \curvedomain\subset\Reals \to \partial\surfimage =
    \curveimage$ a $\surfsign$ peremét indukált, jobbézszabály szerinti
  irányítással. Ekkor:
  $$
    \iint_{\surfimage} \scalar{\rot \rvec v}{\dd \surfvec}
    =
    \oint_{\partial \surfimage} \scalar{\rvec v}{\dd \curvevec}
    \text.
  $$
\end{theorem}

\begin{note}
  Ha $\rvec v$ skalárpotenciálos, akkor az integrál értéke zérus, hiszen
  $\rot \rvec v = \rot \grad \varphi = \rvec 0$.
\end{note}

\begin{definition}[Elemi tértartomány]
  $\volsign: \voldomain \subset \Reals^3 \rightarrow \volimage \subset \Reals^3$
  elemi tértartomány, ha legalább egyszer folytonosan differenciálható
  leképezés. Ekkor
  $$
    \det \left( \DD \volsign(r; s; t) \right)
    = \norma{
      \frac{\partial \volsign(r; s; t)}{\partial r \, \partial s \, \partial t}
    } \neq 0
    \text.
  $$
\end{definition}

\begin{definition}[Térfogat]
  Készítsünk egy olyan beleírt ($c_i$) és körülírt ($C_i$) kockarendszert,
  melyekre igaz, hogy $c_i \cap c_j$ csak lap, él, vagy csúcs lehet. Ekkor
  fennáll, hogy:
  $$
    \bigcup_i c_i
    \subset \volimage \subset
    \bigcup_j C_j
    \text.
  $$
  Finomítsuk minden határon túl ezeket a kockarendszereket. Ha ezek közös
  határértékhez tartanak, akkor:
  $$
    \operatorname{Vol} \volimage
    = \iiint_{\volimage} \dd \volscalar
    = \iiint_{\volimage} \left|
    \det \left( \DD \volsign(r; s; t) \right)
    \right|
    \dd r \dd s \dd t
    \text.
  $$
\end{definition}

\begin{note}
  Pozitív az irányítás, ha $\det \DD \volimage > 0$.
\end{note}

\begin{definition}[Skalármező térfogaton vett skalárértékű integrálja][nobreak]
  Legyen $\volsign: \voldomain \subset \Reals^3 \rightarrow \volimage \subset
    \Reals^3$ irányított, parametrizált, elemi tértartomány. Legyen továbbá
  $\varphi : \Reals^3 \rightarrow \Reals$ folytonos skalármező.
  Ekkor a $\varphi$ térfogaton vett integrálja:
  $$
    \iiint_{\volimage} \varphi \dd \volscalar
    = \iiint_{\voldomain} \varphi \left( \volsign(r; s; t) \right)
    \det \left( \DD \volsign(r; s; t) \right)
    \dd r \dd s \dd t
    \text.
  $$
\end{definition}

\begin{theorem}[Gauss-Osztogradszkij-tétel]
  Legyen $\volsign: \voldomain \subset \Reals^3 \rightarrow \volimage \subset
    \Reals^3$, irányított,parametrizált tértartomány. Legyen továbbá $\rvec v:
    \Reals^3 \rightarrow \Reals^3$ legalább egyszer folytonosan
  differenciálható vektormező. Jelölje a $\partial \volimage =
    \surfimage$ az $\volsign$ peremét indukált irányítással. Ekkor:
  $$
    \iiint_{\volimage} \Div \rvec v \dd \volscalar
    =
    \oiint_{\partial \volimage} \scalar{\rvec v}{\dd \surfvec}
    \text.
  $$
\end{theorem}

\begin{note}
  Ha $\rvec v$ vektorpotenciálos, akkor az integrál értéke zérus, hiszen
  $\Div \rvec v = \Div \rot \rvec u = 0$.
\end{note}

\begin{theorem}[Green-tétel asszimetrikus alakja]
  Legyenek $\varphi; \psi: \mathbb R^3 \rightarrow \mathbb R$ kétszeresen
  folytonos skalármezők, $\volimage \subset \mathbb R^3$ parametrizált,
  irányított tértartomány, $\partial \volimage = \surfimage$ perem indukált
  irányítással. Ekkor:
  $$
    \iiint_{\volimage}
    \psi \, \Delta \varphi +
    \scalar{\grad \psi}{\grad \varphi}
    \dd \volscalar
    =
    \oiint_{\partial \volimage} \scalar{\psi \grad \varphi}{\dd \surfvec}
    \text.
  $$
\end{theorem}

\begin{theorem}[Green-tétel szimmetrikus alakja]
  Legyenek $\varphi; \psi: \mathbb R^3 \rightarrow \mathbb R$ kétszeresen
  folytonos skalármezők, $\volimage \subset \mathbb R^3$ parametrizált,
  irányított tértartomány, $\partial \volimage = \surfimage$ perem indukált
  irányítással. Ekkor:
  $$
    \iiint_{\volimage}
    \psi \, \Delta \varphi + \varphi \, \Delta \psi
    \; \dd \volscalar
    =
    \oiint_{\partial \volimage}
    \scalar{\psi \grad \varphi - \varphi \grad \psi}{\dd \surfvec}
  $$
\end{theorem}