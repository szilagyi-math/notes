\clearpage
\section{Lineáris leképezések}

\begin{definition}[Lineáris leképezés adjungáltja]
  Legyen $(V; \, \scalar{\;}{\;})$ Euklideszi tér, $\varphi: V \rightarrow V$
  lineáris leképezés. A $\varphi^* : V \rightarrow V$ lineáris leképezést
  a $\varphi$ leképezés adjungáltjának hívjuk, ha
  $\forall \rvec v_1; \rvec v_2 \in V$-re:
  $$
    \scalar{\varphi(\rvec v_1)}{\rvec v_2}
    = \scalar{\rvec v_1}{\varphi^*(\rvec v_2)}
  $$
\end{definition}

\begin{statement}
  $\left( \varphi^* \right)^* = \varphi$ -- Idempotencia

  \begin{proof}
    \vspace{-2em}
    $$
      \scalar{\left( \varphi^* \right)^*(\rvec v_1)}{\rvec v_2}
      = \scalar{\rvec v_1}{\varphi^*(\rvec v_2)}
      = \scalar{\varphi(\rvec v_1)}{\rvec v_2}
      \text.
    $$
  \end{proof}
\end{statement}

\begin{note}[$\varphi^*$ mátrix reprezentációja]
  Reprezentálja $\varphi$-t $\rmat A$,
  $\varphi^*$-ot pedig $\rmat A^*$:
  \begin{align*}
    \scalar{\varphi(\rvec v)}{\rvec w}
     & =
    \scalar{\rvec v}{\varphi^*(\rvec w)}
    \\
    \rmat A\rvec v \cdot \rvec w
     & =
    \rvec v \cdot \rmat A^* \rvec w
    \\
    \begin{bmatrix}
      a_{11} & a_{12} \\
      a_{21} & a_{22}
    \end{bmatrix}
    \begin{bmatrix}
      v_1 \\
      v_2
    \end{bmatrix}
    \cdot
    \begin{bmatrix}
      w_1 \\
      w_2
    \end{bmatrix}
     & =
    \begin{bmatrix}
      v_1 \\
      v_2
    \end{bmatrix}
    \cdot
    \begin{bmatrix}
      a_{11}^* & a_{12}^* \\
      a_{21}^* & a_{22}^*
    \end{bmatrix}
    \begin{bmatrix}
      w_1 \\
      w_2
    \end{bmatrix}
    \\
    % \begin{bmatrix}
    %   a_{11} v_1 + a_{12} v_2 \\
    %   a_{21} v_1 + a_{22} v_2
    % \end{bmatrix}
    % \cdot
    % \begin{bmatrix}
    %   w_1 \\
    %   w_2
    % \end{bmatrix}
    %  & =
    % \begin{bmatrix}
    %   v_1 \\
    %   v_2
    % \end{bmatrix}
    % \cdot
    % \begin{bmatrix}
    %   a_{11}^* w_1 + a_{12}^* w_2 \\
    %   a_{21}^* w_1 + a_{22}^* w_2
    % \end{bmatrix}
    % \\[2mm]
    \scalebox{.92}{$
        \rected{$a_{11}$}{green!40!black} v_1 w_1
        + \rected{$a_{12}$}{primaryColor} v_2 w_1
        + \rected{$a_{21}$}{secondaryColor} v_1 w_2
        + \rected{$a_{22}$}{ternaryColor} v_2 w_2
      $}
     & = \scalebox{.92}{$
        \rected{$a_{11}^*$}{green!40!black} w_1 v_1
        + \rected{$a_{12}^*$}{secondaryColor} w_2 v_2
        + \rected{$a_{21}^*$}{primaryColor} w_1 v_2
        + \rected{$a_{22}^*$}{ternaryColor} w_2 v_2
      $}
  \end{align*}
  % \begin{alignat*}{4}
  %   \scalar{\rmat A\rvec v}{\rvec w}
  %    & = \rected{$a_{11}$}{green!40!black} v_1 w_1
  %    & + \rected{$a_{12}$}{primaryColor} v_2 w_1
  %    & + \rected{$a_{21}$}{secondaryColor} v_1 w_2
  %    & + \rected{$a_{22}$}{ternaryColor} v_2 w_2
  %   \\
  %   \scalar{\rmat A^*\rvec w}{\rvec v}
  %    & = \rected{$a_{11}^*$}{green!40!black} w_1 v_1
  %    & + \rected{$a_{12}^*$}{secondaryColor} w_2 v_2
  %    & + \rected{$a_{21}^*$}{primaryColor} w_1 v_2
  %    & + \rected{$a_{22}^*$}{ternaryColor} w_2 v_2
  % \end{alignat*}
  Megállapthatjuk, hogy $\rmat A^* = \rmat A^\T$.
\end{note}

\begin{statement}
  Szimmetrikus leképezés adjungáltja önmaga.
\end{statement}

\begin{statement}[Leképezés felbontása][nobreak]
  Legyen $\varphi \in \operatorname{End}(V)$, ekkor $!\exists$ olyan
  $\mathcal A$ és $\mathcal S$ antiszimmetrikus és szimmetrikus leképezés, ahol
  $\varphi = \mathcal A + \mathcal S$, melyek az endomorfizmusok vektorterét 2
  diszjunkt halmazra bontják:
  $$
    \mathcal A := \frac{\varphi - \varphi^*}{2}
    \qquad\text{és}\qquad
    \mathcal S := \frac{\varphi + \varphi^*}{2}
    \text.
  $$

  \begin{proof}[Unicitás]
    Tegyük fel hogy $\varphi$ előáll $\mathcal A_1 + \mathcal S_1$ és
    $\mathcal A_2 + \mathcal S_2$ összegeként is. Vonjuk ki egymásból
    a két egyenletet, majd vegyük mindkét oldal adjungáltját!
    \begin{align*}
      \mathcal O
       & = \varphi - \varphi
      = \left( \mathcal A_1 - \mathcal A_2 \right)
      + \left( \mathcal S_1 - \mathcal S_2 \right)
      = \mathcal{\overline A} + \mathcal{\overline S}
      \\
      \mathcal O^*
      = \mathcal O
       & = \mathcal{\overline A^*} + \mathcal{\overline S^*}
      = -\mathcal{\overline A} + \mathcal{\overline S}
    \end{align*}
    Az előző két egyenletből következik, hogy
    $\mathcal{O = \overline A = \overline S}$. Feltevésünk hamisnak bizonyult.
  \end{proof}
\end{statement}

\begin{note}[Reguláris mátrix felbontása]
  Egy $\rmat M \in \mathcal M_{n \times n}$ mátrix felbontható szimmetrikus és
  ferdeszimmetrikus (antiszimmetrikus) részekre:
  $$
    \rmat S = \frac{\rmat M + \rmat M^\T}{2}
    \qquad\text{és}\qquad
    \rmat A = \frac{\rmat M - \rmat M^\T}{2}
    \text.
  $$
  Ha mátrixunk $3 \times 3$-as:
  $$
    \rmat M
    = \rmat S + \rmat A
    = \underbrace{\begin{bmatrix}
        a & d & e \\
        d & b & f \\
        e & f & c
      \end{bmatrix}}_{\text{szimmetrikus}}
    + \underbrace{\begin{bmatrix}
        0  & x  & y \\
        -x & 0  & z \\
        -y & -z & 0
      \end{bmatrix}}_{\text{ferdeszimmetrikus}}
    \text.
  $$
  Általános esetben:
  $$
    \dim \rmat A = \frac{n(n + 1)}{2}
    \text,\qquad
    \dim \rmat S = \frac{n(n - 1)}{2}
    \text.
  $$
\end{note}

\begin{note}
  Az antiszimmetrikus leképezések és a
  $V$-beli vektorok között tudunk egy-egyértelmű
  hozzárendelést találni:
  \begin{equation*}
    \rmat A \in \mathcal A
    \hspace{2.5mm} \leftrightarrow \hspace{2.5mm}
    \rvec v \in V
  \end{equation*}

  Keressünk egy olyan $\rvec v$ vektort,
  melyre teljesül az alábbi egyenlet:
  \begin{align*}
    \rmat A\rvec w
     & = \rvec v \times \rvec w
    \\[
    2mm
    ]
    \begin{bmatrix}
      0       & a_{12}  & a_{13} \\
      -a_{12} & 0       & a_{23} \\
      -a_{13} & -a_{23} & 0
    \end{bmatrix}
    \begin{bmatrix}
      w_1 \\
      w_2 \\
      w_3
    \end{bmatrix}
     & =
    \begin{bmatrix}
      v_1 \\
      v_2 \\
      v_3
    \end{bmatrix}
    \times
    \begin{bmatrix}
      w_1 \\
      w_2 \\
      w_3
    \end{bmatrix}
    \\[2mm]
    \begin{bmatrix}
      \rected{$+a_{12}$}{primaryColor}   w_2 \rected{$+a_{13}$}{secondaryColor} w_3 \\
      \rected{$-a_{12}$}{primaryColor}   w_1 \rected{$+a_{23}$}{ternaryColor}   w_3 \\
      \rected{$-a_{13}$}{secondaryColor} w_1 \rected{$-a_{23}$}{ternaryColor}   w_2
    \end{bmatrix}
     & =
    \begin{bmatrix}
      \rected{$v_2$}{secondaryColor} w_3 \rected{$-v_3$}{primaryColor}    w_2 \\
      \rected{$v_3$}{primaryColor}   w_1 \rected{$-v_1$}{ternaryColor}    w_3 \\
      \rected{$v_1$}{ternaryColor}   w_2 \rected{$-v_2$}{secondaryColor}  w_1
    \end{bmatrix}
    \qquad\Rightarrow\qquad
    \rvec v
    =
    \begin{bmatrix}
      -a_{23} \\
      a_{13}  \\
      -a_{12} \\
    \end{bmatrix}
  \end{align*}
\end{note}

\begin{definition}[Vektorinvariáns]
  Egy antiszimmetrikus lineáris transzformáció mindig leírható egy rögzített
  vektorral való keresztszorzással. Ez a vektor a leképezés vektorinvariánsa.
\end{definition}

\begin{note}
  Csak ortonormált, Descartes-féle bázisban számítható az előbbi módszerrel egy
  leképezés vektorinvariánsa.
\end{note}

\begin{definition}[Lineáris transzformáció nyoma][nobreak]
  Egy lineáris transzformáció főátlójában lévő elemek összege minden
  koordinátarendszerben ugyanannyi, tehát a koordináta-transzformáció
  nem befolyásolja. Ezt nevezzük a lineáris leképezés nyomának. (trace / spur)
\end{definition}