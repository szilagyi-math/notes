\clearpage
\section{Felületmenti integrál}

\begin{definition}[Reguláris felület]
  Legyen $\surfimage \subseteq \Reals^3$. Azt mondjuk, hogy az $\surfsign$
  reguláris felület, ha $\forall \rvec p \in \surfimage$ ponthoz megadható
  olyan $\rvec p$-t tartalmazó $V \subset \Reals^3$ nyílt halmaz és
  $\surfsign : \surfdomain \subseteq \Reals^2 \rightarrow \surfimage \cap V$
  leképezés, melyre teljesülnek az alábbiak:
  \begin{itemize}
    \item $\surfsign$ differenciálható homeomorfizmus,

    \item $\surfsign$ immerzió (derivált leképezése injektív).
  \end{itemize}
  Ha ezek teljesülnek, akkor $\surfsign$-t parametrációnak,
  $V \cap \surfimage$-t koordinátakörnyezetnek nevezzük.
\end{definition}

\begin{definition}[Elemi felület]
  A $\surfsign: \surfdomain \subseteq \Reals^2 \rightarrow \surfimage \subseteq
    \Reals^3$ elemi felület, ha $\surfsign$ legalább egyszer differenciálható és
  injektív.
\end{definition}

\begin{note}
  $\partial(\; [a; b] \times [a; b] \;)$ a paramétertartomány pereme.

  A felület irányítható, ha megadható rajta egy differenciálható
  egységvektormező.

  \begin{minipage}{.5\textwidth}
    \centering
    \begin{tikzpicture}[
        scale=2/3,
        every pin/.style={
            fill=yellow!50!white,
            rectangle,
            rounded corners=3pt,
          },
        small dot/.style={
            fill=red!60!black,
            circle,
            scale=0.4
          },
      ]
      \begin{axis}[
          view={-45}{45}
        ]
        \addplot3[
          opacity = 0.8,
          surf,
          colorbar,
          colormap/PuBu,
          shader=interp,
          domain=-1:1,
          domain y=-1.3:1.3,
          samples=10
        ]{exp(-x^2-y^2)};

        \draw[-to, ultra thick, green!40!magenta] (0,0,1) -- (0,0,1.5);
        \draw[-to, ultra thick, green!80!magenta] (0,0,1) -- (0,0,0.5);

        \draw[-to, ultra thick, green!40!magenta] (-.25,.25,.88) -- (-.5,0.25,1.4);
        \draw[-to, ultra thick, green!80!magenta] (-.25,.25,.88) -- (0,0.25,.36);

        \draw[-to, ultra thick, green!40!magenta] (.25,-.25,.88) -- (.5,-.25,1.2);
        \draw[-to, ultra thick, green!80!magenta] (.25,-.25,.88) -- (0 ,-.25,.56);

        \node [small dot,pin=210:{Belső pont}] at (-0.5, -0.5, 0.6 )  {};
        \node [small dot,pin=90:{Perempont}]   at (0   , -1.3, 0.18)  {};
      \end{axis}
    \end{tikzpicture}

    \sftitle{Irányítás szemléltetése}
  \end{minipage}\begin{minipage}{.5\textwidth}
    \centering
    \begin{tikzpicture}[
        declare function={f(\x,\y)=exp(-\x^2-\y^2);},
        scale=2/3,
      ]
      \begin{axis}[
          view={135}{45},
        ]
        \addplot3[
          opacity = 1,
          surf,
          colorbar,
          colormap/PuBu,
          % shader=interp,
          domain=-1:1,
          domain y=-1.3:1.3,
          samples=14
        ]{f(x,y)};

        \draw[-to, ultra thick, primaryColor]   (0,0,1) -- (-1,0,.9)
        node[
          midway,
          above left=.2cm and -0.1cm,
          black,fill=yellow!50!white,
          rectangle,
          rounded corners=3pt
        ] {$\displaystyle\pdv{\surfvec}{s}$};

        \draw[-to, ultra thick, secondaryColor] (0,0,1) -- (0,1,1.1)
        node[
          midway,
          below=.2cm,
          black,fill=yellow!50!white,
          rectangle,
          rounded corners=3pt
        ] {$\displaystyle\pdv{\surfvec}{t}$};
        s

        \draw[dotted, ultra thick, primaryColor] (0,1,1.1) -- (-1,1,1);
        \draw[dotted, ultra thick, yellow!40!magenta] (-1,0,.9) -- (-1,1,1);

        \draw[fill=white!90!yellow,fill opacity=0.6]
        (0,0,1) -- (-1,0,.9) -- (-1,1,1) -- (0,1,1.1) -- cycle
        node [above right=-0.2cm and 1.2cm, opacity=1] {$A$};
      \end{axis}
    \end{tikzpicture}

    \sftitle{Elemi felület}
  \end{minipage}
\end{note}

\begin{definition}[Skalármező skalárértékű felületmenti integrálja][nobreak]
  Legyen $\varphi: \Reals^3 \rightarrow \Reals$ skalármező, $\surfsign:
    \surfdomain \subseteq \Reals^2 \rightarrow \surfimage \subseteq \Reals^3$.
  Ekkor finomítsuk minden határon túl a
  $$
    \sum_i \varphi \left( \xi_i; \eta_i; \zeta_i \right) \Delta \surfscalar_i
  $$
  összeget:
  $$
    \int_{\surfimage} \varphi(\coordvec) \dd \surfscalar
    \text.
  $$
  Számítása:
  \begin{gather*}
    \iint_{\surfdomain} \varphi(\surfsign(s; t)) \norma{
      \pdv{\surfsign}{s}
      \times
      \pdv{\surfsign}{t}
    } \dd s \dd t
    \quad \rightarrow \quad
    \text{ha a felület paraméterezve van,}
    \\
    \iint_{\surfdomain} \varphi(x; y; \varPhi(x; y)) \sqrt{
      1
      + \left( \partial_x \varPhi \right)^2
      + \left( \partial_y \varPhi \right)^2
    } \dd x \dd y
    \quad \rightarrow \quad
    \text{ha } z=\varPhi(x; y) \text{ alakban van.}
  \end{gather*}
\end{definition}

\begin{example}
  Integráljuk a $\varphi(\coordvec) = x^2 + y^2$ skalármezőt az egységgömb
  $z > 0$ részén!

  \boxrule

  Az egységgömb paraméterezése:
  $$
    \surfsign(s;t) = \begin{bmatrix}
      \sin s \cos t \\
      \sin s \sin t \\
      \cos s
    \end{bmatrix}
    \text,
    \qquad
    s \in \left[0; \pi/2\right]
    \text,
    \qquad
    t \in [0; 2\pi]
    \text,
    \qquad
    \rvec n = \norma{\pdv{\surfsign}{s} \times \pdv{\surfsign}{t}} = \sin s
    \text.
  $$
  A skalármező átparaméterezése:
  $$
    \varphi(\surfsign(s; t)) = \sin^2 s \cos^2 t + \sin^2 s \sin^2 t = \sin^2 s
    \text.
  $$
  Az integrál kiszámítása:
  \begin{align*}
    \int_{\surfimage} \varphi \dd \surfscalar
     & = \int_0^{\pi/2} \int_0^{2\pi} \sin^2 s \sin s \dd t \dd s
    = 2\pi \int_0^{\pi/2} (1 - \cos^2 s) \sin s \dd s
    \\
     & = 2\pi \int_{0}^{1} (1 - u^2) \dd u
    = 2\pi \left[ u - \frac{u^3}{3} \right]_{0}^{1}
    = \frac{4\pi}{3}
    \text.
  \end{align*}
\end{example}

\begin{definition}[Skalármező vektorértékű felületmenti integrálja]
  Felület implicit megadása esetén $\left( z = \varPhi(x; y) \right)$:
  $$
    \int_{\surfimage} \varphi(\coordvec) \dd \surfvec
    =
    \iint \varphi(x; y; \varPhi(x; y))
    \begin{bmatrix}
      \pm \partial_x \varPhi \\
      \pm \partial_y \varPhi \\
      \mp 1
    \end{bmatrix}
    \dd x \dd y
  $$
\end{definition}

\begin{definition}[Vektormező skalárértékű felületmenti integrálja]
  \begin{minipage}{.375\textwidth}
    \centering
    \begin{tikzpicture}[
        declare function={f(\x,\y)=exp(-\x^2-\y^2);},
        scale=4/7,
      ]
      \begin{axis}[
          view={100}{45},
          axis lines=none,
        ]
        \addplot3[
          opacity = 1,
          surf,
          colorbar,
          colormap/PuBu,
          % shader=interp,
          domain=-1:1,
          domain y=-1.3:1.3,
          samples=12
        ]{f(x,y)};

        \draw[-to, ultra thick, green!40!magenta]   (0,0,1) -- (0,0,1.6)
        node[
          midway,
          left=.2cm,
          black,fill=yellow!50!white,
          rectangle,
          rounded corners=3pt
        ] {$\rvec n_i$};

        \draw[-to, ultra thick, yellow!40!magenta] (0,0,1) -- (-.33,1.33,1.2)
        node[
          midway,
          below=.2cm,
          black,fill=yellow!50!white,
          rectangle,
          rounded corners=3pt
        ] {$\rvec v_i$};
      \end{axis}
    \end{tikzpicture}
  \end{minipage}\begin{minipage}{.625\textwidth}
    $$
      \rvec v: \Reals^3 \rightarrow \Reals^3
      \quad
      \surfsign: \surfdomain \subset \Reals^2 \rightarrow \surfimage \subset \Reals^3
    $$
    Finomítsuk minden határon túl a
    $
      \sum_i \scalar{\rvec v_i}{\rvec n_i}
    $
    összeget:
    $$
      \int_{\surfimage} \scalar{\rvec v(\coordvec)}{\dd \rvec S}
      =
      \iint_{\surfdomain} \scalar{
        \rvec v(\surfsign(s; t))
      }{
        \norma{\pdv{\surfsign}{s} \times \pdv{\surfsign}{t}}
      }
      \text,
    $$
    ahol $\dd \surfvec = \uvec n \dd \surfscalar$.
  \end{minipage}
\end{definition}

\begin{definition}[Vektormező vektorértékű felületmenti integrálja]
  $$
    \int_{\surfimage} \rvec v(\coordvec) \times \dd \rvec S
    =
    \iint_{\surfdomain} \rvec v(\surfsign(s; t)) \times \left(
    \pdv{\surfsign}{s} \times \pdv{\surfsign}{t}
    \right) \dd s \dd t
  $$
\end{definition}