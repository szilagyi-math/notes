% \\([\w]+)\{(\w)\} -> \\$1 $2
\clearpage
\section{Alapfogalmak}

\begin{definition}[Konvektor]
  Legyen $V$ egy $\Reals$ feletti vektortér. $V^* := \operatorname{Hom}
    (V; \Reals)$ elemeit ($\alpha: V \rightarrow \Reals$) lineáris
  funkcionáloknak, lineáris formáknak, vagy 1-formáknak nevezzük.
  Mivel $\alpha$ lineáris leképezés, ezért
  $$
    \alpha(a \rvec v + b \rvec w)
    = a \, \alpha(\rvec v) + b \, \alpha(\rvec w)
    \text{ teljesül.}
  $$
\end{definition}

\begin{definition}[Duális tér]
  Legyen $\alpha; \beta \in V^*$, $\rvec v \in V$, $a \in \Reals$. A fenti
  módon teljesülnek az alábbiak:
  \begin{itemize}
    \item $(\alpha + \beta)(\rvec v) := \alpha(\rvec v) + \beta(\rvec v)$,
    \item $(a \, \alpha)(\rvec v) := a \, \alpha(\rvec v)$.
  \end{itemize}
  Ekkor $V^*$ vektortérré tehető, $V$ vektortér duális terének nevezzük.
\end{definition}

\begin{note}[Bázis jelölése]
  \bullet \;\; ${\left\{ \uvec e_1; \uvec e_2; \dots; \uvec e_n \right\}}$
  -- ortonormált / standard bázis

  \bullet \;\; ${\left\{ \rvec b_1; \rvec b_2; \dots \rvec b_n \right\}}$
  -- tetszőleges bázis
\end{note}

\begin{note}[Einstein-féle konvenció]
  $!\exists \left( r^1; r^2; \dots; r^n \right)$,
  melyre teljesül, hogy
  $$
    \rvec v
    = r^1 \rvec b_1
    + r^2 \rvec b_2
    + \dots
    + r^n \rvec b_n
    = \sum_{j=1}^n r^j \rvec b_j
    = {r^j \rvec b_j}
  $$
  Vezessük be a következő 1-formát:
  $\omega^i: V \rightarrow \Reals$,
  $\; \forall \, \rvec v \in V: \omega^i(\rvec v) = r^i$
  Ekkor $\rvec v$ felírható az alábbi alakban:
  $$
    \rvec v
    = \underbrace{\omega^1(\rvec v)}_{r^1} \rvec b_1
    + \underbrace{\omega^2(\rvec v)}_{r^2} \rvec b_2
    + \dots
    + \underbrace{\omega^n(\rvec v)}_{r^n} \rvec b_2
  $$
\end{note}

\begin{statement}
  A fent definiált $\omega^i: V \rightarrow \Reals$ 1-formák
  ($i \in {1;\; 2;\; \dots;\; n}$) bázist alkotnak $V^*$-ban.

  \begin{minipage}{\textwidth}
    \begin{proof}[Egzisztencia]
      Legyen $\alpha: V \rightarrow \Reals$ 1-forma:
      \begin{gather*}
        \alpha \left(
        \omega^1(\rvec v) \, \rvec{b_1}
        + \dots
        + \omega^n(\rvec v) \, \rvec{b_n}
        \right)
        = \omega^1(\rvec v) \, \alpha(\rvec b_1)
        + \dots
        + \omega^n(\rvec v) \, \alpha(\rvec b_n)
        \\
        \alpha(\rvec v)
        = \sum_{j=1}^n \omega^j(\rvec v) \, \alpha(\rvec b_j)
        \\
        \alpha
        = \underbrace{\alpha(\rvec b_1)}_{r_1} \, \omega^1
        + \dots
        + \underbrace{\alpha(\rvec b_n)}_{r_n} \, \omega^n
        = \sum_{j=1}^n r_j \, \omega^j
      \end{gather*}

      $r_i$ nem speciális, tetszőleges 1-forma felírható így.
    \end{proof}
  \end{minipage}

  \begin{proof}[Unicitás]
    Tegyük fel, hogy:
    \vspace{-1em}
    \begin{alignat*}{9}
      \alpha & = r_1 \, \omega^1 & \;+\;
             & r_2 \, \omega^2   & \;+\;
             & \dots             & \;+\;
             & r_n \, \omega^n
      \text,                             \\
      \alpha & = s_1 \, \omega^1 & \;+\;
             & s_2 \, \omega^2   & \;+\;
             & \dots             & \;+\;
             & s_n \, \omega^n
      \text.
    \end{alignat*}
    Vonjuk ki egymásból a 2 egyenletet:
    $$
      \mathcal O
      = (r_1 - s_1) \omega^1
      + (r_2 - s_2) \omega^2
      + \dots
      + (r_n - s_n) \omega^n
    $$
    Ekkor $\mathcal O$ egy 1-forma, melynek bármely vektort a nullvektorba
    visz, azaz
    $$
      \mathcal O: V \rightarrow \Reals
      \quad
      \mathcal O(\rvec v) = 0
      \quad
      \forall \rvec v \in V
      \qquad \Leftrightarrow \qquad
      r_i = s_i \hspace{5mm} \forall \; i \text{-re}
      \text.
    $$
    Ezzel ellentmondásra jutottunk, tehát nem igaz a feltevés.
  \end{proof}
\end{statement}

\begin{note}[Kovariáns és kontravariáns koordináták]
  \begin{minipage}{.5\textwidth}
    \begin{center}
      \begin{tikzpicture}[ultra thick, scale=2/3]
        % grid lines
        \draw[dashed,secondaryColor!50, thick]
        (-0.25,-0.5) -- (1.75, 3.5)
        ( 2.25,-0.5) -- (4.25, 3.5)
        (-0.5,  0  ) -- (3.5 , 0  )
        ( 0.5,  3  ) -- (5   , 3  )
        ;

        % scaled basis vectors r¹b₁ and r²b₂
        \draw[-to,draw=ternaryColor]
        (0,0) -- (1.5,3)   node[above left]  {$r^1\rvec b_1$};
        \draw[-to,draw=ternaryColor]
        (0,0) -- (2.5,0)   node[below right] {$r^2\rvec b_2$};

        % basis vectors b₁ and b₂
        \draw[-to,draw=secondaryColor]
        (0,0) -- (.75,1.5) node[left]        {$\rvec b_1$};
        \draw[-to,draw=secondaryColor]
        (0,0) -- (1.5,0)   node[below left]  {$\rvec b_2$};

        % main vector v
        \draw[-to,draw=primaryColor]
        (0,0) -- (4,3)     node[above right] {$\rvec v$};
      \end{tikzpicture}

      \textbf{Kovariáns koordináták}
    \end{center}
  \end{minipage}\begin{minipage}{.5\textwidth}
    \begin{center}
      \begin{tikzpicture}[ultra thick, scale=2/3]
        % O - origin, AB - coordinates, E - endpoint of v
        \tkzDefPoints{0/0/O/, 4/2/E, 1/2/A, 1.5/0/B}

        % right angle
        \tkzDefPointBy[projection = onto O--A](E) \tkzGetPoint{P}
        \tkzDefPointBy[projection = onto O--B](E) \tkzGetPoint{Q}

        % grid lines
        \tkzDrawLines[add=.15 and .15, dashed, secondaryColor!50, thick](O,P E,P O,Q E,Q)

        % scaled basis vectors r₁b₁ and r₂b₂
        \draw [-to, ultra thick, draw=ternaryColor]
        (O) -- (P) node[left, black]        {$r_1 \rvec b_1$};
        \draw [-to, ultra thick, draw=ternaryColor]
        (O) -- (Q) node[below right, black] {$r_2 \rvec b_2$};

        \draw [-to, draw=secondaryColor]
        (O) -- (A) node[left, black]        {$\rvec b_1$};
        \draw [-to, draw=secondaryColor]
        (O) -- (B) node[below left, black]  {$\rvec b_2$};

        \draw [-to, draw=primaryColor]
        (O) -- (E) node[above right, black] {$\rvec v$};
      \end{tikzpicture}


      \textbf{Kontravariáns koordináták}
    \end{center}
  \end{minipage}

  \vspace{2em}

  $\rvec v = r^1 \rvec b_1 + r^2 \rvec b_2$, ahol $(r^1 ; \; r^2)$ a $\rvec v$
  vektor kontravariáns koordinátái a $\left\{ \rvec b_1 ; \; \rvec b_2 \right\}$
  bázisban.

  $r_1$ és $r_2$ pedig $\rvec v$ kovariáns koordinátái, melyek a következőképpen
  számíthatóak:
  $$
    r_i = \frac{\scalar{\rvec v}{\rvec b_i}}{\norma{\rvec b_i}}
    \cdot \frac{\rvec b_i}{\norma{\rvec b_i}}
    = \underbrace{
      \frac{\scalar{\rvec v}{\rvec b_i}}{\scalar{\rvec b_i}{\rvec b_i}}
    }_{r_i}
    \cdot \rvec b_i
  $$
\end{note}

\begin{statement}[][nobreak]
  Kovariáns és kontravariáns koordináták ortonormált
  $\left\{ \uvec e_1; \uvec e_2 ; \dots; \uvec e_n \right\}$
  bázisban megegyeznek.

  \begin{proof}
    % $$
    %   \rvec v = \sum_{j=1}^n r^j \uvec e_j
    %   \quad \Rightarrow \quad
    %   \begin{array}{l}
    %     r^j = \dfrac{
    %       \scalar{\rvec v}{\uvec e_j}
    %     }{
    %       \scalar{\uvec e_j}{\uvec e_j}
    %     } = \scalar{\rvec v}{\uvec e_j}
    %     \\[3mm]
    %     r_j = \scalar{
    %       \displaystyle\sum_{i=1}^n r^i \, \uvec e_i
    %     }{\uvec e_j} \; = \displaystyle\sum_{i=1}^n r^i
    %     \scalar{\uvec e_i}{\uvec e_j}
    %     = r^j
    %   \end{array}
    % $$
  \end{proof}
  \vspace*{9em}
  % LOOKS CRAP, SO COMMENTED OUT XD
\end{statement}