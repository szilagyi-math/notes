\clearpage
\section{Differenciáloperátorok}

\begin{blueBox}
  Legyen $\vfunc f: V \rightarrow V$ függvény ($\dim V = n$), melynek vegyük a
  deriváltját:
  \begin{gather*}
    \vfunc f \left( x_1; x_2; \dots ; x_n \right)
    =
    \begin{bmatrix}
      f_1 \left( x_1; x_2; \dots ; x_n \right) \\
      f_2 \left( x_1; x_2; \dots ; x_n \right) \\
      \vdots                                   \\
      f_n \left( x_1; x_2; \dots ; x_n \right) \\
    \end{bmatrix}
    \text,
    \\
    \DD f
    =
    \begin{bmatrix}
      \partial_1 f_1 & \partial_2 f_1 & \dots  & \partial_n f_1 \\
      \partial_1 f_2 & \partial_2 f_2 & \dots  & \partial_n f_2 \\
      \vdots         & \vdots         & \ddots & \vdots         \\
      \partial_1 f_n & \partial_2 f_n & \dots  & \partial_n f_n
    \end{bmatrix}
    =
    \begin{bmatrix}
      \grad^\T f_1 \\
      \grad^\T f_2 \\
      \vdots       \\
      \grad^\T f_n \\
    \end{bmatrix}
    \in
    \mathcal M_{n \times n}
  \end{gather*}
  Definiáljuk az alábbi fogalmakat:
  \begin{itemize}
    \item $\rot \vfunc f := \DD f - \DD f^*$
          -- rotáció,

    \item $\Div \vfunc f := \operatorname{tr} \left( \DD f \right)$
          -- divergencia.
  \end{itemize}
  $V = \Reals^3$ esetén:
  $$
    \rot \vfunc f
    =
    \begin{bmatrix}
      0                               &
      \partial_2 f_1 - \partial_1 f_2 &
      \partial_3 f_1 - \partial_1 f_3
      \\
      \partial_1 f_2 - \partial_2 f_1 &
      0                               &
      \partial_3 f_2 - \partial_2 f_3
      \\
      \partial_1 f_3 - \partial_3 f_1 &
      \partial_2 f_3 - \partial_3 f_2 &
      0
    \end{bmatrix}
    \text.
  $$
  A mátrix vektorinvariánsa:
  \begin{equation*}
    \rvec v
    =
    \begin{bmatrix}
      \partial_2 f_3 - \partial_3 f_2 \\
      \partial_3 f_1 - \partial_1 f_3 \\
      \partial_1 f_2 - \partial_2 f_1
    \end{bmatrix}
    =
    \begin{bmatrix}
      \partial_1 \\
      \partial_2 \\
      \partial_3
    \end{bmatrix}
    \times
    \begin{bmatrix}
      f_1 \\
      f_2 \\
      f_3
    \end{bmatrix}
    =
    \nabla \times f
    \text.
  \end{equation*}

  $\nabla$ - Nabla operátor (formális differenciáloperátor)
  -- nem vektor, de aként viselkedik.
\end{blueBox}

\begin{note}[Gradiens, divergencia és rotáció számítása]
  \vspace{-1.5em}
  \begin{align*}
    \grad \varphi & = \nabla \varphi
    \\
    \Div \rvec v  & = \scalar{\nabla}{\rvec v}
    \\
    \rot \rvec v  & = \nabla \times \rvec v
  \end{align*}
\end{note}

\begin{note}[Differenciáloperátorok kompbinálása][nobreak]
  Nem értelmezhető:
  \vspace{-.5em}
  $$
    \grad \grad \text, \quad
    \grad \rot \text, \quad
    \Div \Div \text, \quad
    \rot \Div \text.
  $$
  Laplace-operátor:
  $$
    \Div \grad \varphi = \nabla^2 \varphi = \Delta \varphi
  $$
  Tetszőleges $\rvec v$ vektormező és $\varphi$ skalármező esetén:
  \vspace{-.5em}
  \begin{align*}
    \Div \rot \rvec v  & \equiv 0 \text,     \\
    \rot \grad \varphi & \equiv \nvec \text.
  \end{align*}
  \vspace{-2em}
\end{note}

\begin{definition}[Skalárpotenciálosság]
  Egy $\rvec v: V \rightarrow V$ vektormező skalárpotenciálos, ha létezik olyan
  $\varphi: V \rightarrow \Reals$ skalármező, hogy $\rvec v = \grad \varphi$.
\end{definition}

\begin{definition}[Vektorpotenciálosság]
  Egy $\rvec v: V \rightarrow V$ vektormező vektorpotenciálos, ha létezik olyan
  $\rvec u: V \rightarrow V$ vektormező, hogy $\rvec v = \rot \rvec u$.
\end{definition}

\begin{theorem}
  Legyen $\rvec v: V \rightarrow V$ mindenhol értelmezett, legalább egyszer
  differenciálható vektormező. Ekkor:
  \begin{itemize}
    \item $\rvec v$ skalárpotenciálos
          $\;\Leftrightarrow\;$
          $\rot \rvec v = \nvec$,
          hiszen $\rot \grad \varphi \equiv \nvec$,
    \item $\rvec v$ vektorpotenciálos
          $\;\Leftrightarrow\;$
          $\Div \rvec v = 0$,
          hiszen $\Div \rot \rvec u \equiv 0$.
  \end{itemize}

  \begin{proof}[$\Rightarrow$ könnyű, $\Leftarrow$ nehéz]
    Ha $\rvec v$ skalárpotenciálos $\;\Rightarrow\;$
    $\exists \varphi : V \rightarrow \Reals$,
    hogy $\rvec v = \grad \varphi$, ekkor
    $\rot \rvec v = \rot \grad \varphi \equiv \nvec$

    Ha $\rvec v$ vektorpotenciálos $\;\Rightarrow\;$
    $\exists \rvec u : V \rightarrow V$,
    hogy $\rvec v = \rot \rvec u$, ekkor
    $\Div \rvec v = \Div \rot \rvec u \equiv 0$
  \end{proof}
\end{theorem}

\begin{blueBox}[][nobreak]
  $! \varPhi; \varPsi : \Reals^3 \rightarrow \Reals$ skalármezők,
  $\rvec u; \rvec v; \rvec w : \Reals^3 \rightarrow \Reals^3$
  vektormezők, $\lambda; \mu \in \Reals$ pedig skalárok.
  \begin{itemize}
    \item Teljesül a linearitás:
          \vspace{-.5em}
          \begin{alignat*}{4}
            \grad & (\lambda \, \varPhi && + \mu \, \varPsi) && = \lambda \, \grad \varPhi && + \mu \, \grad \varPsi
            \\
            \rot  & (\lambda \, \rvec v && + \mu \, \rvec w) && = \lambda \, \rot \rvec v  && + \mu \, \rot \rvec w
            \\
            \Div  & (\lambda \, \rvec v && + \mu \, \rvec w) && = \lambda \, \Div \rvec v  && + \mu \, \Div \rvec w
          \end{alignat*}

    \item Zérusság:
          \vspace{-.5em}
          \begin{alignat*}{1}
            \rot \grad \varPhi & \equiv \nvec
            \\
            \Div \rot \rvec v  & \equiv 0
          \end{alignat*}

    \item Deriválási szabályokhoz hasonló:
          \vspace{-.5em}
          \begin{align*}
            \grad \left( \varPhi \, \varPsi \right)
             & = \varPhi \, \grad \varPsi
            + \varPsi \, \grad \varPhi
            \\
            \Div \left( \varPhi \, \rvec v \right)
             & = \varPhi \, \Div \rvec v \,
            + \scalar{\rvec v}{\grad \varPhi}
            \\
            \rot \left( \varPhi \, \rvec v \right)
             & = \varPhi \, \rot \rvec v
            - \rvec v \times \grad \varPhi
          \end{align*}

    \item Egyéb szabályok:
          \vspace{-.5em}
          \begin{align*}
            \rot \rot \rvec v
             & =\grad \Div \rvec v
            - \Delta \rvec v
            \\
            \rot \left( \rvec u \times \rvec v \right)
             & = \rvec u \, \Div \rvec v
            - \rvec v \, \Div \rvec u
            + (\DD \rvec u) \rvec v
            - (\DD \rvec v) \rvec u
            \\
            \Div \left( \rvec u \times \rvec v \right)
             & = \; \scalar{\rvec v}{\rot \rvec u}
            - \scalar{\rvec u}{\rot \rvec v}
            \\
            \grad \left( \scalar{\rvec u}{\rvec v} \right)
             & = (\DD \rvec u) \rvec v
            + (\DD \rvec v) \rvec u
            + \rvec v \times \rot \rvec u
            + \rvec u \times \rot \rvec v
          \end{align*}
  \end{itemize}
\end{blueBox}

\begin{note}[3-dimenziós Levi--Civita-szimbólum]
  $$
    \varepsilon_{ijk} =
    \begin{cases}
      +1         & \text{ha } (i;j;k) \text{ az } (1;2;3) \text{ páros permutációja}    \\
      -1         & \text{ha } (i;j;k) \text{ az } (1;2;3) \text{ páratlan permutációja} \\
      \phantom-0 & \text{ha } i = j \text{, vagy } j = k \text{, vagy } k = i
    \end{cases}
    \text.
  $$
  Vektoriális szorzat esetében
  $$
    \bigl(\rvec v\bigr)_i
    = \bigl(\rvec x \times \rvec y\bigr)_i
    = \sum_{j=1}^3 \sum_{k=1}^3 \varepsilon_{ijk} x_j y_k
    = \begin{bmatrix}
      \varepsilon_{123} x_2 y_3 + \varepsilon_{132} x_3 y_2 \\
      \varepsilon_{231} x_3 y_1 + \varepsilon_{213} x_1 y_3 \\
      \varepsilon_{312} x_1 y_2 + \varepsilon_{321} x_2 y_1
    \end{bmatrix} = \begin{bmatrix}
      x_2 y_3 - x_3 y_2 \\
      x_3 y_1 - x_1 y_3 \\
      x_1 y_2 - x_2 y_1
    \end{bmatrix}
    \text.
  $$
  Minden koordinátában hat tag szerepelne, viszont:
  \begin{enumerate}
    \item $\varepsilon_{223} = \varepsilon_{232} = \varepsilon_{323} = \varepsilon_{332} = 0$,
    \item $\varepsilon_{113} = \varepsilon_{131} = \varepsilon_{311} = \varepsilon_{311} = 0$,
    \item $\varepsilon_{112} = \varepsilon_{121} = \varepsilon_{211} = \varepsilon_{221} = 0$.
  \end{enumerate}
  A nemzérus tagok pedig:
  \begin{enumerate}
    \item $\varepsilon_{123} = +1$, $\varepsilon_{132} = -1$,
    \item $\varepsilon_{231} = +1$, $\varepsilon_{213} = -1$,
    \item $\varepsilon_{312} = +1$, $\varepsilon_{321} = -1$.
  \end{enumerate}
\end{note}

\begin{example}[Linearitásos azonosságok bizonyítása]
  \begin{enumerate}
    \item \underline{
            $\grad(\lambda\varPhi+\mu\varPsi)=\lambda\grad\varPhi+\mu\grad\varPsi$
          }
          $$
            \bigl(\grad(\lambda\varPhi+\mu\varPsi)\bigr)_i
            = \partial_i(\lambda\varPhi+\mu\varPsi)
            = \lambda\,\partial_i\varPhi+\mu\,\partial_i\varPsi
            = \bigl(\lambda\grad\varPhi+\mu\grad\varPsi\bigr)_i
            \text.
          $$

    \item \underline{
            $\rot(\lambda\rvec v+\mu\rvec w)=\lambda\rot\rvec v+\mu\rot\rvec w$
          }
          $$
            \bigl(\rot(\lambda\rvec v+\mu\rvec w)\bigr)_i
            = \varepsilon_{ijk}\partial_j(\lambda v_k+\mu w_k)
            = \lambda\,\varepsilon_{ijk}\partial_jv_k+\mu\,\varepsilon_{ijk}\partial_jw_k
            = \bigl(\lambda\rot\rvec v+\mu\rot\rvec w\bigr)_i
            \text.
          $$

    \item \underline{
            $\Div(\lambda\rvec v+\mu\rvec w)=\lambda\Div\rvec v+\mu\Div\rvec w$
          }
          $$
            \bigl(\Div(\lambda\rvec v+\mu\rvec w)\bigr)_i
            = \partial_i(\lambda v_i+\mu w_i)
            = \lambda\,\partial_iv_i+\mu\,\partial_iw_i
            = \bigl(\lambda\Div\rvec v+\mu\Div\rvec w\bigr)_i
            \text.
          $$
  \end{enumerate}
\end{example}

\begin{example}[Zérusságos azonosságok bizonyítása][nobreak]
  \begin{enumerate}
    \item \underline{$\rot\grad\varPhi\equiv\nvec$}
          $$
            \bigl(\rot\grad\varPhi\bigr)_i=\varepsilon_{ijk}\partial_j\partial_k\varPhi
            = -\varepsilon_{ijk}\partial_k\partial_j\varPhi=0,
          $$
          mert $\partial_j\partial_k$ szimmetrikus, $\varepsilon_{ijk}$ pedig
          antiszimmetrikus $j,k$ indexekre.

    \item \underline{$\Div\rot\rvec v\equiv0$}
          $$
            \bigl(\Div\rot\rvec v\bigr)_i=\partial_i\varepsilon_{ijk}\partial_jv_k
            = \varepsilon_{ijk}\partial_i\partial_jv_k
            = -\varepsilon_{ijk}\partial_j\partial_iv_k=0,
          $$
          mert $\partial_i\partial_j$ szimmetrikus, $\varepsilon_{ijk}$ pedig
          antiszimmetrikus $i,j$ indexekre.
  \end{enumerate}
\end{example}

\begin{example}[Deriválási szabályokhoz hasonló azonosságok bizonyítása]
  \begin{enumerate}
    \item \underline{$\grad(\varPhi\varPsi)=\varPhi\grad\varPsi+\varPsi\grad\varPhi$}
          $$
            \bigl(\grad(\varPhi\varPsi)\bigr)_i
            = \partial_i(\varPhi\varPsi)
            = \varPhi\,\partial_i\varPsi+\varPsi\,\partial_i\varPhi
            = \bigl(\varPhi\grad\varPsi+\varPsi\grad\varPhi\bigr)_i
            \text.
          $$

    \item \underline{$\Div(\varPhi\rvec v)=\varPhi\Div\rvec v+\scalar{\rvec v}{\grad\varPhi}$}
          $$
            \bigl(\Div(\varPhi\rvec v)\bigr)_i
            = \partial_i(\varPhi v_i)
            = \varPhi\,\partial_iv_i+\scalar{\rvec v}{\partial_i\grad\varPhi}
            = \bigl(\varPhi\Div\rvec v+\scalar{\rvec v}{\grad\varPhi}\bigr)_i
            \text.
          $$

    \item \underline{$\rot(\varPhi\rvec v)=\varPhi\rot\rvec v-\rvec v\times\grad\varPhi$}
          $$
            \begin{aligned}
              \bigl(\rot(\varPhi\rvec v)\bigr)_i
               & = \varepsilon_{ijk}\partial_j(\varPhi v_k)
              \\
               & = \varepsilon_{ijk}(\varPhi\,\partial_jv_k+v_k\,\partial_j\varPhi)
              \\
               & = \varPhi\,\varepsilon_{ijk}\partial_jv_k+\varepsilon_{ijk}v_k\,\partial_j\varPhi
              \\
               & = \varPhi\,\varepsilon_{ijk}\partial_jv_k-\varepsilon_{ijk}v_j\,\partial_k\varPhi
              = \bigl(\varPhi\rot\rvec v-\rvec v\times\grad\varPhi\bigr)_i
              \text.
            \end{aligned}
          $$
  \end{enumerate}
\end{example}

\begin{example}[Egyéb szabályok bizonyítása][nobreak]
  \vspace{10.5cm}
  % TOO COMPLICATED TO FIT INTO THIS PAGE, BUT ISN'T COMPLICATED ENOUGH TO FILL
  % THE NEXT PAGE AS WELL.
\end{example}
