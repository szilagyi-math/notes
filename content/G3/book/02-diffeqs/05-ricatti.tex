\clearpage
\section{Ricatti-féle differenciálegyenletek}

\begin{definition}[Ricatti-féle differenciálegyenletek]
  A Ricatti-típusú diffegyenletek elsőrendű, másodfokú, nemlineáris egyenletek,
  amelyek az alábbi alakban írhatóak fel:
  $$
    y'(x) + p(x)y(x) = q(x)y^2(x) + h(x)
    \text.
  $$
\end{definition}

\begin{note}
  Ha $q(x) = 0$, akkor a Ricatti-egyenlet lineáris.

  Ha $h(x) = 0$, akkor Bernoulli-típusú egyenletet kapunk.
\end{note}

\begin{blueBox}[Ricatti-féle differenciálegyenlet megoldása][nobreak]
  Egy Ricatti-egyenlet általában nem oldható meg integrálással. A standard
  módszer akkor működik közvetlenül, ha ismerjük az egyenlet $y_p$ partikuláris
  megoldását.

  \underline{\textbf{Bernoulli-típusra redukálás, majd linearizálás}}

  Alkalmazzuk az alábbi eltolást:
  $$
    y = y_p + z
    \text.
  $$
  Ekkor, felhasználva hogy
  $$
    y_p' + p y_p = q y_p^2 + h
    \text,
  $$
  az egyenlet $z$-re:
  $$
    z' + p z = 2q y_p z + q z^2
    \quad\Leftrightarrow\quad
    z' = (2qy_p - p) z + q z^2
    \text.
  $$
  Az egyenlet már Bernoulli-típusú, használhatjuk a korábban tanult
  helyettesítést:
  $$
    z = \frac{1}{u}
    \quad\Rightarrow\quad
    z' = -\frac{u'}{u^2}
  $$
  Behelyettesítve:
  $$
    -\frac{u'}{u^2} = (2qy_p - p)\frac{1}{u} + q\frac{1}{u^2}
    \quad\Rightarrow\quad
    -u' = (2qy_p - p)u + q
    \text.
  $$
  Az egyenlet már lineáris, megoldható a korábban tanult módszerekkel:
  $$
    u' + \alpha\,u = \beta,
    \qquad
    \text{, ahol}
    \begin{cases}
      \;\;\alpha := 2q\,y_p - p,
      \\
      \;\;\beta := -q
    \end{cases}
  $$

  \underline{\textbf{Egylépéses linearizálás}}

  Alkalmazzuk az alábbi helyettesítést:
  $$
    y = \frac{1}{u} + y_p
    \quad\Rightarrow\quad
    y' = -\frac{u'}{u^2} + y_p'
    \text.
  $$
  Ezt visszahelyettesítve ugyanarra az alakra jutnánk, mint az előző módszer
  során.
\end{blueBox}

\clearpage
\begin{example}[Oldjuk meg az alábbi Ricatti-féle differenciálegyenletet:][nobreak]
  $$
    y' + e^x \, y = e^{2x} \, y^2 - e^{-x}
    \text,\qquad
    y_p = e^{-x}
  $$

  \boxrule

  \underline{\textbf{Bernoulli-típusra redukálás, majd linearizálás}}

  Eltolás a partikuláris megoldással:
  $$
    y = y_p + z = e^{-x} + z,
    \qquad
    y' = -e^{-x} + z'
    \text.
  $$
  Behelyettesítve az egyenletbe:
  $$
    -e^{-x} + z' + e^x(e^{-x} + z) = e^{2x}(e^{-x} + z)^2 - e^{-x}
    \text.
  $$
  Az egyszerűsítés után:
  $$
    z' = e^x z + e^{2x} z^2
    \text.
  $$
  Most linearizáljuk az egyenletet $u = 1 / z$ helyettesítéssel:
  $$
    \left(\frac1z\right)' = \frac{e^x}{u} + \frac{e^{2x}}{u^2}
    \quad\Rightarrow\quad
    -\frac{u'}{u^2} = \frac{e^x}{u} + \frac{e^{2x}}{u^2}
    \quad\Rightarrow\quad
    u' + e^x u = - e^{2x}
    \text.
  $$
  A linearizált egyenlet homogén megoldása:
  $$
    u_p(x)
    = C e^{-\int e^x \dd x}
    = C \underbrace{e^{-e^x}}_{U(x)}
    \text.
  $$
  Az általános megoldás az eddigiekhez képest egy más módszerrel is megkapható.
  Osz\-szuk el az egyenlet mindkét oldalát $U(x)$-szel:
  $$
    e^{e^x} u' + e^x e^{e^x} u = -e^{2x} e^{e^x}
    \text.
  $$
  Az egyenlet bal oldalán $U(x) \cdot u'(x)$ szorzatfüggvény deriváltja
  található, vagyis:
  $$
    \left(e^{e^x} u\right)' = -e^{2x} e^{e^x}
    \quad\Rightarrow\quad
    e^{e^x} u
    = -\int e^{2x} e^{e^x} \dd x
    = -\int e^{e^x + 2x} \dd x
  $$
  Éljünk a $t = e^{e^x}$ helyettesítéssel, ekkor $x = \ln \ln t$, $e^x = \ln t$,
  $\dd t = e^{e^x} \cdot e^x \dd x = e^{e^x + 1}$.
  $$
    e^{e^x} u
    = -\int e^{2x} e^{e^x} \frac{\dd t}{e^{e^x + 1}}
    = -\int e^x \dd t
    = -\int \ln t \dd t
    = -t \ln t + t
    = -e^{e^x} e^x + e^{e^x} + C
  $$
  Az általános megoldás $u$-ra:
  $$
    u = e^{-e^x} \cdot \left[e^{e^x}(1 - e^x) + C\right]
    = 1 - e^x + C e^{-e^x}
    \text.
  $$
  Visszahelyettesítve $z$-re, majd $y$-ra:
  $$
    z = \frac1u
    = \frac{1}{u} = \frac{1}{1 - e^x + C e^{-e^x}}
    \quad\Rightarrow\quad
    y = e^{-x} + z
    = e^{-x} + \frac{1}{1 - e^x + C e^{-e^x}}
    \text.
  $$
\end{example}