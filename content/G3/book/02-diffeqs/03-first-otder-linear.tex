\clearpage
\section{Elsőrendű lineáris differenciálegyenletek}

\begin{definition}[Elsőrendű lineáris differenciálegyenlet]
  Az $a(x) y' + b(x) y + c(x) = 0$ alakú differenciálegyenletet, ahol
  $a;b;c : I \subset \Reals \rightarrow \Reals$, $a(x) \not \equiv 0$
  egyváltozós, valós, folytonos függvények, elsőrendű lineáris
  differenciálegyenletnek nevezzük.\

  Gyakran a következő alakban írjuk fel őket:
  $$
    y' + p(x) y = q(x)
    \text.
  $$
\end{definition}

\begin{blueBox}[Elsőrendű lineáris differenciálegyenlet megoldása]
  \underline{\textbf{Homogén megoldás}}: Ha $q(x) \equiv 0$, akkor a DE az
  alábbi alakra redukálódik:
  $$
    \odv{y}{x} + p(x) y = 0
    \text.
  $$
  Rendezzük a differenciálegyenletet, majd integráljuk az egyenlet mindkét
  oldalát:
  \begin{align*}
    \frac{\dd y}{y}      & = -p(x) \dd x
    \\
    \int \frac{\dd y}{y} & = -\int p(x) \dd x
    \\
    \ln y                & = -\int p(x) \dd x + \ln C
    \\
    y_h                  & = C e^{-\int p(x) \dd x}
  \end{align*}

  \underline{\textbf{Partikuláris megoldás}}: Konstans variálással:

  A homogén megoldásnál kiszámított $y_h$ függvényben tekintsünk a konstans
  $C$-re $x$-től függő függvényként. Keressük a partikuláris megoldást az
  alábbi alakban:
  $$
    y_p(x) = C(x) \underbrace{e^{-\int p(x) \dd x}}_{Y(x)} = C(x) Y(x)
    \text.
  $$
  Számítsuk ki a $y_p$ függvény deriváltját:
  $$
    y_p' = C'(x) Y(x) + C(x) Y'(x)
    \text.
  $$
  Helyettesítsük be a $y_p$-t és $y_p'$-t az eredeti differenciálegyenletbe:
  \begin{gather*}
    y' + p(x) y = q(x)
    \\
    C'(x) Y(x) + C(x) Y'(x) + p(x) C(x) Y(x) = q(x)
    \\
    C(x) \underbrace{
      \left( Y'(x) + p(x) Y(x) \right)
    }_{= 0\text{, homogén egyenlet}} + C'(x) Y(x) = q(x)
  \end{gather*}
  Ezek alapján $C(x)$ értéke:
  $$
    C'(x) Y(x) = q(x)
    \quad \Rightarrow \quad
    C(x) = \int \frac{q(x)}{Y(x)} \dd x
    \text.
  $$

  \underline{\textbf{Általános megoldás}}: A homogén és a partikuláris megoldás
  összege:
  $$
    y(x)
    = y_h(x) + y_p(x)
    = C e^{-\int p(x) \dd x} + \int \frac{q(x)}{Y(x)} \dd x Y(x)
    \text.
  $$
\end{blueBox}

\begin{example}[Oldjuk meg az alábbi elsőrendű lineáris differenciálegyenletet:][nobreak]
  $$
    y' - \frac{2}{x} y = x^2, \quad x > 0
  $$
  \boxrule

  \underline{\textbf{Homogén megoldás}}:
  $$
    y' - \frac{2}{x} y = 0
    \quad \Rightarrow \quad
    \odv{y}{x} = \frac{2}{x} y
    \quad \Rightarrow \quad
    \int\frac{\dd y}{y} = \int\frac{2}{x} \dd x
    \quad \Rightarrow \quad
    \ln y = 2 \ln x + \ln C
  $$
  A logaritmus azonosságok felhasználásával:
  $$
    y_h(x) = C x^2
    \text.
  $$
  \underline{\textbf{Partikuláris megoldás}}: Tekintsük $C$-t $x$-től függőnek:
  \begin{align*}
    y(p)  & = C(x) x^2 \text,             \\
    y'(p) & = C'(x) x^2 + 2 C(x) x \text.
  \end{align*}
  Helyettesítsük be a $y(p)$-t és $y'(p)$-t az eredeti differenciálegyenletbe:
  \begin{gather*}
    C'(x) x^2 + 2 C(x) x - \frac{2}{x} C(x) x^2 = x^2
    \\
    C'(x) x^2 + 2 C(x) x - 2 C(x) x = x^2
    \\
    C'(x) x^2 = x^2
    \\
    C'(x) = 1
    \\
    C(x) = x + K
  \end{gather*}
  Vagyis a partikuláris megoldás:
  $$
    y_p(x) = (x + K) x^2 = x^3 + K x^2
    \text,
  $$
  amelyben $K$-t tetszőlegesen megválaszthatjuk, hiszen $C$ a homogén
  $C x^2$ a homogén megoldásban már szerepel. Legyen $K = 0$.

  \underline{\textbf{Általános megoldás}}:
  A homogén és a partikuláris megoldás összege:
  $$
    y(x) = C x^2 + x^3
    \text.
  $$
\end{example}