\clearpage
\section{Egzakt differenciálegyenletek}

\begin{definition}[Egzakt differenciálegyenlet]
  A $P(x;y) \dd x + Q(x;y) \dd y = 0$ differenciálegyenlet -- ahol
  $P;Q : \Reals^2 \rightarrow \Reals$ folytonosan differenciálható függvények
  -- egzaktnak nevezzük, ha teljesül az alábbi feltétel:
  $$
    \pdv{P(x;y)}{y} = \pdv{Q(x;y)}{x}
    \text.
  $$
  Ekkor $\exists F : \Reals^2 \rightarrow \Reals$, melyre igaz, hogy:
  \begin{gather*}
    \pdv{F(x;y)}{x} = P(x;y)
    \quad \text{és} \quad
    \pdv{F(x;y)}{y} = Q(x;y)
    \\
    \pdv{F(x;y)}{x}   \dd y
    + \pdv{F(x;y)}{y} \dd x
    = 0
    \\
    F(x;y) = C
  \end{gather*}
\end{definition}

\begin{example}[Oldjuk meg az alábbi egzakt differenciálegyenletet!][nobreak]
  $$
    \underbrace{(2xy + y)}_{P(x; y)} \dd x + \underbrace{(x^2 + x)}_{Q(x; y)} \dd y = 0
  $$
  \boxrule

  Ellenőrizzük, hogy valóban egzakt-e a differenciálegyenlet:
  $$
    \left\{
    \begin{array}{l}
      \displaystyle\pdv{P}{y}  = 2x + 1
      \\[.33em]
      \displaystyle \pdv{Q}{x} = 2x + 1
    \end{array}
    \right.
    \qquad \Rightarrow \qquad
    \pdv{P}{y} = \pdv{Q}{x}
    \text,
  $$
  tehát egzakt. Integráljuk $P$-t $x$ szerint, $Q$-t pedig $y$ szerint:
  \begin{align*}
    F(x;y) & = \int P \dd x = \int (2xy + y) \dd x
    = x^2 y + xy + C_y(y)
    \text,
    \\
    F(x;y) & = \int Q \dd y = \int (x^2 + x) \dd y
    = x^2 y + xy + C_x(x)
    \text,
  \end{align*}
  ahol $C_y(y)$ és $C_x(x)$ rendre $y$-tól és $x$-tól függő függvények,
  jelen esetben nullának válasthatóak. A megoldás tehát:
  $$
    x^2 y + xy = C
    \text.
  $$
\end{example}

\begin{note}
  Az előző példát más módszerekkel is megoldhattuk volna, például szeparábilis
  differenciálegyenletként:
  $$
    \int\frac{\dd y}{y}
    = - \int\frac{(2x + 1)\dd x}{x^2 + x}
    \quad \Rightarrow \quad
    \ln y = -\ln(x^2 + x) + K
    \quad \Rightarrow \quad
    y = \frac{C}{x^2 + x}
    \text.
  $$
\end{note}

\begin{blueBox}[Differenciálegyenlet egzakttá tevése][nobreak]
  Előfordulhat, hogy a $P(x;y)\dd x + Q(x;y) \,\dd y = 0$ differenciálegyenlet
  nem egzakt, de integráló tényező segítségével egzakttá tehető.

  Tegyük fel, hogy $\exists M : \Reals^2 \rightarrow \Reals$, hogy
  $M(x;y) P(x;y) \dd x + M(x;y) Q(x;y) \dd y = 0$ differenciálegyenlet már
  egzakt, azaz:
  \begin{gather*}
    \pdv{MP}{y} = \pdv{MQ}{x}
    \\[.33em]
    \pdv{M}{y} P
    + M \pdv{P}{y}
    = \pdv{M}{x} Q
    + M \pdv{Q}{x}
    \\[.33em]
    Q \pdv{M}{x}
    - P \pdv{M}{y}
    + M \left(
    \pdv{Q}{x}
    - \pdv{P}{y}
    \right) = 0
  \end{gather*}
  Tegyük fel, hogy $M$ egyváltozós:

  \boxrule

  $!M(x)$, ekkor a differenciálegyenlet az alábbi alakra redukálódik:
  $$
    Q \pdv{M}{x} + M \left( \pdv{Q}{x} - \pdv{P}{y} \right) = 0
    \text.
  $$
  Ez már $M(x)$-re szeparábilis:
  \begin{gather*}
    \frac{1}{M} \dd M = \frac{1}{Q} \left( \pdv{P}{y} - \pdv{Q}{x} \right) \dd x
    \\[.33em]
    \ln M = \int \frac{\partial_y P - \partial_x Q}{Q} \dd x
    \qquad\Rightarrow\quad
    M(x) = C e^{\int \frac{\partial_y P - \partial_x Q}{Q} \dd x}
    \text.
  \end{gather*}
  Itt $C$ tetszőlegesen megválasztható, ezért célszerű $1$-nek választani.

  \boxrule

  $!M(y)$, ebben az esetben pedig a differenciálegyenlet az alábbi alakot ölit:
  $$
    P \pdv{M}{y}
    + M \left(
    \pdv{Q}{x}
    - \pdv{P}{y}
    \right) = 0
    \text.
  $$
  Ez már $M(y)$-re szeparábilis:
  \begin{gather*}
    \frac{1}{M} \dd M = \frac{1}{P} \left( \pdv{Q}{x} - \pdv{P}{y} \right) \dd y
    \\[.33em]
    \ln M = \int \frac{\partial_y Q - \partial_x P}{P} \dd y
    \qquad\Rightarrow\quad
    M(y) = Ce^{\int \frac{\partial_y Q - \partial_x P}{P} \dd y}
    \text.
  \end{gather*}
  $C$ ebben az esetben is tetszőlegesen megválasztható.
\end{blueBox}

\begin{note}
  Előfordulhat, hogy a differenciálegyenlet nem egzakt, és egyváltozós
  multiplikátorral sem tehetó azzá. Ilyen például:
  $$
    (x^2 + xy) \dd x + (y^2 + xy) \dd y = 0
    \text.
  $$
\end{note}

\begin{example}[Tegyük egzakttá az alábbi differenciálegyenlet!][nobreak]
  $$
    (y^3 + y) x^2 \dd x
    + \frac{2y^3 + 3y^2 + 2y + 1}{3} \, x^3 \dd y
    = 0
  $$
  \boxrule

  A parciális deriváltak:
  $$
    \left\{
    \begin{array}{l}
      \displaystyle\pdv{P}{y} = x^2 (3y^2 + 1)
      \\[.33em]
      \displaystyle\pdv{Q}{x} = x^2 (2y^3 + 3y^2 + 2y + 1)
    \end{array}
    \right.
    \qquad \Rightarrow \qquad
    \pdv{P}{y} \neq \pdv{Q}{x}
    \text.
  $$
  A differenciálegyenlet valóban nem egzakt. Viszont megfigyelhetjük, hogy
  mind a parciális deriváltakban, mind $P(x; y)$-ben szerepel egy $x^2$-es
  szorzó. Ezek az $y$-től függő multiplikátoros képletben kiejtik egymást:
  $$
    \ln M(y)
    = \int \frac{\partial_y Q - \partial_x P}{P} \dd y
    = \int \frac{2y^3 + 2y}{y^3 + y} \dd y
    = \int 2 \dd y
    = 2y(+ C)
    \quad\Rightarrow\quad
    M(y) = e^{2y}
    \text.
  $$
  Ezek után a differenciálegyenlet már egzakt:
  $$
    \underbrace{e^{2y} (y^3 + y) x^2}_{P'} \dd x
    + \underbrace{e^{2y} \frac{2y^3 + 3y^2 + 2y + 1}{3} x^3}_{Q'} \dd y
    = 0
    \text.
  $$
  \vspace{-1em}
  Ellenőrizzük a parciális deriváltakat:
  $$
    \left\{
    \begin{array}{l}
      \displaystyle\pdv{P'}{y} = x^2\,e^{2y}\,\left[
        2(y^3 + y) + 3y^2 + 1
        \right]
      \\[.33em]
      \displaystyle\pdv{Q'}{x} = x^2\,e^{2y}\,(2y^3+3y^2+2y+1)
    \end{array}
    \right.
    \qquad \Rightarrow \qquad
    \pdv{P'}{y} = \pdv{Q'}{x}
    \text.
  $$
  Most már integrálhatjuk $P'$-t $x$ szerint, $Q'$-t pedig $y$ szerint:
  \begin{align*}
    F(x;y) & = \int P' \dd x
    = \int e^{2y} (y^3 + y) x^2 \dd x
    = \frac{e^{2y} \, x^3}{3} \left( y^3 + y \right) + C_y(y)
    \text,
    \\
    F(x;y) & = \int Q' \dd y
    = \frac{x^3}{3} \int e^{2y} (2y^3 + 3y^2 + 2y + 1) \dd y
    = \frac{e^{2y} \, x^3}{3} \left( y^3 + y \right) + C_x(x)
    \text.
  \end{align*}
  Ahol $C_y(y)$ és $C_x(x)$ rendre $y$-tól és $x$-tól függő függvények,
  jelen esetben nullának válaszhatóak. A megoldás tehát:
  $$
    \frac{e^{2y} \, x^3}{3} \left( y^3 + y \right) = C
    \text.
  $$
\end{example}

\begin{note}[][nobreak]
  Az előző példában szereplő differenciálegyenlethez, csak $x$-től függő
  multiplikátor nem létezik, hiszen az
  $$
    \ln M(x)
    = \int \frac{\partial_x P - \partial_y Q}{Q} \dd x
    = \int \frac{x^2\left(-2y^3 - 2y\right)}{x^3\left(2y^3 + 3y^2 + 2y + 1\right)/3} \dd x
  $$
  integrálban az integrálandó tag $y$-tól is függ.
\end{note}