\documentclass{szb-solution}

\title{Ismétlés, operátorok}
\area{Vektoranalízis}
\subject{Matematika G3}
\subjectCode{BMETE94BG03}
\date{Utoljára frissítve: \today}
\docno{1}

\begin{document}
\maketitle

\subsection{Leképezés vizsgálata}

Adja meg a $\varphi : \Reals^3 \to \Reals^3$ leképezés mátrixát a
standard normális, illetve az $\rvec b_1(1; 0; 0)$, $\rvec b_2(1; 1; 0)$
és $\rvec b_3(1; 1; 1)$ vektorok alkotta bázisban. Adja meg a leképezés
magterének és képterének dimenzióját is!
$$
  \varphi : \Reals^3 \to \Reals^3
  \qquad
  \begin{bmatrix}
    x \\ y \\ z
  \end{bmatrix}
  \mapsto
  \begin{bmatrix}
    2x - y \\ y + z \\ y - z
  \end{bmatrix}
$$

A leképezés mátrixa a standard bázisban:
$$
  \varphi \begin{bmatrix}
    1 \\ 0 \\ 0
  \end{bmatrix} = \begin{bmatrix}
    2 \\ 0 \\ 0
  \end{bmatrix}
  \text,\quad
  \varphi \begin{bmatrix}
    0 \\ 1 \\ 0
  \end{bmatrix} = \begin{bmatrix}
    -1 \\ 1 \\ 1
  \end{bmatrix}
  \text,\quad
  \varphi \begin{bmatrix}
    0 \\ 0 \\ 1
  \end{bmatrix} = \begin{bmatrix}
    0 \\ 1 \\ -1
  \end{bmatrix}
  \quad\Rightarrow\quad
  \rmat A = \begin{bmatrix}
    2 & -1 & 0  \\
    0 & 1  & 1  \\
    0 & 1  & -1
  \end{bmatrix}
  \text.
$$

A bázistranszformáció mátrixa, illetve annak inverze:
$$
  \rmat T = \begin{bmatrix}
    \rvec b_1 & \rvec b_2 & \rvec b_3
  \end{bmatrix} = \begin{bmatrix}
    1 & 1 & 1 \\
    0 & 1 & 1 \\
    0 & 0 & 1
  \end{bmatrix}
  \text,\qquad
  \rmat T^{-1} = \begin{bmatrix}
    1 & -1 & 0  \\
    0 & 1  & -1 \\
    0 & 0  & 1
  \end{bmatrix}
  \text.
$$

A leképezés mátrixa az új bázisban:
$$
  \hat{\rmat A} = \rmat T^{-1} \rmat A \rmat T
  = \begin{bmatrix}
    1 & -1 & 0  \\
    0 & 1  & -1 \\
    0 & 0  & 1
  \end{bmatrix} \begin{bmatrix}
    2 & -1 & 0  \\
    0 & 1  & 1  \\
    0 & 1  & -1
  \end{bmatrix} \begin{bmatrix}
    1 & 1 & 1 \\
    0 & 1 & 1 \\
    0 & 0 & 1
  \end{bmatrix} = \begin{bmatrix}
    2 & 0 & -1 \\
    0 & 0 & 2  \\
    0 & 1 & 0
  \end{bmatrix}
  \text.
$$

A leképezés mátrixának rangja maximális ($\rg \varphi = 3$), hiszen
$$
  \det \rmat A = 2 \cdot \begin{vmatrix}
    1 & 1  \\
    1 & -1
  \end{vmatrix} - (-1) \cdot \begin{vmatrix}
    0 & 1  \\
    0 & -1
  \end{vmatrix} + 0 \cdot \begin{vmatrix}
    0 & 1 \\
    0 & 1
  \end{vmatrix} = 2 \cdot (-1 -1) + 1 \cdot (0) + 0 = -4 \neq 0
  \text.
$$

A magtér dimenziója a rang-nullitás tétel alapján:
$$
  \dim \ker \varphi = \dim \Reals^3 - \rg \varphi = 3 - 3 = 0
  \text.
$$

\subsection{Mátrix felbontása}

Bontsa fel az $\rmat A$ mátrixot szimmetrikus és antiszimmetrikus komponensekre!
$$
  \rmat A = \begin{bmatrix}
    1 & 1 & 0 \\
    2 & 2 & 1 \\
    1 & 4 & 3
  \end{bmatrix}
  \qquad\Rightarrow\qquad
  \rmat A^\T = \begin{bmatrix}
    1 & 2 & 1 \\
    1 & 2 & 4 \\
    0 & 1 & 3
  \end{bmatrix}
$$
A szimmetrikus komponens:
$$
  \rmat A_\text{s} = \frac{\rmat A + \rmat A^\T}{2} = \frac{1}{2} \begin{bmatrix}
    2 & 3 & 1 \\
    3 & 4 & 5 \\
    1 & 5 & 6
  \end{bmatrix} = \begin{bmatrix}
    1   & 3/2 & 1/2 \\
    3/2 & 2   & 5/2 \\
    1/2 & 5/2 & 3
  \end{bmatrix}
  \text.
$$
Az antiszimmetrikus komponens:
$$
  \rmat A_\text{as} = \frac{\rmat A - \rmat A^\T}{2} = \frac{1}{2} \begin{bmatrix}
    0 & -1 & -1 \\
    1 & 0  & -3 \\
    1 & 3  & 0
  \end{bmatrix} = \begin{bmatrix}
    0   & -1/2 & -1/2 \\
    1/2 & 0    & -3/2 \\
    1/2 & 3/2  & 0
  \end{bmatrix}
  \text.
$$

\subsection{Jacobi-mátrix}

Adja meg a $\varphi$ és $\psi$ leképezések Jacobi-mátrixát!

\begin{alignat*}{9}
  \varphi         & : \Reals^3 \to \Reals^2
  \quad
  \begin{bmatrix}
    x \\ y \\ z
  \end{bmatrix}
  \mapsto
  \begin{bmatrix}
    x^2 + z \\ y z^2
  \end{bmatrix}
  \qquad\qquad
                  &
  \psi            & : \Reals^2 \to \Reals^2
  \quad
  \begin{bmatrix}
    x \\ y
  \end{bmatrix}
  \mapsto
  \begin{bmatrix}
    \sin \ln (x y^2) \\ \sqrt{e^{xy} + \tan y}
  \end{bmatrix}
  \\
  \rmat J_\varphi & = \begin{bmatrix}
                        2x & 0   & 1   \\
                        0  & z^2 & 2yz
                      \end{bmatrix}
                  &
  \rmat J_\psi    & = \begin{bmatrix}
                        \dfrac{\cos \ln (x y^2)}{x}              &
                        \dfrac{2 \cos \ln (x y^2)}{y}              \\[5mm]
                        \dfrac{y\,e^{xy}}{2\sqrt{e^{xy}+\tan y}} &
                        \dfrac{x\,e^{xy}+1 / \cos^2 y}{2\sqrt{e^{xy}+\tan y}}
                      \end{bmatrix}
\end{alignat*}

\subsection{Gradiens}

Adja meg az alábbi leképezések gradienseit!
($C \in \Reals, \rvec a \in \Reals^3$)
\begin{enumerate}[a)]
  \item $f (\coordvec) = C \cdot \coordvec^2$
        $$
          \grad f
          = C \cdot \grad \scalar{\coordvec}{\coordvec}
          = C \cdot \grad (x^2 + y^2 + z^2)
          = C \cdot \begin{bmatrix}
            2x \\ 2y \\ 2z
          \end{bmatrix}
          = 2C \cdot \coordvec
        $$

  \item $g (\coordvec) = |\coordvec|$
        $$
          \grad g
          = \grad \sqrt{x^2 + y^2 + z^2}
          = \frac{1}{2\sqrt{x^2 + y^2 + z^2}} \cdot \begin{bmatrix}
            2x \\ 2y \\ 2z
          \end{bmatrix} = \frac{1}{\sqrt{x^2 + y^2 + z^2}} \cdot \begin{bmatrix}
            x \\ y \\ z
          \end{bmatrix} = \frac{\coordvec}{|\coordvec|}
        $$

  \item $h (\coordvec) = \scalar{\rvec a}{\coordvec}$
        $$
          \grad h
          = \grad (a_1 x + a_2 y + a_3 z)
          = \begin{bmatrix}
            a_1 \\ a_2 \\ a_3
          \end{bmatrix}
          = \rvec a
        $$
\end{enumerate}

\subsection{Divergencia és rotáció Jacobi-mátrix alapján}

Adja meg a $\rvec v (\coordvec)$ vektormező divergenciáját és rotációját!
Mely halmazokon forrás-, illetve örvénymentes a mező?
$$
  \rvec v(\coordvec) = \ijk{x^2 - y^2}{y^2 - z^2}{z^2 - x^2}
$$

A vektormező Javobi-mátrixa:
$$
  \rmat J = \begin{bmatrix}
    2x  & -2y & 0   \\
    0   & 2y  & -2z \\
    -2x & 0   & 2z
  \end{bmatrix}
$$

A divergencia:
$$
  \Div \rvec v = \operatorname{tr} \rmat J = 2x + 2y + 2z
  \text.
$$
A vektormező a $2x + 2y + 2z = 0$ síkon forrásmentes, hiszen itt a divergencia
nulla.

A rotáció:
$$
  \rot \rvec v = \operatorname{axl}(\rmat J - \rmat J^\T) = \begin{bmatrix}
    J_{32} - J_{23} \\
    J_{13} - J_{31} \\
    J_{21} - J_{12}
  \end{bmatrix} = \begin{bmatrix}
    0 - (-2z) \\
    0 - (-2x) \\
    0 - (-2y)
  \end{bmatrix} = \begin{bmatrix}
    2z \\ 2x \\ 2y
  \end{bmatrix}
  \text.
$$
A vektormező csak az origóban örvénymentes, hiszen itt a rotáció nullvektor.

\subsection{Divergencia és rotáció a Nabla-operator segítségével}

Adja meg az alábbi vektormezők divergenciáját és rotációját!
($C \in \Reals$, $\rvec a \in \Reals^3$)
\begin{enumerate}
  \item $\rvec u(\coordvec) = C \cdot \coordvec$
        \begin{align*}
          \Div \rvec u
           & = \scalar{
            \begin{bmatrix}
              \partial_x \\ \partial_y \\ \partial_z
            \end{bmatrix}
          }{
            \begin{bmatrix}
              Cx \\ Cy \\ Cz
            \end{bmatrix}
          } = C + C + C = 3C
          \\
          \rot \rvec u
           & =
          \begin{bmatrix}
            \partial_x \\ \partial_y \\ \partial_z
          \end{bmatrix}
          \times
          \begin{bmatrix}
            Cx \\ Cy \\ Cz
          \end{bmatrix}
          =
          \begin{bmatrix}
            \partial_y Cz - \partial_z Cy \\
            \partial_z Cx - \partial_x Cz \\
            \partial_x Cy - \partial_y Cx
          \end{bmatrix}
          =
          \begin{bmatrix}
            0 \\ 0 \\ 0
          \end{bmatrix}
        \end{align*}
        Az $\rvec u(\coordvec)$ vektormező sehol sem forrásmentes, viszont
        mindenhol örvénymentes.

  \item $\rvec v(\coordvec) = \grad \|\coordvec\|$
        \begin{align*}
          \Div \rvec v
           & = \scalar{
            \begin{bmatrix}
              \partial_x \\ \partial_y \\ \partial_z
            \end{bmatrix}
          }{
            \frac{1}{\sqrt{x^2 + y^2 + z^2}}
            \cdot
            \begin{bmatrix}
              x \\ y \\ z
            \end{bmatrix}
          }
          = \frac{3}{\sqrt{x^2 + y^2 + z^2}}
          + \left( -\frac{1}{2} \right) \cdot \frac{
            2x^2 + 2y^2 + 2z^2
          }{
            (x^2 + y^2 + z^2)^{3/2}
          }
          \\
           & = \frac{3}{\sqrt{x^2 + y^2 + z^2}}
          - \frac{x^2 + y^2 + z^2}{(x^2 + y^2 + z^2)^{3/2}}
          = \frac{3}{\|\coordvec\|} - \frac{\|\coordvec\|}{\|\coordvec\|^{3/2}}
          = \frac{3}{\|\coordvec\|} - \frac{1}{\|\coordvec\|}
          = \frac{2}{\|\coordvec\|}
          \\[2mm]
          \rot \rvec v
           & =
          \begin{bmatrix}
            \partial_x \\ \partial_y \\ \partial_z
          \end{bmatrix}
          \times \left(
          \frac{1}{\sqrt{x^2 + y^2 + z^2}} \cdot
          \begin{bmatrix}
            x \\ y \\ z
          \end{bmatrix}
          \right)
          = \frac{1}{(x^2 + y^2 + z^2)^{3/2}} \cdot
          \begin{bmatrix}
            -yz + zy \\
            xz - zx  \\
            -xy + yx
          \end{bmatrix} = \begin{bmatrix}
                            0 \\ 0 \\ 0
                          \end{bmatrix}
        \end{align*}
        Az $\rvec v(\coordvec)$ sehol sem forrásmentes, viszont mindenhol
        örvénymentes. A divergencia az origóban nincs értelmezve.

  \item $\rvec w(\coordvec) = \rvec a \cdot \ln |\coordvec|$
        \begin{align*}
          \Div \rvec w
           & = \scalar{
            \begin{bmatrix}
              \partial_x \\ \partial_y \\ \partial_z
            \end{bmatrix}
          }{
            \rvec a \cdot \ln \sqrt{x^2 + y^2 + z^2}
          }
          = \frac{1}{x^2 + y^2 + z^2} \cdot \left(
          a_1 x + a_2 y + a_3 z
          \right) = \frac{\scalar{\rvec a}{\coordvec}}{|\coordvec|^2}
          \\[2mm]
          \rot \rvec w
           & =
          \begin{bmatrix}
            \partial_x \\ \partial_y \\ \partial_z
          \end{bmatrix}
          \times
          \left(
          \rvec a \cdot \ln \sqrt{x^2 + y^2 + z^2}
          \right)
          = \frac{1}{x^2 + y^2 + z^2} \cdot
          \begin{bmatrix}
            a_3 y - a_2 z \\
            a_1 z - a_3 x \\
            a_2 x - a_1 y
          \end{bmatrix}
          = \frac{\coordvec \times \rvec a}{|\coordvec|^2}
        \end{align*}
        A $\rvec w(\coordvec)$ vektormező az $\scalar{\rvec a}{\coordvec} = 0$
        síkon forrásmentes, az $\coordvec = t \cdot \rvec a$ egyenesen
        örvénymentes ($t \in \Reals$). $\rvec w(\coordvec)$ az origóban
        nincs értelmezve.
\end{enumerate}

\end{document}