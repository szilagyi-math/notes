
\documentclass[a4paper, 12pt]{scrartcl}

\usepackage{math-practice}

\area{Vektoranalízis}
\title{Operátorok}
\subject{Matematika G3}
\subjectCode{BMETE94BG03}
\date{Utoljára frissítve: \today}
\docno{1}

\begin{document}
\maketitle

\subsection{Elméleti Áttekintő}

\begin{definition}[Nabla -- operátor]
  Az $\mathbb R^n$-beli Descartes koordináta-rendszerben, ahol $\rvec x = (x_1;
    x_2; \dots; x_n)$ egy tetszőleges pont koordinátái, a standard bázis pedig
  $\{ \uvec e_1; \uvec e_2; \dots; \uvec e_n \}$ a Nabla egy olyan formális
  differenciáloperátor, melynek koordinátái a parciális derivált operátorok,
  vagyis:
  $$
    \nabla = \sum_{i = 1}^n \uvec e_i \pdv{}{x_i}
    =
    \begin{pmatrix}
      \displaystyle\pdv{}{x_1} &
      \displaystyle\pdv{}{x_2} &
      \hdots                   &
      \displaystyle\pdv{}{x_n}
    \end{pmatrix}^\T
    \text.
  $$
  % A három dimenziós térben az operátor az alábbi alakot veszi fel:
  % $$
  %   \nabla = \begin{pmatrix}
  %     \displaystyle\pdv{}{x} &
  %     \displaystyle\pdv{}{y} &
  %     \displaystyle\pdv{}{z}
  %   \end{pmatrix}^\T
  %   \text.
  % $$
\end{definition}

\begin{blueBox}
  \sftitle{Differenciáloperátorok:}

  Legyen $\rvec v(\rvec r) : \mathbb R^3 \to \mathbb R^3$ vektormező,
  $\varphi(\rvec r): \mathbb R^3 \to \mathbb R$ skalármező, ahol
  $\rvec r$ az $\mathbb R^3$-beli Descartes koordináta-rendszerben
  $\rvec r = (x; y; z)$.
  \begin{center}
    \def\arraystretch{1.5}
    \newenvironment{bm}{\bgroup\renewcommand*{\arraystretch}{1.1}\begin{bmatrix}}{\end{bmatrix}\egroup}
    \newcommand{\dspl}[3]{\begin{bm}#1\\#2\\#3\end{bm}}
    \newcommand\nablavec{\dspl{\partial_x}{\partial_y}{\partial_z}}

    \begin{tabular}{*{3}{>{\centering\arraybackslash}p{3.5cm}}}
      \def\arraystretch{1}
      % &
      \bfseries Rotáció
       & \bfseries Divergencia
       & \bfseries Gradiens
      \\
      \hline
      % Jelölés & 
      $\rot \rvec v$
       & $\Div \rvec v$
       & $\grad \varphi$
      \\
      % Operátor & 
      $\nabla \times \rvec v$
       & $\scalar{\nabla}{\rvec v}$
       & $\nabla \cdot \varphi$
      \\
      % Számítás &
      $\nablavec \times \dspl{v_x}{v_y}{v_z}$
       & $\scalar{\nablavec}{\dspl{v_x}{v_y}{v_z}}$
       & $\dspl{\partial_x \varphi}{\partial_y \varphi}{\partial_z \varphi}$
      \\
      % Ért. tart. & 
      $\mathcal D_{\rvec v} = \mathbb R^3$
       & $\mathcal D_{\rvec v} = \mathbb R^3$
       & $\mathcal D_{\varphi} = \mathbb R^3$
      \\
      % Ért. készl. &
      $\mathcal R_{\rvec v} = \mathbb R^3$
       & $\mathcal R_{\rvec v} = \mathbb R^3$
       & $\mathcal R_{\varphi} = \mathbb R$
      \\
      % Ért. készl. &
      $\mathcal R_{\rot \rvec v} = \mathbb R^3$
       & $\mathcal R_{\Div \rvec v} = \mathbb R$
       & $\mathcal R_{\grad \varphi} = \mathbb R^3$
      \\
      % \hline
    \end{tabular}
  \end{center}
\end{blueBox}

\begin{note}
  Speciális esetek:
  \begin{itemize}
    \item ha $\Div \rvec v = 0$, akkor a vektromező forrásmentes,
    \item ha $\rot \rvec v = \nvec$, akkor a vektromező örvénymentes.
  \end{itemize}
\end{note}

\begin{definition}[Laplace -- operátor]
  A Laplace-operátor egy másodrendű differenciáloperátor az $n$ dimenziós
  térben. Megadja egy skalármező gradiensének divergenciáját, azaz:
  \[
    \triangle \varphi
    =
    \scalar{\nabla}{\nabla} \varphi
    =
    \Div \grad \varphi
    \text.
  \]
\end{definition}

\clearpage
\subsection{Feladatok}

\begin{enumerate}
  \item Adja meg a $\varphi : \Reals^3 \to \Reals^3$ leképezés mátrixát a
        standard normális, illetve az $\rvec b_1(1; 0; 0)$, $\rvec b_2(1; 1; 0)$
        és $\rvec b_3(1; 1; 1)$ vektorok alkotta bázisban. Adja meg a leképezés
        megterének és képterének dimenzióját is!
        $$
          \varphi : \Reals^3 \to \Reals^3
          \qquad
          \begin{bmatrix}
            x \\ y \\ z
          \end{bmatrix}
          \mapsto
          \begin{bmatrix}
            2x - y \\ y + z \\ y - z
          \end{bmatrix}
        $$

  \item Bontsa fel az $\rmat A$ mátrixot szimmetrikus és antiszimmetrikus
        komponensekre!
        $$
          \rmat A = \begin{bmatrix}
            1 & 1 & 0 \\
            2 & 2 & 1 \\
            1 & 4 & 3
          \end{bmatrix}
        $$

  \item Adja meg a $\varphi$ és $\psi$ leképezés Jacobi-mátrixát!
        $$
          \varphi : \Reals^3 \to \Reals^2
          \quad
          \begin{bmatrix}
            x \\ y \\ z
          \end{bmatrix}
          \mapsto
          \begin{bmatrix}
            x^2 + z \\ y z^2
          \end{bmatrix}
          \qquad\qquad
          \psi : \Reals^2 \to \Reals^2
          \quad
          \begin{bmatrix}
            x \\ y
          \end{bmatrix}
          \mapsto
          \begin{bmatrix}
            \sin \ln x^2 y \\ \sqrt{e^{xy} + \tan y}
          \end{bmatrix}
        $$

  \item Adja meg az alábbi leképezések gradienseit!
        ($C \in \Reals$)
        \begin{multicols}{3}
          \begin{enumerate}
            \item $f (\rvec r) = C \cdot \rvec r^2$,
            \item $g (\rvec r) = |\rvec r|$
            \item $h (\rvec r) = C \cdot \rvec r$
          \end{enumerate}
        \end{multicols}

  \item Adja meg a $\rvec v (\rvec r)$ vektormező divergenciáját és rotációját!
        Mely halmazokon forrás-, illetve örvénymentes a mező?
        $$
          \rvec v(\rvec r)
          = (x^2 - y^2) \rvec i
          + (y^2 - z^2) \rvec j
          + (z^2 - x^2) \rvec k
        $$

  \item Adja meg az alábbi vektormezők divergenciáját és rotációját!
        ($C \in \Reals$, $\rvec a \in \Reals^3$)
        \begin{multicols}{3}
          \begin{enumerate}
            \item $\rvec v_1(\rvec r) = C \cdot \rvec r$,
            \item $\rvec v_2(\rvec r) = \grad |\rvec r|$,
            \item $\rvec a \cdot \ln |\rvec r|$.
          \end{enumerate}
        \end{multicols}

  \item Hattassa a Laplace-operátort a $\rvec v (\rvec r) = |\rvec a| \rvec r +
          \rvec a |\rvec r|$ vektormezőre! ($\rvec a \in \Reals^3$)
\end{enumerate}
\end{document}