\documentclass[a4paper, 12pt]{scrartcl}

\usepackage{math-solution}

\newcommand\coordv{\rvec r}
\newcommand\arcv{\rvec r}
\newcommand\surfv{\rvec \varrho}

\title{Integrálátalakító tételek}
\area{Többváltozós analízis}
\subject{Matematika G3}
\subjectCode{BMETE94BG03}
\date{Utoljára frissítve: \today}
\docno{6}

\begin{document}
\allowdisplaybreaks

\maketitle

\begin{enumerate}
  \item Egy automata raktárrendszer ferromágneses manipulátora egy előre
        számított mágneses potenciálmezőben mozog. A számított Joule-potenciál a
        munkatérben
        $$
          \varphi(\coordv) = 2x^2 y + 3yz
          \text.
        $$
        A csípőkaron lévő, $Q = 1\,\text{C}$ ekvivalens töltéssel modellezett
        végfogót a vezérlőnek az
        $$
          A(0;0;0) \rightarrow B(2;1;3)
        $$
        pontok között kell mozgatnia. Mennyi munkát végez a mágneses tér a
        végfogón, és miért nem kell törődnünk az útvonal pontos alakjával?
        ($\rvec F = -Q \grad \varphi$, a munka az erő pálya menti integrálja)

  \item Egy elektromágneses rendszer mágneses térerősségét a
        $$
          \rvec B(\coordv) = \ijk{y \, \sin x}{z^2 \, \cos y - \cos x}{b_3}
        $$
        vektormező írja le. Határozza meg $b_3$ értékét, ha tudjuk, hogy
        a vektormező bármely zárt görbe mentén vett integrálja zérus!
        (Vagyis mikor lesz $\rot \rvec B = \nvec$)

  \item Egy $R = 1$ sugarú, kör keresztmetszetű, $z$ tengellyel egybeeső
        szimmetriavonalú hengerben áramló folyadék sebességét a
        $$
          \rvec v(\coordv) = \ijk{2xy + z}{x^2 + z}{y - x}
        $$
        vektormező írja le. Adja meg a $z = 1$ síkban lévő keresztmetszet
        menti cirkulációt!
        (A cirkuláció a vektormező zárt görbe menti integrálja.)

  \item Egy fotonikus chipeket hordozó wafer-darabot egy ellipszoid alakú
        Faraday-kalitkába rögzítenek. A kalitka belsejében lineáris
        feszültségelosztással ($\rvec E(\coordv) = \coordv$) térerőt állítanak
        elő. Számolja ki a Faraday-kalitka belsejében lévő nettó töltést, ha
        $\varepsilon_0 = 8,85 \times 10^{-12} \, \text{F/m}$, az ellipszoid
        egyenlete pedig:
        $$
          \frac{(x - 2)^2}{5} + \frac{(y + 3)^2}{19} + \frac{(z - 1)^2}{4} = 1
          \text.
        $$

  \item Egy drón IMU-modulját teljes egészében kitöltő, hővezető műgyanta gömb
        alakú, sugara $R = 0,02 \, \text{m}$, A vezérlő egység folyamatosan
        hőt disszipál, az állandósult hő\-mér\-sék\-let-mező jó közelítéssel
        $$
          \varphi(\coordv) = T_c - \alpha \rvec r^2
          \text, \quad
          \alpha = 1,3 \times 10^5 \,\text{K/m}^2
          \text.
        $$
        Becsülje meg, mekkora teljes hőáram távozik a burkolaton át, ha
        $$
          q_{\text{hő}}
          = -\lambda \oint_{\partial V} \scalar{\grad \varphi}{\dd \rvec S}
          \text,\quad
          \lambda = 0,2 \, \text{W/(m K)}
          \text.
        $$
\end{enumerate}

\end{document}

