\section{Fogalmak, definíciók}

\begin{definition}[Sorozat]
  A pozitív egész számok halmazán értelmezett $a_n:\mathbb{N}\rightarrow
    \Reals$ függvényt \textbf{valós sorozat}nak hívjuk,
  az $a_n:\mathbb{N}\rightarrow \mathbb{C}$ függvényt \textbf{komplex
    sorozat}nak nevezzük.
\end{definition}

\begin{definition}[Konvergencia]
  Az $(a_n)$ sorozatot konvergensnek mondjuk, ha $\exists a \in \Reals$ valós
  szám, hogy $\forall \varepsilon > 0$ esetén $\exists N(\varepsilon)$
  küszöbszám, hogy $|a_n - a| < \varepsilon$, ha $n > N(\varepsilon)$.
  Jelölése:
  \[
    \lim_{n \rightarrow \infty} a_n = a
    \text{, ahol $a$ a sorozat határértéke.}
  \]
\end{definition}

\begin{definition}[Divergencia]
  Az $(a_n)$ sorozatot divergensnek mondjuk, ha nem konvergens.
\end{definition}

\begin{note}
  Egy konvergens sorozatnak pontosan egy határértéke van.
\end{note}

\begin{note}
  Szükséges és elégséges feltételek egy sorozat konvergenciájára.

  Következmény: ha egy sorozatban véges sok elemet megváltoztatunk, vagy egy
  sorozathoz véges sok elemet hozzáveszünk, vagy belőle véges sok elemet
  elveszünk, akkor az sem a sorozat határértékét, sem a konvergenciáját nem
  változtatja meg.
\end{note}

\begin{definition}[Sorozat korlátossága]
  \begin{itemize}
    \item Az $(a_n)$-t \textbf{alulról korlátos}nak nevezzük,
          ha értékkészlete alulról korlátos.
    \item Az $(a_n)$-t \textbf{felülről korlátos}nak nevezzük,
          ha értékkészlete felülről korlátos.
    \item Az $(a_n)$ sorozat \textbf{korlátos},
          ha alulról és felülről is korlátos.
  \end{itemize}
\end{definition}

\begin{note}
  Konvergens sorozat korlátos. (Az állítás megfordítása nem igaz.)
\end{note}

\vspace{-.5em}
\begin{definition}[Műveletek sorozatokkal]
  Legyenek $(a_n)$ és $(b_n)$ sorozatok, $\lambda \in \Reals$, ekkor:
  \begin{itemize}
    \item $(a_n) + (b_n) := (a_n + b_n)$,
    \item $\lambda \cdot (a_n) := (\lambda \cdot a_n)$,
    \item $(a_n)\cdot (b_n):=(a_n\cdot b_n)$,
    \item $(a_n) / (b_n) = (a_n / b_n)$, ha $b_n \neq 0$.
  \end{itemize}
\end{definition}

\begin{note}
  Legyenek $(a_n)$ és $(b_n)$ konvergens sorozatok $a_n \rightarrow a$,
  $b_n \rightarrow b$, ha $n \rightarrow \infty $ és legyen ${\lambda \in
      \Reals}$, ekkor ezen sorozatok összege, számszorosa, szorzata és
  hányadosa is konvergens, és:
  \begin{itemize}
    \item $a_n + b_n \rightarrow{a + b}$,
    \item $\lambda \cdot a_n \rightarrow \lambda \cdot a$,
    \item $a_n \cdot b_n\rightarrow a \cdot b$,
    \item $(a_n / b_n) \rightarrow (a / b)$, ha $b \neq 0$
  \end{itemize}
\end{note}

% TODO: FIX
\begin{note}
  Konvergens sorozat jeltartó, ha
  \[
    \lim_{n \rightarrow\infty} a_n=a\neq 0
    \text,
  \]
  akkor $\exists N_0$ index, hogy $\sgn a_n = \sgn a $, ha $n > N_0$.

  Következmény:
  \[
    \lim_{n \rightarrow \infty} a_n = a
    \text{ és }
    \lim_{n \rightarrow \infty} b_n = b
    \text{ és }
    a_n \geq b_n
    \quad \Rightarrow \quad
    a \geq b
    \text,
  \]
  azaz a határátmenet rendezéstartó.
\end{note}

\begin{theorem}[Rendőr tétel]
  Tegyük fel, hogy $(a_n)$, $(b_n)$ és $(x_n)$ sorozatokra teljesül, hogy $a_n
    \leq x_n \leq b_n : \forall n$-re vagy $n > N_0$, továbbá
  \[
    \lim_{n \rightarrow \infty} a_n = \lim_{n \rightarrow \infty} b_n = a
    \text{, ekkor:}
    \lim_{n \rightarrow \infty} x_n = a
    \text.
  \]

  % Következmény:
  % Ha $(a_n)$ és $(b_n)$ sorozatok nullsorozatok, akkor a rendőr tételből
  % következik, hogy szorzatuk is nullsorozat.
\end{theorem}

\begin{note}
  Ha $(a_n)$ és $(b_n)$ sorozatok nullsorozatok, akkor a rendőr tételből
  következik, hogy szorzatuk is nullsorozat.
\end{note}

\begin{note}
  Ha $(a_n)$ sorozat konvergens és határértéke $a$, akkor
  \[
    \lim\limits_{n \rightarrow \infty} |a_n|=|a|
    \text.
  \]
  Visszafelé ez nem igaz (csak nullsorozatokra).
\end{note}

\begin{definition}[Kibővített valós számok halmaza]
  Az $\Reals_b := \Reals \cup \{-\infty; \infty\}$ halmazt kibővített valós
  számok halmazának nevezzük.
\end{definition}


\begin{definition}[Sorozat határértéke $\pm\infty$]
  Azt mondjuk, hogy az $(a_n)$ határértéke $\infty$, ha $\forall K \in \Reals$
  esetén $\exists N_K$ : $a_n > K$, ha $n > N_K$.

  Azt mondjuk, hogy az $(a_n)$ határértéke $-\infty$, ha $\forall K \in \Reals$
  esetén $\exists N_K$ : $a_n < K$, ha $n > N_K$.
\end{definition}

% \begin{definition}[Sorozat határértéke $-\infty$]
%   Azt mondjuk, hogy az $(a_n)$ határértéke $-\infty$, ha $\forall K \in \Reals$
%   esetén $\exists N_K$ : $a_n < K$, ha $n > N_K$.
% \end{definition}

\begin{definition}[Sorozat monotonitása]
  Az $(a_n)$ sorozat monotonitása:
  \begin{itemize}
    \item monoton növekvő, ha $a_n \geq a_{n-1}$,
    \item monoton csökkenő, ha $a_n \leq a_{n-1}$,
    \item szigorúan monoton növekvő, ha $a_n > a_{n-1}$,
    \item szigorúan monoton csökkenő, ha $a_n < a_{n-1}$.
  \end{itemize}
\end{definition}

\begin{note}
  Ha az $(a_n)$ monoton növekvő, akkor alulról korlátos, illetve ha monoton
  csökkenő, akkor felülről korlátos.
\end{note}

\begin{statement}
  \begin{enumerate}
    \item Monoton, korlátos sorozat konvergens.
    \item Monoton, nem korlátos sorozatnak van határértéke.
    \item Ha egy sorozat divergens, akkor vagy nem létezik a határértéke, vagy a
          határértéke $\infty$ vagy $-\infty$.
  \end{enumerate}
\end{statement}

\begin{definition}[Részsorozat]
  $!(k_n)$ a természetes számok egy szigorúan növekvő sorozata és $(a_n)$ egy
  valós számsorozat, ekkor a $b_n = a_{k_n}$ sorozatot az $a_n$ sorozat $k_n$
  indexsorozathoz tartozó részsorozatának nevezzük.
\end{definition}

\begin{theorem}[Sorozat határértékének létezése]
  Ha az $a_n$ sorozatnak van határértéke, akkor bármely részsorozatának is van
  határértéke, és ez a két határérték megegyezik.
\end{theorem}

\begin{theorem}[Monoton részsorozat létezése]
  Bármely sorozatnak van monoton részsorozata.
\end{theorem}

\begin{theorem}[Bolzano--Weierstrass-tétel]
  Minden korlátos sorozatnak van konvergens részsorozata.
\end{theorem}

\begin{definition}[Limesz szuperior és inferior]
  Az $(a_n)$ sorozat limesz szuperiorjának nevezzük az alábbi mennyiséget:
  \[
    \lim \sup a_n
    = \overline{\lim} \; a_n
    := \lim_{n \rightarrow \infty} \sup \{\;
    a_{n + 1}; a_{n + 2}; \dots
    \;\}
    \text.
  \]

  Az $(a_n)$ sorozat limesz inferiorjának nevezzük az alábbi mennyiséget:
  \[
    \lim \inf a_n
    = \underline{\lim} \; a_n
    := \lim_{n \rightarrow \infty} \inf \{\;
    a_{n + 1}; a_{n + 2}; \dots
    \;\}
    \text.
  \]
\end{definition}

% \begin{note}
%   Sorozat limesz szuperiorja sok esetben egybeesik a határértékkel.
% \end{note}

% \begin{definition}[Limesz inferior]
%   Az $(a_n)$ sorozat limesz inferiorjának nevezzük az alábbi mennyiséget:
%   \[
%     \lim \inf a_n
%     = \underline{\lim} \; a_n
%     = \lim_{n \rightarrow \infty} \inf \{\;
%     a_{n + 1}; a_{n + 2}; \dots
%     \;\}
%     \text.
%   \]
% \end{definition}

\begin{definition}[Cauchy-sorozat]
  Az $(a_n)$ sorozatot Cauchy-sorozatnak nevezzük, ha $\forall \varepsilon > 0$
  esetén $\exists N(\varepsilon)$, hogy $|a_n - a_m| < \varepsilon$, ha $n, m >
    N(\varepsilon)$.
\end{definition}

\begin{theorem}[Cauchy-féle konvergencia kritérium]
  Az $(a_n)$ konvergens $\Leftrightarrow$ ha Cauchy-sorozat.
\end{theorem}