\vfill
\begin{questions}[section.5.4]
  \begin{enumerate}
    \item Definiálja a függvény fogalmát! Mi az értelmezési tartomány? Mi az
          értékkészlet?
    \item Mit jelent az, hogy egy függvény
          \begin{itemize}
            \item Monoton,
            \item Szigorúan monoton,
            \item Korlátos,
            \item Páros, páratlan,
            \item Periodikus?
          \end{itemize}
    \item Adjon példát a fentebb említett tulajdonsággel rendelkező, nem
          rendelkező függvényekre!
    \item Hogyan kapható meg az $f(x)$ függvényből
          \begin{itemize}
            \item $f(x+b)$, ha $b>0$,
            \item $f(x)+b$, ha $b>0$,
            \item $a \cdot f(x)$,
            \item $f(a \cdot x)$,
            \item $f(|x|)$,
            \item $|f(x)|$.
          \end{itemize}
    \item Mit jelent a függvény zérushelye?
    \item Vázolja fel a középiskolában megtanult függvényeket!
    \item Mikor nevezünk egy függvényt összetettnek? Adjon példát! Mutassa be a
          külső-, belső függvény fogalmát!
    \item Mikor mondjuk, hogy egy függvény invertálható?
    \item Hogyan kapjuk meg az eredeti függvény képéből annak inverzét, ha
          létezik?
    \item Hogyan viszonyul az inverz függvény monotonitása az eredeti függvény
          monotonitásához?
    \item Mutassa be a trigonometrikus függvények inverzeit!
    \item Mutassa be az hiperbolikus függvényeket és azok inverzeit!
    \item Definiálja a függvény határértékét az $x = a$ pontban és plusz-,
          illetve mínusz végtelenben!
    \item Definiálja a folytonosságot!
  \end{enumerate}
\end{questions}