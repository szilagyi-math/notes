\mainChapter{Függvények}

\begin{tikzpicture}[
    remember picture,
    overlay,
    % ultra thick,
    shift={($(current page.south west) + (\innerMargin-20mm, 7.5mm)$)},
  ]

  \draw[gray!40, thin] (-0.75,-.75) grid (\paperwidth,\paperheight);

  \coordinate (BL) at (\innerMargin-0.25cm,10.75cm-\innerMargin);
  \coordinate (TR) at ($(\textwidth+5mm,\paperheight-\innerMargin-15.25cm)+(BL)$);

  % tan
  \draw[ultra thick, primaryColor, domain=-1.55:1.45, samples=100, smooth]
  plot ({\x*3+5}, {6-tan(\x r)});
  \draw[ultra thick, primaryColor, domain=1.6:4.6, samples=100, smooth]
  plot ({\x*3+5}, {6-tan(\x r)});
  \draw[ultra thick, primaryColor, domain=4.75:7, samples=100, smooth]
  plot ({\x*3+5}, {6-tan(\x r)});

  % horizontal line
  \draw[ultra thick, gray, dashed] (-1,6) -- (22,6);

  % vertical asymptotes
  \draw[ultra thick, gray, dashed] (0.29,\paperheight) -- ++(0,-\paperheight-1cm);
  \draw[ultra thick, gray, dashed] (9.71,\paperheight) -- ++(0,-\paperheight-1cm);
  \draw[ultra thick, gray, dashed] (19.14,\paperheight) -- ++(0,-\paperheight-1cm);



  \fill[white, fill opacity=.9] (BL) rectangle (TR);


  % sin
  % \draw[ultra thick, secondaryColor, domain=-1.5:20.5, samples=200, smooth]
  % plot (\x, {2+sin(\x r)});

  % 1 / x
  % \draw[ultra thick, ternaryColor, domain=0.01:5, samples=500, smooth]
  % plot ({\x+19}, {1/\x + 6});
  % \draw[ultra thick, ternaryColor, domain=-20:-0.1, samples=500, smooth]
  % plot ({\x+19}, {1/\x + 6});


\end{tikzpicture}

\bgroup
\color{gray!50!black}
\sffamily

A matematikában a függvények alapvető fogalmak, amelyek a változók közötti
kapcsolatokat írják le. Egy függvény egy szabály, amely egy adott halmaz egy
eleméhez pontosan egy elemet rendel egy másik halmazból. Az előbbi halmazt
értelmezési tartománynak, az utóbbit értékkészletnek hívjuk. A függvényeket
gyakran ábrázoljuk grafikusan, az $x$-tengelyen a független változó, az
$y$-tengelyen pedig a függő változó értékei feltüntetve. A függvényekkel
kapcsolatos alapvető fogalmak megértése elengedhetetlen a matematika további
területeinek tanulmányozásához.

Középiskolai tanulmányaink során megismertük a legfontosabb függvényosztályokat:
a lineáris, a kvadratikus és egyéb hatványfüggvényeket, a trigonometrikus, az
exponenciális és a logaritmikus függvényeket, foglalkoztunk
függvénytranszformációkkal. Ebben a fejezetben áttekintjük a függvények
legfontosabb jellemzőit, amelyek a matematikai analízis alapkövei,
tanulmányozásuk nemcsak a matematikai elméletben, hanem a gyakorlati
alkalmazásokban is kulcsfontosságú, hiszen a függvények segítségével
modellezhetjük a világ különböző jelenségeit.

\chaptertoc
\egroup

\clearpage