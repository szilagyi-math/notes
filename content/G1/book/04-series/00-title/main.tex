\mainChapter{Sorok}

\begin{tikzpicture}[
    remember picture,
    overlay,
    ultra thick,
    shift={($(current page.south west) + (\innerMargin+75mm, 4.25cm)$)},
    % scale=1/4,
  ]
  \def\triangles#1{%
    \pgfmathsetmacro\currentVal{int(#1)}
    \begin{scope}[transparency group, fill opacity=0.2]
      \foreach \x in {\currentVal,...,0} {
          \pgfmathsetmacro\currentRadius{pow(2, \x) / sqrt(3) / 8}
          \pgfmathsetmacro\currentRotation{mod(\x, 2) * 180}
          \pgfmathsetmacro\currentMod{int(mod(\x, 3))}

          \ifnum\currentMod=2
            \def\fillcolor{primaryColor}
          \fi
          \ifnum\currentMod=0
            \def\fillcolor{secondaryColor}
          \fi
          \ifnum\currentMod=1
            \def\fillcolor{ternaryColor}
          \fi

          \draw [thick, fill=\fillcolor] (\currentRotation+90:\currentRadius) --
          (\currentRotation+210:\currentRadius) --
          (\currentRotation+330:\currentRadius) -- cycle;
        }
    \end{scope}

    \foreach \x in {0,...,\currentVal} {
        \pgfmathsetmacro\currentRadius{pow(2, \x) / sqrt(3) / 8}
        \pgfmathsetmacro\currentRotation{mod(\x, 2) * 180}
        \pgfmathsetmacro\currentMod{int(mod(\x, 3))}

        \ifnum\currentMod=2
          \def\drawcolor{primaryColor}
        \fi
        \ifnum\currentMod=0
          \def\drawcolor{secondaryColor}
        \fi
        \ifnum\currentMod=1
          \def\drawcolor{ternaryColor}
        \fi

        \draw [ultra thick, draw=\drawcolor, rounded corners=2pt] (\currentRotation+90:\currentRadius) --
        (\currentRotation+210:\currentRadius) --
        (\currentRotation+330:\currentRadius) -- cycle;
      }
  }

  \node at (0,-17) {
    \begin{tikzpicture}[scale=1.85]
      \triangles{9}
    \end{tikzpicture}
  };
\end{tikzpicture}

\bgroup
\color{gray!50!black}
\sffamily

A numerikus sorok olyan matematikai objektumok, amelyek végtelen sok valós
számot adnak össze. Egy adott számsorozat tagjainak részletösszegeit sorba
rendezve kaphatjuk meg őket. Vizsgálatuk célja, hogy megállapítsuk, hogy a sorok
összegezhetők-e, azaz hogy a részletösszegek sorozata konvergens-e.

Ebben a fejezetben a numerikus sorok definiálását követően megismerjük alapvető
tulajdonságaikat szükséges, szükséges és elégséges feltételeket fogalmazunk meg
a konvergenciára, amely a sorok viselkedésének alapvető jellemzője. Foglalkozni
fogunk olyan nevezetes numerikus sorokkal, melyek különleges tulajdonságaik vagy
gyakori előfordulásuk miatt kiemelt figyelmet érdemelnek.

\chaptertoc
\egroup

\clearpage