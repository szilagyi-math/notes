\section{Határozatlan integrál}\label{section.7.1}

\vfill

\begin{definition}[Primitív függvény]
  Legyen $f: I \rightarrow \Reals$, ekkor az $F : I \rightarrow \Reals$
  függvényt az $f$ függvény primitív függvényének nevezzük $I$-n, ha $F$
  differenciálható $I$-n és $F'(x) = f(x) \quad \forall x \in I$.
\end{definition}

\vfill

\begin{note}
  Ha az $F$ függvény primitív függvénye az $f$-nek, akkor $G(x) = F(x) + C$
  ugyancsak primitív függvénye az $f$-nek, ahol $C \in \Reals$.
\end{note}

\vfill

\begin{note}
  Ha $F$ és $G$ is $f$ primitív függvényei, akkor $\exists C \in \Reals$, hogy
  $F(x) = G(x) + C$.
\end{note}

\vfill

\begin{definition}[Határozatlan integrál]
  Az $f$ primitív függvényeinek összességét $f$ határozatlan integráljának
  nevezzük az $I$-n. Jelölése:
  \[
    \int f(x) \dd x = F(x) + C
    \text.
  \]
\end{definition}

\vfill

\begin{blueBox}
  Az $\; 1,\; x,\; e^x,\; \ln x,\; \sin x,\; \arcsin x \;$ függvényekből a négy
  alapművelet, az összetett függvényképzés és a nyílt halmazra való leszűkítés
  véges sokszori alkalmasával keletkező függvényeket \textbf{elemi
    függvényeknek} nevezzük.
  % \[
  %   1 \text, \quad
  %   x \text, \quad
  %   e^x \text, \quad
  %   \ln x \text, \quad
  %   \sin x \text, \quad
  %   \arcsin x \text.
  % \]

  A $\cos x$, $\tan x$, $\cot x$, $\sinh x$, $\cosh x$, $\tanh x$, $\coth x$,
  $\arccos x$, $\arctan x$, $\arccot x$, $\arccosh x$, $\arccosh x$, $\arctanh
    x$, $\arccoth x$, $x^\alpha$, polinom, racionális függvények elemi
  függvények.

  A négy alapművelet, az összetett függvényképzés, a nyílt halmazra való
  leszűkítés megőrzi a differenciálhatóságot, tehát az elemi függvények az
  értelmezési tartományuk belső pontjában differenciálhatóak. A primitív
  függvények megkeresése azonban kivezet az elemi függvények köréből, ez
  indokolja az \textbf{elemien integrálható függvények} elnevezés bevezetését.

  \textbf{Elemien integrálható függvény} olyan elemi függvény, amelynek primitív
  függvénye ugyancsak elemi függvény.

  Elemien integrálható függvények például:
  \[
    x^2 \text, \quad
    \frac{1}{x + 2} \text, \quad
    \frac{1}{x^2 + 1} \text.
  \]

  Nem elemien integrálható függvény például:
  \[
    e^{-x^2} \text, \quad
    \frac{\sin x}{x} \text, \quad
    \frac{1}{\ln x} \text.
  \]
\end{blueBox}

\clearpage