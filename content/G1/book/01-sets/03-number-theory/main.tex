\clearpage
\section{A számfogalom kiépítése}\label{sec-01-03}

\begin{blueBox}
  \bgroup
  \sffamily\bfseries Peano-axiómák:
  \egroup

  Legyen $ \mathbb N \neq \emptyset$, $\mathbb N$-t a természetes számok
  halmazának, elemeit természetes számoknak mondjuk, ha teljesülnek az alábbiak:
  \begin{enumerate}
    \item legyen adva egy $\varphi : \mathbb N \rightarrow \mathbb N$ leképezés,
    \item $\varphi$ injektív : $\varphi(a) = \varphi(b) \Rightarrow a = b$,
    \item $\exists$ $\mathbb N$-nek egy kitüntetett eleme, ez a $0$,
    \item a $0$-nak nincs ősképe, azaz $\nexists n \in \mathbb N : \varphi(n) =
            0$,
    \item a teljes indukció elve teljesül, azaz ha $H \subseteq \mathbb N$ és
          \begin{enumerate}
            \item $0 \in H$,
            \item $n \in H \Rightarrow \varphi(n) \in H$,
          \end{enumerate}
          akkor $H = \mathbb N$.
  \end{enumerate}
\end{blueBox}

\begin{blueBox}
  A természetes számok halmazát ekvivalenciarelációkkal ellátva megkapjuk a
  középiskolában megismert számhalmazokat:
  \begin{itemize}
    \item $\mathbb Z$ : az egész számok halmaza
          ($\mathbb N \times \mathbb N$),
    \item $\mathbb Q$ : a racionális számok halmaza
          ($\mathbb Z \times \mathbb Z$),
    \item $\mathbb Q^*$ : az irracionális számok halmaza,
    \item $\Reals$ : a valós számok halmaza
          ($\mathbb Q \cup \mathbb Q^*$).
  \end{itemize}

  \begin{center}
    \begin{tikzpicture}[ultra thick, draw=primaryColor]
      % SETS WITH LABELS
      \draw         (0.00,0) ellipse (1.5 and 1)    node[right=0.75cm] {$\mathbb N$};
      \draw         (0.75,0) ellipse (2.5 and 1.67) node[right=1.75cm] {$\mathbb Z$};
      \draw         (1.50,0) ellipse (3.5 and 2.33) node[right=2.75cm] {$\mathbb Q$};
      \draw[dashed] (2.25,0) ellipse (4.5 and 3)    node[right=3.75cm] {}           ;
      \draw         (3.00,0) ellipse (5.5 and 3.67) node[right=4.75cm] {$\Reals   $};

      % HELPER LINES FOR IRRATIONALS AND TRANSCENDENTALS
      \draw[dashed, gray, very thick] (8.50,0) -- ++(0,-4.85);
      \draw[dashed, gray, very thick] (6.75,0) -- ++(0,-3.90);
      \draw[dashed, gray, very thick] (5.00,0) -- ++(0,-4.85);

      \begin{scope}[font=\scriptsize]
        % IRRATIONALS
        \draw[to-to, draw=secondaryColor, very thick]
        (5.00,-4.70) -- ++(3.50,0)
        node [midway, above] {irracionális};

        % TRANSCENDENTALS
        \draw[to-to, draw=secondaryColor, very thick]
        (6.75,-3.75) -- ++(1.75,0)
        node [midway, above] {transzcendens};
      \end{scope}

      % NATURAL EXAMPLES
      \node (N) at (-0.85,0) {$1$};
      \node[above right] at (N.100) {$0$};
      \node[below right] at (N.260) {$2$};

      % INTEGERS EXAMPLES
      \node (Z) at (2,+0.5) {$-1$};
      \node (Z) at (2,-0.5) {$-2$};

      % RATIONALS EXAMPLES
      \node (Q) at (3.75,+0.5) {$\sfrac{1}{2}$};
      \node (Q) at (3.75,-0.5) {$\sfrac{2}{3}$};

      % IRRATIONALS EXAMPLES
      \node (I) at (5.75,+0.5) {$\sqrt{2}$};
      \node (I) at (5.75,-0.5) {$\frac{1 + \sqrt5}{2}$};

      % TRANSCENDENTALS EXAMPLES
      \node (T) at (7.5,+0.5) {$\pi$};
      \node (T) at (7.5,-0.5) {$e$};
    \end{tikzpicture}
  \end{center}
\end{blueBox}

\begin{note}
  A \textbf{transzcendens} számok olyan irracionális, valós számok, amelyek
  nem algebraiak, azaz nem valamely racionális együtthatós polinom gyökei. Ilyen
  szám pélául a $\pi$.
\end{note}

\clearpage
\begin{blueBox}
  \sftitle{A valós számok axiómarendszere:}

  Értelmezzük két bináris műveletet, az összeadást ($+$) és a szorzást
  ($\cdot$), valamint egy relációt ($>$).

  \bgroup
  \def\arraystretch{2}
  \newcounter{tctr}
  \begin{tabular}{
      @{\stepcounter{tctr}\makebox[2.25em][r]{\arabic{tctr}.\;\;}}
      >{$}l<{$}
      >{\makebox[2.5em][c]{$\sim$}}l
    }
    a + b = b + a
     & $+$ kommutatív,
    \\
    (a + b) + c = a + (b + c)
     & $+$ asszociatív,
    \\
    \exists! 0 \in \Reals : a + 0 = a
     & $+$ egységelem,
    \\
    \forall a \in \Reals : \exists -a \in \Reals : a + (-a) = 0
     & $+$ inverz elem,
    \\
    a \cdot b = b \cdot a
     & $\cdot$ kommutatív,
    \\
    (a \cdot b) \cdot c = a \cdot (b \cdot c)
     & $\cdot$ asszociatív,
    \\
    \exists! 1 \in \Reals : a \cdot 1 = a
     & $\cdot$ egységelem,
    \\
    \forall a \in \Reals \setminus \{0\} : \exists a^{-1} \in \Reals :
    a \cdot a^{-1} = 1
     & $\cdot$ inverz elem,
    \\
    a \cdot (b + c) = a \cdot b + a \cdot c
     & disztributivitás,
    \\
    \forall a, b \in \Reals : a < b \vee a = b \vee b < a
     & trichotómia,
    \\
    \forall a, b, c \in \Reals : a < b \land b < c \Rightarrow a < c
     & $<$ tranzitivitás,
    \\
    \forall a, b, c \in \Reals : a < b \Rightarrow a + c < b + c
     & $+$ monotonitás,
    \\
    \forall a, b, c \in \Reals : a < b \land 0 < c \Rightarrow
    a \cdot c < b \cdot c
     & $\cdot$ monotonitás,
    \\
    \forall a \in \Reals : \exists n \in \mathbb N : a < n
     & Arkhimédész-féle rendezés,
    \\
    a_n \leq a_{n+1} \land b_n \geq b_{n+1}: \bigcap\limits_{n = 1}^{\infty}
    \left[ a_n; b_n \right] \neq \emptyset
     & Cantor-axióma,
  \end{tabular}
  \egroup
\end{blueBox}

\begin{blueBox}
  \begin{itemize}
    \item $2 - 4$: csoport,
    \item $1 - 4$: Abel-csoport,
    \item $1 - 9$: test,
    \item $1 - 13$: rendezett test,
    \item $1 - 14$: arkhimédészien rendezett test,
    \item $1 - 15$: teljes rendezett test.
  \end{itemize}
\end{blueBox}

\begin{statement}
  A $\mathbb Q$ és $\mathbb Q^*$ sűrű.
\end{statement}