\clearpage
\section{Alapfogalmak, alapműveletek}\label{sec-01-01}

\begin{blueBox}
  \textbf{Alapfogalmak}:
  \begin{itemize}
    \item axióma / posztulátum,
    \item definíció,
    \item nem definiált alapfogalom,
    \item állítás / tétel / lemma / segédtétel.
  \end{itemize}
\end{blueBox}

\begin{blueBox}
  A \textbf{halmaz} egy nem definiált alapfogalom:
  \begin{itemize}
    \item A halmazokat nagybetűvel jelöljük: $A$, $B$, \dots
    \item Az elemeket kisbetűvel: $a$, $b$, \dots
    \item Halmaz \textbf{eleme} jelölés: $\in$, pl.: $x \in Y$, $x$ eleme az $Y$
          halmaznak.
    \item Halmaznak \textbf{nem eleme}: $\notin$, pl.: $x \notin Y$, $x$ nem
          eleme az $Y$ halmaznak.
  \end{itemize}
\end{blueBox}

\begin{note}
  Egy halmaz akkor \textbf{jól megadott}, ha bármely elemről eldönthető, hogy
  hozzá tartozik-e a halmazhoz, vagy nem.
\end{note}

\begin{definition}[Üreshalmaz]
  Azt a halmazt, amelynek egyetlen eleme sincs, \textbf{üreshalmaz}nak nevezzük,
  jele: $\emptyset$.
\end{definition}

\begin{center}
  \begin{note}
    A \textbf{nemüres halmaz}: olyan halmaz, melynek legalább egy eleme van.
  \end{note}
\end{center}

\begin{note}
  A halmazok megadási módjai:
  \begin{itemize}
    \item \textbf{utasítással}: $A := \{ \text{A 180 cm-nél magasabb emberek} \}$,
    \item \textbf{felsorolással}: $B := \{ 1, 2, 3, 4, 5 \}$.
  \end{itemize}
\end{note}

\begin{note}
  \sftitle{Nevezetes halmazok:}
  \vspace{-1em}
  \begin{multicols}{2}
    \begin{itemize}
      \item $\mathbb N$ -- természetes számok halmaza,
      \item $\mathbb Z$ -- egész számok halmaza,
      \item $\mathbb Q$ -- racionális számok halmaza,
      \item $\mathbb Q^*$ -- irracionális számok halmaza,
      \item $\mathbb R$ -- valós számok halmaza,
      \item $\mathbb C$ -- komplex számok halmaza.
    \end{itemize}
  \end{multicols}
\end{note}

\begin{definition}[Részhalmaz]
  Legyenek $A$ és $B$ halmazok. Ha $A$ minden eleme eleme $B$-nek is, akkor azt
  mondjuk, hogy az $A$ a $B$ részhalmaza, jele: $\subseteq$ vagy $\subset$
  (valódi részhalmaza).
\end{definition}

\begin{note}
  $A = B$, ha $A \subset B$ és $B \subset A$ is teljesül (kölcsönös
  tartalmazás).
\end{note}

\begin{statement}
  Legyenek $A$, $B$, $C$ tetszőleges halmazok, ekkor teljesülnek az alábbiak:
  \begin{enumerate}
    \item $A \subset A$, azaz minden halmaz része önmagának
          (\textbf{reflexív}),
    \item $A \subset B$ és $B \subset A$, akkor $A = B$
          (\textbf{antiszimmetrikus}),
    \item $A \subset B$ és $B \subset C$, akkor $A \subset C$
          (\textbf{tranzitív}).
  \end{enumerate}
\end{statement}

\begin{definition}[Unió, metszet, különbség]
  Legyenek $A$ és $B$ az $X$ alaphalmaz részhalmazai, ekkor:
  \begin{center}
    \def\arraystretch{1.5}
    \begin{tabular}
      {>{$}r<{$} >{-}c<{-} l}
      A \cup B := \Big\{\; x \in X \;\Big|\; x \in A \lor x \in B \;\Big\}
       &  & \textbf{unió}, egyesítés,
      \\
      A \cap B := \Big\{\; x \in X \;\Big|\; x \in A \land x \in B \;\Big\}
       &  & \textbf{metszet},
      \\
      A \setminus B := \Big\{\; x \in X \;\Big|\; x \in A \land x \notin B \;\Big\}
       &  & \textbf{különbség},
    \end{tabular}
  \end{center}
\end{definition}

\begin{learnMore}[Halmazműveletek és logikai műveletek közötti kapcsolat]
  \centering

  \begin{tikzpicture}[thick, scale=.95]
    % GATES
    \foreach \name/\desc/\logic/\shift in {%
        or port/VAGY/+/0cm,%
        and port/ÉS/\cdot/2cm,%
        xor port/XOR/\oplus/6cm%
      }{
        % GATE
        \node[\name, secondaryColor] (\name) at (0,-\shift) {};

        % LABELS
        \draw (\name.in 1) to[short, -o] ++(-0.25,0) node[left]  {$A$};
        \draw (\name.in 2) to[short, -o] ++(-0.25,0) node[left]  {$B$};
        \draw (\name.out)  to[short, -o] ++(+0.25,0) node[right] {$A \logic B$};

        % DESCRIPTION
        \node[right] at (2,-\shift) {-- \quad\desc};

        % SIMILARITY
        \node at (-3.15,-\shift) {$\sim$};
      }

    % SETS
    \def\radius{0.65cm}
    \def\lcirc{-6cm}
    \def\rcirc{-5.25cm}

    % UNION FILL
    \fill[primaryColor!50] (\lcirc,0) circle (\radius);
    \fill[primaryColor!50] (\rcirc,0) circle (\radius);

    % INTERSECTION FILL
    \begin{scope}
      \clip[] (\lcirc,-2) circle (\radius);
      \fill[primaryColor!50] (\rcirc,-2) circle (\radius);
    \end{scope}

    % DIFFERENCE FILL
    \fill[primaryColor!50] (\lcirc,-4) circle (\radius);
    \fill[gray!10] (\rcirc,-4) circle (\radius);

    % SYMMETRIC DIFFERENCE FILL
    \begin{scope}
      \fill[primaryColor!50] (\lcirc,-6) circle (\radius);
      \fill[primaryColor!50] (\rcirc,-6) circle (\radius);
      \clip[] (\lcirc,-6) circle (\radius);
      \fill[gray!10] (\rcirc,-6) circle (\radius);
    \end{scope}

    % CIRCLES AND LABELS
    \foreach \desc/\shift in {%
        Unió/0cm,%
        Metszet/2cm,%
        Különbség/4cm,%
        Szimmetrikus differencia/6cm%
      } {
        % SET A
        \draw[draw=primaryColor, ultra thick] (\lcirc,-\shift) circle (\radius)
        node[left=6mm] {$A$};
        % SET B
        \draw[draw=primaryColor, ultra thick] (\rcirc,-\shift) circle (\radius)
        node[right=6mm] {$B$};

        % LABELS
        \node (D) at (-7.75,-\shift) {--};
        \node[left, text width=3.25cm] at (D) {\desc};
      }
  \end{tikzpicture}
\end{learnMore}

\begin{definition}[Diszjunkt halmaz]
  Két halmaz diszjunkt, ha metszetük az üreshalmaz.
\end{definition}

\begin{definition}[Komplementer halmaz]
  Ha $A \subset B$, akkor az $A$ halmaznak a $B$-re vonatkozó komplementere:
  $B \setminus A$, jele: $\overline A$.
\end{definition}

\begin{statement}
  Halmaz komplenterének komplementere önmaga, vagyis
  \[
    \overline{\overline A} = A
    \text.
  \]
\end{statement}

\begin{theorem}[Halmazműveletek tulajdonságai]
  Legyenek $A, B, C \in X$
  \vspace{-1em}
  \begin{center}
    \begin{tabular}{>{$}c<{$} c >{$}c<{$}}
      A \cup B = B \cup A
       & kommunutatív
       & A \cap B = B \cap A
      \\
      A \cup (B \cup C) = (A \cup B ) \cup C
       & asszociatív
       & A \cap (B \cap C) = (A \cap B) \cap C
      \\
      A \cup A = A
       & idempotens
       & A \cap A = A
      \\
      \scalebox{.9}{$(A \cup B) \cap C = (A \cap C) \cup (B \cap C)$}
       & disztributív
       & \scalebox{.9}{$(A \cap B) \cup C = (A \cup C) \cap (B \cup C)$}
      \\
      A \cup \emptyset = A
       &                                                                 % Name?
       & A \cap \emptyset = \emptyset
      \\
      A \cup \overline A = X
       &                                                                 % Name?
       & A \cap \overline A = \emptyset
      \\
      \overline{A \cup B} = \overline A \cap \overline B
       & De Morgan
       & \overline{A \cap B} = \overline A \cup \overline B
      \\
    \end{tabular}
  \end{center}

  \begin{proof}[De Morgan-azonosságok]
    \vspace{-2em}
    \begin{multicols}{2}
      \begin{gather*}
        x \in \overline{A \cup B}
        \\
        \downarrow
        \\
        x \notin A \cup B
        \\
        x \notin A \land x \notin B
        \\
        x \in \overline A \land x \in \overline B
        \\
        x \in \overline A \cap \overline B
      \end{gather*}

      \begin{gather*}
        x \in \overline{A \cap B}
        \\
        \downarrow
        \\
        x \notin A \cap B
        \\
        x \notin A \lor x \notin B
        \\
        x \in \overline A \lor x \in \overline B
        \\
        x \in \overline A \cup \overline B
      \end{gather*}
    \end{multicols}
  \end{proof}
\end{theorem}

\begin{definition}[Hatványhalmaz]
  Egy $A$ halmaz összes részhalmazainak halmazát az $A$ halmaz hatványhalmazának
  nevezzük.
\end{definition}

\begin{statement}
  Egy $A$ véges halmaz összes részhalmazainak száma: $2^{|A|}$.

  \begin{proof}
    A binomiális tétel:
    \[
      (a + b)^n = \sum_{k=0}^{n} \binom{n}{k} \, a^{n-k} \, b^k
      \text.
    \]
    Vegyük észre, hogy a binomiális tételben $a = b = 1$, és $n = |A|$ esetén:
    \[
      2^n
      = (1 + 1)^n
      = \sum_{k=0}^{n} \binom{n}{k} \, 1^{n-k} \, 1^k
      = \sum_{k=0}^{n} \binom{n}{k}
      = \underbracket{
        \binom n0 + \binom n1 + \binom n2 + \dots + \binom nn
      }_\text{az összes részhalmaz száma}
      \text.
    \]
  \end{proof}
\end{statement}

\clearpage