\section{Relációk, leképezések, függvények}

\begin{definition}[Descartes-szorzat]
  Az $A$ és $B$ halmazok Descartes-szorzatán az $A$ és $B$ halmaz elemeiből álló
  \textbf{összes rendezett elempár}ok halmazát értjük:
  \[
    A \times B := \Big\{\;
    (a; b) \;\Big|\; (a \in A) \land (b \in B)
    \;\Big\}
    \text.
  \]
\end{definition}

\begin{example}
  Legyen $A = \{1;2\}$ és $B = \{a;b\}$, ekkor az $A \times B$
  Descartes-szorzat:
  \[
    A \times B = \Big\{\;
    (1; a); (1; b); (2; a); (2; b)
    \;\Big\}
    \text.
  \]

\end{example}

\begin{definition}[Binér reláció]
  Az $A \times B$ szorzathalmaz $T \subset A \times B$ részhalmazát az $A$ és
  $B$ közötti binér (kételemű) relációnak hívjuk. Ha $(a; b) \in T$, akkor azt
  mondjuk, hogy $a$ és $b$ relációban vannak, és ezt $aTb$-vel jelöljük.
\end{definition}

\begin{definition}[Reláció értelmezési tartománya, értékkészlete és inverze]
  Legyen $ T\subset A\times B$ egy reláció, ekkor
  \begin{center}
    \def\arraystretch{1.5}
    \addtolength{\tabcolsep}{-0.25em}
    \begin{tabular}{
        >{$}r<{: = \Big\{$}
        >{$}c<{$}
        >{$\Big|$}c
        >{$}c<{$}
        >{$\Big\}$}c
        >{--}c
        l
      }
      \Domain_T
       & a \in A
       &         & \exists b \in B: (a;b) \in T
       &         &                              & a reláció értelmességi tartománya,
      \\
      \Range_T
       & b \in B
       &         & \exists a \in A: (a;b) \in T
       &         &                              & a reláció értékkészlete,
      \\
      T^{-1}
       & (b;a)
       &         & (a;b) \in T
       &         &                              & a reláció inverze.
    \end{tabular}

  \end{center}
\end{definition}

\begin{definition}[Ekvivalenciareláció]
  Legyen $A \neq \emptyset$, a $T \subset A \times A$ relációt ekvivalencia%
  relációnak mondjuk, ha teljesülnek az alábbiak:
  \begin{itemize}
    \item \textbf{reflexivitás} -- $\forall A \in A$ esetén $(a; a) \in T$,
    \item \textbf{szimmetria} -- ha $(a; b) \in T$, akkor $(b; a) \in T$,
    \item \textbf{tranzitivitás} -- ha $(a; b) \in T$ és $(b; c) \in T$, akkor
          $(a; c) \in T$.
  \end{itemize}
\end{definition}

\begin{theorem}[Ekvivalencia osztályok]
  Minden $A \times A$ halmazon adott ekvivalenciareláció diszjunkt halmazokra
  bontja fel az A halmazt, ezeket a diszjunkt halmazokat ekvivalencia%
  osztályoknak nevezzük.
\end{theorem}

\begin{example}
  \samepage
  Két természetes szám relációban van egymással, ha hárommal osztva azonos
  maradékot adnak.
  \begin{center}
    \begin{tikzpicture}[ultra thick]
      \node[
        circle,
        minimum size=3.5cm,
        draw=primaryColor,
        fill=primaryColor!10
      ] (C) at (0,0) {};

      \coordinate (1) at (80:1.75);
      \coordinate (2) at (100:1.75);

      \draw[primaryColor, fill=primaryColor!10]
      ($(2)+(1mm,6mm)$)
      arc (90:180:1mm)
      -- (2)
      arc (100:80:1.75)
      -- ++(0,5mm)
      arc (0:90:1mm)
      -- cycle
      ;

      \node at (0,2.025) {$\mathbb N$};


      \draw[secondaryColor] (170:1.75) .. controls (40:.35) .. (290:1.75)
      coordinate[pos=0.6] (A);

      \draw[secondaryColor] (A) .. controls (30:.75) .. (90:1.75);

      \foreach \angle/\label in {110/0, 230/1, 345/2}{
          \node at (\angle:1) {$\overline\label$};
        }
    \end{tikzpicture}
  \end{center}
\end{example}

\begin{definition}[Függvény]
  A $T \subset A \times B$ binér relációt leképezésnek/függvénynek mondjuk, ha
  \[
    (a; b) \in T \land (a; c)\in T \Rightarrow b = c
    \text.
  \]
  Jelölés: $f: A \rightarrow B$, ahol $A$ az értelmezési tartomány ($\Domain_f$)
  és $B$ az értékkészlet ($\Range_f$).
\end{definition}

\begin{definition}[Bijekció]
  Az $f : A \rightarrow B$ kölcsönösen egyértelmű (egy-egyértelmű, bijektív), ha
  \begin{itemize}
    \item \textbf{injektív}, vagyis $f(a_1) = f(a_2) \Rightarrow a_1 = a_2$,
          valamint
    \item \textbf{szürjektív}, vagyis $\forall b \in B$ esetén $\exists a \in A:
            f(a) = b$.
  \end{itemize}
\end{definition}

\begin{note}
  Ha az $f: A \rightarrow B$ bijektív, akkor az $f^{-1}: B \rightarrow A$
  leképezést $f$ \textbf{inverz leképezés}ének hívjuk.
\end{note}

\begin{example}
  Az $f: \Reals \rightarrow (0; +\infty), x \mapsto e^x$ függvény bijektív,
  inverze a természetes alapú logaritmus: $f^{-1}: (0; +\infty) \rightarrow
    \Reals, x \mapsto \ln x$.

  \begin{center}
    \begin{tikzpicture}[ultra thick]
      % COORDINATE SYSTEM
      \draw[-to, draw=primaryColor] (-2.5,0) -- (3,0) node[above left] {$x$};
      \draw[-to, draw=primaryColor] (0,-2.5) -- (0,3) node[below left] {$y$};

      \draw[very thick, gray, dashed] (-2,-2) -- (2,2);

      % PLOT EXP(X)
      \draw plot[domain=-2.25:0.85, samples=25, smooth, draw=secondaryColor]
      (\x, {exp(\x)})
      node[above right] {$e^x$}
      ;
      % PLOT LN(X)
      \draw plot[domain=0.10:2.5, samples=25, smooth, draw=secondaryColor]
      (\x, {ln(\x)})
      node[above right] {$\ln x$}
      ;
    \end{tikzpicture}
  \end{center}
\end{example}

% \clearpage