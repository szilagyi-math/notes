\section{Bevezetés}\label{section.6.1}

\begin{definition}[Differenciahányados]
  Legyen $f: I \subset \Reals \rightarrow \Reals$ értelmezve az $x \in I$
  pontban és annak környezetében. Ha $x \neq a$, akkor az
  \[
    \frac{f(x) - f(a)}{x - a}
  \]
  hányadost különbséghányadosnak vagy differenciahányadosnak nevezzük.
  Jelölése:
  \[
    \frac{\Delta f(a)}{\Delta x}
    \quad \text{ vagy } \quad
    \frac{f(a + \Delta x) - f(a)}{\Delta x}
    \text.
  \]
\end{definition}

\begin{definition}[Differenciálhányados]
  Ha létezik és véges a
  \[
    \lim_{x \rightarrow a} \frac{f(x) - f(a)}{x - a}
  \]
  határérték, akkor azt az $f$ függvény $a$ pontbeli differenciálhányadosának,
  vagy az $a$ pontbeli deriváltjának mondjuk. Jelölése:
  \[
    f'(a)
    \quad \text{ vagy } \quad
    \odv{f(a)}{x}
    \text.
  \]
\end{definition}

\begin{blueBox}
  Az $f$ függvény $a$ pontbeli \textbf{érintő}jének \textbf{egyenlete} onnan
  következik, hogy $f' = m$, ahol $m$ a meredekséget jelöli, az $y = m \cdot x
    + b$ egyenes egyenletéből levezetve:
  \[
    f(a) = f'(a) \cdot a + b
    \quad \rightarrow \quad
    b = f(a) - f'(a) \cdot a
    \text,
  \]
  és mivel
  \begin{gather*}
    (a; f(a)) \in y = m \cdot x + b
    \\
    \downarrow
    \\
    y = f'(a) \cdot x + f(a) - f'(a) \cdot a
  \end{gather*}
  Ebből átalakítva:
  \[
    \boxed{
      y = f'(a)\cdot (x-a) + f(a)
      \text.
    }
  \]
  Az $(a; f(a))$ pontbeli \textbf{normális egyenlete}:
  \begin{gather*}
    M = \frac{-1}{f'(a)}
    \text, \quad \text{és} \quad
    (a; f(a)) \in Y = M \cdot X + B
    \text,
    \\
    \boxed{
      y = \frac{-1}{f'(a)} \cdot (x-a) + f(a)
      \text.
    }
  \end{gather*}
\end{blueBox}

\begin{definition}[Derivált]
  Legyen $f : I \subset \Reals \rightarrow \Reals$ értelmezve $a$ pontban és
  annak egy környezetében. Ekkor $f$-t differenciálhatónak mondjuk  $a$ pontban,
  ha $\exists A \in \Reals$ és $\varepsilon : \Reals \rightarrow \Reals$
  függvény, melyre
  \[
    \lim\limits_{x\to a} \varepsilon(x) = 0
    \text{, hogy }
    f(x) - f(a) = A \cdot (x-a) + \varepsilon(x)\cdot (x-a)
    \text.
  \]
\end{definition}

\begin{definition}[Jobboldali derivált]
  Legyen $f:[a; b] \rightarrow \Reals$, ekkor a
  \[
    \lim_{\phantom{{}^{+}}x \to a^+} \frac{f(x)-f(a)}{x-a}
  \]
  határértékét, ha létezik és véges, az $f$ függvény $a$ pontbeli jobboldali
  deriváltjának mondjuk.
\end{definition}

\begin{definition}[Baloldali derivált]
  Legyen $f:[a; b] \rightarrow \Reals$, ekkor a
  \[
    \lim_{\phantom{{}^{-}}x \to a^-} \frac{f(x)-f(a)}{x-a}
  \]
  határértékét, ha létezik és véges, az $f$ függvény $a$ pontbeli baloldali
  deriváltjának mondjuk.
\end{definition}

\begin{note}
  Az $f: I\subset \Reals \rightarrow \Reals$ függvény az $x_0\in I$ pontban
  differenciálható $\Leftrightarrow$, ha ott balról és jobbról is
  differenciálható és a bal és jobb oldali deriváltak egyenlőek.
\end{note}

\begin{example}
  $f(x) = |x|$ függvény az $x = 0$ pontban:
  \[
    \lim_{\phantom{{}^{+}}x \to 0^+} f(x) = \frac{x-0}{x-0} = 1
    \quad \text{és} \quad
    \lim_{\phantom{{}^{+}}x \to 0^-} f(x) = \frac{-x-0}{x-0} = -1
    \text.
  \]
  A két derivált nem egyezik meg, tehát az $x = 0$ pontban a függvény nem
  deriválható. A folytonosság szükséges, de nem elégséges feltétele a
  differenciálhatóságnak.
\end{example}

\begin{theorem}
  Ha az $f$ függvény differenciálható egy adott pontban, akkot ott folytonos is.
\end{theorem}

\begin{definition}[Függvény differenciálhatósága nyílt intervallumon]
  Az $f$ függvény differenciálható az $(a; b)$ intervallumon, ha annak minden
  pontjában differenciálható.
\end{definition}

\begin{definition}[Függvény differenciálhatósága zárt intervallumon]
  Az $f$ függvény differenciálható az $[a; b]$ intervallumon, ha
  differenciálható az $(a; b)$ intervallumon, továbbá jobbról differenciálható
  $a$-ban és balról differenciálható $b$-ben.
\end{definition}

\begin{definition}[Differenciálhányados függvény]
  Az $f$ függvény differenciahányados függvényének nevezzük és $f'$-vel
  jelöljük azt a függvényt, amelynek értelmezési tartománya azon pontok
  halmaza, ahol $f$ differenciálható és $\forall x \in \Domain_{f'}$ esetén
  a függvényérték $f'(x)$.
\end{definition}

\begin{definition}[Másodrendű derivált]
  Ha az $f'$ függvény differenciálható az $x_0$ helyen, akkor a derivált az
  $f$ függvény $x_0$ pontbeli másodrendű deriváltjának nevezzük, jele:
  \[
    \odv[order = 2]{f(x_0)}{x}
    \quad\text{vagy}\quad
    f''(x_0)
    \text.
  \]
  Ha $f''(x_0)$ létezik, akkor azt is mondjuk, hogy $f$ kétszer differenciálható
  $x_0$-ban.
\end{definition}

\begin{note}
  Hasonlóan értelmezhetőek a többedrendű deriváltakat is. Harmadik derivált
  utána jelölés $f^{(k)}$.
\end{note}

\begin{example}
  \sftitle{Példák függvények deriváltjára:}
  \begin{enumerate}
    % \item $f(x) = c
    %         \phantom{xx^n}\rightarrow\quad
    %         \lim\limits_{x \rightarrow x_0} \dfrac{f(x)-f(x_0)}{x-x_0} =
    %         \lim\limits_{x \rightarrow x_0} \dfrac{c-c}{x-x_0} = 0$,
    \item $f(x) = x
            \phantom{cx^n}\rightarrow\quad
            \lim\limits_{x \rightarrow x_0} \dfrac{f(x)-f(x_0)}{x-x_0} =
            \lim\limits_{x \rightarrow x_0} \dfrac{x-x_0}{x-x_0} = 1$,
    \item $f(x) = x^n
            \phantom{cx}\rightarrow\quad
            \lim\limits_{x \rightarrow x_0} \dfrac{f(x)-f(x_0)}{x-x_0} =
            \lim\limits_{x \rightarrow x_0} \dfrac{x^n-x_0^n}{x-x_0} =$\\
          $\phantom{f(x) = cxx^n\rightarrow\quad}
            \lim\limits_{x \rightarrow x_0}
            \dfrac{(x-x_0)(x^{n-1}+x^{n-2}x_0+\ldots+x_0^{n-1})}{x-x_0} =$\\
          $\phantom{f(x) = cxx^n\rightarrow\quad}
            \lim\limits_{x \rightarrow x_0}
            \underbracket{x^{n-1}+x^{n-2}x_0+\ldots+x_0^{n-1}}_{n\text{ darab}} =
            n\cdot x_0^{n-1}$.

  \end{enumerate}
\end{example}

\begin{statement}
  \sftitle{Függvény konstansszorosának deriváltja}

  Legyen $f: I \subset \Reals \rightarrow \Reals$ differenciálható az $x_0$
  pontban, ekkor $c\cdot f$ is differenciálható $x_0$-ban és
  \[
    (c \cdot f(x_0))' = c\cdot f'(x_0)
    \text{, ahol }
    c \in \Reals
    \text.
  \]
\end{statement}

\begin{mdframed}[
    style=statement,
    nobreak=true,
  ]
  \sftitle{Összegfüggvény deriváltja}

  Legyenek $f$ és $g$ differenciálható az $x_0 \in \Domain_f \cap \Domain_g$
  pontban, ekkor $f + g$ is differenciálható $x_0$-ban és
  \[
    (f+g)'(x_0) = f'(x_0)+ g'(x_0)
    \text.
  \]
\end{mdframed}


\begin{statement}
  \sftitle{Szorzatfüggvény deriváltja}

  Legyenek $f$ és $g$ differenciálható az $x_0 \in \Domain_f \cap \Domain_g$
  pontban, ekkor $f \cdot g$ is differenciálható $x_0$-ban és
  \[
    (f\cdot g)'(x_0) = f'(x_0)\cdot g(x_0) + f(x_0)\cdot g'(x_0)
    \text.
  \]
\end{statement}

\begin{statement}
  \sftitle{Hányadosfüggvény deriváltja}

  Legyenek $f$ és $g$ differenciálható az $x_0 \in \Domain_f \cap \Domain_g$
  pontban, ekkor $f/g$ is differenciálható $x_0$-ban és:
  \[
    \frac{f'}{g'}(x_0) = \frac{
      f'(x_0)\cdot g(x_0) - f(x_0)\cdot g'(x_0)
    }{
      (g(x_0))^2
    }
    \text.
  \]
\end{statement}

\begin{statement}
  \sftitle{Láncszabály}

  Legyen $f$ differenciálható az $x_0 \in \Domain_f$ pontban és $g$
  differenciálható az $f(x_0)\in \Domain_g$ pontban, ekkor $f \circ g$ is
  differenciálható $x_0$-ban és:
  \[
    (g \circ f(x_0))' = g'(f(x_0))\cdot f'(x_0)
    \text.
  \]
\end{statement}

\begin{definition}[Inverz függvény]
  Legyen $f: I \subset \Reals \rightarrow J \subset \Reals$ és $J:= f(I)$,
  kölcsönösen egyértelmű leképezése $I$-nek $f$-re, ekkore az $f^{-1} : J
    \rightarrow I$ függvényt $f$ inverzének hívjuk, ha $\forall y \in J$ esetén
  $\exists x \in I$, hogy $y = f(x)$.

  \begin{center}
    \begin{tikzpicture}[ultra thick]
      \draw[primaryColor, fill=primaryColor!10] (0,1) circle [radius=1.5];
      \draw[secondaryColor, fill=secondaryColor!10] (6,1) circle [radius=1.5];

      \node at (0,3) {$I$};
      \node at (6,3) {$J$};

      \draw [->] (0,1.5)
      arc [radius=4.5, start angle=130, end angle= 50]
      node [above, midway] {$f$};

      \draw [->] (6,0.5)
      arc [radius=4.5, start angle=-50, end angle= -130]
      node [above, midway] {$f^{-1}$};
    \end{tikzpicture}
  \end{center}
\end{definition}


\begin{theorem}
  Legyen $f: [a;b] \rightarrow \Reals$ szigorúan monoton, folytonos függvény
  $[a;b]$ intervallumon, ekkor teljesülnek az alábbiak:
  \begin{itemize}
    \item $f$ értékkészlete:
          \begin{itemize}
            \item $[f(a), f(b)]$, ha szigorúan monoton növekvő,
            \item $[f(a), f(b)]$, ha szigorúan monoton csökkenő.
          \end{itemize}
    \item $f$ invertálható.
    \item $f$ inverze is szogorúan monoton, éppen olyan értelemben, ahogyan $f$.
    \item az $f^{-1}$ folytonos $[f(a), f(b)]$-n vagy $[f(b), f(a)]$-n.
  \end{itemize}
\end{theorem}

\begin{statement}
  Inverz függvény deriválási szabály: Legyen $f: I \subset \Reals \rightarrow
    \Reals$ szigorúan monoton és folytonos az $x_0$ pont egy környezetében,
  továbbá $f'(x_0) \neq 0$, ekkor $f$ inverze is differenciálható a $b = f(x_0)$
  pontban és:
  \[
    (f^{-1}(b))' = \frac{1}{f'(x_0)}
  \]
\end{statement}