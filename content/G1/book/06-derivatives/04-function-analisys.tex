\section{Differenciálható függvények vizsgálata}\label{section.6.4}

\begin{theorem}
  Ha $f : I \subset \Reals \rightarrow \Reals$ differenciálható a
  $J \subset I $ intervallumon, akkor ahhoz, hogy $f$ a $J$-n monoton növekvő
  legyen, szükséges és elégséges feltétel, hogy $\forall x \in J$ esetén
  $f'(x)\geq 0$ fennálljon.
\end{theorem}

\begin{theorem}
  Ha $f : I \subset \Reals \rightarrow \Reals$ differenciálható a
  $J \subset I $ intervallumon, akkor ahhoz, hogy $f$ a $J$-n monoton csökkenő
  legyen, szükséges és elégséges feltétel, hogy $\forall x \in J$ esetén
  $f'(x)\leq 0$ fennálljon.
\end{theorem}

\begin{note}
  Ha $f'(x) \geq 0$ $\forall x \in J$ esetén és véges sok pont kivételével
  $f'(x)\geq 0$, akkor $f$ a $J$ intervallumon szigorúan monoton növekvő.
\end{note}

\begin{theorem}
  Ha $f$ a $J$ intervallumon differenciálható, akkor ahhoz, hogy a $J$-n konvex
  legyen, szükséges és elégséges feltétel, hogy $f'(x)$ a $J$-n monoton növekvő
  legyen.
\end{theorem}

\begin{theorem}
  Ha $f$ a $J$ intervallumon differenciálható, akkor ahhoz, hogy a $J$-n konkáv
  legyen, szükséges és elégséges feltétel, hogy $f'(x)$ a $J$-n monoton csökkenő
  legyen.
\end{theorem}

\begin{theorem}
  Ha $f$ a $J$-n kétszer differenciálható, akkor ahhoz, hogy itt konvex legyen,
  szükséges és elégséges feltétel, hogy $f''(x)\geq 0$ $\forall x\in J$.
\end{theorem}

\begin{theorem}
  Ha $f$ a $J$-n kétszer differenciálható, akkor ahhoz, hogy itt konkáv legyen,
  szükséges és elégséges feltétel, hogy $f''(x)\leq 0$ $\forall x\in J$.
\end{theorem}

\begin{theorem}[Taylor-formula]
  Legyen $f$ $[a; b]$ $\rightarrow \Reals $ és $n$ természetes szám, tegyük fel,
  hogy $f^{(n)}$ folytonos az $[a; b]$ intervallumon és differenciálható az
  $(a; b)$ intervallumon, ekkor $\forall x$ és $\alpha \in [a; b]$ mellett
  $\exists \xi$ $x$ és $\alpha$ között úgy, hogy
  \[
    f(x)=\sum\limits_{k=0}^n \frac{f^{(k)}(\alpha)}{k!}\cdot
    (x-\alpha)^k+\frac{f^{(n+1)}(\xi)}{(n+1)!}\cdot (x-\alpha)^{n+1}
    \text.
  \]
\end{theorem}

\begin{definition}[Szélsőérték számítás]
  Legyen $f$ $I \subset \Reals \rightarrow \Reals $ és $a\in I$, azt mondjuk,
  hogy:
  \begin{itemize}
    \item $f$-nek $a$ pontban lokális
          \begin{itemize}
            \item maximuma van, ha $\exists \delta>0$, hogy $f(x)\leq f(a)$, ha
                  $x\in K(a,\delta)$,
            \item minimuma van, ha $\exists \delta>0$, hogy $f(x)\geq f(a)$, ha
                  $x\in K(a,\delta)$.
          \end{itemize}
    \item $f$-nek $a$ pontban
          \begin{itemize}
            \item szigorú lokális maximuma van, ha $f(x)<f(a)$,
            \item szigorú lokális minimuma van, $f(x)>f(a)$.
          \end{itemize}
    \item $f$-nek $a$ pontban
          \begin{itemize}
            \item abszolút maximuma van, ha $f(x)\leq f(a)$, $\forall x\in I$,
            \item abszolút minimuma van, ha $f(x)\geq f(a)$, $\forall x\in I$.
          \end{itemize}
    \item $f$-nek $a$ pontban
          \begin{itemize}
            \item szigorú abszolút maximuma van, ha $f(x)<f(a)$, $\forall x\in I
                    \setminus \{a\}$,
            \item szigorú abszolút minimuma van, ha $f(x)>f(a)$, $\forall x\in I
                    \setminus \{a\}$.
          \end{itemize}
  \end{itemize}
\end{definition}

\begin{note}
  Zárt intervallumon valós értékű folytonos függvény felveszi a szuprémumát,
  illetve infimumát függvényértékként.
\end{note}

\begin{theorem}[Szélsőérték létezésének szükséges feltétele]
  Ha $f$: $I\in\Reals \rightarrow \Reals $ differenciálható $I$-n és $f$-nek az
  $\alpha \in \inner I$ pontban szélsőértéke van, akkor $f'(\alpha)=0$.
\end{theorem}

\begin{theorem}[Szélsőérték létezésének elégséges feltétele]
  Ha $f: I \in\Reals \rightarrow \Reals $ differenciálható $I$-n és $f$-nek az
  $\alpha \in \inner I$, akkor ha $\exists r>0$, hogy:
  \begin{itemize}
    \item $f'(x)\geq0$, ha $x\in (\alpha-r;\alpha)$ és
          $f'(x)\leq0$, ha $x\in (\alpha;\alpha+r)$,\\
          akkor $f$-nek lokális maximuma van $\alpha$-ban,
    \item $f'(x)\leq0$, ha $x\in (\alpha;\alpha+r)$
          és $f'(x)\geq0$, ha $x\in (\alpha-r;\alpha)$,\\
          akkor $f$-nek lokális minimuma van $\alpha$-ban.
  \end{itemize}
\end{theorem}

\begin{theorem}
  Legyen $f$ $[a; b] \rightarrow \Reals$ $a$-ban jobbról, $b$-ben balról
  differenciálható függvény.
  \begin{itemize}
    \item Ha $f'(b)>0$, akkor $f$-nek $b$-ben szigorú helyi maximuma van.
    \item Ha $f'(b)<0$, akkor $f$-nek $b$-ben szigorú helyi minimuma van.
    \item Ha $f'(a)>0$, akkor $f$-nek $a$-ban szigorú helyi minimuma van.
    \item Ha $f'(a)<0$, akkor $f$-nek $a$-ban szigorú helyi maximuma van.
  \end{itemize}
\end{theorem}

\begin{theorem}
  Legyen $f: I\subset \Reals \rightarrow \Reals $, $n>1$-szer differenciálható
  $I$-n, $\alpha \in \inner I $, ha
  \[
    f'(\alpha)=f''(\alpha)=\dots =f^{(n-1)}
    (\alpha)=0
    \quad \text{és}\quad
    f^{(n)}(\alpha)\neq 0
    \text,
  \]
  akkor, ha
  \begin{itemize}
    \item $n$ páratlan, akkor nincs szélsőérték $\alpha$-ban,
    \item $n$ páros, akkor van szélsőérték $\alpha$-ban, és
          \begin{itemize}
            \item $f^{(n)}(b)>0$ esetén lokális minimuma van,
            \item $f^{(n)}(b)<0$ esetén lokális maximuma van.
          \end{itemize}
  \end{itemize}
\end{theorem}

\begin{theorem}
  Legyen $f: I\subset \Reals \rightarrow \Reals $, $n>1$-szer differenciálható
  az $I$-n, $\alpha \in \inner I $, valamint
  \[
    f'(\alpha)=f''(\alpha)=\dots =f^{(n-1)}
    (\alpha)=0
    \quad \text{és}\quad
    f^{(n)}(\alpha)\neq 0
    \text,
  \]
  akkor, ha
  \begin{itemize}
    \item $f^{(n)}(b)>0$ és $n$ páratlan, akkor szigorú helyi maximuma van
          $b$-ben,
    \item $f^{(n)}(b)>0$ és $n$ páros, akkor szigorú helyi minimuma van
          $b$-ben.
  \end{itemize}
\end{theorem}

\begin{definition}
  Legyen $f : I \rightarrow \Reals $ és $\alpha \in \inner I$, $\alpha$-t $f$
  inflexiós pontjának mondjuk, ha $\exists \delta >0$, hogy $f$ függvény
  $(\alpha,\alpha+ \delta)$ intervallumon konvex és az $(\alpha - \delta,
    \alpha)$ intervallumon konkáv, vagy ha  $(\alpha,\alpha+ \delta)$
  intervallumon konkáv és az $(\alpha - \delta,\alpha)$ intervallumon konvex.
\end{definition}

\begin{note}
  Ha $f$ kétszer differenciálható és $\alpha$ inflexiós pont, akkor
  $f''(\alpha)=0$.
\end{note}

\begin{definition}
  Ha van olyan $y=A\cdot x+B$ lineáris függvény, melyre
  \[
    \lim\limits_{x \rightarrow \infty} f(x)-(A\cdot x+B)=0
    \text{, vagy }
    \lim\limits_{x \rightarrow -\infty} f(x)-(A\cdot x+B)=0
    \text,
  \]
  akkor az $y=A\cdot x+B$ egyenest az $f$ függvény aszimptotájának nevezzük.
\end{definition}

\begin{note}
  \sftitle{Aszimptota létezésének szükséges feltétele:}

  Ha $f$-nek létezik a fenti definícióban szereplő aszimptotája, akkor
  \[
    \lim_{x \rightarrow \infty} \frac{f(x)}{x}=A
    \text,
    \quad\text{vagy}\quad
    \lim_{x \rightarrow -\infty} \frac{f(x)}{x}=A
    \text.
    \quad
    (A \in \Reals)
  \]
\end{note}
\begin{note}
  \sftitle{Aszimptota létezésének elégséges feltétele:}
  \[
    \lim\limits_{x \rightarrow \infty} (f(x)-A\cdot x)\; \text{vagy}
    \lim\limits_{x \rightarrow \infty} (f(x)-A\cdot x)
  \]
  határérték végessége, ami a $B$ értékét adja.
\end{note}

\begin{theorem}[L'Hôpital-szabály]
  Legyenek $f$ és $g$ differenciálhatóak az $\alpha \in \Reals_b$ pont egy
  környezetében ($\alpha$-ban nem szükségképpen), továbbá $g(x) \neq 0$ és
  $g'(x) \neq 0$ és
  \[
    \lim\limits_{x \rightarrow \alpha} f(x) = \lim\limits_{x \rightarrow \alpha} g(x)=0
    \text{, vagy}
    \lim\limits_{x \rightarrow \alpha} |f(x)|=\lim\limits_{x \rightarrow \alpha}|g(x)|=\infty
    \text,
  \]
  \[
    \text{ekkor, ha}
    \lim\limits_{x \rightarrow \alpha} \frac{f'(x)}{g'(x)}=B
    \text{, akkor }
    \lim\limits_{x \rightarrow \alpha} \frac{f(x)}{g(x)}=B
    \text.
  \]
\end{theorem}