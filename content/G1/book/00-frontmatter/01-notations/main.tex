\chapter{Jelölések}

\begin{blueBox}
  Ez egy egyszerű szövegdoboz.
\end{blueBox}

\begin{note}
  Ez egy megjegyzés.
\end{note}

\begin{statement}
  Ez egy állítás.
\end{statement}

\begin{example}
  Ez egy példa.
\end{example}

\begin{learnMore}
  Ez egy kitekintés.
\end{learnMore}

\begin{definition}
  Ez egy definíció.
\end{definition}

\begin{theorem}
  Ez egy tétel.
\end{theorem}

\begin{mdframed}[style=questions, frametitle={\color{white}Felkészülést segítő kérdések}]
  Ezek segítenek a tanulásban.
\end{mdframed}

% \newcommand{\tname}[1]{\marginnote{\sffamily\bfseries\color{primaryColor}#1}}
\newcommand{\mfrac}[2]{\frac{{#1}^{\mathstrut}}{{#2}_{\mathstrut}}}
\newcommand{\mwrap}[1]{{#1}^{\mathstrut}_{\mathstrut}}
\newenvironment{notations}[1]{%
  \checkoddpage%
  \ifoddpage%
    % \def\tcbleft{0pt}%
    % \def\tcbright{2pt}%
    \newcolumntype{U}{r<{\hspace{2mm}}}
  \else%
    % \def\tcbleft{2pt}%
    % \def\tcbright{0pt}%
    \newcolumntype{U}{l}
  \fi%
  % \begin{tcolorbox}[
  %     frame hidden,
  %     colframe=white,
  %     boxsep=0pt,
  %     left=0pt,
  %     right=0pt,
  %     top=12mm,
  %     bottom=0pt,
  %     arc=0pt,
  %     outer arc=0pt,
  %   ]
  \begin{center}
    \def\arraystretch{1.55}
    \setlength\arrayrulewidth{1pt}
    \arrayrulecolor{primaryColor}
    \begin{tabular}
      {
      |
      >{\centering\arraybackslash$\displaystyle}m{3.5cm}<{$}
      |
      >{\centering\arraybackslash}m{5cm}
      |
      >{\centering\arraybackslash$\displaystyle}m{6cm}<{$}
      |
      }
      \hline
      \multicolumn{3}{|U|}{
        \sffamily\Large\bfseries\color{white}\cellcolor{primaryColor}
        #1
      }
      \\
      \hline
      \cellcolor{gray!25}\text{\sffamily\color{primaryColor}\textbf{Jel}}
       & \cellcolor{gray!25}\text{\sffamily\color{primaryColor}\textbf{Megnevezés}}
       & \cellcolor{gray!25}\text{\sffamily\color{primaryColor}\textbf{Példa}}
      \\
      \hline
      }{%
      \\
      \hline
    \end{tabular}
  \end{center}
  % \end{tcolorbox}
}

\clearpage

\begin{notations}{Logikai szimbólumok}
  \land       & és               & p \land q
  \\\hline
  \lor        & vagy             & p \lor q
  \\\hline
  \forall     & minden / bármely & \forall x \in X
  \\\hline
  \exists     & létezik          & \exists x \in X
  \\\hline
  % \exists!    & biztosan létezik & \exists! x \in X
  % \\\hline
  \not\exists & nem létezik      & \not\exists x \in X
  % \\\hline
  % !           & legyen           & !x \in X
\end{notations}
\hfill
\begin{notations}{Egyenlőség, relációk}
  =      & egyenlő              & 2 + 2 = 4
  \\\hline
  \neq   & nem egyenlő          & 2 \neq 3
  \\\hline
  \equiv & ekvivalens           & 2 \equiv 2
  \\\hline
  <      & kisebb               & 2 < 3
  \\\hline
  \leq   & kisebb vagy egyenlő  & 2 \leq 3
  \\\hline
  >      & nagyobb              & 3 > 2
  \\\hline
  \geq   & nagyobb vagy egyenlő & 3 \geq 2
\end{notations}
\hfill
\begin{notations}{Műveletek}
  a + b       & összeg        & 2 + 3 = 5
  \\\hline
  a - b       & különbség     & 5 - 3 = 2
  \\\hline
  a \cdot b   & szorzat       & 2 \cdot 3 = 6
  \\\hline
  a / b       & hányados      & 6 / 3 = 2
  \\\hline
  a^b         & hatvány       & 2^3 = 8
  \\\hline
  \sqrt{a}    & négyzetgyök   & \sqrt{4} = 2
  \\\hline
  \sqrt[n]{a} & $n$-edik gyök & \sqrt[3]{8} = 2
  \\\hline
  a!          & faktoriális   & 3! = 3 \cdot 2 \cdot 1 = 6
\end{notations}

\clearpage

\begin{notations}{Halmazok és halmazműveletek}
  \emptyset, \{\} & üreshalmaz
                  & A := \{ k \mid k \in \mathbb N \land k < 0 \}
  \\\hline
  \mathbb N       & természetes számok halmaza
                  & 1, 2, 3, \dots
  \\\hline
  \mathbb Z       & egész számok halmaza
                  & \dots, -2, -1, 0, 1, 2, \dots
  \\\hline
  \mathbb Q       & racionális számok halmaza
                  & 2/3 \in \mathbb{Q}
  \\\hline
  \mathbb Q^*     & irracionális számok halmaza
                  & \pi \in \mathbb{Q}
  \\\hline
  \mathbb R       & valós számok halmaza
                  & \sqrt{2} \in \mathbb{R}
  \\\hline
  \mathbb C       & komplex számok halmaza
                  & i \in \mathbb{C}
  \\\hline
  A, B, C         & halmazok
                  & A = \{1; 2; 3\}
  \\\hline
  a, b, c         & halmazok elemei
                  & x \in A
  \\\hline
  \in             & eleme
                  & \iu \in \mathbb C
  \\\hline
  \notin          & nem eleme
                  & \pi \notin \mathbb Q
  \\\hline
  \sim            & ekvivalencia
                  & A \sim B
  \\\hline
  \subseteq       & részhalmaza
                  & \{1\} \subseteq \{1; 2\}
  \\\hline
  \subset         & valódi részhalmaza
                  & A \subset B \Leftrightarrow A \subseteq B \land A \neq B
  \\\hline
  \overline A     & komplementer halmaz
                  & \{x \in X \mid x \notin A\}
  \\\hline
  \cup            & unió
                  & \{x \in X \mid x \in A \lor x \in B\}
  \\\hline
  \cap            & metszet
                  & \{x \in X \mid x \in A \land x \in B\}
  \\\hline
  \setminus       & kivonás
                  & \{x \in X \mid x \in A \land x \notin B\}
\end{notations}
\hfill
\begin{notations}{Intervallumok}
  [a; b] & zárt intervallum                         & [0; 1]
  \\\hline
  (a; b) & nyílt intervallum                        & (0; 1)
  \\\hline
  [a; b) & balról zárt, jobbról nyitott intervallum & [0; 1)
  \\\hline
  (a; b] & balról nyitott, jobbról zárt intervallum & (0; 1]
\end{notations}

\clearpage

% \begin{notations}{Konstansok}
%   \pi & pi                 & \pi \approx 3.14159
%   \\\hline
%   e   & Euler-féle szám    & e \approx 2.71828
%   \\\hline
%   \iu & imaginárius egység & \iu^2 = -1
% \end{notations}
% \vfill
\begin{notations}{Komplex számok}
  \mathbb C
   & komplex számok halmaza
   & z \in \mathbb C
  \\\hline
  \iu
   & imaginárius egység
   & \iu^2 = -1
  \\\hline
  z
   & komplex szám
   & z = 3 + 4 \iu
  \\\hline
  z = a + b \iu
   & algebrai alak
   & \iRe \{z\} = a, \iIm \{z\} = b
  \\\hline
  \overline z = a - b \iu
   & konjugált
   & \overline{3 + 4\iu} = 3 - 4\iu
  \\\hline
  \iRe \{ z \}
   & valós rész
   & z = 3 + 2\iu \;\rightarrow\; \iRe \{z\} = 3
  \\\hline
  \iIm \{ z \}
   & képzetes rész
   & z = 1 + 4\iu \;\rightarrow\; \iIm \{z\} = 4
  \\\hline
  | z |
   & abszolút érték / hossz
   & | z | = \sqrt{a^2 + b^2}
  \\\hline
  \arg \{z\}
   & argumentum
   & \arg \{z\} = \arctan(b/a)
  \\\hline
  z = r (\cos \varphi + \iu \sin \varphi)
   & trigonometrikus alak
   & | z | = r, \arg \{z\} = \varphi
  \\\hline
  z = r e^{\iu \varphi}
   & exponenciális alak
   & z = r e^{\iu \varphi} = r \exp(\iu \varphi)
\end{notations}
\vfill
\begin{notations}{Sorozatok, sorok}
  (a_n)
   & numerikus sorozat
   & a_n = \mfrac{1}{n}
  \\\hline
  \lim_{n \rightarrow \infty} a_n = a
   & sorozat határértéke
   & \lim_{n \rightarrow \infty} \mfrac{1}{n} = 0
  \\\hline
  a_n = a_{n - 1} + d
   & számtani sorozat
   & a_n = a_{n - 1} + 2
  \\\hline
  a_n = a_{n - 1} \cdot q
   & mértani sorozat
   & a_n = a_{n - 1} \cdot 2
  \\\hline
  \mwrap\sum a_n
   & numerikus sor
   & \mwrap{\sum \frac{1}{n}}
  \\\hline
  L = \sum_{n = 0}^{\infty} a_n
   & sor összege
   & L = \mwrap{\sum_{n = 0}^{\infty}} \frac{1}{n} = \infty
  \\\hline
  \mwrap\sum a \cdot r^n
   & \makecell{geometriai sor                               \\ {(${|r| < 1}$ esetén)}}
   & \mwrap{\sum_{n = 0}^{\infty}} \frac{1}{2^n}
  = \frac{1}{1 - r}
  = \frac{1}{1 - \sfrac{1}{2}}
  = 2
\end{notations}

\clearpage

\begin{notations}{Függvények}
  f: \Domain \rightarrow \Range, x \mapsto y
   & $f$ függvény
   & f: \Reals \rightarrow \Reals, x \mapsto x^2
  \\\hline
  \Domain_f
   & értelmezési tartomány
   & \Domain_f = \Reals
  \\\hline
  \Range_f
   & értékkészlet
   & \Range_f = [0; +\infty)
  \\\hline
  \Domain \rightarrow \Range
   & értékkészlet hozzárendelése az értelmezési tartományhoz
   & f: \Reals \rightarrow \Reals
  \\\hline
  x \mapsto y
   & függvényértékek hozzárendelése az ősképekhez
   & f: x \mapsto x^2
  \\\hline
  f^{-1}
   & inverz függvény
   & \text{ha } f(3) = 5 \text{, akkor } f^{-1}(5) = 3
  \\\hline
  f \circ g
   & összetett függvény
   & f(x) = e^x, g(x) = x^2 : {(f \circ g)(x) = e^{x^2}}
  \\\hline
  \lim_{x \rightarrow a} f(x) = A
   & függvény határértéke
   & \lim_{x \rightarrow 0} \mfrac{\sin x}{x} = 1
  \\\hline
  \lim_{x \rightarrow a^{+}} f(x) = A
   & jobboldali
   & \lim_{x \rightarrow 0^{+}} \mfrac{1}{x} = +\infty
  \\\hline
  \lim_{x \rightarrow a^{-}} f(x) = A
   & baloldali
   & \lim_{x \rightarrow 0^{-}} \mfrac{1}{x} = -\infty
\end{notations}
\vfill
\begin{notations}{Nevezetes függvények}
  e^x, \exp x
   & exponenciális függvény
   & x \mapsto e^x
  \\\hline
  \ln x
   & természetes alapú logaritmus
   & x \mapsto \ln x
  \\\hline
  a^x
   & hatványfüggvény
   & x \mapsto 2^x
  \\\hline
  \log_a x
   & $a$ alapú logaritmus
   & x \mapsto \log_2 x
  \\\hline
  % \lg x
  %  & tízes alapú logaritmus
  %  & \lg 100 = 2
  % \\\hline
  \makecell{\sin, \cos,             \\ \tan, \cot}
   & szögfüggvények
   & x \mapsto \sin x
  \\\hline
  \makecell{\arcsin, \arccos,       \\ \arctan, \arccot}
   & inverz szögfüggvények
   & x \mapsto \arcsin x
  \\\hline
  \makecell{\sinh, \cosh,           \\ \tanh, \coth}
   & hiperbolikus függvények
   & x \mapsto \sinh x
  \\\hline
  \makecell{\arcsinh, \arccosh,     \\ \arctanh, \arccoth}
   & inverz hiperbolikus függvények
   & x \mapsto \arcsinh x
\end{notations}

\clearpage

\begin{notations}{Kalkulus}
  f'(x), f''(x), f^{(n)}(x)
   & első, második és $n$-edik derivált (Lagrange-féle jelölés)
   & f'(x) = \lim_{h \to 0} \frac{f(x + h) - f(x)}{h}
  \\\hline
  \odv{f}{x}, \odv[order=2]{f}{x}, \odv[order=n]{f}{x}
   & első, második és $n$-edik derivált (Leibniz-féle jelölés)
   & f'(x) = \odv{f}{x}
  \\\hline
  \dot f, \ddot f, \overset{n}{\dot f}
   & első, második és $n$-edik derivált (Newton-féle jelölés)
   & \dot f = \odv{f}{t}
  \\\hline
  \mwrap{\int_a^b} f(x) \dd x
   & Riemann-integrál
   & \mwrap{\int_0^1} x^2 \dd x = \frac{1}{3}
  \\\hline
  \mwrap{\int} f(x) \dd x
   & határozatlan integrál
   & \mwrap{\int} f(x) \dd x = F(x) + C
  \\\hline
  F
   & $f$ primitív függvénye
   & F'(x) = f(x)
  \\\hline
  f \in \mathcal R [a; b]
   & $f$ Riemann-integrálható az $[a; b]$ intervallumon
   & x^2 \in \mathcal R (-\infty; +\infty)
\end{notations}