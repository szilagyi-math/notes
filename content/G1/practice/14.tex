\documentclass[a4paper, 12pt]{scrartcl}

\usepackage{math-practice}

\title{Numerikus sorok}
\area{Sorok}
\subject{Matematika G1}
\subjectCode{BMETE94BG01}
\date{Utoljára frissítve: \today}
\docno{14}

\begin{document}
\maketitle

\subsection{Elméleti Áttekintő}

\clearpage
\subsection{Feladatok}

\begin{enumerate}
  % 1
  \item Bizonyítsuk be a konvergencia definíciója alapján, hogy az alábbi sorok
        konvergensek vagy divergensek!
        \begin{multicols}{2}
          \begin{enumerate}
            \item $\displaystyle
                    \sum_{n = 1}^\infty \frac{2^n + 3^n}{6^n}
                  $

            \item $\displaystyle
                    \sum_{n = 1}^\infty \frac{1}{n (n - 1)}
                  $

            \item $\displaystyle
                    \sum_{n = 2}^\infty \ln \left( 1 - \frac{1}{n^2} \right)
                  $

            \item $\displaystyle
                    \sum_{n = 1}^\infty \frac{a^n}{n^k}
                  $
          \end{enumerate}
        \end{multicols}

        % 2
  \item A Cauchy-féle konvergenciakritérium alapján bizonyítsuk be, hogy az
        alábbi sor konvergens!
        \[
          \sum_{n = 1}^\infty \frac{n}{n^3 + n^2 + 1}
        \]

        % 3
        % \item A majoráns vagy minoráns kritérium, a hágyados- vagy gyökteszt illetve
        %       az integrálkritérium segítségével vizsgáljuk meg az alábbi sorok
        %       konvergenciáját!
  \item Vizsgáljuk meg az alábbi sorok konvergenciáját!
        \begin{multicols}{3}
          \begin{enumerate}
            \item $\displaystyle
                    \sum_{n = 1}^\infty \frac{2n^3 - 16}{n^5 + n}
                  $

            \item $\displaystyle
                    \sum_{n = 1}^\infty \frac{(\cos \sfrac{\pi}{2})^n}{n^n + 1}
                  $

            \item $\displaystyle
                    \sum_{n = 2}^\infty \frac{1}{\ln n}
                  $

            \item $\displaystyle
                    \sum_{n = 1}^\infty \left( 1 - \frac1n \right)^n
                  $

            \item $\displaystyle
                    \sum_{n = 1}^\infty \frac{1}{\sqrt{n(n+1)}}
                  $

            \item $\displaystyle
                    \sum_{n = 1}^\infty \frac{2n^2}{(2 + \sfrac{1}{n})^n}
                  $

            \item $\displaystyle
                    \sum_{n = 1}^\infty \left(
                    \frac{n - 1}{n +1}
                    \right)^{n(n - 1)}
                  $

            \item $\displaystyle
                    \sum_{n = 1}^\infty \frac{n}{e^n}
                  $

            \item $\displaystyle
                    \sum_{n = 1}^\infty (-1)^{n + 1} \frac{n}{n^2 + 1}
                  $

            \item $\displaystyle
                    \sum_{n = 1}^\infty \frac{\sin n}{\sqrt[3]{n^4}}
                  $

            \item $\displaystyle
                    \sum_{n = 1}^\infty (-1)^n \frac{n - 1}{n(n + 1)}
                  $

            \item $\displaystyle
                    \sum_{n = 1}^\infty (-1)^n \frac{n - 1}{n(n + 1)}
                  $
          \end{enumerate}
        \end{multicols}

        % 4
  \item Konvergens-e a $\sum a_n$ és $\sum b_n$ összegsora, ha
        \[
          \sum a_n = \sum_{n = 1}^\infty \frac{1+n}{3^n}
          \hspace{2cm}
          \sum b_n = \sum_{n = 1}^\infty \frac{(-1)^n - n}{3^n}
          \text.
        \]

        % 5
  \item Konvergens-e a $\sum a_n$ és $\sum b_n$ különbségsora, ha
        \[
          \sum a_n = \sum_{n = 1}^\infty \frac{1}{2n - 1}
          \hspace{2cm}
          \sum b_n = \sum_{n = 1}^\infty \frac{1}{2n}
          \text.
        \]

        % 6
  \item Konvergens-e a $\sum a_n$ és $\sum b_n$ Cauchy-szorzata, ha
        \[
          \sum a_n = \sum_{n = 1}^\infty \frac{1}{n\sqrt{n}}
          \hspace{2cm}
          \sum b_n = \sum_{n = 1}^\infty \frac{1}{2^{n - 1}}
          \text.
        \]

        % 7
\end{enumerate}

% \\underline\{(\w)\}

\end{document}
