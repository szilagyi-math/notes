\documentclass[a4paper, 12pt]{scrartcl}

\usepackage{math-practice}

\title{Térgeometriai alakzatok}
\area{Analitikus geometria}
\subject{Matematika G1}
\subjectCode{BMETE94BG01}
\date{Utoljára frissítve: \today}
\docno{2}

\begin{document}
\maketitle

\subsection{Elméleti Áttekintő}

\begin{blueBox}
  \begin{minipage}{.575\textwidth}
    \sftitle{Egyenes 2D-ben}:
    \begin{alignat*}{9}
      \rvec n   & (A; B)     &  & \text{ -- egyenes normálvektora}        \\
      \rvec r   & (x; y)     &  & \text{ -- tetszőleges pont helyvektora} \\
      \rvec r_0 & (x_0; y_0) &  & \text{ -- $P_0$ fixpont helyvektora}
    \end{alignat*}
  \end{minipage}\begin{minipage}{.425\textwidth}
    \centering
    \begin{tikzpicture}[ultra thick]
      % Coordinate system
      \coordinate (O) at (0, 0);
      \draw[draw=ternaryColor, ->, thick] (O) -- (4, 0) node[above left] {$x$};
      \draw[draw=ternaryColor, ->, thick] (O) -- (0, 2) node[below left] {$y$};

      % Line
      \draw[draw=secondaryColor, thick] (-.5, .5) -- (3.5, 1.25)
      node[right] {$e$}
      coordinate[pos=.3] (P0)
      coordinate[pos=.65] (P)
      coordinate[pos=.85] (N)
      ;

      % Points
      \node[above] at (P0) {$\rvec r_0$};
      \node[above left] at (P) {$\rvec r$};

      % Vectors
      \draw[draw=primaryColor, ->] (O) -- (P0);
      \draw[draw=primaryColor, ->] (O) -- (P);

      \draw[draw=primaryColor, ->] (N) -- ++(-0.1875,1)
      coordinate (Q)
      node[midway, right] {$\rvec n$}
      ;

      % Right angle
      \draw pic["$\cdot$", draw, angle eccentricity=.5, angle radius=4mm, thick]
        {angle=Q--N--P}
      ;
    \end{tikzpicture}
  \end{minipage}
\end{blueBox}

\begin{blueBox}
  \sftitle{Az egyenes egyenlete}:
  \begin{gather*}
    (\rvec r_0 - \rvec r) \cdot \rvec n = 0
    \\[2mm]
    \rvec r \cdot \rvec n = \rvec r_0 \cdot \rvec n
    \\[2mm]
    Ax + By = \underbrace{Ax_0 + By_0}_{=: -C}
    \quad\rightarrow\quad
    Ax + By + C = 0
  \end{gather*}
\end{blueBox}

\begin{blueBox}
  \sftitle{Hesse-féle normálalak}:
  \[
    \frac{Ax + By + C}{\sqrt{A^2 + B^2}} = 0
  \]

  A Hesse-féle normálalakot úgy kapjuk, hogy az egyenes normálvektorát
  egységhosszúságúra normáljuk.
\end{blueBox}

\begin{blueBox}
  \sftitle{Két egyenes viszonya}:

  Két egyenes által \textbf{közbezárt szög} a két egyenes normálvektorai által
  bezárt szög.

  Ebből következik, hogy ha a normálvektorok által bezárt szög $90^\circ$,
  akkor a két egyenes merőleges egymásra. Ha a normálvektorok párhuzamosak,
  akkor a két egyenes is párhuzamos.
\end{blueBox}

\begin{blueBox}
  \sftitle{Pont és egyenes távolsága}:

  \begin{minipage}{.625\textwidth}
    Adott egy $e$ egyenes Hesse-féle normálalakja és egy $P_0(x_0; y_0)$ pont.
    Ekkor a pont és az egyenes távolsága:
    \[
      d = \left|
      \frac{Ax_0 + By_0 + C}{\sqrt{A^2 + B^2}}
      \right|
      \text.
    \]
  \end{minipage}\begin{minipage}{.375\textwidth}
    \centering
    \begin{tikzpicture}[ultra thick]
      % Coordinate system
      \coordinate (O) at (0, 0);
      \draw[draw=ternaryColor, ->, thick] (O) -- (4, 0) node[above left] {$x$};
      \draw[draw=ternaryColor, ->, thick] (O) -- (0, 2) node[below left] {$y$};

      % Line
      \draw[draw=secondaryColor, thick] (-.5,0.25) -- ++(4,1)
      node[right] {$e$}
      coordinate[pos=.5] (P)
      coordinate (E)
      ;

      %  Distance
      \draw[draw=primaryColor]
      (P) -- ++(-.375,1.5)
      coordinate (P0)
      node[right=2mm] {$P_0$}
      node[midway, left] {$d$}
      ;

      % Point
      \draw[fill=primaryColor] (P0) circle (0.1);

      % Right angle
      \draw pic["$\cdot$", draw, angle eccentricity=.5, angle radius=4mm, thick]
        {angle=E--P--P0}
      ;
    \end{tikzpicture}
  \end{minipage}

  A formula csak azon pontokra ad zérus értéket, amelyek rajta vannak az
  egyenesen.
\end{blueBox}

\begin{blueBox}
  \begin{minipage}{.6\textwidth}
    \sftitle{Egyenes 3D-ben}:
    \begin{alignat*}{9}
      \rvec v   & (a; b; c)       &  & \text{ -- egyenes irányvektora}         \\
      \rvec r   & (x; y; z)       &  & \text{ -- tetszőleges pont helyvektora} \\
      \rvec r_0 & (x_0; y_0; z_0) &  & \text{ -- $P_0$ fixpont helyvektora}
    \end{alignat*}
    % \begin{gather*}
    %   (\rvec r - \rvec r_0) \;||\; \rvec v \text{ (irányvektor)}
    %   \\
    %   \Downarrow
    %   \\
    %   (\rvec r - \rvec r_0) \times \rvec v = \nvec
    % \end{gather*}
  \end{minipage}\begin{minipage}{.4\textwidth}
    \centering
    \begin{tikzpicture}[scale=.4, ultra thick]
      % Coordinate system
      \coordinate (O) at (0,0,0);
      \draw[draw=ternaryColor, ->, thick] (O) -- (8,0,0) node[below left] {$x$};
      \draw[draw=ternaryColor, ->, thick] (O) -- (0,4,0) node[below left] {$y$};
      \draw[draw=ternaryColor, ->, thick] (O) -- (0,0,4) node[below left] {$z$};

      % Line
      \draw[draw=secondaryColor, thick] (-2.5, 1, 0) coordinate (E) -- (7, 3.25, 0.75)
      node[pos=.05, above] {$e$}
      coordinate[pos=1] (V)
      coordinate[pos=.5] (P0)
      coordinate[pos=.8] (P)
      ;

      % Vectors
      \draw[draw=primaryColor, ->] (O) -- (P0)
      node[above] {$\rvec r_0$}
      ;
      \draw[draw=primaryColor, ->] (O) -- (P)
      node[above] {$\rvec r$}
      ;
      \draw[draw=primaryColor, ->] (V) -- ++($0.3*(V)-0.3*(E)$)
      node[midway, above] {$\rvec v$}
      ;
    \end{tikzpicture}
  \end{minipage}
\end{blueBox}

\begin{blueBox}
  \sftitle{3D egyenes paraméteres alakja}:

  \begin{minipage}{.45\textwidth}
    \begin{alignat*}{9}
      \rvec r_0 & (x_0; y_0; z_0) &  & \text{ -- $P_0$ fixpont helyvektora} \\
      \rvec v   & (a; b; c)       &  & \text{ -- egyenes irányvektora}      \\
      t         & \in \Reals      &  & \text{ -- paraméter}
    \end{alignat*}
  \end{minipage}\begin{minipage}{.55\textwidth}
    \[
      \rvec r(t) = \rvec r_0 + t \rvec v
      \quad\rightarrow\quad
      \begin{cases}
        \;x(t) = x_0 + a t \\
        \;y(t) = y_0 + b t \\
        \;z(t) = z_0 + c t
      \end{cases}
    \]
  \end{minipage}

  A paraméteres egyenletekből $t$-t kifejezve is megadhatjuk az egyenest:
  \[
    (t = \; ) \;
    \frac{x - x_0}{a} = \frac{y - y_0}{b} = \frac{z - z_0}{c}
    \text.
  \]
\end{blueBox}

\begin{blueBox}
  \sftitle{Két egyenes viszonya}:

  Két egyenes által közbezárt szög a két egyenes irányvektorai által bezárt
  szög.
\end{blueBox}

\begin{blueBox}
  \sftitle{Pont és egyenes távolsága}:

  \begin{minipage}{.55\textwidth}
    Egy $\rvec r_1$ irányvektorú $P_1$ pont, és egy $\rvec r(t) = \rvec r_0 + t
      \rvec v$ egyenletű egyenes távolsága
    \[
      d = \left|
      \frac{\rvec v \times (\rvec r_0 - \rvec r_1)}{|\rvec v|}
      \right|
    \]
  \end{minipage}\begin{minipage}{.45\textwidth}

    \centering
    \begin{tikzpicture}[scale=.4, ultra thick]
      % Coordinate system
      \coordinate (O) at (0,0,0);
      \draw[draw=ternaryColor, ->, thick] (O) -- (8,0,0) node[above left] {$x$};
      \draw[draw=ternaryColor, ->, thick] (O) -- (0,5,0) node[below left] {$y$};
      \draw[draw=ternaryColor, ->, thick] (O) -- (0,0,4) node[below left] {$z$};

      % Line
      \draw[draw=secondaryColor, thick] (-2.5, 2.5, 1) coordinate (E) -- (8, 2.5, 8)
      node[pos=.05, below] {$e$}
      coordinate[pos=1] (V)
      coordinate[pos=.55] (P)
      ;

      % Vectors
      \draw[draw=primaryColor, ->] (V) -- ++($0.3*(V)-0.3*(E)$)
      node[midway, below] {$\rvec v$}
      ;

      % Point
      \coordinate (Q) at (2,2.5,-2);
      \node [right] at (Q) {$P_1$};
      \draw[draw=primaryColor] (Q) -- (P)
      node[midway, above left] {$d$}
      ;
      \draw[fill=primaryColor] (Q) circle (0.1);

      % Right angle
      \draw pic["$\cdot$", draw, angle eccentricity=.55, angle radius=4mm, thick]
        {angle=V--P--Q}
      ;
    \end{tikzpicture}
  \end{minipage}
\end{blueBox}

\begin{blueBox}
  \sftitle{Két egyenes távolsága}:

  \begin{minipage}{.6\textwidth}
    Az $e_1: \rvec p_1(t_1) = \rvec r_1 + t_1 \rvec v_1$ és $e_2: \rvec p_2(t_2)
      = \rvec r_2 + t_2 \rvec v_2$ egyenesek távolsága
    \[
      d = \Bigg|
      (\rvec r_2 - \rvec r_1) \cdot
      \underbrace{\frac{(\rvec v_1 \times \rvec v_2)}{|\rvec v_1 \times \rvec v_2|}}_{\uvec n_T}
      \Bigg| = \left|
      (\rvec r_2 - \rvec r_1) \cdot \uvec n_T
      \right|
      \text,
    \]
    ahol $\uvec n_T$ egy olyan egységvektor, amely merőleges mindkét egyenesre.
    (normál transzverzális)
  \end{minipage}\begin{minipage}{.4\textwidth}
    \centering
    \begin{tikzpicture}[ultra thick]
      % Coordinate system
      \coordinate (O) at (0,0,0);
      \draw[draw=ternaryColor, ->, thick] (O) -- (4,0,0) node[above left] {$x$};
      \draw[draw=ternaryColor, ->, thick] (O) -- (0,3,0) node[below left] {$y$};
      \draw[draw=ternaryColor, ->, thick] (O) -- (0,0,2) node[below left] {$z$};

      % 2 lines
      \draw[draw=secondaryColor, thick] (-1, 2) -- (4, 3)
      coordinate (V1)
      node[pos=.05, below] {$e_1$}
      coordinate[pos=.5] (P1)
      ;
      \draw[draw=secondaryColor, thick] (-1.5, 1) -- (3.75, -.5)
      coordinate[pos=0] (V2)
      node[pos=.05, below] {$e_2$}
      coordinate[pos=.5] (P2)
      ;

      % Distance
      \draw[draw=primaryColor] (P1) -- (P2)
      node[midway, left] {$d$}
      ;

      % Right angles
      \draw
      pic["$\cdot$", draw, angle eccentricity=.5, angle radius=4mm, thick]
        {angle=P2--P1--V1}
      pic["$\cdot$", draw, angle eccentricity=.5, angle radius=4mm, thick]
        {angle=P1--P2--V2}
      ;
    \end{tikzpicture}
  \end{minipage}

  Amennyiben $e_1$ és $e_2$ párhuzamosak, akkor a távolságukat a pont és egyenes
  távolságának képletével számolhatjuk.
\end{blueBox}

\begin{blueBox}
  \begin{minipage}{.575\textwidth}
    \sftitle{Sík 3D-ben}:
    \begin{alignat*}{9}
      \rvec n   & (A; B; C)       &  & \text{ -- sík normálvektor}             \\
      \rvec r   & (x; y; z)       &  & \text{ -- tetszőleges pont helyvektora} \\
      \rvec r_0 & (x_0; y_0; z_0) &  & \text{ -- $P_0$ fixpont helyvektora}
    \end{alignat*}
  \end{minipage}\begin{minipage}{.425\textwidth}
    \centering
    \begin{tikzpicture}[ultra thick]
      % Coordinate system
      \coordinate (O) at (0,0,0);
      \draw[draw=ternaryColor, ->, thick] (O) -- (4,0,0) node[above left] {$x$};
      \draw[draw=ternaryColor, ->, thick] (O) -- (0,2.25,0) node[below left] {$y$};
      \draw[draw=ternaryColor, ->, thick] (O) -- (0,0,1.5) node[right=2mm] {$z$};

      % Plane at y = 1.5
      \draw[draw=secondaryColor, fill=secondaryColor, fill opacity=.35, thick]
      (-1, 1.25, -.5) -- (4, 1.25, -.5) -- (4, 1.25, 2) -- (-1, 1.25, 2) -- cycle
      ;

      % Vectors
      \draw[draw=primaryColor, ->, ultra thick] (O) -- (-.5, 1.25, 1)
      node[midway, below left] {$\rvec r_0$}
      ;
      \draw[draw=primaryColor, ->, ultra thick] (O) -- (2, 1.25, 1)
      node[midway, below right] {$\rvec r$}
      ;
      \draw[draw=primaryColor, ->, ultra thick]
      (3, 1.25, .5) coordinate (N)
      -- ++(0, 1, 0) coordinate (Q)
      node[pos=.7, right] {$\rvec n$}
      ;

      \coordinate (S) at (2, 1.25, .5);
      \draw[gray, dashed] (S) -- (N);

      % Right angle
      \draw pic["$\cdot$", draw, angle eccentricity=.5, angle radius=4mm, thick]
        {angle=Q--N--S};

    \end{tikzpicture}
  \end{minipage}
\end{blueBox}

\begin{blueBox}
  \sftitle{Sík egyenlete}:

  A sík tetszőleges $\rvec r$ pontjára igaz, hogy
  \begin{gather*}
    (\rvec r - \rvec r_0) \cdot \rvec n = 0
    \text,\\
    \rvec r \cdot \rvec n = \rvec r_0 \cdot \rvec n =: -D
    \text,\\
    Ax + By + Cz + D = 0
    \text.
  \end{gather*}
\end{blueBox}

\begin{blueBox}
  \sftitle{Hesse-féle normálegyenlet}:

  \[
    \left|\frac{
      Ax + By + Cz + D
    }{
      \sqrt{A^2 + B^2 + C^2}
    }\right| = 0
  \]
\end{blueBox}

\begin{blueBox}
  \sftitle{Két sík viszonya}:

  Két sík által bezárt szög a síkok normálvektorai által bezárt szög.
\end{blueBox}

\begin{blueBox}
  \sftitle{Sík és egyenes döféspontja}:

  Egy $s: Ax + By+ Cz + D = 0$ sík és egy $e: \rvec r(t) = \rvec r_0 + t \rvec
    v$ egyenes döféspontjait meghatározhatjuk, ha megoldjuk a következő
  egyenletrendszert:
  \[
    \begin{cases}
      \; x = x_0 + a t \\
      \; y = y_0 + b t \\
      \; z = z_0 + c t
    \end{cases}
    \; \text{ és} \qquad
    Ax + By + Cz + D = 0
    \text.
  \]

  A megoldások száma alapján:
  \begin{center}
    \begin{tikzpicture}[thick]
      % 0 solutions
      \draw[primaryColor, ultra thick, xshift=-5.25cm] (-.75, 1.35, 0) -- ++(4.5, 0, 0);

      %  1 solution
      \draw[primaryColor, ultra thick] (2.25, 0, 0.75) -- ++(-1, 2, 0)
      coordinate[pos=.5] (Q)
      ;
      \draw[fill=primaryColor] (Q) circle (0.075);

      % infinite solutions
      \draw[primaryColor, ultra thick, xshift=5.25cm] (-.75, 1, 0.75) -- ++(4.5, 0, 0);

      \foreach \xs/\sol in {-5.25/{$0$},0/{$1$},5.25/{$\infty$}} {
      \begin{scope}[xshift=\xs cm]
        % Coordinate system
        \coordinate (O) at (0,0,0);
        \draw[draw=ternaryColor, ->, thick] (O) -- (3,0,0) node[below left] {$x$};
        \draw[draw=ternaryColor, ->, thick] (O) -- (0,2,0) node[below left] {$y$};
        \draw[draw=ternaryColor, ->, thick] (O) -- (0,0,1.5) node[right=2mm] {$z$};

        % Plane at y = 1
        \draw[draw=secondaryColor, fill=secondaryColor, fill opacity=.35]
        (0, 1, -.5) -- (3, 1, -.5) -- (3, 1, 2) -- (0, 1, 2) -- cycle
        ;

        % Solutions
        \node at (1.5,-1) {\sol{} megoldás};
      \end{scope}
      }

    \end{tikzpicture}
  \end{center}
\end{blueBox}

\begin{blueBox}
  \sftitle{Sík és egyenes által bezárt szög}:

  Egy sík és egy egyenes által bezárt szög a sík normálvektora és az egyenes
  irányvektora által bezárt szöggel egyenlő.
\end{blueBox}

\begin{blueBox}
  \begin{minipage}{.6\textwidth}
    \sftitle{Sík és pont távolsága}:\\[3mm]
    Egy $P_0(x_0; y_0;z_0)$ pont és egy $s: Az + By + Cz + D = 0$ sík távolsága:
    \[
      d = \left|
      \frac{Ax_0 + By_0 + Cz_0 + D}{\sqrt{A^2 + B^2 + C^2}}
      \right|
      \text.
    \]
  \end{minipage}\begin{minipage}{.4\textwidth}
    \centering
    \begin{tikzpicture}[ultra thick]
      % Coordinate system
      \coordinate (O) at (0,0,0);
      \draw[draw=ternaryColor, ->, thick] (O) -- (3,0,0) node[below left] {$x$};
      \draw[draw=ternaryColor, ->, thick] (O) -- (0,2.5,0) node[below left] {$y$};
      \draw[draw=ternaryColor, ->, thick] (O) -- (0,0,1.5) node[right=2mm] {$z$};

      % Plane at y = 1
      \draw[draw=secondaryColor, fill=secondaryColor, fill opacity=.35, thick]
      (0, 1, -.5) -- (3, 1, -.5) -- (3, 1, 2) -- (0, 1, 2) -- cycle
      ;

      % Point
      \coordinate (P) at (1.5, 2.5, 0.75);
      \draw[fill=primaryColor] (P) circle (0.075);

      % Distance
      \draw[draw=primaryColor] (P)
      node[above right] {$P_0$}
      -- ++(0, -1.5, 0)
      node[midway, right] {$d$}
      coordinate (Q)
      ;

      % Right angle
      \draw[dashed, gray] (Q) -- ++(-1,0,0) coordinate (R);
      \draw pic["$\cdot$", draw, angle eccentricity=.5, angle radius=4mm, thick]
        {angle=P--Q--R};
    \end{tikzpicture}
  \end{minipage}
\end{blueBox}

\begin{blueBox}
  \sftitle{Két sík metszésvonala}:

  \begin{minipage}{.65\textwidth}
    A metszésvonal irányvektora mindkét sík normálvektorára merőleges:
    \[
      \rvec v = \rvec n_1 \times \rvec n_2
      \text.
    \]
    Ezen kívül szükségünk van még egy tetszőleges pontra, amely rajta van
    mindkét síkon. Ezt megkaphatjuk úgy, hogy az egyik koordinátát fixáljuk,
    és a másik kettőt kiszámítjuk a 2 sík egyenletéből. (Pl. $z = 0$)
  \end{minipage}\begin{minipage}{.35\textwidth}
    \centering
    \begin{tikzpicture}[ultra thick]
      % Coordinate system
      \coordinate (O) at (0,0,0);
      \draw[draw=ternaryColor, ->, thick] (O) -- (3,0,0) node[below left] {$x$};
      \draw[draw=ternaryColor, ->, thick] (O) -- (0,2.5,0) node[below left] {$y$};
      \draw[draw=ternaryColor, ->, thick] (O) -- (0,0,1.5) node[right=2mm] {$z$};

      \begin{scope}[transparency group, fill opacity=.35]
        % Plane at y = 1
        \fill[fill=secondaryColor]
        (0, 1.25, -.5) -- (3, 1.25, -.5) -- (3, 1.25, 2) -- (0, 1.25, 2) -- cycle
        ;

        % Plane at x = 1.5
        \fill[fill=secondaryColor]
        (1.5, 0, -.5) -- (1.5, 2.5, -.5) -- (1.5, 2.5, 2) -- (1.5, 0, 2) -- cycle
        ;
      \end{scope}


      % Plane at y = 1
      \draw[draw=secondaryColor, thick]
      (0, 1.25, -.5) -- (3, 1.25, -.5) -- (3, 1.25, 2) -- (0, 1.25, 2) -- cycle
      ;

      % Plane at x = 1.5
      \draw[draw=secondaryColor, thick]
      (1.5, 0, -.5) -- (1.5, 2.5, -.5) -- (1.5, 2.5, 2) -- (1.5, 0, 2) -- cycle
      ;

      % Line (Yes, not in 3D xd)
      \draw[draw=primaryColor] (0,-0.25) -- ++(45:3.5);
    \end{tikzpicture}
  \end{minipage}
\end{blueBox}

\clearpage
\subsection{Feladatok}

\begin{enumerate}
  % 1
  \item Számítsa ki az $e_1: 3x - 4y - 10 = 0$ és az $e_2: 6x - 8y + 5 = 0$
        egyenes távolságát!

        % 2
  \item Írja fel azon egyenesnek az egyenletét, amely átmegy a $P(-2;5;6)$ és a
        $Q(7; -1; 3)$ pontokon!

        % 3
  \item Határozza meg az $\alpha$ paramétert, ha ismert, hogy az alábbi
        egyenesek metszik egymást!
        \[
          \frac{x + 2}{2} = \frac{y    }{-3} = \frac{z - 1}{4}
          \hspace{2cm}
          \frac{x - 3}{a} = \frac{y - 1}{4 } = \frac{z - 7}{2}
        \]

        % 4
  \item Határozza meg az alábbi egyenesek távolságát!
        \[
          \begin{cases}
            \; x_1(t) =  2 + 3 t_1 \\
            \; y_1(t) = -1 + 4 t_1 \\
            \; z_1(t) =      2 t_1
          \end{cases}
          \hspace{2cm}
          \begin{cases}
            \; x_2(t) = 7 + 6 t_2 \\
            \; y_2(t) = 1 + 8 t_2 \\
            \; z_2(t) = 3 + 4 t_2
          \end{cases}
        \]

  \item Adott két egyenes. Határozza meg a távolságukat és normáltraszverzálisuk
        egyenletrendszerét!
        \[
          \begin{cases}
            \; x_1(t) = -7 + 3 t \\
            \; y_1(t) =  4 - 2 t \\
            \; z_1(t) =  4 + 3 t
          \end{cases}
          \hspace{2cm}
          \begin{cases}
            \; x_2(t) =   1 +   t \\
            \; y_2(t) =  -8 + 2 t \\
            \; z_2(t) = -12 -   t
          \end{cases}
        \]

        % 6
  \item Vizsgálja meg, hogy a $P(0;-1;2)$, a $Q(2;-1;1)$ és az $R(4;3;-2)$
        pontok egy egyenesbe esnek-e! Ha nem, akkor írja fel az általuk
        kifeszített sík egyenletét!

        % 7
  \item Határozza meg az $\alpha$ paramétert, ha ismert, hogy az $e$ egyenes és
        az $s$ sík párhuzamos egymással!
        \[
          e: \frac{x - 1}{2} = \frac{y - 2}{2} = \frac{z}{\alpha}
          \hspace{2cm}
          s: x + 3y - 2\alpha z = 0
        \]

        % 8
  \item Számítsa ki az $e$ egyenes és az $s$ sík metszéspontját!
        \[
          e: x - 1 = \frac{y + 1}{-2} = \frac{z}{6}
          \hspace{2cm}
          s: 2x + 3y + z - 1 = 0
        \]

        % 9
  \item Adja meg az $s_1$ és $s_2$ síkok metszésvonalának egyenletrendszerét!
        \[
          s_1: x - 2y + 3z - 4 = 0
          \hspace{2cm}
          s_2: 3x + 2y - 5z - 4 = 0
        \]

        % 10
  \item Igazolja, hogy az alábbi három síknak egy közös pontja van. Írja fel
        ezen a ponton átmenő síkot, amely párhuzamos az $x + y + 2z = 0$ síkkal!
        \[
          \begin{cases}
            \; s_1: 2x + y - z - 2 = 0 \\
            \; s_2: x - 3y + z + 1 = 0 \\
            \; s_3: x + y + z - 3 = 0
          \end{cases}
        \]
\end{enumerate}

% \\underline\{(\w)\}

\end{document}