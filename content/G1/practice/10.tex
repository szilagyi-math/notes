\documentclass[a4paper, 12pt]{scrartcl}

\usepackage{math-practice}

\title{Integrálszámítás I}
\area{Kalkulus}
\subject{Matematika G1}
\subjectCode{BMETE94BG01}
\date{Utoljára frissítve: \today}
\docno{10}

\begin{document}
\maketitle

\subsection{Elméleti Áttekintő}

\begin{blueBox}
  \sftitle{Alapintegrálok}:

  \centering
  \newcommand{\cs}{\rule{0pt}{6mm}}
  \newcommand{\dint}{\displaystyle\int}
  \newcommand{\dmint}{\displaystyle\int^{\mathstrut}}
  \newcommand{\fr}{%
    \hline
    f(x) & F(x) \\
    \hline
  }
  \newenvironment{intTabular}{%
    \begin{tabular}[b]{| *{2}{>{\centering\arraybackslash$}p{2cm}<{$}} |}
      }{%
      \hline
    \end{tabular}
  }
  \def\arraystretch{1.4}
  \begin{intTabular}
    \multicolumn{2}{c}{$\dmint f(x) \dd x = F(x) + C$}              \\[6mm]
    %
    \fr
    %
    k                                       & kx                                 \\[1mm]
    x^{\alpha}                              & \dfrac{x^{\alpha + 1}}{\alpha + 1} \\[3mm]
    %
    \hline
    % %
    \cs
    \dfrac{1^{\mathstrut}}{x}               & \ln |x|                            \\[3mm]
    e^x                                     & e^x                                \\
    a^x                                     & \dfrac{a^x}{\ln a}                 \\[3mm]
    %
  \end{intTabular}
  \hfill
  \begin{intTabular}
    %
    \fr
    %
    \sin x                                  & - \cos x                           \\
    \cos x                                  & \sin x                             \\
    \dfrac{1}{\cos^2 x}                     & \tan x                             \\[3mm]
    \dfrac{1}{\sin^2 x}                     & - \cot x                           \\[3mm]
    %
    \hline
    %
    \cs
    \dfrac{1^{\mathstrut}}{\sqrt{1 - x^2}}  & \arcsin x                          \\[3mm]
    \dfrac{-1}{\sqrt{1 - x^2}}              & \arccos x                          \\[3mm]
    \dfrac{1}{1 + x^2}                      & \arctan x                          \\[3mm]
  \end{intTabular}
  \hfill
  \begin{intTabular}
    %
    \fr
    %
    \sinh x                                 & \cosh x                            \\
    \cosh x                                 & \sinh x                            \\
    \dfrac{1}{\cosh^2 x}                    & \tanh x                            \\[3mm]
    \dfrac{1}{\sinh^2 x}                    & -\coth x                           \\[3mm]
    %
    \hline
    %
    \cs
    \dfrac{1^{\mathstrut}}{\sqrt{x^2 + 1}}  & \arcsinh x                         \\[3mm]
    \dfrac{1}{\sqrt{x^2 - 1}}               & \arccosh x                         \\[3mm]
    \dfrac{1}{1 - x^2}                      & \arctanh x                         \\[3mm]
  \end{intTabular}
\end{blueBox}

\begin{blueBox}
  \sftitle{Integrálás tulajdonságai}:

  \begin{tabular}{l<{:} >{$\displaystyle}l<{$}}
    Linearitás                            & \int (\lambda f + \mu g) = \lambda \int f + \mu \int g \\[3mm]
    Lineáris az integrációs intervallumra & \int_a^b f = \int_a^c f + \int_c^b f                   \\
  \end{tabular}
\end{blueBox}

\begin{blueBox}
  \sftitle{Speciális helyettesítéses módszerek}:

  \begin{itemize}
    \item $\displaystyle \int f^\alpha \cdot f' = \frac{f^{\alpha + 1}}
            {\alpha + 1}$, ahol $\alpha \neq -1$,
    \item $\displaystyle \int \frac{f'}{f} = \ln |f| + C$,
    \item $\displaystyle \int e^f \cdot f' = e^f + C$,
    \item $\displaystyle \int (f \circ g) \cdot g' \dd x = \left( \int f \right)
            \circ g$.
  \end{itemize}
\end{blueBox}

\clearpage
\subsection{Feladatok}

\begin{enumerate}
  \item Végezze el az $f(x) = x \cdot e^{\sfrac{-1}{x}}$ függvény vizsgálatát!

  \item Végezze el a $g(x) = \sin^2 x - 2 \sin x$ függvény vizsgálatát!

  \item Végezze el az alábbi határozatlan integrálok számítását!
        \begin{enumerate}
          \item $\displaystyle
                  \int x^2 (x^2 - 1) + 2 \sqrt{x \sqrt{x \sqrt{x}}} \dd x
                $

          \item $\displaystyle
                  \int \frac{x^2 - 4x + 7}{x - 2} \dd x
                $

          \item $\displaystyle
                  \int \tan^2 x \dd x
                $

          \item $\displaystyle
                  \int \frac{\ln 2}{\sqrt{2 + 2x^2}} + \frac{e^{3x} + 1}{e^x + 1} \dd x
                $
        \end{enumerate}

  \item Határozza meg az alábbi integrálok értékét a helyettesítéses módszer
        segítségével!
        \begin{enumerate}
          \item $\displaystyle
                  \int (x^3 + 2x) \cdot \cos\left(x^4 + 4x^2\right) \dd x
                $

          \item $\displaystyle
                  \int \frac{\sqrt[3]{\tan x}}{\cos^2 x} \dd x
                $

          \item $\displaystyle
                  \int \frac{x}{x^4 + 1} \dd x
                $

          \item $\displaystyle
                  \int \sin^3(2x + 1) \cdot \cos(2x + 1) \dd x
                $

          \item $\displaystyle
                  \int \frac{x}{4 + x^2} \dd x
                $

          \item $\displaystyle
                  \int \frac{1}{\sin x \cdot \cos x} \dd x
                $

          \item $\displaystyle
                  \int \frac{1}{\tan x} \dd x
                $

          \item $\displaystyle
                  \int \frac{1}{x \cdot \ln x} \dd x
                $

          \item $\displaystyle
                  \int \frac{2x + 1}{x^2 + 2} \dd x
                $
        \end{enumerate}
\end{enumerate}

% \\underline\{(\w)\}

\end{document}
