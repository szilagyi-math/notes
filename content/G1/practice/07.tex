\documentclass[a4paper, 12pt]{scrartcl}

\usepackage{math-practice}

\title{Differenciálás I}
\area{Kalkulus}
\subject{Matematika G1}
\subjectCode{BMETE94BG01}
\date{Utoljára frissítve: \today}
\docno{7}

\begin{document}
\maketitle

\subsection{Elméleti Áttekintő}

\begin{definition}[Differenciálhányados]
  Ha létezik és véges a
  \[
    \lim_{x \rightarrow a} \frac{f(x) - f(a)}{x - a}
  \]
  határérték, akkor azt az $f$ függvény $a$ pontbeli differenciálhányadosának,
  vagy az $a$ pontbeli deriváltjának mondjuk.

  Jelölése:
  \[
    f'(a)
    \quad \text{ vagy } \quad
    \odv{f(a)}{x}
    \text.
  \]
\end{definition}

\begin{note}
  A differenciálhányados létezésének \textbf{szükséges feltétel}e, hogy az $f$
  függvény \textbf{folytonos} legyen az $a$ pontban.
\end{note}

\begin{blueBox}
  \sftitle{Fontosabb függvények és deriváltjaik}:

  \begin{minipage}{.3\textwidth}
    \def\arraystretch{1.5}\centering
    \begin{tabular}{c c}
      $f(x)$   & $f'(x)$                                                   \\ \toprule
      $e^x$    & $e^x$                                                     \\[2mm]
      $\ln x $ & $\dfrac{1}{x} \vphantom{\dfrac{x^{\sfrac{1}{n} - 1}}{n}}$ \\ \bottomrule
    \end{tabular}
  \end{minipage}\hfill\begin{minipage}{.3\textwidth}
    \def\arraystretch{1.5}\centering
    \begin{tabular}{c c}
      $f(x)$     & $f'(x)$                                                         \\ \toprule
      $a^x$      & $a^x \ln a$                                                     \\[2mm]
      $\log_a x$ & $\dfrac{1}{x \ln a} \vphantom{\dfrac{x^{\sfrac{1}{n} - 1}}{n}}$ \\ \bottomrule
    \end{tabular}
  \end{minipage}\hfill\begin{minipage}{.3\textwidth}
    \def\arraystretch{1.5}\centering
    \begin{tabular}{c c}
      $f(x)$        & $f'(x)$                           \\ \toprule
      $x^n$         & $n \cdot x^{n - 1}$               \\[2mm]
      $\sqrt[n]{x}$ & $\dfrac{x^{\sfrac{1}{n} - 1}}{n}$ \\ \bottomrule
    \end{tabular}
  \end{minipage}
\end{blueBox}

\begin{note}
  A konstans függvény deriváltja zérus.
\end{note}

\begin{blueBox}
  \sftitle{Trigonometrikus függvények és deriváltjaik}:

  \begin{minipage}{0.45\textwidth}
    \begin{center}
      \begin{tikzpicture}[ultra thick, scale=.85]
        \draw[gray, thick] (-2.75,0)--(3,0);
        \draw[gray, thick] (0,-2.75)--(0,3);

        \draw  (0,0) circle [radius=2.5];

        \draw[primaryColor] (0,0)--(40:4);

        % sin
        \draw[draw=secondaryColor] (40:2.5) -- (40:2.5 |- 0,0)
        coordinate (t)
        node[left, pos=.7] {$\sin x$}
        ;

        % cos
        \draw[draw=secondaryColor] (t) -- (0,0)
        node[below, midway] {$\cos x$}
        ;

        % tan
        \draw[draw=secondaryColor] (0:2.5) -- ++(0,2.08) coordinate (t)
        node[right, midway] {$\tan x$}
        ;

        % cot
        \draw[draw=secondaryColor] (90:2.5) -- ++(3,0) coordinate (t)
        node[above, midway] {$\cot x$}
        ;
      \end{tikzpicture}
    \end{center}
  \end{minipage}%
  \begin{minipage}{0.275\textwidth}
    \def\arraystretch{1.5}
    \begin{tabular}{c | c}
      $f(x)$   & $f'(x)$                                      \\ \toprule
      $\sin x$ & $\cos x \vphantom{\dfrac{1}{\sqrt{1-x^2}}}$  \\ [2mm]
      $\cos x$ & $-\sin x \vphantom{\dfrac{1}{\sqrt{1-x^2}}}$ \\ [2mm]
      $\tan x$ & $\dfrac{1}{\cos^2 x}$                        \\ [2mm]
      $\cot x$ & $-\dfrac{1}{\sin^2 x}$                       \\ \bottomrule
    \end{tabular}
  \end{minipage}%
  \begin{minipage}{0.275\textwidth}
    \def\arraystretch{1.5}
    \begin{tabular}{c | c}
      $f(x)$      & $f'(x)$                               \\ \toprule
      $\arcsin x$ & $\dfrac{1}{\sqrt{1-x^2}}$             \\ [2mm]
      $\arccos x$ & $-\dfrac{1}{\sqrt{1-x^2}}$            \\ [2mm]
      $\arctan x$ & $\dfrac{1}{1+x^2 \vphantom{\cos^2}}$  \\ [2mm]
      $\arccot x$ & $-\dfrac{1}{1+x^2 \vphantom{\sin^2}}$ \\ \bottomrule
    \end{tabular}
  \end{minipage}
\end{blueBox}

\begin{blueBox}
  \sftitle{Hiperbolikus függvények és deriváltjaik}:

  \begin{minipage}{0.35\textwidth}
    \begin{center}
      \begin{tikzpicture}[ultra thick, scale=.85]
        \draw[gray, thick] (-2.75,0)--(2.75,0);
        \draw[gray, thick] (0,-2.75)--(0,2.75);

        % unit hyperbola helper
        \draw[gray, dashed, thick] (2.5, 2.5) -- (-2.5, -2.5);
        \draw[gray, dashed, thick] (2.5, -2.5) -- (-2.5, 2.5);

        % \draw[domain=0.275:3.5, rotate=-45, samples=75, scale=1/1.141] plot (\x,{1/\x});
        % \draw[domain=0.275:3.5, rotate=135, samples=75] plot (\x,{1/\x});

        \draw[domain=0:1.5] plot ({cosh(\x)}, {sinh(\x)});
        \draw[domain=0:1.5] plot ({cosh(\x)}, {-sinh(\x)});

        \draw[domain=0:1.5] plot ({-cosh(\x)}, {sinh(\x)});
        \draw[domain=0:1.5] plot ({-cosh(\x)}, {-sinh(\x)});

        % cosh(1) = 1.543
        % sinh(1) = 1.175

        \draw[draw=secondaryColor] (0, 1.175)
        -- (1.543, 1.175) node [midway, above] {$\cosh x$}
        -- (1.543, 0) node [midway, right] {$\sinh x$}
        ;
      \end{tikzpicture}
    \end{center}
  \end{minipage}%
  \begin{minipage}{0.25\textwidth}
    \def\arraystretch{1.5}
    \begin{tabular}{c | c}
      $f(x)$    & $f'(x)$                                      \\ \toprule
      $\sinh x$ & $\cosh x \vphantom{\dfrac{1}{\sqrt{x^2-1}}}$ \\ [2mm]
      $\cosh x$ & $\sinh x \vphantom{\dfrac{1}{\sqrt{x^2-1}}}$ \\ [2mm]
      $\tanh x$ & $\dfrac{1}{\cosh^2 x}$                       \\ [2mm]
      $\coth x$ & $-\dfrac{1}{\sinh^2 x}$                      \\ \bottomrule
    \end{tabular}
  \end{minipage}%
  \begin{minipage}{0.4\textwidth}
    \def\arraystretch{1.5}
    \begin{tabular}{c | l}
      $f(x)$       & $\;\;\;f'(x)$                                             \\ \toprule
      $\arcsinh x$ & $\dfrac{1}{\sqrt{x^2+1}}$                                 \\ [2mm]
      $\arccosh x$ & $\dfrac{1}{\sqrt{x^2-1}} \;\;\;\, (x > 1)$                \\ [2mm]
      $\arctanh x$ & $\;\;\dfrac{1}{1-x^2 \vphantom{\cosh^2}} \quad (|x| < 1)$ \\ [2mm]
      $\arccoth x$ & $\;\;\dfrac{1}{1-x^2 \vphantom{\sinh^2}} \quad (|x| > 1)$ \\ \bottomrule
    \end{tabular}
  \end{minipage}
\end{blueBox}

\begin{blueBox}
  \sftitle{Hiperbolikus azonosságok}:
  \begin{alignat*}{9}
    \sinh x   & = \frac{e^x - e^{-x}}{2}
    \hspace{2cm}
              & \cosh x                  & = \frac{e^x + e^{-x}}{2}
    \\[2mm]
    \sinh x   & = - \iu \sin(\iu x)
    \hspace{2cm}
              & \cosh x                  & = \cos(\iu x)
    \\[2mm]
    \sinh^2 x & = \frac{\cosh 2x - 1}{2}
    \hspace{2cm}
              & \cosh^2 x                & = \frac{\cosh 2x + 1}{2}
    \\[2mm]
    \sinh 2x  & = 2 \sinh x \cosh x
    \hspace{2cm}
              & \cosh 2x                 & = \cosh^2 x + \sinh^2 x
    \\[2mm]
    \multicolumn{4}{c}{$1 = \cosh^2 x - \sinh^2 x$}
  \end{alignat*}
\end{blueBox}

\begin{blueBox}
  \sftitle{Műveleti szabályok}:

  \centering
  \setlength{\tabcolsep}{0mm}
  \begin{tabular}{>{\bullet\hskip1ex\bfseries\sffamily}l<{:} >{\hskip1.5em$\displaystyle}c<{$\hskip1.5em} >{$\displaystyle}c<{$ }}
    Konstans kiemelhető                                 &
    (cf)' = cf'                                         &
    \odv{}{x}(cf)) = c \odv{f}{x}
    \\[5mm]
    Összeg- és különbségfüggvény                        &
    (f \pm g)' = f' \pm g'                              &
    \odv{}{x}(f \pm g) = \odv{f}{x} \pm \odv{g}{x}
    \\[5mm]
    Szorzatfüggvény                                     &
    (fg)' = f'g + fg'                                   &
    \odv{}{x}(fg) = \odv{f}{x}g + f\odv{g}{x}
    \\[5mm]
    Hányadosfüggvény                                    &
    \left( \frac{f}{g} \right)' = \frac{f'g - fg'}{g^2} &
    \odv{}{x}\left( \frac{f}{g} \right) = \frac{\displaystyle\odv{f}{x}g - f\odv{g}{x}}{g^2}
  \end{tabular}
\end{blueBox}

\begin{blueBox}
  \sftitle{Láncszabály}:

  A láncszabály segítségével összetett függvényeket tudunk differenciálni.
  Az összefüggést három különböző jelölésmóddal is felírhatjuk:
  \[
    (f(g(x)))' = f'(g(x)) \cdot g'(x)
    \quad \text{vagy} \quad
    (f \circ g)' = (f' \circ g) \cdot g'
    \quad \text{vagy} \quad
    \odv{f(g)}{x} = \odv{f(g)}{g} \cdot \odv{g}{x}
  \]
\end{blueBox}

% \begin{note}
%   Határozzuk meg az $\ln \sin x$ függvény deriváltját a láncszabály
%   segítségével!

%   Jelen esetben a külső függvény $f(x) = \ln x$, a belső függvény pedig
%   $g(x) = \sin x$. Ezek deriváljai: $f'(x) = 1/x$ és $g'(x) = \cos x$.
%   Ezek alapján az összetett függvény deriváltja:
%   \[
%     (\ln \sin x)' = \frac{1}{\sin x} \cdot \cos x
%     = \frac{\cos x}{\sin x}
%     = \cot x
%     \text.
%   \]
% \end{note}

\clearpage

\begin{blueBox}
  \sftitle{Elemi átalakításos módszer}:

  Előfordulhat olyan eset, hogy $(f(x))^{g(x)}$ alakú függvényeket kell
  differenciálni. Ebben az esetben az alábbi átalakítást alkalmazzuk:
  \[
    (f(x))^{g(x)} = e^{\ln (f(x))^{g(x)}} = e^{g(x) \ln f(x)}
    \text.
  \]

  Az $e^{g(x) \ln f(x)}$ függvény deriváltja a láncszabály segítségével:
  \begin{align*}
    \left((f(x))^{g(x)}\right)'
     & =\left(e^{g(x) \ln f(x)}\right)'                                            \\
     & = e^{g(x) \ln f(x)} \left( g'(x) \ln f(x) + g(x) \frac{f'(x)}{f(x)} \right) \\
     & = (f(x))^{g(x)} \left( g'(x) \ln f(x) + g(x) \frac{f'(x)}{f(x)} \right)
    \text.
  \end{align*}
\end{blueBox}

\begin{note}
  Határozzuk meg az $\ln^x x$ függvény deriváltját!

  Az $\ln^x x$ függvényt $e^{x \ln \ln x}$ alakra hozva, a láncszabály
  segítségével differenciálható:
  \begin{align*}
    \left(\ln^x x\right)'
     & = \left(e^{x \ln \ln x}\right)'                                                        \\
     & = e^{x \ln \ln x} \left( \ln \ln x + x \cdot \frac{1}{\ln x} \cdot \frac{1}{x} \right) \\
     & = \ln^x x \left( \ln \ln x + \frac{1}{\ln x} \right)
    \text.
  \end{align*}
\end{note}

\begin{blueBox}
  \sftitle{Geometriai alkalmazás}:

  Az $f$ függvény $a$ pontbeli \textbf{érintő}jének \textbf{egyenlete} onnan
  következik, hogy $f' = m$, ahol $m$ a meredekséget jelöli, az $y = m \cdot x
    + b$ egyenes egyenletéből levezetve:
  \[
    f(a) = f'(a) \cdot a + b
    \quad \rightarrow \quad
    b = f(a) - f'(a) \cdot a
    \text,
  \]
  és mivel
  \begin{gather*}
    (a; f(a)) \in y = m \cdot x + b
    \\
    \downarrow
    \\
    y = f'(a) \cdot x + f(a) - f'(a) \cdot a
  \end{gather*}
  Ebből átalakítva:
  \[
    \boxed{
      y = f'(a)\cdot (x-a) + f(a)
      \text.
    }
  \]
  Az $(a; f(a))$ pontbeli \textbf{normális egyenlete}:
  \begin{gather*}
    M = \frac{-1}{f'(a)}
    \text, \quad \text{és} \quad
    (a; f(a)) \in y = M \cdot x + B
    \text,
    \\
    \boxed{
      y = \frac{-1}{f'(a)} \cdot (x-a) + f(a)
      \text.
    }
  \end{gather*}
\end{blueBox}

\clearpage
\subsection{Feladatok}

\begin{enumerate}
  \item A differenciálhányados definíciója segítségével határozza meg az
        $f(x) = x^n$ függvény deriváltját az $x = x_0$ pontban!

  \item Differenciálhatóak-e az alábbi függvények az $x_0 = 0$ pontban?
        \begin{multicols}{2}
          \begin{enumerate}
            \item $\displaystyle
                    f(x) = \begin{cases}
                      \sin^2 x\text, & \text{ha } x \leq 0 \\
                      x^2 \text,     & \text{ha } x > 0    \\
                    \end{cases}
                  $

            \item $\displaystyle
                    f(x) = \begin{cases}
                      \arctan x \text,   & \text{ha } x > 0 \\
                      0 \text,           & \text{ha } x = 0 \\
                      x^3 + x + 1 \text, & \text{ha } x < 0 \\
                    \end{cases}
                  $
          \end{enumerate}
        \end{multicols}

  \item Adjon példát olyan függvényekre, melyek $\forall x \in \Reals$ valós
        számra értelmezve vannak és teljesül, hogy \dots
        \begin{enumerate}
          \item $f$ mindenhol folytonos, de az $x_0 = 1$ pontban nem
                differenciálható,
          \item $f$ mindenhol differenciálható, de az $x_0 = 1$ pontban nem
                folytonos,
          \item $f$ mindenhol differenciálható és $f'$ is folytonos,
          \item $f$ mindenhol differenciálható, de $f'$ az $x_0 = 0$ pontban nem
                az.
        \end{enumerate}

  \item Mutassa meg, hogy az alábbi függvényre igaz, hogy bár differenciálható
        az $x_0 = 0$ pontban, viszont létezik az $x_0$ tetszőlegesen kis
        környezetében olyan pont, ahol nem differenciálható.
        \[
          f(x) = \begin{cases}
            x^2 \left| \sin \sfrac1x \right|\text, & \text{ha } x \neq 0 \\
            0\text,                                & \text{ha } x = 0
          \end{cases}
        \]

  \item Differenciálja az alábbi függvényeket!
        \begin{enumerate}
          \item $\displaystyle
                  f(x) = (6x^7 + 7x^4 + 2x^2)^5 + \sin^2 x + \cos^2 x
                $

          \item $\displaystyle
                  g(x) = \ln x \cdot e^x + x^2 \cot x + x^{-1/3}
                $

          \item $\displaystyle
                  h(x) = \frac{
                    (3x + x^2) \cdot \sinh x \cdot \arctan x
                  }{
                    (1 + \cos x) \cdot \arctanh \pi
                  }
                $

          \item $\displaystyle
                  i(x) = \sqrt[3]{\ln \cos^2 x^4}
                $

          \item $\displaystyle
                  j(x) = \ln \arcsin \sqrt{\frac{x^2 + 3}{e^{2x}}}
                $

          \item $\displaystyle
                  k(x) = x^x
                $

          \item $\displaystyle
                  l(x) = (x^2 + 1)^{\sin x}
                $
        \end{enumerate}

  \item Határozza meg az alábbi függvények $n$-edik deriváltját!
        \begin{multicols}{2}
          \begin{enumerate}
            \item $f(x) = x^m$
            \item $g(x) = \sin x$
          \end{enumerate}
        \end{multicols}

  \item Írja fel az $f(x) = 2x^3 + 3 \sqrt{x} - \sfrac{3}{2x^2}$ függvény
        $x_0 = 1$ pontban lévő érintő egyenesének egyenletét! Adja meg az
        érintőre merőleges egyenes egyenletét is!

  \item Határozza meg azon pontok halmazát, melyekben az $x^2 + y^2 = 25$ kör
        érintője párhuzamos a $3x - 4y + 7 = 0$ egyenessel.
\end{enumerate}

% \\underline\{(\w)\}

\end{document}