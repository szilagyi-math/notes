\documentclass[a4paper, 12pt]{scrartcl}

\usepackage{math-practice}

\title{Differenciálás I}
\area{Kalkulus}
\subject{Matematika G1}
\subjectCode{BMETE94BG01}
\date{Utoljára frissítve: \today}
\docno{7}

\begin{document}
\maketitle

\subsection{Elméleti Áttekintő}

\begin{definition}[Differenciálhányados]
  Ha létezik és véges a
  \[
    \lim_{x \rightarrow a} \frac{f(x) - f(a)}{x - a}
  \]
  határérték, akkor azt az $f$ függvény $a$ pontbeli differenciálhányadosának,
  vagy az $a$ pontbeli deriváltjának mondjuk.

  Jelölése:
  \[
    f'(a)
    \quad \text{ vagy } \quad
    \odv{f(a)}{x}
    \text.
  \]
\end{definition}

\begin{note}
  A differenciálhányados létezésének \textbf{szükséges feltétel}e, hogy az $f$
  függvény \textbf{folytonos} legyen az $a$ pontban.
\end{note}

\begin{blueBox}
  \sftitle{Geometriai alkalmazás}:

  Az $f$ függvény $a$ pontbeli \textbf{érintő}jének \textbf{egyenlete} onnan
  következik, hogy $f' = m$, ahol $m$ a meredekséget jelöli, az $y = m \cdot x
    + b$ egyenes egyenletéből levezetve:
  \[
    f(a) = f'(a) \cdot a + b
    \quad \rightarrow \quad
    b = f(a) - f'(a) \cdot a
    \text,
  \]
  és mivel
  \begin{gather*}
    (a; f(a)) \in y = m \cdot x + b
    \\
    \downarrow
    \\
    y = f'(a) \cdot x + f(a) - f'(a) \cdot a
  \end{gather*}
  Ebből átalakítva:
  \[
    \boxed{
      y = f'(a)\cdot (x-a) + f(a)
      \text.
    }
  \]
  Az $(a; f(a))$ pontbeli \textbf{normális egyenlete}:
  \begin{gather*}
    M = \frac{-1}{f'(a)}
    \text, \quad \text{és} \quad
    (a; f(a)) \in Y = M \cdot x + B
    \text,
    \\
    \boxed{
      y = \frac{-1}{f'(a)} \cdot (x-a) + f(a)
      \text.
    }
  \end{gather*}
\end{blueBox}


\clearpage
\subsection{Feladatok}

\begin{enumerate}
  \item A differenciálhányados definíciója segítségével határozza meg az
        $f(x) = x^n$ függvény deriváltját az $x = x_0$ pontban!

  \item Differenciálhatóak-e az alábbi függvények az $x_0 = 0$ pontban?
        \begin{multicols}{2}
          \begin{enumerate}
            \item $\displaystyle
                    f(x) = \begin{cases}
                      \sin^2 x\text, & \text{ha } x \leq 0 \\
                      x^2 \text,     & \text{ha } x > 0    \\
                    \end{cases}
                  $

            \item $\displaystyle
                    f(x) = \begin{cases}
                      \arctan x \text,   & \text{ha } x > 0 \\
                      0 \text,           & \text{ha } x = 0 \\
                      x^3 + x + 1 \text, & \text{ha } x < 0 \\
                    \end{cases}
                  $
          \end{enumerate}
        \end{multicols}

  \item Adjon példát olyan függvényekre, melyek $\forall x \in \Reals$ valós
        számra értelmezve vannak és teljesül, hogy \dots
        \begin{enumerate}
          \item $f$ mindenhol folytonos, de az $x_0 = 1$ pontban nem
                differenciálható,
          \item $f$ mindenhol differenciálható, de az $x_0 = 1$ pontban nem
                folytonos,
          \item $f$ mindenhol differenciálható és $f'$ is folytonos,
          \item $f$ mindenhol differenciálható, de $f'$ az $x_0 = 0$ pontban nem
                az.
        \end{enumerate}

  \item Mutassa meg, hogy az alábbi függvényre igaz, hogy bár differenciálható
        az $x_0 = 0$ pontban, viszont létezik az $x_0$ tetszőlegesen kis
        környezetében olyan pont, ahol nem differenciálható.
        \[
          f(x) = \begin{cases}
            x^2 \left| \sin \sfrac1x \right|\text, & \text{ha } x \neq 0 \\
            0\text,                                & \text{ha } x = 0
          \end{cases}
        \]

  \item Differenciálja az alábbi függvényeket!
        \begin{enumerate}
          \item $\displaystyle
                  f(x) = (6x^7 + 7x^4 + 2x^2)^5 + \sin^2 x + \cos^2 x
                $

          \item $\displaystyle
                  g(x) = \ln x \cdot e^x + x^2 \cot x + x^{-1/3}
                $

          \item $\displaystyle
                  h(x) = \frac{
                    (3x + x^2) \cdot \sinh x \cdot \arctan x
                  }{
                    (1 + \cos x) \cdot \arctanh \pi
                  }
                $

          \item $\displaystyle
                  i(x) = \sqrt[3]{\ln \cos^2 x^4}
                $

          \item $\displaystyle
                  j(x) = \ln \arcsin \sqrt{\frac{x^2 + 3}{e^{2x}}}
                $

          \item $\displaystyle
                  k(x) = x^x
                $

          \item $\displaystyle
                  l(x) = (x^2 + 1)^{\sin x}
                $
        \end{enumerate}

  \item Határozza meg az alábbi függvények $n$-edik deriváltját!
        \begin{multicols}{2}
          \begin{enumerate}
            \item $f(x) = x^m$
            \item $g(x) = \sin x$
          \end{enumerate}
        \end{multicols}

  \item Írja fel az $f(x) = 2x^3 + 3 \sqrt{x} - \sfrac{3}{2x^2}$ függvény
        $x_0 = 1$ pontban lévő érintő egyenesének egyenletét! Adja meg az
        érintőre merőleges egyenes egyenletét is!

  \item Határozza meg azon pontok halmazát, melyekben az $x^2 + y^2 = 25$ kör
        érintője párhuzamos a $3x - 4y + 7 = 0$ egyenessel.
\end{enumerate}

% \\underline\{(\w)\}

\end{document}