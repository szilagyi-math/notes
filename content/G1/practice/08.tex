\documentclass[a4paper, 12pt]{scrartcl}

\usepackage{math-practice}

\title{Differenciálás II}
\area{Kalkulus}
\subject{Matematika G1}
\subjectCode{BMETE94BG01}
\date{Utoljára frissítve: \today}
\docno{8}

\begin{document}
\maketitle

\subsection{Elméleti Áttekintő}

\begin{blueBox}
  \sftitle{Implicit függvények differenciálása}:

  A korábbiakban $y = f(x)$ alakú függvényeket vizsgáltunk. Az ilyen függvények
  az $x \mapsto f(x)$ hozzárendelés alapján egyértelműen megadják, hogy az egyes
  ősképekhez ($x \in \Domain_f$) milyen értékek tartoznak ($y \in \Range_f$).

  Előfordulhat azonban olyan eset, amikor nehéz, vagy éppen lehetetlen
  \textbf{explicit alakban} megadni egy görbét. Ennek a problémának a
  feloldására vezessük be az \textbf{implicit függvények} fogalmát. Az ilyen
  függvények esetén az ősképek és képek közötti kapcsolatot az $F(x; y) = 0$
  egyenlettel adhatjuk meg.

  Ilyen függvények differenciálásakor mindig az összetett függvények deriválási
  szabályait kell alkalmazni:
  \[
    \left[ g(y) \right]' = g'(y) \cdot y'
  \]
\end{blueBox}

\begin{note}
  Implicit függvény deriváltjai a parciális deriváltak segítségével is
  meghatározhatóak. Parciális deriválás során az $F(x;y)$ függvényre úgy
  tekintük, mintha az $x$ és $y$ változói függetlenek lennének egymástól. Az
  $x$ szerinti parciális derivált esetén $y$-t, az $y$ szerinti parciális
  derivált esetén pedig $x$-et konstansként kezeljük.

  Az $x$ szertini derivált:
  \[
    \frac{\partial F}{\partial x} + \frac{\partial F}{\partial y} \cdot y' = 0
    \quad \rightarrow \quad
    y'
    = \odv{y}{x}
    = -\frac{\partial F / \partial x}{\partial F / \partial y}
    = -\frac{F'_x}{F'_y}
    \text.
  \]
  Az $y$ szerinti derivált pedig:
  \[
    \frac{\partial F}{\partial x} \cdot x' + \frac{\partial F}{\partial y} = 0
    \quad \rightarrow \quad
    x'
    = \odv{x}{y}
    = -\frac{\partial F / \partial y}{\partial F / \partial x}
    = -\frac{F'_y}{F'_x}
  \]
  % A második deriváltak:
  % \[
  %   y''
  %   = \odv{}{x} \left( -\frac{\partial F / \partial x}{\partial F / \partial y} \right)
  %   = - \frac{
  %     \odv{}{x} \left( \pdv{F}{X} \right) \cdot \pdv{F}{y}
  %     - \pdv{F}{x} \cdot \odv{}{x} \left( \pdv{F}{y} \right)
  %   }{
  %     \pdv[2]{F}{y}
  %   } = \frac{
  %     F''_x \cdot F'_y - F''_xy \cdot F'x
  %   }{
  %     F''y
  %   }
  % \]
  % A láncszabály alapján:
  % \[
  %   \odv{}{x} \left( \pdv{F}{x} \right)
  %   = \pdv[2]{F}{x} + \pdv{F}{x,y} \cdot y'
  %   \quad \text{és} \quad
  %   \odv{}{x} \left( \pdv{F}{y} \right)
  %   = \pdv{F}{x,y} + \pdv{F}{y,y} \cdot y'
  % \]
\end{note}

\begin{note}
  Az $(f(x))^{g(x)}$ típusú függvényeket az implicit függvény deriválási
  szabályai szerint is differenciálhatjuk.
  \begin{gather*}
    y = f(x)^{g(x)}
    \quad \rightarrow \quad
    \ln y = g(x) \ln f(x)
    \quad \rightarrow \quad
    \frac{y'}{y} = g'(x) \ln f(x) + \frac{g(x) f'(x)}{f(x)}
    \\
    \Downarrow
    \\
    y' = f(x)^{g(x)} \left( g'(x) \ln f(x) + \frac{g(x) f'(x)}{f(x)} \right)
  \end{gather*}
\end{note}

\begin{note}
  Határozzuk meg az $\ln^x x$ függvény deriváltját!
  \begin{gather*}
    y = \ln^x x
    \quad \rightarrow \quad
    \ln y = x \ln \ln x
    \quad \rightarrow \quad
    \frac{y'}{y} = 1 \cdot \ln \ln x + x \cdot \frac{1}{\ln x} \cdot \frac{1}{x}
    \\
    \Downarrow
    \\
    y' = \ln^x x \left( \ln \ln x + \frac{1}{\ln x} \right)
  \end{gather*}

  A feladat az előző gyakorlaton tanult módszerrel is megoldható:
  \[
    y = e^{x \ln \ln x}
    \quad \rightarrow \quad
    y' = e^{x \ln \ln x} \left( 1 \cdot \ln \ln x + x \cdot \frac{1}{\ln x} \cdot \frac{1}{x} \right)
    = \ln^x x \left( \ln \ln x + \frac{1}{\ln x} \right)
    \text.
  \]
\end{note}

\begin{blueBox}
  \sftitle{Inverz függvény differenciálása}:

  Függvény invertálása során a függvény görbéjét tükrözzük az $y = x$ egyenesre.
  Jele: $f^{-1}(x)$. Amennyiben az eredeti függvény differenciálható az $x_0$
  pontban, és $f'(x_0) \neq 0$, akkor az inverz függvény deriváltja az
  $y_0 = f(x_0)$ pontban:
  \[
    \left. \odv{f^{-1}(x)}{x} \right|_{f(x_0)} = \frac{1}{f'(x_0)}
  \]
\end{blueBox}

\begin{note}
  Az inverz függvény létezésének szükséges feltétele, hogy az eredeti függvény
  bijektív legyen.
\end{note}

\begin{blueBox}
  \sftitle{Paraméteresen megadott függvények differenciálása}:

  Paraméteresen megadott függvények esetén egy paraméterünk ($t$), viszont kettő
  egyenletünk ($x(t)$ és $y(t)$) van. Az $x$ szerinti derivált:
  \[
    \odv{y}{x}
    = \odv{y}{t} \cdot \odv{t}{x}
    = \frac{\sfrac{\dd y}{\dd t}}{\sfrac{\dd x}{\dd t}}
    = \frac{\dot y}{\dot x}
    \text.
  \]
  Az $y$ szerinti derivált pedig ennek a reciproka:
  \[
    \odv{x}{y}
    = \odv{x}{t} \cdot \odv{t}{y}
    = \frac{\sfrac{\dd x}{\dd t}}{\sfrac{\dd y}{\dd t}}
    = \frac{\dot x}{\dot y}
    \text.
  \]
  A másosik deriváltak:
  \[
    \odv[2]{y}{x}
    = \odv{y'}{x}
    = \odv{y'}{t} \cdot \odv{t}{x}
    = \frac{\dot{(y')}}{\dot x}
    = \frac{\odv{}{t}\left(\odv{y}{x}\right)}{\dot x}
    = \frac{\odv{}{t}\left(\frac{\dot y}{\dot x}\right)}{\dot x}
    = \frac{\ddot y \dot x - \dot y \ddot x}{\dot x^3}
  \]
  \[
    \odv[2]{x}{y}
    = \odv{x'}{y}
    = \odv{x'}{t} \cdot \odv{t}{y}
    = \frac{\dot{(x')}}{\dot y}
    = \frac{\odv{}{t}\left(\odv{x}{y}\right)}{\dot y}
    = \frac{\odv{}{t}\left(\frac{\dot x}{\dot y}\right)}{\dot y}
    = \frac{\ddot x \dot y - \dot x \ddot y}{\dot y^3}
  \]
\end{blueBox}

\begin{theorem}[L'Hôpital-szabály]
  Legyenek $f$ és $g$ differenciálhatóak az $\alpha \in \Reals_b$ pont egy
  környezetében ($\alpha$-ban nem szükségképpen), továbbá $g(x) \neq 0$ és
  $g'(x) \neq 0$ és
  \[
    \lim\limits_{x \rightarrow \alpha} f(x) = \lim\limits_{x \rightarrow \alpha} g(x)=0
    \text{, vagy}
    \lim\limits_{x \rightarrow \alpha} |f(x)|=\lim\limits_{x \rightarrow \alpha}|g(x)|=\infty
    \text,
  \]
  \[
    \text{ekkor, ha}
    \lim\limits_{x \rightarrow \alpha} \frac{f'(x)}{g'(x)}=B
    \text{, akkor }
    \lim\limits_{x \rightarrow \alpha} \frac{f(x)}{g(x)}=B
    \text.
  \]
\end{theorem}

\begin{theorem}[Rolle-tétel]
  Legyen $f$ folytonos $[a; b]$ intervallumon és differenciálható $(a; b)$
  intervallumon, továbbá $f(a) = f(b) = 0$, ekkor létezik $ \xi \in (a; b)$,
  melyre teljesül, hogy
  \[
    f'(\xi) = 0
    \text.
  \]
\end{theorem}

\begin{theorem}[Lagrange-féle középértéktétel]
  Legyen $f : I \subset R \to R$ folytonos $[a; b]$ intervallumon és
  differenciálható $(a; b)$ intervallumon, ekkor létezik olyan $\delta \in
    (a; b)$ hogy
  \[
    f'(\delta) = \frac{f(b)-f(a)}{b-a}
    \text.
  \]
\end{theorem}

\begin{theorem}[Cauchy-féle középértéktétel]
  Legyen $f$ és $g$ függvények folytonosak $[a; b]$ intervallumon és
  differenciálhatóak $(a; b)$ intervallumon, valamint tegyük fel, hogy $g'(x)
    \neq 0$ bármely $x \in (a; b)$ esetén. Ekkor létezik olyan $\eta \in
    (a; b)$
  hogy
  \[
    \frac{f(b)-f(a)}{g(b)-g(a)} = \frac{f'(\eta)}{g'(\eta)}
    \text.
  \]
\end{theorem}

\clearpage
\subsection{Feladatok}

\begin{enumerate}
  \item Határozza meg az alábbi függvény deriváltjait!
        ($y' = \dd y / \dd x$ és $x' = \dd x / \dd y$)
        \[
          F(x; y) = x^4 y + 5 y^2 x- 4 = 0
        \]

  \item Határozza meg az alábbi függvény első és második deriváltjait, valamint
        az érintőjének egyenletét a $P(1;1)$ pontban!
        \[
          \ln y + xy = 1
        \]

  \item Határozza meg az $x^2 + y^2 = 25$ kör azon pontjait, amelyekben a kör
        érintőjének meredeksége $3/4$!

  \item Írja fel az $f(x) = 5 x^3 + x - 7$ függvény inverzét, és annak
        deriváltját! Adja meg ennek értékét az $f(x_0)$ pontban, ha $x_0 = 1$!

  \item Határozza meg az alábbi paraméteresen megadott függvény $x$ szerinti
        első és második deriváltját. Mekkora lesz az érintő meredeksége a
        $t = \pi / 6$-hoz tartozó pontban?
        \[
          \begin{cases}
            x(t) = e^t \cos t \\
            y(t) = e^t \sin t
          \end{cases}
        \]

  \item Határozza meg az alábbi paraméteresen megadott kör azon pontjait,
        ahol az érintő meredeksége $3/4$!
        \[
          \begin{cases}
            x(t) = 5 \cos t \\
            y(t) = 5 \sin t
          \end{cases}
        \]

  \item Határozza meg az alábbi határértékeket a L'Hôpital szabály segítségével!
        \begin{multicols}{2}
          \begin{enumerate}
            \item $\displaystyle
                    \lim_{x \rightarrow 3} = \frac{3^x - x^3}{x - 3}
                  $

            \item $\displaystyle
                    \lim_{x \rightarrow 3^{+}} \frac{\ln(x - 3)}{\ln(e^x - e^3)}
                  $

            \item $\displaystyle
                    \lim_{x \rightarrow 0} x \ln x
                  $

            \item $\displaystyle
                    \lim_{x \rightarrow 0} x^x
                  $

            \item $\displaystyle
                    \lim_{x \rightarrow 0} \left(
                    \frac{\cos x}{x} - \frac{1}{\sin x}
                    \right)
                  $
          \end{enumerate}
        \end{multicols}

  \item Vizsgálja meg, hogy alkalmazható-e a L'Hôpital szabály az alábbi
        határértékek kiszámítására! Ha igen, alkalmazza, ha nem, indokolja meg!
        \[
          \lim_{x \rightarrow 0} \frac{x^2 \sin(\sfrac{1}{x})}{\sin x}
        \]

  \item A Rolle-féle középértéktétel segítségével bizonyítsa be, hogy az
        $f(x) = 3x^5 + 15x - 2$ függvénynek egy valós gyöke van!

  \item Határozza meg az alábbi függvény lokális szélsőértékeit!
        \[
          f(x) = \frac{x^3}{x^2 - x - 2}
        \]
\end{enumerate}

% \\underline\{(\w)\}

\end{document}