\documentclass[a4paper, 12pt]{scrartcl}

\usepackage{math-practice}

\area{Analitikus geometria}
\title{Vektorok}
\subject{Matematika G1}
\subjectCode{BMETE94BG01}
\date{Utoljára frissítve: \today}
\docno{1}

\begin{document}
\maketitle

\subsection{Elméleti Áttekintő}

\begin{definition}[Vektor]
  Egy $(v_1, v_2, v_3)$ valós számokból alló rendezett számhármast a térben
  $(\Reals^3)$ vektornak nevezünk. Jelölése: $\rvec v$ (nyomtatott szöveg),
  $\underline v \; / \; \vec v$ (kézzel írott szöveg).
\end{definition}

\begin{note}
  A vektorok geometriai értelemben olyan irányított szakaszok, melyeknek
  hossza és iránya van.
\end{note}

\begin{blueBox}
  \begin{minipage}{.45\textwidth}
    \sftitle{Vektorok megadása}:\\[3mm]
    Egy tetszőleges $\rvec v \; (v_1; v_2; v_3)$ vektor a standard normális
    bázisban
    \[
      \rvec v = v_1 \ibase + v_2 \jbase + v_3 \kbase
      \text.
    \]
  \end{minipage}\begin{minipage}{.25\textwidth}
    \begin{center}
      \begin{tikzpicture}[z={(210:1)},x={(-30:1)}, scale=0.4]
        % 3d axis
        \draw[->, draw=ternaryColor, thick]
        (0,0,0) -- (4,0,0) node[anchor=south west]{$x$};
        \draw[->, draw=ternaryColor, thick]
        (0,0,0) -- (0,4,0) node[right=2mm]        {$y$};
        \draw[->, draw=ternaryColor, thick]
        (0,0,0) -- (0,0,4) node[anchor=south east]{$z$};

        % unit vectors
        \draw[->, draw=primaryColor, ultra thick]
        (0,0,0) -- (2,0,0) node[anchor=south west]{$\ibase$};
        \draw[->, draw=primaryColor, ultra thick]
        (0,0,0) -- (0,2,0) node[right=2mm]        {$\jbase$};
        \draw[->, draw=primaryColor, ultra thick]
        (0,0,0) -- (0,0,2) node[anchor=south east]{$\kbase$};
      \end{tikzpicture}
    \end{center}
  \end{minipage}\begin{minipage}{.30\textwidth}
    \begin{align*}
      \ibase & = (1; 0; 0) \\
      \jbase & = (0; 1; 0) \\
      \kbase & = (0; 0; 1)
    \end{align*}
  \end{minipage}
\end{blueBox}

\begin{blueBox}
  \sftitle{Vektorok típusai}:

  \begin{itemize}
    \item \textbf{kötött vektor}: fix kezdőponttal rendelkezik,
    \item \textbf{szabad vektor}: nincs fix kezdőpontja,
    \item \textbf{helyvektor}: olyan kötött vektor, amelynek kezdőpontja az
          origó.
  \end{itemize}
\end{blueBox}

\begin{blueBox}
  \sftitle{Vektor hossza}:
  \[
    |\rvec v| = \sqrt{v_1^2 + v_2^2 + v_3^2}
    \text.
  \]

  \begin{itemize}
    \item Ha $|\rvec v| = 0$, akkor $\rvec v$ \textbf{nullvektor}.
          (Jele: $\nvec$)
    \item Ha $|\rvec v| = 1$, akkor $\rvec v$ \textbf{egységvektor}.
  \end{itemize}
\end{blueBox}

\begin{note}
  A nullvektor iránya nem definiált.
\end{note}

\begin{blueBox}
  \sftitle{Egy adott $\rvec v$ vektorhoz tartozó egységvektor}:
  \[
    \uvec e_v
    = \frac{\rvec v}{|\rvec v|}
    = \begin{pmatrix}
      \dfrac{v_1}{|\rvec v|} &
      \dfrac{v_2}{|\rvec v|} &
      \dfrac{v_3}{|\rvec v|}
    \end{pmatrix}
  \]
\end{blueBox}

\begin{blueBox}
  \begin{minipage}{.6\textwidth}
    \sftitle{Háromszög-egyenlőtlenség}:\\[3mm]
    Minden $\rvec u$, $\rvec v$ vektorpárra igaz, hogy
    \[
      |\rvec u + \rvec v| \leq |\rvec u| + |\rvec v|
      \text.
    \]
  \end{minipage}\begin{minipage}{.4\textwidth}
    \centering
    \begin{tikzpicture}[ultra thick]
      % Coordinate system
      \coordinate (O) at (0,0,0);
      \draw[draw=ternaryColor, ->, thick] (O) -- (3,0,0) node[below left] {$x$};
      \draw[draw=ternaryColor, ->, thick] (O) -- (0,1.5,0) node[below left] {$y$};
      \draw[draw=ternaryColor, ->, thick] (O) -- (0,0,1.5) node[right=2mm] {$z$};

      % Vectors
      \draw[-to, draw=secondaryColor]
      (O) -- (2,0,1)
      coordinate (U)
      node[midway, below]{$\rvec u$}
      ;
      \draw[-to, draw=secondaryColor]
      (U) -- ++ (1.25,1.5,0)
      coordinate (V)
      node[midway, right]{$\rvec v$}
      ;
      \draw[-to, draw=primaryColor]
      (O) -- (V)
      node[midway, above, rotate=20]{$\rvec u + \rvec v$}
      ;
    \end{tikzpicture}
  \end{minipage}
\end{blueBox}

\begin{blueBox}
  \begin{minipage}{.6\textwidth}
    \sftitle{Paralelogramma-szabály}:\\[3mm]
    Ha az $\rvec u$ és $\rvec v$ vektor különböző állású, akkor a két vektor
    összegét megadja az $\rvec u$ és $\rvec v$ vektorokkal, mint oldalakkal
    szerkesztett paralelogrammának azon átlója, amely a közös pontból indul.
  \end{minipage}\begin{minipage}{.4\textwidth}
    \centering
    \begin{tikzpicture}[ultra thick]
      % Coordinate system
      \coordinate (O) at (0,0,0);
      \draw[draw=ternaryColor, ->, thick] (O) -- (3,0,0) node[below left] {$x$};
      \draw[draw=ternaryColor, ->, thick] (O) -- (0,1.5,0) node[below left] {$y$};
      \draw[draw=ternaryColor, ->, thick] (O) -- (0,0,1.5) node[right=2mm] {$z$};


      % Vectors
      \draw[-to, draw=secondaryColor]
      (O) -- (2,0,1)
      coordinate (U)
      node[midway, below]{$\rvec u$}
      ;
      \draw[-to, draw=secondaryColor]
      (O) -- (1.25,1.5,0)
      coordinate (V)
      node[midway, left]{$\rvec v$}
      ;

      % Helper lines
      \coordinate (X) at ($(U)+(V)$);
      \draw[dashed, gray, thick] (U) -- (X) -- (V);

      % Sum
      \draw[-to, draw=primaryColor]
      (O) -- (X)
      node[pos=.5, above, rotate=20]{$\rvec u + \rvec v$}
      ;
    \end{tikzpicture}
  \end{minipage}
\end{blueBox}

\begin{blueBox}
  \begin{minipage}{.65\textwidth}
    \sftitle{Vektor koordinátatengelyekkel bezárt szöge}:
    \vspace{5mm}
    \begin{equation*}
      \cos \varphi_x = \frac{v_1}{|\rvec v|}
      \quad
      \cos \varphi_y = \frac{v_2}{|\rvec v|}
      \quad
      \cos \varphi_z = \frac{v_3}{|\rvec v|}
    \end{equation*}
  \end{minipage}\begin{minipage}{.35\textwidth}
    \centering
    \begin{tikzpicture}[scale=.5, thick]
      % 3d axis
      \draw[->, draw=ternaryColor, thick]
      (0,0,0) -- (6,0,0) node[anchor=south west]{$x$};
      \draw[->, draw=ternaryColor, thick]
      (0,0,0) -- (0,4,0) node[right=2mm]        {$y$};
      \draw[->, draw=ternaryColor, thick]
      (0,0,0) -- (0,0,4) node[anchor=south east]{$z$};

      \def\xlen{4} % 42.03111377
      \def\ylen{3} % 56.14548519
      \def\zlen{2} % 68.19859051

      % vector
      \draw[->, draw=primaryColor, ultra thick]
      (0,0,0) -- (4,3,2)
      coordinate (V)
      ;

      % % bounding box
      % \draw[dashed, draw=gray]
      % % 
      % (V) -- (\xlen,\ylen,0) coordinate (Vxy)
      % (V) -- (\xlen,0,\zlen) coordinate (Vxz)
      % (V) -- (0,\ylen,\zlen) coordinate (Vyz)
      % % 
      % (\xlen,0,0) coordinate (Vx) -- (Vxy) --
      % (0,\ylen,0) coordinate (Vy) -- (Vyz) --
      % (0,0,\zlen) coordinate (Vz) -- (Vxz) --
      % cycle
      % ;

      \def\ecc{2}

      % x angle
      \begin{scope}[
          rotate around x=33.69006753,
          canvas is xy plane at z=0,
          draw=secondaryColor,
        ]
        \draw (\ecc,0) arc (0:42.03111377:\ecc)
        node[midway, right]{$\varphi_x$}
        ;
      \end{scope}

      % y angle
      \begin{scope}[
          rotate around y=63.43494882,
          canvas is yz plane at x=0,
          draw=secondaryColor,
        ]
        \draw (\ecc,0) arc (0:56.14548519:\ecc)
        node[midway, above right]{$\varphi_y$}
        ;
      \end{scope}

      % z angle
      \begin{scope}[
          rotate around z=-53.13010235,
          canvas is zy plane at x=0,
          draw=secondaryColor,
        ]
        \draw (\ecc,0) arc (0:68.19859051:\ecc)
        node[midway, below right]{$\varphi_z$}
        ;
      \end{scope}
    \end{tikzpicture}
  \end{minipage}
\end{blueBox}

\begin{blueBox}
  \sftitle{Kollinearitás}:

  Az $\rvec u$ és $\rvec v$ kollineárisak, ha $\rvec v$ előáll $\rvec u$ és egy
  $\lambda \in \Reals$ szorzataként. Amennyiben $\lambda > 0$, akkor a két
  vektor azonos irányú.
\end{blueBox}

\begin{blueBox}
  \sftitle{Komplanaritás}:

  Tetszőleges számú vektor komplanáris, ha azok egy síkban helyezkednek el.
\end{blueBox}

\begin{definition}[Lineáris függetlenség]
  Egy $\{ \rvec v_1; \rvec v_2; \dots; \rvec v_n \}$ vektorrendszer lineárisan
  független, ha a $ \lambda_1 \rvec v_1 + \lambda_2 \rvec v_2 + \dots + \lambda_n
    \rvec v_n = \nvec $ egyenletnek csak a triviális megoldása létezik. (Azaz
  $\lambda_1 = \lambda_2 = \dots = \lambda_n = 0$.)
\end{definition}

\begin{note}
  \begin{itemize}
    \item A nullvektor minden vektorral lineárisan függő.
    \item Két vektor akkor lineárisan független, ha nem kollineáris.
    \item Ha két vektor nem kollineáris, akkor egyértelműen meghatároznak egy
          síkot, azaz bármely velük koplanáris vektor előállítható a két vektor
          lineáris kombinációjaként.
    \item 3D koordinátarendszerben 3-nál több vektor biztos, hogy lineárisan
          összefüggő. (Feltéve, hogy nincs köztük nullvektor.)
    \item 3 vektor lineárisan független ha nem koplanáris. (3D-ben)
  \end{itemize}
\end{note}

\begin{blueBox}
  \begin{minipage}{.5\textwidth}
    \sftitle{Vektorok összege és különbsége}:
    \begin{align*}
      \rvec u + \rvec v & = (u_1 + v_1; u_2 + v_2; u_3 + v_3) \\
      \rvec u - \rvec v & = (u_1 - v_1; u_2 - v_2; u_3 - v_3)
    \end{align*}
    \begin{itemize}
      \item \textbf{Kommutatív}:
            $\rvec u + \rvec v = \rvec v + \rvec u$

      \item \textbf{Asszociatív}:
            $(\rvec u + \rvec v) + \rvec w = \rvec u + (\rvec v + \rvec w)$
    \end{itemize}
  \end{minipage}\begin{minipage}{.5\textwidth}
    \begin{center}
      \begin{tikzpicture}[scale=2/3, ultra thick]
        % vectors
        \draw[draw=ternaryColor, ->]
        (0,0) -- (2,3) node[midway,anchor=east]{$\rvec u$};
        \draw[draw=ternaryColor, ->]
        (0,0) -- (5,1) node[midway,anchor=north]{$\rvec v$};

        % helper lines
        \draw[dashed, draw=gray, thick]
        (2,3) -- (7,4)
        (5,1) -- (7,4)
        ;

        % sum
        \draw[primaryColor, ->]
        (0,0) -- (7,4) node[anchor=south]{$\rvec u + \rvec v$};

        % difference
        \draw[secondaryColor, ->]
        (5,1) -- (2,3) node[anchor=south]{$\rvec u - \rvec v$};
      \end{tikzpicture}
    \end{center}
  \end{minipage}
\end{blueBox}

\begin{blueBox}
  \textbf{Skalárral való szorzás}:

  Skalárral való szorzás esetén a vektor ($\rvec v$) minden koordinátáját
  megszorozzuk a $\lambda \in \Reals$ skalárral, vagyis:
  \[
    \rvec u = \lambda \rvec v = (\lambda v_1; \lambda v_2; \lambda v_3)
    \text.
  \]
  \textbf{A skalárral való szorzás eredménye egy vektor}, melynek hossza az
  eredeti vektor hosszának skalárszorosa.
\end{blueBox}

\begin{blueBox}
  \sftitle{Vektorok skaláris szorzata}: (Scalar / Dot product)

  Az $\rvec u$ és $\rvec v$ vektorok skaláris szorzata:
  \[
    \rvec u \cdot \rvec v = u_1 v_1 + u_2 v_2 + u_3 v_3
    \text.
  \]
  \textbf{Két vektor skaláris szorzatának eredménye egy skalár.}
\end{blueBox}

\begin{note}
  \sftitle{A skaláris szorzat tulajdonságai}:
  \begin{itemize}
    \item $\rvec u \cdot \rvec v = \rvec v \cdot \rvec u$
          (kommutatív)

    \item $\rvec u \cdot (\rvec v + \rvec w) =
            \rvec u \cdot \rvec v + \rvec u \cdot \rvec w$
          (disztributív)

    \item $\rvec u \cdot \rvec u = |\rvec u|^2$

    \item $\rvec u \cdot \nvec = 0$

    \item $(\lambda \rvec u) \cdot \rvec v = \lambda (\rvec u \cdot \rvec v)$
  \end{itemize}
\end{note}

\begin{note}
  \begin{minipage}{.6\textwidth}
    \sftitle{A skaláris szorzat geometriai jelentése}:\\[3mm]
    A skaláris szorzás segítségével kiszámítható az $\rvec u$ és $\rvec v$
    vektorok közötti szög.
    \[
      \rvec u \cdot \rvec v = |\rvec u| |\rvec v| \cos \varphi
    \]
  \end{minipage}\begin{minipage}{.4\textwidth}
    \centering
    \begin{tikzpicture}[scale=1/2, ultra thick]
      % vectors
      \draw[draw=primaryColor, ->]
      (0,0) -- (2,3)
      node[midway,anchor=east]{$\rvec u$}
      coordinate (7)
      ;
      \draw[draw=primaryColor, ->]
      (0,0) -- (5,1)
      node[midway,anchor=north]{$\rvec v$}
      coordinate (4)
      ;

      \coordinate (0) at (0,0);

      % Angle
      \draw pic["$\varphi$", draw=ternaryColor, angle eccentricity=0.65, angle radius=1.1cm]
          {angle=4--0--7};
    \end{tikzpicture}
  \end{minipage}
\end{note}

\begin{blueBox}
  % \sftitle{Vetület}:

  Az $\rvec u$ vektor $\rvec v$ vektorra vett párhuzamos és merőleges
  komponense:
  \[
    \rvec u_{||} = (\rvec u \cdot \uvec e_v) \, \uvec e_v
    \quad \text{és} \quad
    \rvec u_{\perp} = \rvec u - \rvec u_{||}
    \text,
  \]
  ahol $\uvec e_v$ a $\rvec v$ irányába mutató egységvektor.
\end{blueBox}

\begin{blueBox}
  \sftitle{Vektoriális szorzat / keresztszorzat} (Cross product):

  Az $\rvec u$ és $\rvec v$ vektorok keresztszorzata:
  \[
    \rvec u \times \rvec v
    = \begin{bmatrix}
      u_1 \\ u_2 \\ u_3
    \end{bmatrix} \times \begin{bmatrix}
      v_1 \\ v_2 \\ v_3
    \end{bmatrix}
    = \left|\begin{matrix}
      \ibase & \jbase & \kbase \\
      u_1    & u_2    & u_3    \\
      v_1    & v_2    & v_3
    \end{matrix}\right|
    = \begin{bmatrix}
      u_2 v_3 - u_3 v_2 \\
      u_3 v_1 - u_1 v_3 \\
      u_1 v_2 - u_2 v_1
    \end{bmatrix}
    \text.
  \]

  \textbf{Két vektor keresztszorzatának eredménye egy vektor}, amely merőleges
  mindkét vektorra, iránya pedig a jobbkéz szabály szerinti.
\end{blueBox}

\begin{note}
  \sftitle{A keresztszorzat tulajdonságai}:
  \begin{itemize}
    \item $\rvec v \times \rvec v = \nvec$

    \item $\rvec u \times \rvec v = -\rvec v \times \rvec u$
          (antikommutatív)

    \item $\rvec u \times (\rvec v + \rvec w)
            = \rvec u \times \rvec v + \rvec u \times \rvec w$
          (disztributív)

    \item $\rvec u \times (\lambda \rvec v) = \lambda (\rvec u \times \rvec v)$

    \item $\rvec u \times \rvec v = \nvec$ akkor és csak akkor, ha $\rvec u$ és
          $\rvec v$ kollineárisak, vagy ha valamelyikük nullvektor.
  \end{itemize}
\end{note}

\begin{note}
  \begin{minipage}{.6\textwidth}
    \sftitle{A keresztszorzat geometriai jelentése}:\\[1mm]
    Az $\rvec u \times \rvec v$ vektor hossza megegyezik az $\rvec u$ és $\rvec v$
    vektorok által kifeszített paralelogramma területével.

    \[
      |\rvec u \times \rvec v| = |\rvec u| |\rvec v| \sin \varphi
    \]
  \end{minipage}\begin{minipage}{.4\textwidth}
    \centering
    \begin{tikzpicture}[scale=1/2, ultra thick]
      % vectors
      \draw[draw=primaryColor, ->]
      (0,0) -- (2,3) node[midway,anchor=east]{$\rvec u$};
      \draw[draw=primaryColor, ->]
      (0,0) -- (5,1) node[midway,anchor=north]{$\rvec v$};

      % helper lines
      \draw[dashed, draw=gray, thick]
      (2,3) -- (7,4)
      (5,1) -- (7,4)
      ;

      \begin{pgfonlayer}{bg}
        \fill [primaryColor!20] (0,0) -- (2,3) -- (7,4) -- (5,1) -- cycle;
      \end{pgfonlayer}
    \end{tikzpicture}
  \end{minipage}
\end{note}

\begin{blueBox}
  \sftitle{Vegyesszorzat}:

  Az $\rvec u$, $\rvec v$ és $\rvec w$ vektorok vegyes szorzata:
  \[
    \rvec u \rvec v \rvec w
    = \rvec u \cdot (\rvec v \times \rvec w)
  \]
  \textbf{A vegyesszorzat eredménye egy skalár}.
\end{blueBox}

\begin{note}
  \sftitle{A vegyesszorzat tulajdonságai}:
  \begin{itemize}
    \item $\rvec u \rvec v \rvec w
            = \rvec w \rvec u \rvec v
            = \rvec v \rvec w \rvec u
            = -\rvec v \rvec u \rvec w
            = -\rvec w \rvec v \rvec u
            = -\rvec u \rvec w \rvec v$
          (ciklikus csere)

    \item lineáris mindhárom változójában:
          $(\lambda \rvec u + \mu \rvec v) \rvec w \rvec z
            = \lambda \rvec u \rvec w \rvec z + \mu \rvec v \rvec w \rvec z$

    \item Ha $\rvec u$, $\rvec v$ és $\rvec w$ vektorok egy síkban helyezkednek
          el, akkor vegyesszorzatuk nulla.
  \end{itemize}
\end{note}

\begin{note}
  \sftitle{A vegyesszorzat geometriai jelentése}:

  3 vektor vegyesszorzata megadja az általuk kifeszített paralelepipedon
  térfogatát, illetve az általuk kifeszített tetraéder térfogatának hatszorosát.
\end{note}

\clearpage
\subsection{Feladatok}

\begin{enumerate}
  % 1
  \item Legyen $\rvec u$ és $\rvec v$ két tetszőleges vektor. Milyen $\alpha$ és
        $\beta$ paraméterek esetén lesznek kollineárisak, ha az $\{ \rvec a;
          \rvec b; \rvec c \}$ vektorrendszer lineárisan független?
        \begin{multicols}{2}
          \begin{enumerate}
            \item $
                    \begin{cases}
                      \rvec u = 2 \rvec a + 3 \rvec b \\
                      \rvec v = 4 \rvec a + \alpha \rvec b
                    \end{cases}
                  $

            \item $
                    \begin{cases}
                      \rvec u = 3 \rvec a - 3 \alpha \rvec b + \beta \rvec c \\
                      \rvec v = \rvec a - \alpha \rvec b - \rvec c
                    \end{cases}
                  $
          \end{enumerate}
        \end{multicols}

        % 2
  \item Legyen az $\{ \rvec a; \rvec b; \rvec c \}$ vektorrendszer lineárisan
        független. Lineárisan független lesz-e az $\{ \rvec r; \rvec s;
          \rvec t \}$ vektorrendszer?
        \begin{multicols}{2}
          \begin{enumerate}
            \item $
                    \begin{cases}
                      \rvec r = 3 \rvec a + 2 \rvec b +   \rvec c \\
                      \rvec s = 5 \rvec a - 3 \rvec b - 2 \rvec c \\
                      \rvec t = \nvec
                    \end{cases}
                  $

            \item $
                    \begin{cases}
                      \rvec r = \rvec a + \rvec b + \rvec c \\
                      \rvec s =           \rvec b + \rvec c \\
                      \rvec t = \rvec a           + \rvec c
                    \end{cases}
                  $
          \end{enumerate}
        \end{multicols}

        % 3
  \item Legyen $\rvec a$, $\rvec b$ és $\rvec c$ közös középpontú komplanáris
        vektorok. ($\rvec a$ és $\rvec b$ nem kollineáris) Bizonyítsa be,
        hogy az $\rvec a, \rvec b, \rvec c$ vektorok végpontja akkor és csakis
        akkor esik egy egyenesre, ha $\rvec c = \alpha \rvec a + \beta \rvec b$
        előállításban $\alpha + \beta = 1$.

        % 4
  \item Számítsa ki az $\rvec a (7; -1; 6)$ és $\rvec b (2; 20; 2)$
        vektorok által bezárt szöget!

        % 5
  \item Milyen $z$ esetén lesz a $\rvec b(6; -2; z)$ vektor merőleges az
        $\rvec a(2; -3; 1)$ vektorra?

        % 6
  \item Ha az $\rvec a + 3 \rvec b$ vektor merőleges a $7 \rvec a - 5 \rvec b$
        vektorra, az $\rvec a - 4 \rvec b$ vektor pedig merőleges a $7 \rvec a -
          2 \rvec b$ vektorra, mekkora $\rvec a$ és $\rvec b$ bezárt szögének
        koszinusza?

        % 7
  \item Az $ABCD$ téglalap ismert csúcsainak koordinátái: $A(2; 6; 0)$, $B(1; 2;
          3)$, $C(-2; 8; z)$. Mennyi $z$ értéke? Hol van $D$ pont?

        % 8
  \item Számítsa ki az $\rvec a \times \rvec b$ keresztszorzatot, amennyiben
        $\rvec a (-4; 2; 1)$ és $\rvec b (-2; 7; 8)$.

        % 9
  \item Hozza egyszerűb alakra a $(3 \rvec a - \rvec b) \times (\rvec b +
          3 \rvec a)$ kifejezést!

        % 10
  \item Kollineárisak-e az $\rvec a (-3; 4; 7)$ és $\rvec b (2; 5; 1)$ vektorok?

        % 11
  \item Mekkora az $ABC$ háromszög területe, ha csúcsai: $A(1; 0; 2)$,
        $B(-1; 4; 7)$ és $C(5; -2; 1)$?

        % 12
  \item Igaz-e, hogy ha $\rvec a \times \rvec c = \rvec b \times \rvec c$, akkor
        $\rvec a = \rvec b$?

        % 13
  \item Lehet-e az $\rvec a(6; 2; -3)$ és $\rvec b(-3; 6; -2)$ vektor egy kocka
        egy csúcsából induló élvektorok? Ha igen, határozzuk meg a harmadik élt!

        % 14
  \item Lineárisan független-e az $\rvec a(2; 3; -1)$, $\rvec b(1; -1; 3)$
        és $\rvec c(1; 9; -11)$ vektor?

        % 15
  \item Az $\rvec a$, $\rvec b$ és $\rvec c$ vektorok által kifeszített
        paralelepipedon térfogata $V$. Mekkora az $\rvec r = 2 \rvec a + 3
          \rvec b + 4 \rvec c$, $\rvec s = \rvec a - \rvec b + \rvec c$ és
        $\rvec t = 2 \rvec a + 4 \rvec b - \rvec c$ vektorok által kifeszített
        paralelepipedon térfogata?

        % 16
  \item Milyen $\alpha$ paraméter esetén lesz az $\{ \rvec a, \rvec b, \rvec c \}$
        vektorrendszer lineárisan függő, illetve lineárisan független, ha
        $\rvec a (3; \alpha; 0)$, $\rvec b (0; 3; \alpha)$ és $\rvec c
          (1; 0; -1)$?

        % 17
  \item Határozza meg $\rvec a(-1; 2; 1)$ vektor $\rvec b(1; 2; 2)$ vektorra
        vett vetületét!
\end{enumerate}

% \\underline\{(\w)\}

\end{document}