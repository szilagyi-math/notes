\documentclass[a4paper, 12pt]{scrartcl}

\usepackage{math-practice}

\title{Integrálszámítás II}
\area{Integrálszámítás}
\subject{Matematika G1}
\subjectCode{BMETE94BG01}
\date{Utoljára frissítve: \today}
\docno{11}

\begin{document}
\maketitle

\subsection{Elméleti Áttekintő}

\begin{blueBox}
  A \textbf{parciális integrálás} módszerének bevezetéséhez írjuk fel két
  függvény szorzatának deriváltját:
  \[
    (f(x) \cdot g(x))' = f'(x) \cdot g(x) + f(x) \cdot g'(x)
    \text.
  \]
  Integráljuk $x$ szerint az egyenlet mindkét oldalát:
  \[
    \int (f(x) \cdot g(x))' \dd x = \int f'(x) \cdot g(x) \dd x +
    \int f(x) \cdot g'(x) \dd x
    \text.
  \]
  Az integrálás és a deriválás műveletei egymás inverzei, így az egyenlet bal
  oldala az alábbi alakot ölti:
  \[
    f(x) \cdot g(x) = \int f'(x) \cdot g(x) \dd x + \int f(x) \cdot g'(x) \dd x
  \]
  Rendezzük át az egyenletet:
  \[
    \int f(x) \cdot g'(x) \dd x = f(x) g(x) - \int f'(x) \cdot g(x) \dd x
    \text.
  \]
  Amennyiben bevezetjük az $f(x) = u$, $g(x) = v$, $\dd u = u \dd x$, $\dd v = v
    \dd x$ jelöléseket, akkor megkapjuk a parciális integrálás egy másik gyakran
  használt alakját:
  \[
    \int u \dd v = u v - \int v \dd u
    \text.
  \]
\end{blueBox}

\begin{note}
  A parciális integrálás módszerét az alábbi esetekben érdemes alkalmazni:
  \begin{itemize}
    \item polinom és trigonometrikus/exponenciális függvény szorzatának
          integrálása:
          \[
            \int x \sin x \dd x
            \text,\qquad
            \int x \cos x \dd x
            \text,\qquad
            \int x \, e^x \dd x
            \text,
          \]
    \item exponenciális és trigonometrikus függvények szorzatának integrálása:
          \[
            \int e^x \sin x \dd x
            \text,\qquad
            \int e^x \cos x \dd x
            \text,
          \]
    \item logaritmus függvények integrálása,
    \item egyéb esetek, ahol egy szorzatfüggvényt kell integrálni.
  \end{itemize}
\end{note}

\begin{blueBox}
  Egy \textbf{racionális törtfüggvény} polinomok hányadosaként áll elő.
  Általános alakja:
  \[
    R(x) = \frac{P(x)}{Q(x)}
    \text.
  \]

  Amennyiben a nevező fokszáma kisebb, mint a számlálóé, vagyis ${\deg P(x) \geq
      \deg Q(x)}$, akkor a \textbf{polinomosztás} módszeréhez kell folyamodnunk,
  mely elvégzése után a törtfüggvény az alábbi alakot ölti:
  \[
    R(x) = T(x) + \frac{S(x)}{Q(x)}
    \text,
  \]
  hol $T(x)$ egy újabb polinom, $S(x)$ fokszáma pedig már kisebb, mint $Q(x)$
  fokszáma.

  Ezután \textbf{parciális törtekké} bontjuk a $S(x) / Q(x)$ hányadost, majd
  ezeket, illetve a $T(x)$ polinomot integráljuk.

  Amennyiben a nevező fokszáma nagyobb, mint a számláló fokszáma, vagyis
  $\deg P(x) < \deg Q(x)$, akkor polinomosztás nélkül tudjuk parciális törtekké
  bontani a függvényt.

  A parciális törtekre való bontáshoz az \textbf{algebra alaptételét} használjuk
  fel, miszerint bármely valós együtthatós polinom felbontható első és
  másodrendű kifejezések szorzatára, vagyis
  \[
    p(x) = A
    \cdot
    \underbrace{\prod (x - a_i)}_{\text{valós gyökök}}
    \cdot
    \underbrace{\prod (x^2 + p_i x + q_i)}_{\text{komplex gyökök}}
    \text.
  \]
  Valós gyökök esetén $a_i$ maga a polinom gyöke, míg komplex gyökök esetén
  \[
    x^2 + p_i x + q_i
    = (x - z_i)(x - \overline z_i)
    = x^2 - 2 \iRe(z_i) x + |z_i|^2
    \text.
  \]
  Vegyük például a $p(x) = x^5 - 13 x^4 + 73 x^3 - 193 x^2 + 232 x - 100$
  polinomot, melynek gyökei: $x_1 = 1$, $x_2 = 2$, $x_3 = 2$, $x_4 = 4 + 3i$,
  $x_5 = 4 - 3i$. Ekkor a polinom felbontható:
  \[
    p(x)
    = \underbrace{(x - 1)}_{x_1}
    \cdot \underbrace{(x - 2)^2}_{x_2, x_3}
    \cdot \underbrace{(x^2 - 6x + 25)}_{x_4, x_5}
    \text.
  \]
\end{blueBox}

\begin{example}
  Hozzuk $R(x) = T(x) + S(x) / Q(x)$ alakra ($\deg S < \deg Q$) az $R(x) =
    P(x) / Q(x)$ függvényt, ahol ${P(x)=x^3 - 12x^2 - 42}$ és $Q(x) = x - 3$.
  Végezzük el a polinomosztást:
  \[
    \def\arraycolsep{4pt}
    \begin{array}{rrr}
      \begin{array}{l}
        \\{\color{primaryColor}x^2} \cdot (x - 3) \rightarrow
      \end{array}
       &
      \begin{array}{lrlll}
          & (x^3 & -12x^2 & {\color{gray}+0x}\phantom2 & -42)               \\
        - & (x^3 & -3x^2  & {\color{gray}+0x}\phantom2 & {\color{gray}+0x}) \\ \hline
      \end{array}
       &
      \begin{array}{llll}
        \hspace{-8pt}\div & (x-3) & = &
        {\color{primaryColor}x^2}
          {\color{secondaryColor}\;-\;9x}
          {\color{ternaryColor}\;-\;27}
        \text.                          \\\\
      \end{array}
      \\
      \begin{array}{l}
        \\{\color{secondaryColor}-9x} \cdot (x - 3) \rightarrow
      \end{array}
       &
      \begin{array}{lrll}
          & -9x^2\phantom2  & {\color{gray}+0x} & -42                        \\
        - & (-9x^2\phantom2 & +27x              & {\color{gray}+0\phantom2}) \\ \hline
      \end{array}
       &
      \\
      \begin{array}{l}
        \\{\color{ternaryColor}-27} \cdot (x - 3) \rightarrow
      \end{array}
       &
      \begin{array}{lrl}
          & -27x  & -42  \\
        - & (-27x & +81) \\ \hline
      \end{array}
       &
      \\
       &
      -123
    \end{array}
  \]
  Az eredmény tehát:
  \[
    R(x) = x^2 - 9x - 27 - \frac{123}{x - 3}
    \text.
  \]
\end{example}

\begin{blueBox}
  \sftitle{Félszöges tangens helyettesítés}:

  Amennyiben trigonometrikus ($\sin x$, $\cos x$) függvényekből álló racionális
  törtfüggvényeket szeretnénk integrálni, akkor az alábbi helyettesítés
  alkalmazásával közönséges, $t$-től függő racionális törtfüggvényeket kapunk:
  \[
    t = \tan \frac{x}{2}
    \quad \rightarrow \quad
    \dd x = \frac{2 \dd t}{1 + t^2}
    \text.
  \]
  Ilyen esetben a $\sin x$ és $\cos x$ trigonometrikus függvényeket a következő
  módon helyettesítjük:
  \[
    \sin x = \frac{2t}{1 + t^2}
    \quad \text{és} \quad
    \cos x = \frac{1 - t^2}{1 + t^2}
    \text.
  \]
  Egy ilyen integrál általános alakja:
  \[
    \int R(\sin x; \cos x) \dd x =
    \int R\left(
    \frac{2t}{1 + t^2}; \frac{1 - t^2}{1 + t^2}
    \right) \frac{2 \dd t}{1 + t^2}
    \text.
  \]
\end{blueBox}

\begin{note}
  \sftitle{Félszöges tangens helyettesítés levezetése}:

  Használjuk az alábbi trigonometrikus azonosságokat:
  \begin{align*}
    \sin x & = 2 \sin (\sfrac{x}{2}) \cos (\sfrac{x}{2}) \text,     \\[1mm]
    \cos x & = \cos^2 (\sfrac{x}{2}) - \sin^2 (\sfrac{x}{2}) \text, \\[1mm]
    1      & = \cos^2 (\sfrac{x}{2}) + \sin^2 (\sfrac{x}{2}) \text.
  \end{align*}

  Ezek alapján a $\sin x$ és $\cos x$ trigonometrikus függvényeket a következő
  módon helyettesíthetjük:
  \begin{align*}
    \sin x & = \frac{\sin x}{1} = \frac{
      2 \sin (\sfrac{x}{2}) \cos (\sfrac{x}{2})
    }{
      \cos^2 (\sfrac{x}{2}) + \sin^2 (\sfrac{x}{2})
    } \stackrel{*}{=} \frac{
      2 \tan (\sfrac{x}{2})
    }{
      1 + \tan^2 (\sfrac{x}{2})
    } = \frac{
      2t
    }{
      1 + t^2
    }
    \text, \qquad \left(*: \cdot \; \frac{1/\cos^2(\sfrac{1}{x})}{1/\cos^2(\sfrac{1}{x})} \right)
    \\[1mm]
    \cos x & = \frac{\cos x}{1}= \frac{
      \cos^2 (\sfrac{x}{2}) - \sin^2 (\sfrac{x}{2})
    }{
      \cos^2 (\sfrac{x}{2}) + \sin^2 (\sfrac{x}{2})
    } \stackrel{*}{=} \frac{
      1 - \tan^2 (\sfrac{x}{2})
    }{
      1 + \tan^2 (\sfrac{x}{2})
    } = \frac{
      1 - t^2
    }{
      1 + t^2
    }
    \text. \qquad \left(*: \cdot \; \frac{1/\cos^2(\sfrac{1}{x})}{1/\cos^2(\sfrac{1}{x})} \right)
  \end{align*}

  Végül pedig $t = \tan (\sfrac{x}{2})$ alapján:
  \[
    x = 2 \arctan t
    \quad \rightarrow \quad
    \odv{x}{t} = \frac{2}{1 + t^2}
    \quad \rightarrow \quad
    \dd x = \frac{2 \dd t}{1 + t^2}
  \]
\end{note}

\begin{example}
  \sftitle{A koszekáns integrálása}:
  \[
    \int \csc x \dd x
    = \int \frac{1}{\sin x} \dd x
    = \int \frac{1 + t^2}{2t} \frac{2 \dd t}{1 + t^2}
    = \int \frac{\dd t}{t}
    = \ln |t| + C
    = \ln \left|\tan \frac{x}{2}\right| + C
  \]
\end{example}


\begin{learnMore}[Félszöges tangens helyettesítés geometriai levezetése]
  \begin{minipage}{.5\textwidth}
    % \centering
    \begin{tikzpicture}[scale=2.5, thick]
      \draw[->, shorten >=-5mm, shorten <=-5mm]
      (-1,0) -- (1,0) node[below right] {$\xi$};
      \draw[->, shorten >=-5mm, shorten <=-5mm]
      (0,-1) -- (0,1) node[above left] {$\eta$};

      \draw (0,0) circle (1);

      \coordinate (O) at (0,0);
      \coordinate (A) at (50:1);
      \coordinate (B) at (-1,0);
      \coordinate (C) at (0,0.46630766); % (tan(25/180*pi))
      \coordinate (D) at (1,1.19175359); % (tan(50/180*pi))
      \coordinate (E) at (O-|A);

      % Big triangle
      \draw [ultra thick, draw=primaryColor]
      (O)
      -- (A) node [midway, above, rotate=50] {$1$}
      -- (E) node [midway, above, rotate=90] {$\sin x$}
      -- cycle node[midway, below] {$\cos x$}
      ;

      % Small triangle
      \draw [ultra thick, draw=secondaryColor]
      (B)
      -- (O) node[below, midway] {$1$}
      -- (C) node[left, midway] {$t$}
      -- cycle % node[above, midway, rotate=25] {$\sqrt{1+t^2}$}
      ;

      % Draw line with slope t
      \draw[
        dashed, draw=ternaryColor, ultra thick,
        shorten >=-20mm, shorten <=-10mm
      ] (B) -- (A)
      % node [above=10mm, right=-14mm] {$e: \eta = t(\xi + 1)$}
      node [above=8mm, right=10mm] {$e$}
      ;

      % Mark important points
      \draw[fill=primaryColor, thick] (B)
      node[above left] {$(-1;0)$}
      circle (0.03)
      ;
      \draw[fill=primaryColor, thick] (A)
      node[below=1mm, right=2mm] {$(\xi_2; \eta_2)$}
      circle (0.03)
      ;

      % Half angle
      \draw pic [
          draw,
          "$\sfrac{x}{2}$",
          angle eccentricity=.75,
          angle radius=1.5cm,
        ] {angle = O--B--C};

      % Full angle
      \draw pic [
          draw,
          "$x$",
          angle eccentricity=.65,
          angle radius=1cm,
        ] {angle = E--O--A};

      % Circle equation
      % \node[below left] at (250:1) {$k: \xi^2 + \eta^2 = 1$};
      \node[below left] at (240:1) {$k$};
    \end{tikzpicture}
  \end{minipage}\hfill\begin{minipage}{.475\textwidth}
    A $k$ egységkör egyenlete a $\xi\eta$ koordinátarendszerben:
    \[
      k: \xi^2 + \eta^2 = 1
      \text.
    \]
    Az $e$ egyenes átmegy a $(-1;0)$ ponton, meredeksége pedig $t$.
    Egyenlete:
    \[
      e: \eta = t(\xi + 1)
      \text.
    \]
    Helyettesítsük be az egyenes egyenletét a kör egyenletébe:
    \[
      \xi^2 + (t(\xi + 1))^2 = 1
    \]
  \end{minipage}

  Fejezzük ki a $\xi$, majd $\eta$ koordinátákat a $t$ függvényében!
  \[
    0
    = \xi^2 + t^2(\xi + 1)^2 - 1
    = \xi^2 + t^2\xi^2 + 2t^2\xi + t^2 - 1
    = (1 + t^2)\xi^2 + 2t^2\xi + t^2 - 1
  \]

  Használjuk a másodfokú egyenlet megoldóképletét!
  \[
    \xi_{12} = \frac{
      - 2t^2 \pm \sqrt{4t^4 - 4(1 + t^2)(t^2 - 1)}
    }{
      2(1 + t^2)
    } = \frac{
      -t^2 \pm \sqrt{(t^4 - (t^4 - 1))}
    }{
      1 + t^2
    } = \frac{
      \pm 1 - t^2
    }{
      1 + t^2
    }
  \]

  Az egyenlet egyik megoldásából visszakaphatjuk a $(-1;0)$ pontot:
  \[
    \xi_1 = \frac{-1 - t^2}{1 + t^2} = -1
    \quad \rightarrow \quad
    \eta_1 = t(\xi_1 + 1) = t(-1 + 1) = 0
    \text.
    \hspace{41mm}
  \]

  A másik megoldásból pedig a $(\xi_2; \eta_2)$ pontot:
  \[
    \xi_2 = \frac{1 - t^2}{1 + t^2}
    \quad \rightarrow \quad
    \eta_2 = t(\xi_2 + 1) = t\left(
    \frac{1 - t^2}{1 + t^2} + 1
    \right) = t\left(
    \frac{1 - t^2 + 1 + t^2}{1 + t^2}
    \right) = \frac{2t}{1 + t^2}
    \text.
  \]
  A kék háromszög alapján:
  \[
    \tan \frac{x}{2} = \frac{t}{1}
    \quad \rightarrow \quad
    t = \tan \frac{x}{2}
    \text.
  \]
  A piros háromszög alapján:
  \[
    \sin x = \eta_2 = \frac{2t}{1 + t^2}
    \text,
    \hspace{1cm}
    \cos x = \xi_2 = \frac{1 - t^2}{1 + t^2}
    \text.
  \]
\end{learnMore}

\begin{note}
  A két háromszög bejelölt szögeinek aránya kerületi és
  középponti szögek tételéből következik, amely kimondja, hogy adott körben
  adott ívhez tartozó kerületi szög mindig fele az ívhez tartozó középponti
  szögnek.
\end{note}


\begin{blueBox}
  \sftitle{Félszöges tangens hiperbolikusz helyettesítés}:

  A trigonometrikus függvényekhez nagyon hasonló ez az eset is, viszont itt
  hiperbolikus ($\sinh x$, $\cosh x$) függvényekből álló racionális
  törtfüggvényeket szeretnénk integrálni. A helyettesítés:
  \[
    u = \tanh \frac{x}{2}
    \quad \rightarrow \quad
    \dd x = \frac{2 \dd u}{1 - u^2}
  \]
  Ilyen esetben a $\sinh x$ és $\cosh x$ hiperbolikus függvényeket a következő
  módon helyettesítjük:
  \[
    \sinh x = \frac{2u}{1 - u^2}
    \quad \text{és} \quad
    \cosh x = \frac{1 + u^2}{1 - u^2}
    \text.
  \]
  Egy ilyen integrál általános alakja:
  \[
    \int R(\sinh x; \cosh x) \dd x =
    \int R\left(
    \frac{2u}{1 - u^2}; \frac{1 + u^2}{1 - u^2}
    \right) \frac{2 \dd u}{1 - u^2}
    \text.
  \]
\end{blueBox}

\begin{note}
  \sftitle{Félszöges tangens hiperbolikusz helyettesítés levezetése}:

  Használjuk az alábbi hiperbolikus azonosságokat:
  \begin{align*}
    \sinh x & = 2 \sinh (\sfrac{x}{2}) \cosh (\sfrac{x}{2}) \text,     \\[1mm]
    \cosh x & = \cosh^2 (\sfrac{x}{2}) + \sinh^2 (\sfrac{x}{2}) \text, \\[1mm]
    1       & = \cosh^2 (\sfrac{x}{2}) - \sinh^2 (\sfrac{x}{2}) \text.
  \end{align*}

  Ezek alapján a $\sinh x$ és $\cosh x$ hiperbolikus függvényeket a következő
  módon helyettesíthetjük:
  \begin{align*}
    \sinh x & = \frac{\sinh x}{1} = \frac{
      2 \sinh (\sfrac{x}{2}) \cosh (\sfrac{x}{2})
    }{
      \cosh^2 (\sfrac{x}{2}) - \sinh^2 (\sfrac{x}{2})
    } = \frac{
      2 \tanh (\sfrac{x}{2})
    }{
      1 - \tanh^2 (\sfrac{x}{2})
    } = \frac{
      2u
    }{
      1 - u^2
    }
    \text,
    \\[2mm]
    \cosh x & = \frac{\cosh x}{1}= \frac{
      \cosh^2 (\sfrac{x}{2}) + \sinh^2 (\sfrac{x}{2})
    }{
      \cosh^2 (\sfrac{x}{2}) - \sinh^2 (\sfrac{x}{2})
    } = \frac{
      1 + \tanh^2 (\sfrac{x}{2})
    }{
      1 - \tanh^2 (\sfrac{x}{2})
    } = \frac{
      1 + u^2
    }{
      1 - u^2
    }
    \text.
  \end{align*}

  Végül pedig $u = \tanh (\sfrac{x}{2})$ alapján:
  \[
    x = 2 \arctanh u
    \quad \rightarrow \quad
    \odv{x}{u} = \frac{2}{1 - u^2}
    \quad \rightarrow \quad
    \dd x = \frac{2 \dd u}{1 - u^2}
  \]
\end{note}

\begin{example}
  \sftitle{A koszekáns hiperbolikusz integrálása}:
  \[
    \int \operatorname{csch} x \dd x
    = \int \frac{1}{\sinh x} \dd x
    = \int \frac{1 - u^2}{2u} \frac{2 \dd u}{1 - u^2}
    = \int \frac{\dd u}{u}
    = \ln |u| + C
    = \ln \left|\tanh \frac{x}{2}\right| + C
  \]
\end{example}

\begin{learnMore}[Félszöges tangens hiperbolikusz helyettesítés geometriai levezetése]
  \begin{minipage}{.5\textwidth}
    % \centering
    \begin{tikzpicture}[scale=2.25, thick]
      % Coordinate system
      \draw[->]
      (-1.35,0) -- (2,0) node[below left] {$\xi$};
      \draw[->]
      (0,-1.5) -- (0,1.75) node[below left] {$\eta$};

      % Unit hyperbola 
      \draw[domain=1:1.75, samples=100, smooth, -to]
      plot ({\x}, {sqrt(\x^2 - 1)});
      \draw[domain=1:1.75, samples=100, smooth, -to]
      plot ({\x}, {-sqrt(\x^2 - 1)});

      % Coordinates
      \coordinate (O) at (0,0);
      \coordinate (A) at (1.54308063,1.17520119); % (cosh(1), sinh(1))
      \coordinate (B) at (-1,0);
      \coordinate (C) at (0,0.46211716); % (0,tanh(0.5))
      \coordinate (D) at (1,0.76159416); % (1,tanh(1))
      \coordinate (E) at (O-|A);

      % Hyperbolic functions
      \draw [ultra thick, draw=primaryColor]
      (O)
      -- (E) node [pos=.35, below] {$\cosh x$}
      -- (A) node [midway, below, rotate=90] {$\sinh x$}
      % -- cycle
      % (O)
      % -| (D) node [pos=.7125, above, rotate=90] {$\tanh x$}
      % (O)
      % -- (A)
      ;

      % Slope of the line
      \draw [ultra thick, draw=secondaryColor]
      (B)
      -- (O) node[below, midway] {$1$}
      -- (C) node[left, midway] {$u$}
      -- cycle
      ;

      % Draw line with slope t
      \draw[
        dashed, draw=ternaryColor, ultra thick,
        shorten >=-10mm, shorten <=-10mm
      ] (B) -- (A)
      % node [above=10mm, right=-14mm] {$e: \eta = t(\xi + 1)$}
      node [above=6mm, right=6mm] {$e$}
      ;

      % Mark important points
      \draw[fill=primaryColor, thick] (B)
      node[above=1mm] {$(-1;0)$}
      circle (0.06)
      ;
      \draw[fill=primaryColor, thick] (A)
      node[above left] {$(\xi_2; \eta_2)$}
      circle (0.06)
      ;
    \end{tikzpicture}
  \end{minipage}\hfill\begin{minipage}{.475\textwidth}
    Az egységhiperbola egyenlete a $\xi\eta$ koordinátarendszerben:
    \[
      h: \xi^2 - \eta^2 = 1
      \text.
    \]
    Az $e$ egyenes átmegy a $(-1;0)$ ponton, meredeksége pedig $u$.
    Egyenlete:
    \[
      e: \eta = u(\xi + 1)
      \text.
    \]
    Helyettesítsük be az egyenes egyenletét a hiperbola egyenletébe:
    \[
      \xi^2 - (u(\xi + 1))^2 = 1
      \text.
    \]
  \end{minipage}

  Fejezzük ki a $\xi$, majd $\eta$ koordinátákat a $u$ függvényében!
  \[
    0
    = \xi^2 - u^2(\xi + 1)^2 - 1
    = \xi^2 - u^2\xi^2 - 2u^2\xi - u^2 - 1
    = (1 - u^2)\xi^2 - 2u^2\xi - u^2 - 1
  \]
  Használjuk a másodfokú egyenlet megoldóképletét!
  \[
    \xi_{12} = \frac{
      2u^2 \pm \sqrt{4u^4 + 4(1 - u^2)(u^2 + 1)}
    }{
      2(1 - u^2)
    } = \frac{
      u^2 \pm \sqrt{(u^4 + (1 - u^4))}
    }{
      1 - u^2
    } = \frac{
      \pm 1 + u^2
    }{
      1 - u^2
    }
  \]
  Az egyenlet egyik megoldásából visszakaphatjuk a $(-1;0)$ pontot:
  \[
    \xi_1 = \frac{-1 + u^2}{1 - u^2} = -1
    \quad \rightarrow \quad
    \eta_1 = u(\xi_1 + 1) = u(-1 + 1) = 0
    \text.
    \hspace{43mm}
  \]
  A másik megoldásból pedig a $(\xi_2; \eta_2)$ pontot:
  \[
    \xi_2 = \frac{1 + u^2}{1 - u^2}
    \quad \rightarrow \quad
    \eta_2 = u(\xi_2 + 1) = u\left(
    \frac{1 + u^2}{1 - u^2} + 1
    \right) = u\left(
    \frac{1 + u^2 + 1 - u^2}{1 - u^2}
    \right) = \frac{2u}{1 - u^2}
    \text.
  \]
  Hasonló háromszögek alapján:
  \[
    u
    = \frac{\sinh x}{1 + \cosh x}
    = \tanh \frac{x}{2}
    \text.
  \]
  Az egységhiperbola parametrikus egyenlete alapján:
  \[
    \cosh x = \xi_2 = \frac{1 + u^2}{1 - u^2}
    \quad \text{és} \quad
    \sinh x = \eta_2 = \frac{2u}{1 - u^2}
    \text.
  \]
\end{learnMore}

\begin{note}
  \[
    \tanh \frac{x}{2}
    % = \frac{\sinh(\sfrac{x}{2})}{\cosh(\sfrac{x}{2})}
    = \frac{e^{x/2} - e^{-x/2}}{e^{x/2} + e^{-x/2}}
    \cdot \frac{e^{x/2} + e^{-x/2}}{e^{x/2} + e^{-x/2}}
    = \frac{e^x - e^{-x}}{2 + e^x + e^{-x}}
    = \frac{(e^x - e^{-x})/2}{1 + (e^x + e^{-x})/2}
    = \frac{\sinh x}{1 + \cosh x}
  \]
\end{note}

\begin{blueBox}
  \sftitle{Speciális helyettesítések összefoglaló}:
  \begin{itemize}
    \item $R(\sin x; \cos x)$
          \[
            t = \tan \frac{x}{2}
            \quad
            \dd x = \frac{2 \dd t}{1 + t^2}
            \quad
            \sin x = \frac{2t}{1 + t^2}
            \quad
            \cos x = \frac{1 - t^2}{1 + t^2}
          \]

    \item $R(\sinh x; \cosh x)$
          \[
            u = \tanh \frac{x}{2}
            \quad
            \dd x = \frac{2 \dd u}{1 - u^2}
            \quad
            \sinh x = \frac{2u}{1 - u^2}
            \quad
            \cosh x = \frac{1 + u^2}{1 - u^2}
          \]

    \item $R(e^x; e^{2x}; \dots)$
          \[
            t = e^x
            \quad
            \dd x = \frac{\dd t}{t}
          \]

    \item $R(x; \sqrt{1 - x^2})$
          \[
            x = \sin t
            \quad
            t = \arcsin x
            \quad
            \dd x = \sqrt{1 - x^2} \cdot \dd t
          \]\[
            1 = \cos^2 t + \sin^2 t
          \]

    \item $R(x; \sqrt{x^2 + 1})$
          \[
            x = \sinh t
            \quad
            t = \arcsinh x
            \quad
            \dd x = \sqrt{x^2 + 1} \cdot \dd t
          \]\[
            1 = \cosh^2 t - \sinh^2 t
          \]

    \item $R(x^{\sfrac{a}{c}}; x^{\sfrac{b}{c}}; \dots)$
          \[
            x = t^c
            \quad
            \dd x = c x^{1 - \sfrac{1}{c}} \dd t
          \]
  \end{itemize}
\end{blueBox}

\begin{note}
  A $t = \tan(\sfrac{x}{2})$ és $u = \tanh(\sfrac{x}{2})$ helyettesítésekhez
  tartozó levezetéseket nem szükséges fejből tudni, csupán a megértés érdekében
  szerepelnek az elméleti áttekintőben.
\end{note}

\clearpage
\subsection{Feladatok}

\begin{enumerate}
  \item Határozza meg az alábbi integrálok értékét!
        (Ajánlott módszer: parciális integrálás.)
        \begin{enumerate}
          \item $\displaystyle
                  \int x \cos x \dd x
                $

          \item $\displaystyle
                  \int (x^2 - 1) \sin 3x \dd x
                $

          \item $\displaystyle
                  \int \ln x \dd x
                $

          \item $\displaystyle
                  \int x \arctan x \dd x
                $

          \item $\displaystyle
                  \int e^x \sin x \dd x
                $

          \item $\displaystyle
                  \int \sin^2 x \dd x
                $

          \item $\displaystyle
                  \int e^{\arccos x} \dd x
                $
        \end{enumerate}

  \item Integrálja az alábbi racionális törtfüggvényeket!
        \begin{enumerate}
          \item $\displaystyle
                  \int \frac{x^3 - 9x^2 + 27x - 26}{x^2 - 7x + 12} \dd x
                $

          \item $\displaystyle
                  \int \frac{x^3 - 2x^2 + 4}{x^3 (x - 2)^2} \dd x
                $

          \item $\displaystyle
                  \int \frac{3x - 2}{x^2 + 4x + 8} \dd x
                $
        \end{enumerate}

  \item Határozza meg az alábbi integrálok értékét!
        (Ajánlott módszer: helyettesítéses integrálás.)
        \begin{enumerate}
          \item $\displaystyle
                  \int \frac{1}{5 + 3 \cos x} \dd x
                $

          \item $\displaystyle
                  \int \frac{1}{1 + \cosh x + 2 \sinh x}
                $

          \item $\displaystyle
                  \int \sqrt{\frac{x}{1 - x}} \dd x
                $

          \item $\displaystyle
                  \int \frac{1}{\sqrt x (1 + \sqrt[3] x)}
                $

          \item $\displaystyle
                  \int \frac{e^x + 2}{e^x + e^{2x}} \dd x
                $
        \end{enumerate}
\end{enumerate}

% \\underline\{(\w)\}

\end{document}
