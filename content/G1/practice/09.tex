\documentclass[a4paper, 12pt]{scrartcl}

\usepackage{math-practice}

\title{Differenciálás II}
\area{Kalkulus}
\subject{Matematika G1}
\subjectCode{BMETE94BG01}
\date{Utoljára frissítve: \today}
\docno{9}

\begin{document}
\maketitle

\subsection{Elméleti Áttekintő}

\begin{blueBox}
  \sftitle{Szöveges feladatok}

  Ezen a gyakorlaton olyan szöveges feladatokkal fogunk foglalkozni, amelyekben
  valamilyen szélsőértéket kell meghatároznunk.

  Tudjuk, hogy egy $f$ függvénynek az értelmezési tartományának egy $x_0$
  pontjában akkor van szélsőértéke, ha $f'(x_0) = 0$ és $f'(x)$ előjelet vált
  az $x_0$ pontban, vagy $f''(x_0) \neq 0$.

  Ezen feladatok esetén fontos, hogy a feladat elolvasása után a szöveg alapján
  felírjuk az alapösszefüggéseket. Ezután meg kell határoznunk azt a függvényt,
  amelynek a szélsőértékét keressük. Miután meghatároztuk a függvény
  szélsőértékeit, ellenőriznünk kell, hogy valóban szélsőértéke-e.

  % \begin{itemize}
  %   \item az alapösszefüggésel felírása a feladat alapján,
  %   \item azon függvény/változó kijelölése, amelyenek a szélsőértékét keressük,
  %   \item azon változó kijelölése, amelynek függvényében keressük a
  %         szélsőértéket,
  %   \item cél azon pont megtalálása, ahol $f'(x_0)=0$,
  %   \item ellenőrzés, hogy tényleg szélsőértéke-e?
  % \end{itemize}

  % Ezen tárgy keretében csak egyváltozós függvényekkel fogunk foglalkozni.
\end{blueBox}

\begin{blueBox}
  \sftitle{Teljes függvényvizsgálat}

  Cél, hogy a lehető legtöbb információt megtudjuk az adott függvényről!
  \begin{enumerate}
    \item Értelmezési tartomány
          (Hol nincs értelmezve?)
    \item Zérushelyek
          ($x$ tengelymetszet)
    \item Paritás, periodicitás
          ($f(x) = f(-x)$ -- páros, $f(x) = -f(-x)$ -- páratlan, $f(x) = f(x +
            kp)$, ahol $k \in \mathbb Z$ -- periodikus)
    \item Határérték
          ($\pm \infty$-ben valamint az értélmezési tartomány egyéb szélein,
          szakadási pontokban.)
    \item Monotonitás
          ($f'(x) > 0$ -- nő, $f'(x) < 0$ -- csökken)
    \item Lokális/Globális szélsőértékek
          ($f'(x)$ előjelet vált)
    \item Konvexitás, konkávitás (inflexió)
          ($f''(x) > 0$ -- konvex, $f''(x) < 0$ -- konkáv)
    \item Táblázat
          (Fel kell benne tüntetni a szakadási pontokat, szélsőértékeket,
          inflexiós pontokat)
    \item Aszimptóták keresése ($a = mx + b$ alakban)
          \[
            m = \lim_{x \to \pm\infty} \frac{f(x)}{x}
            \quad \text{és} \quad
            b = \lim_{x \to \pm\infty} f(x) - mx
          \]
    \item Ábrázolás
    \item Értékkészlet leolvasása az ábráról
  \end{enumerate}
\end{blueBox}

\clearpage
\subsection{Feladatok}

\begin{enumerate}
  \item Határozza meg az 1 literes felül nyitott legkisebb felszínű hengert!

  \item Határozza meg a legnagyobb térfogatú $h$ alkotójú kúpot!

  \item Határozza meg az $r$ sugarú körbe írt legnagyobb területű derékszögű
        négyszöget!

  \item Egy $a$ szélességű csatornából derékszögben kinyúlik egy $b$ szélességű
        csatorna. Határozza meg mekkora azon gerenda hossza, amely befordítható
        egyik csatornából a másikba!
        % Keressük meg azt az $\alpha$ szöget, amikor az $A$ pontban átmenő
        % egyenes hossza a legkisebb. Ez lesz a lehető legnagyobb deszka hossza.

  \item A gazda épp a kocsmában mulat, mikor neje felhívja, hogy hol van.
        (Természetesen titokban ment meccset nézni). A gazda, nehogy lebukjon,
        azt hazudja, hogy a szomszédnál van és sietve indul haza. Azonban, hogy
        a kocsmaszagot lemossa magárol, elhatározza, hogy megfürdik a patakban.
        Milyen úton halad, ha a lehető leggyorsabban akar hazaérni?
        \begin{center}
          \begin{tikzpicture}[ultra thick]
            % River
            \draw[secondaryColor] (-4,0) -- ++(8,0);
            \draw[secondaryColor] (-4,-0.5) -- ++(8,0);

            % Home
            \node[draw=secondaryColor, minimum height=6mm, minimum width=6mm] (H) at (-3,1.25) {H};
            \draw[primaryColor] (H.45) -- ++(-0.3,0.25) -- (H.135) -- cycle;

            \draw[dashed, draw=ternaryColor] (H) -- (-3,0) node[midway, left] {$0,25 \, \mathrm{km}$};

            % Pub
            \node[draw=secondaryColor, minimum height=6mm, minimum width=6mm] (K) at (3,3) {K};
            \draw[primaryColor] (K.45) -- ++(-0.3,0.25) -- (K.135) -- cycle;

            \draw[dashed, draw=ternaryColor] (K) -- (3,0) node[midway, right] {$0,75 \, \mathrm{km}$};

            \draw[dashed, draw=ternaryColor] (-3,.25) -- (3,.25) node[midway, above] {$1 \, \mathrm{km}$};
          \end{tikzpicture}
        \end{center}

  \item Végezzük el az $f(x) = \dfrac{2x^2}{x^2 - 9}$ függvény teljes
        vizsgálatát!
\end{enumerate}

% \\underline\{(\w)\}

\end{document}
