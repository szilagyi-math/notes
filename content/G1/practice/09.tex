\documentclass[a4paper, 12pt]{scrartcl}

\usepackage{math-practice}

\title{Differenciálás III}
\area{Kalkulus}
\subject{Matematika G1}
\subjectCode{BMETE94BG01}
\date{Utoljára frissítve: \today}
\docno{9}

\begin{document}
\maketitle

\subsection{Elméleti Áttekintő}

\begin{blueBox}
  \sftitle{Teljes függvényvizsgálat lépései}:

  \begin{enumerate}
    \itemsep-0.33em
    \item Értelmezési tartomány ($\Domain_f$)\\
          Zérushelyek ($x$ tengelymetszet)\\
          Paritás ($f(x) = f(-x)$ -- páros, $f(x) = -f(-x)$ -- páratlan)\\
          Periodicitás ($f(x) = f(x + kp)$, ahol $k \in \mathbb Z$)\\
          Határérték ($\pm \infty$-ben, szakadási pontokban, határpontokban)

    \item $f'(x)$ vizsgálata: monotonitás, lokális szélsőértékek
          \begin{itemize}
            \itemsep-0.66em
            \item $f'(x) > 0$ -- monoton nő
            \item $f'(x) < 0$ -- monoton csökken
          \end{itemize}

    \item $f''(x)$ vizsgálata: konvexitás, konkávitás, inflexiós pontok
          \begin{itemize}
            \itemsep-0.66em
            \item $f''(x) > 0$ -- konvex
            \item $f''(x) < 0$ -- konkáv
          \end{itemize}

    \item Lineáris aszimptoták keresése:
          \begin{itemize}
            \itemsep-0.5em
            \item Az $x = a$ egyenes függőleges aszimptota, ha
                  $\displaystyle\lim_{x \rightarrow a^{+}} = \pm \infty$, vagy
                  $\displaystyle\lim_{x \rightarrow a^{-}} = \pm \infty$.

            \item Az $y = b$ egyenes vízszintes aszimptota, ha
                  $\displaystyle\lim_{x \rightarrow \pm \infty} = b$.

            \item Ferde aszimptotákat $y = mx + b$ alakban keressük, ahol
                  \[
                    m = \lim_{x \rightarrow \pm \infty} \frac{f(x)}{x}
                    \quad \text{és} \quad
                    b = \lim_{x \rightarrow \pm \infty} f(x) - mx
                    \text.
                  \]
          \end{itemize}

    \item Táblázat készítése, ábrázolás és értékkészlet leolvasása az ábráról
  \end{enumerate}
\end{blueBox}

\begin{theorem}[Lokális szélsőérték]
  Ha az $f$ függvény deriválható az értelmezési tartományának egy
  $x_0$ belső pontjában, akkor az $x_0$-beli lokális szélsőérték
  létezésének
  \begin{itemize}
    \item szükséges feltétele:
          $f'(x_0) = 0$,

    \item elégséges feltétele:
          \begin{enumerate}
            \item $f'(x_0) = 0$ és $f'$ előjelet vált az $x_0$-ban
            \item ha $f$ második deriváltja is létezik az $x_0$-ban, akkor
                  $f''(x_0) \neq 0$.
                  \begin{itemize}
                    \item Ha $f''(x_0) > 0$, akkor $f$-nek lokális minimuma van
                          az $x_0$-ban.
                    \item Ha $f''(x_0) < 0$, akkor $f$-nek lokális maximuma van
                          az $x_0$-ban.
                  \end{itemize}
          \end{enumerate}
  \end{itemize}
\end{theorem}

\begin{theorem}[Inflexiós pont]
  Ha az $f$ függvény kétszer deriválható az értelmezési tartományának egy
  $x_0$ belső pontjában, akkor az $x_0$-beli inflexiós pont létezésének
  \begin{itemize}
    \item szükséges feltétele:
          $f''(x_0) = 0$,

    \item elégséges feltétele:
          $f''(x)$ előjelet vált az $x_0$-ban, vagy $f'''(x_0) \neq 0$.
  \end{itemize}
\end{theorem}

\begin{blueBox}
  \sftitle{Szöveges feladatok}

  Ezen a gyakorlaton olyan szöveges feladatokkal fogunk foglalkozni, amelyekben
  valamilyen szélsőértéket kell meghatároznunk.

  Tudjuk, hogy egy $f$ függvénynek az értelmezési tartományának egy $x_0$
  pontjában akkor van szélsőértéke, ha $f'(x_0) = 0$ és $f'(x)$ előjelet vált
  az $x_0$ pontban, vagy $f''(x_0) \neq 0$.

  Ezen feladatok esetén fontos, hogy a feladat elolvasása után a szöveg alapján
  felírjuk az alapösszefüggéseket. Ezután meg kell határoznunk azt a függvényt,
  amelynek a szélsőértékét keressük. Miután meghatároztuk a függvény
  szélsőértékeit, ellenőriznünk kell, hogy valóban szélsőértéke-e.

  % \begin{itemize}
  %   \item az alapösszefüggésel felírása a feladat alapján,
  %   \item azon függvény/változó kijelölése, amelyenek a szélsőértékét keressük,
  %   \item azon változó kijelölése, amelynek függvényében keressük a
  %         szélsőértéket,
  %   \item cél azon pont megtalálása, ahol $f'(x_0)=0$,
  %   \item ellenőrzés, hogy tényleg szélsőértéke-e?
  % \end{itemize}

  % Ezen tárgy keretében csak egyváltozós függvényekkel fogunk foglalkozni.
\end{blueBox}

\vfill

\subsection{Feladatok}

\begin{enumerate}
  \item Végezze el az $f(x) = \dfrac{2x^2}{x^2 - 9}$ függvény teljes
        vizsgálatát!

  \item Határozza meg az 1 literes felül nyitott legkisebb felszínű hengert!

  \item Határozza meg a legnagyobb térfogatú $h$ alkotójú kúpot!

  \item Határozza meg az $r$ sugarú körbe írt legnagyobb területű derékszögű
        négyszöget!

  \item Egy $a$ szélességű csatornából derékszögben kinyúlik egy $b$ szélességű
        csatorna. Határozza meg mekkora azon gerenda hossza, amely befordítható
        egyik csatornából a másikba!
        % Keressük meg azt az $\alpha$ szöget, amikor az $A$ pontban átmenő
        % egyenes hossza a legkisebb. Ez lesz a lehető legnagyobb deszka hossza.

  \item A gazda épp a kocsmában mulat, mikor neje felhívja, hogy hol van.
        (Természetesen titokban ment meccset nézni). A gazda, nehogy lebukjon,
        azt hazudja, hogy a szomszédnál van és sietve indul haza. Azonban, hogy
        a kocsmaszagot lemossa magárol, elhatározza, hogy megfürdik a patakban.
        Milyen úton halad, ha a lehető leggyorsabban akar hazaérni?
        \begin{center}
          \begin{tikzpicture}[ultra thick]
            % River
            \draw[secondaryColor] (-4,0) -- ++(8,0);
            \draw[secondaryColor] (-4,-0.5) -- ++(8,0);

            % Home
            \node[draw=secondaryColor, minimum height=6mm, minimum width=6mm] (H) at (-3,1.25) {H};
            \draw[primaryColor] (H.45) -- ++(-0.3,0.25) -- (H.135) -- cycle;

            \draw[dashed, draw=ternaryColor] (H) -- (-3,0) node[midway, left] {$0,25 \, \mathrm{km}$};

            % Pub
            \node[draw=secondaryColor, minimum height=6mm, minimum width=6mm] (K) at (3,3) {K};
            \draw[primaryColor] (K.45) -- ++(-0.3,0.25) -- (K.135) -- cycle;

            \draw[dashed, draw=ternaryColor] (K) -- (3,0) node[midway, right] {$0,75 \, \mathrm{km}$};

            \draw[dashed, draw=ternaryColor] (-3,.25) -- (3,.25) node[midway, above] {$1 \, \mathrm{km}$};
          \end{tikzpicture}
        \end{center}
\end{enumerate}

% \\underline\{(\w)\}

\end{document}
