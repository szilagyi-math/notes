\documentclass[a4paper, 12pt]{scrartcl}

\usepackage{math-practice}

\title{Integrálszámítás III}
\area{Integrálszámítás}
\subject{Matematika G1}
\subjectCode{BMETE94BG01}
\date{Utoljára frissítve: \today}
\docno{12}

\begin{document}
\maketitle

\subsection{Elméleti Áttekintő}

% \begin{blueBox}
%   Trigonometrikus integrálást
%   \[
%     f(x) = \sin^n x \cos^m x
%   \]
%   alakú függvényekre tudunk alkalmazni. Integrálás során a trigonometrikus
%   azonosságokat használjuk fel, hogy a függvényt könnyebben integrálható alakra
%   hozzuk. A legfontosabb azonosságok:
%   \begin{align*}
%     \sin^2 x & = \frac{1 - \cos 2x}{2} \text, \\
%     \cos^2 x & = \frac{1 + \cos 2x}{2} \text, \\
%     1        & = \cos^2 x + \sin^2 x \text.   \\
%   \end{align*}
% \end{blueBox}

\begin{blueBox}
  \sftitle{Trigonometrikus integrálás}:

  Trigonometrikus függvények integrálásakor a tanult trigonometrikus
  azonosságokat kell alkalmaznunk. Ezek közül a legfontosabbak:
  \begin{align*}
    1        & = \sin^2 x + \cos^2 x \text,   \\
    \sin^2 x & = \frac{1 - \cos 2x}{2} \text, \\
    \cos^2 x & = \frac{1 + \cos 2x}{2} \text, \\
    \sin 2x  & = 2 \sin x \cos x \text,       \\
    \cos 2x  & = \cos^2 x - \sin^2 x \text.
  \end{align*}

  Amennyiben a szögfüggvény fokszáma páros, akkor a függvényt a fenti
  trigonometrikus azonosságok segítségével át tudjuk alakítani.

  Amennyiben a szögfüggvény fokszáma páratlan ($2k + 1$), akkor azt felbontjuk
  egy $2k$-s és egy $1$-es szögfüggvény szorzataként, majd a már páros fokszámú
  tagot az előbbi módszerrel tudjuk integrálni.
\end{blueBox}

\begin{blueBox}
  \sftitle{Határozott integrál}:

  Egy függvény $[a; b]$ intervallumon vett határozott integrálja a
  Newton-Leibniz formula alapján
  \[
    \int_a^b f(x) \dd x = F(b) - F(a)
    \text,
  \]
  ahol $F(x)$ az $f(x)$ primitív függvénye.
\end{blueBox}

\begin{note}
  A határozott integrálás során a határozatlan integrálásnál tanult
  összefüggéseket alkalmazhatjuk. Azonban két integrálási technikánál különösen
  figyelnünk kell az integrálási tartományra:
  \begin{itemize}
    \item Parciális integrálás: $\displaystyle
            \int_a^b f g' = \Big[ f \, g \Big]_a^b - \int_a^b f' g
          $,

    \item Helyettesítéses integrálás: $\displaystyle
            \int_a^b f(x) \dd x = \int_{\varphi^{-1}(a)}^{\varphi^{-1}(b)} f(\varphi(t)) \cdot  \varphi'(t) \dd t
          $.
  \end{itemize}
\end{note}

\clearpage
\begin{blueBox}
  \sftitle{Görbe alatti terület}:

  A határozott integrál segítségével a függvény görbéje és az $x$-tengely által
  bezárt \textbf{előjeles} területet tudjuk meghatározni. Amennyiben a függvény
  képe a tengely alatt van, akkor a terület negatív előjelű lesz.

  \begin{minipage}{.45\textwidth}
    \centering
    \begin{tikzpicture}[thick]
      \draw[opacity=0, name path=x] (1,0) -- (3,0);

      \draw [samples=150, smooth, domain=1:3, ultra thick, draw=primaryColor, name path=f]
      plot (\x, {(sin(\x*240)/4+\x)/2 + 1})
      ;

      \tikzfillbetween[
        of=f and x,
        on layer=main,
      ]{primaryColor!25}

      \draw [->, draw=ternaryColor] (-.5,0) -- (5,0) node [below left] {$x$};
      \draw [->, draw=ternaryColor] (0,-.5) -- (0,3) node [below left] {$y$};

      \draw [samples=150, domain=-.5:4, ultra thick, draw=primaryColor, to-to]
      plot (\x, {(sin(\x*240)/4+\x)/2 + 1})
      ;

      \draw[draw=gray, dashed, thick] (1,2.75) -- ++(0,-2.75) node[below] {$a\vphantom{b}$};
      \draw[draw=gray, dashed, thick] (3,2.75) -- ++(0,-2.75) node[below] {$b$};

      \node at (2,1) {$A$};
    \end{tikzpicture}
    \[
      A = \int_a^b f(x) \dd x
    \]
  \end{minipage}\hfill\begin{minipage}{.45\textwidth}
    \centering
    \begin{tikzpicture}[thick]
      \draw[opacity=0, name path=x] (1,0) -- (3,0);

      \draw [samples=150, smooth, domain=1:3, ultra thick, draw=primaryColor, name path=f]
      plot (\x, {(\x - 1)*(\x - 2)*(\x - 3)*1.5})
      ;

      \tikzfillbetween[
        of=f and x,
        split,
        every segment no 0/.style={primaryColor!25},
        every segment no 1/.style={secondaryColor!25},
        on layer=main,
      ]{}

      \draw [->, draw=ternaryColor, thick] (-.5,0) -- (5,0) node [below left] {$x$};
      \draw [->, draw=ternaryColor, thick] (0,-1.5) -- (0,2) node [below left] {$y$};

      \draw [samples=150, smooth, domain=0.75:3.25, ultra thick, draw=primaryColor, to-to]
      plot (\x, {(\x - 1)*(\x - 2)*(\x - 3)*1.5})
      ;

      \node [above left] at (1,0) {$a$};
      \node [above right] at (2,0) {$c$};
      \node [below right] at (3,0) {$b$};
    \end{tikzpicture}
    \[
      A = \int_a^c f(x) \dd x - \int_c^b f(x) \dd x
    \]
  \end{minipage}
\end{blueBox}

\begin{blueBox}
  \begin{minipage}{.55\textwidth}
    \sftitle{Két görbe által bezárt terület}:
    \\[4mm]
    Két függvény által bezárt területet a két függvény különbségének
    integrálásával tudjuk meghatározni:
    \[
      A = \int_a^b |f(x) - g(x)| \dd x
      \text.
    \]
  \end{minipage}\hfill\begin{minipage}{.4\textwidth}
    \centering
    \begin{tikzpicture}[thick]
      \draw [->, draw=ternaryColor] (-.5,0) -- (5,0) node [below left] {$x$};
      \draw [->, draw=ternaryColor] (0,-.5) -- (0,3) node [below left] {$y$};

      \draw [samples=150, smooth, name path=f, domain=1:3, ultra thick, draw=primaryColor]
      plot (\x, {(sin(\x*240)/4+\x)/2 + 1})
      ;

      \draw [samples=150, smooth, name path=g, domain=1:3, ultra thick, draw=primaryColor]
      plot (\x, {(cos(\x*240)/6+\x)/2})
      ;

      \tikzfillbetween[
        of=f and g,
        on layer=main,
      ]{primaryColor!25}

      \draw [samples=150, to-to, domain=-.5:4, ultra thick, draw=primaryColor]
      plot (\x, {(sin(\x*240)/4+\x)/2 + 1})
      node [right] {$f(x)$}
      ;

      \draw [samples=150, to-to, domain=-.5:4, ultra thick, draw=primaryColor]
      plot (\x, {(cos(\x*240)/6+\x)/2})
      node [right] {$g(x)$}
      ;

      \draw[draw=gray, dashed, thick] (1,2.75) -- ++(0,-2.75) node[below] {$a\vphantom{b}$};
      \draw[draw=gray, dashed, thick] (3,2.75) -- ++(0,-2.75) node[below] {$b$};

      \node at (2,1.5) {$A$};
    \end{tikzpicture}
  \end{minipage}
\end{blueBox}

\begin{blueBox}
  \sftitle{Paraméteres görbe által meghatározott görbevonalú trapéz terület}:

  Egy $\gamma: (x(t); y(t))$ görbe $t_1$ és $t_2$ paraméterpontok közötti
  görbevonalú trapéz területe
  \[
    A = \int_{t_1}^{t_2} \left| \dot x(t) \cdot y(t) \right| \dd t
    \text.
  \]
\end{blueBox}

\begin{example}
  Határozzuk meg a ciklois egy $t \in [0; 2\pi]$ intervallumhoz tartozó
  görbevonalú trapéz területét!

  \begin{minipage}{.5\textwidth}
    A ciklois paraméteres egyenlete:

    \[
      \begin{aligned}
        x(t) & = t - \sin t \text, \\
        y(t) & = 1 - \cos t \text.
      \end{aligned}
    \]
  \end{minipage}\begin{minipage}{.5\textwidth}
    \centering
    \begin{tikzpicture}
      \draw[opacity=0, name path=x] (0,0) -- (4,0);

      \draw [samples=150, smooth, name path=f, domain=0:2*pi, ultra thick, draw=primaryColor]
      plot ({1/2 * (\x - sin(deg(\x)))}, {1/2 * (1 - cos(deg(\x)))})
      ;

      \tikzfillbetween[
        of=f and x,
        on layer=main,
      ]{primaryColor!25}

      \draw [-to, draw=ternaryColor, thick] (-.5,0) -- (5,0) node [below left] {$x$};
      \draw [-to, draw=ternaryColor, thick] (0,-.5) -- (0,2) node [below left] {$y$};

      \draw [samples=150, domain=-1.5:9, ultra thick, draw=primaryColor, to-to, smooth]
      plot ({1/2 * (\x - sin(deg(\x)))}, {1/2 * (1 - cos(deg(\x)))})
      ;

      \node at (1.57079633,.45) {$A$};
    \end{tikzpicture}
  \end{minipage}

  \[
    A
    = \int_{0}^{2\pi} \left| \dot x(t) \cdot y(t) \right| \dd t
    = \int_{0}^{2\pi} (1 - \cos t)^2 \dd t
    = \int_{0}^{2\pi} 1 - 2 \cos t + \cos^2 t \dd t
    = \dots
    = 3\pi
  \]
\end{example}


\clearpage
\subsection{Feladatok}

\begin{enumerate}
  \item Határozza meg az alábbi trigonometrikus integrálok értékét!
        %   \begin{multicols}{3}
        \begin{enumerate}
          \item $\displaystyle
                  \int \cos^3 x \sin x \dd x
                $

          \item $\displaystyle
                  \int \cos^5 x \dd x
                $

          \item $\displaystyle
                  \int \sin^4 x \cos^2 x \dd x
                $
        \end{enumerate}
        %   \end{multicols}

  \item Oldja meg az alábbi összetett integrálási feladatokat!
        \begin{enumerate}
          \item $\displaystyle
                  \int \sin \sqrt x \dd x
                $

          \item $\displaystyle
                  \int \frac{\ln \ln x}{x} \dd x
                $

          \item $\displaystyle
                  \int |x| \dd x
                $

          \item $\displaystyle
                  \int \frac{\ln x + 1}{x^x - 1} \dd x
                $

          \item $\displaystyle
                  \int (x^2 - 3x + 2) \sqrt{2x - 1} \dd x
                $
        \end{enumerate}

  \item Határozzuk meg az alábbi határozott integrálok értékét!
        \begin{enumerate}
          \item $\displaystyle
                  \int_0^{2\pi} \cos x \dd x
                $

          \item $\displaystyle
                  \int_0^1 x \sinh x \dd x
                $

          \item $\displaystyle
                  \int_{-3}^3 \sqrt{9 - x^2} \dd x
                $
        \end{enumerate}

  \item Határozza meg az $f(x) = (x + 1) \, x \, (x - 2)$ függvény és az $x$-tengely
        által bezárt geometriai területet!

  \item Adja meg az $f(x) = x^4$ és a $g(x) = 3x^2 - 2$ függvények által bezárt
        terület nagyságát!

  \item Adja meg egy $a$ sugarú körvonal ($x(t) = a \cos t$, $y(t) = a \sin t$)
        alapján a $t \in [0; 2\pi]$ intervallumhoz tartozó görbevonalú trapéz
        területét!
\end{enumerate}

% \\underline\{(\w)\}

\end{document}
