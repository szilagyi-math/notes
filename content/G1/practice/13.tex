\documentclass[a4paper, 12pt]{scrartcl}

\usepackage{math-practice}

\title{Integrálszámítás III}
\area{Integrálszámítás}
\subject{Matematika G1}
\subjectCode{BMETE94BG01}
\date{Utoljára frissítve: \today}
\docno{13}

\begin{document}
\maketitle

\subsection{Elméleti Áttekintő}

\begin{blueBox}
  \sftitle{Improprius integrál}:

  A Riemann-integrál definíció szerint akkor használható, hogyha az integrációs
  intervallum véges, valamint ezen intervallumon az integrálandó függvény is
  korlátos. Előfordulhat azonban olyan eset, hogy
  \begin{itemize}
    \item végtelen tartományon szeretnénk integrálni,
    \item az integrálandó függvény nem korlátos az integrálási tartományban.
  \end{itemize}

  \begin{minipage}{.5\textwidth}
    \centering
    \begin{tikzpicture}[ultra thick]
      \draw[-to] (-1,0) -- (5,0) node[below left] {$x$};
      \draw[-to] (-.5,-.5) -- (-.5,3) node[below left] {$y$};

      \draw[opacity=0, name path=x] (-.5,0) -- (5,0);

      \draw [
        domain=1:6.5,
        samples=100,
        primaryColor,
        name path=p,
        -to,
      ] plot (\x-1.5, {2/\x});

      \tikzfillbetween[
        of=p and x,
        on layer=bg,
      ]{primaryColor!10}

      \node at (3.5,1.5) {$\displaystyle
          \lim_{\beta \rightarrow \infty} \int_a^\beta f(x) \dd x
        $};
    \end{tikzpicture}

    Végtelen tartomány
  \end{minipage}\begin{minipage}{.5\textwidth}
    \centering
    \begin{tikzpicture}[ultra thick]
      \draw[-to] (-1,0) -- (5,0) node[below left] {$x$};
      \draw[-to] (-.5,-.5) -- (-.5,3) node[below left] {$y$};

      \draw[opacity=0, name path=x] (0.25,0) -- (3,0);
      \draw[draw=gray, dashed] (0.25,0) node[below] {$a$} -- ++(0,3);
      \draw[draw=gray, dashed] (3,0) -- ++(0,.25);

      \draw [
        domain=.25:3,
        samples=100,
        primaryColor,
        name path=p,
        to-
      ] plot (\x, {.75/\x});

      \tikzfillbetween[
        of=p and x,
        on layer=bg,
      ]{primaryColor!10}

      \node at (3.5,1.5) {$\displaystyle
          \lim_{\alpha \rightarrow a^{+}} \int_\alpha^b f(x) \dd x
        $};
    \end{tikzpicture}

    Nem korlátos függvény
  \end{minipage}

  Ezekben az esetekben az improprius integrált hívhatjuk segítségül.
\end{blueBox}

\begin{blueBox}
  \sftitle{Ívhossz számítása integrálással}:

  Egy görbét háromféleképpen is definiálhatunk:
  \begin{itemize}
    \item explicit alakban: $y = f(x)$,
    \item paraméteres alapban: $x = x(t)$, $y = y(t)$,
    \item polárkoordináta-rendszerben: $r = r(\varphi)$.
  \end{itemize}

  Az ívhossz számítására az alábbi képletet használhatjuk:
  \begin{itemize}
    \item explicit: $\displaystyle
            L = \int_a^b \sqrt{1 + \left( f'(x) \right)^2} \dd x
          $

    \item paraméteres: $\displaystyle
            L = \int_a^b \sqrt{ \dot x^2 + \dot y^2} \dd t
          $

    \item polár: $\displaystyle
            L = \int_a^b \sqrt{ {\dot r}^2 (\varphi) + r^2(\varphi)} \dd \varphi
          $
  \end{itemize}
\end{blueBox}

\begin{blueBox}
  \sftitle{Forgástestek térfogata és felszíne}:

  Egy görbe $x$ tengely körüli megforgatásával egy forgástestet kapunk.
  \def\arraycolsep{12pt}
  \[
    \begin{array}{lll}
      \text{explicit:}    & \displaystyle V = \pi \int_a^b f^2(x) \dd x
                          & \displaystyle A = 2\pi \int_a^b f(x) \sqrt{1 + (f'(x))^2} \dd x
      \\[7mm]
      \text{paraméteres:} & \displaystyle V = \pi \int_a^b y^2(t) \dot x(t) \dd t
                          & \displaystyle A = 2\pi \int_a^b y(t) \sqrt{{\dot x}^2(t) + {\dot y}^2(t)} \dd t
    \end{array}
  \]
\end{blueBox}

\begin{note}
  Fontos figyelembe venni, hogy a fenti képletek csak a forgástestek palástjának
  felszínét adják eredményül.

  Abban az esetben, hogyha például egy csonkakúpnak a felszínét szeretnénk
  meghatározni, akkor az alső és felső alapok felületét hozzá kell adni
  az előbbi képletekkel kapott eredményhez.
\end{note}

\begin{blueBox}
  \sftitle{Görbeív súlypontja}:

  Egy konstans sűrűségű görbe súlypontjainak koordinátáit az alábbi képletekkel
  számíthatjuk ki:
  \[
    \begin{array}{lll}
      \text{explicit:}    & \displaystyle S_x = \frac{1}{L} \int_a^b x \sqrt{1 + (f'(x))^2} \dd x
                          & \displaystyle S_y = \frac{1}{L} \int_a^b f(x) \sqrt{1 + (f'(x))^2} \dd x
      \\[7mm]
      \text{paraméteres:} & \displaystyle S_x = \frac{1}{L} \int_a^b x(t) \sqrt{{\dot x}^2(t) + {\dot y}^2(t)} \dd t
                          & \displaystyle S_y = \frac{1}{L} \int_a^b y(t) \sqrt{{\dot x}^2(t) + {\dot y}^2(t)} \dd t
    \end{array}
  \]
\end{blueBox}

\begin{blueBox}
  \sftitle{Görbe által meghatározott síktartomány súlypontja}:

  Egy görbe és az $x$ tengely által meghatározott síktartomány súlypontjainak
  koordinátáit az alábbi képletekkel számíthatjuk ki:
  \def\arraycolsep{20pt}
  \[
    \begin{array}{lll}
      \text{explicit:}    & \displaystyle S_x = \dfrac{\displaystyle\int_a^b x f(x) \dd x}{\displaystyle\int_a^b f(x) \dd x}
                          & \displaystyle S_y = \dfrac{\dfrac{1}{2} \displaystyle\int_a^b f^2(x) \dd x}{\displaystyle\int_a^b f(x) \dd x}
      \\[15mm]
      \text{paraméteres:} & \displaystyle S_x = \dfrac{\displaystyle\int_a^b x(t) \, y(t) \, \dot x(t) \dd t}{\displaystyle\int_a^b y(t) \, \dot x(t) \dd t}
                          & \displaystyle S_y = \dfrac{\dfrac{1}{2} \displaystyle\int_a^b y^2(t) \, \dot x(t) \dd t}{\displaystyle\int_a^b y(t) \, \dot x(t) \dd t}
    \end{array}
  \]
\end{blueBox}

\clearpage
\subsection{Feladatok}

\begin{enumerate}
  \item Milyen $a$ és $b$ paraméterek választása esetén lesz az adott integrál
        értéke minimális?
        \[
          \int_a^b (x^4 - 2x^2) \dd x
        \]

  \item Határozza meg az alábbi határozott integrálok értékeit!
        \begin{multicols}{3}
          \begin{enumerate}
            \item $\displaystyle
                    \int_0^\infty \frac{1}{1 + x^2} \dd x
                  $

            \item $\displaystyle
                    \int_0^1 \frac{1}{\sqrt{x}} \dd x
                  $

            \item $\displaystyle
                    \int_{-2}^0 \frac{1}{x + 2} \dd x
                  $

            \item $\displaystyle
                    \int_1^\infty \frac{1}{x \sqrt{x - 1}} \dd x
                  $

            \item $\displaystyle
                    \int_0^\infty e^{-x} \cos x \dd x
                  $

            \item $\displaystyle
                    \int_0^1 \ln x \dd x
                  $
          \end{enumerate}
        \end{multicols}

  \item Adja meg az $f(x) = x^2$ függvény görbéjének ívhosszát az $x \in [0, 2]$
        intervallumon!

  \item Számítsa ki a ciklois egy ívének hosszát! ($x(t) = a \cdot (t - \sin t)$,
        $y(t) = a \cdot (1 - \cos t)$, $t \in [0, 2\pi]$)

  \item Számítsa ki annak a csonkakúpnak a térfogatát, melynek alapja egy
        $R = 5$ sugarú kör, teteje egy $r = 2$ sugarú kör, magassága pedig
        $h = 6$!

  \item Vezesse le a gömb térfogatának képletét!

  \item Adja meg az $f(x) = x^3$ görbe, valamint az $y = 0$ és $x = 1$
        egyenesek által határolt rész súlypontjának koordinátáit!
\end{enumerate}

% \\underline\{(\w)\}

\end{document}
