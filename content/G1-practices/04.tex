\documentclass[a4paper, 12pt]{scrartcl}

\usepackage{math-practice}

\title{Numerikus sorozatok I}
\area{Sorozatok}
\subject{Matematika G1}
\subjectCode{BMETE94BG01}
\date{Utoljára frissítve: \today}
\docno{4}

\begin{document}
\maketitle

\subsection{Elméleti Áttekintő}

\begin{definition}[Sorozat]
  A pozitív egész számok halmazán értelmezett $a_n : \mathbb N \rightarrow
    \Reals$ függvényt \textbf{valós számsorozat}nak hívjuk.

  Az $a_n:\mathbb N \rightarrow \mathbb C $ függvényt \textbf{komplex
    számsorozat}nak nevezzük.

\end{definition}

\begin{definition}[Konvergencia]
  Az $(a_n)$ sorozatot konvergensnek mondjuk, ha $\exists a \in \Reals$ valós
  szám, hogy $\forall \varepsilon > 0$ esetén $\exists N(\varepsilon)$
  küszöbszám, hogy ha $n > N(\varepsilon)$, akkor $|a_n - a| < \varepsilon$.
  Jelölése:
  \[
    \lim_{n \rightarrow \infty} a_n = a
    \text{, ahol $a$ a sorozat határértéke.}
  \]
\end{definition}

\begin{definition}[Divergencia]
  Az $(a_n)$ sorozatot divergensnek mondjuk, ha nem konvergens.
\end{definition}

\begin{definition}[Torlódási pont]
  Az $(a_n)$ sorozatnak torlódási pontja van az $a \in \Reals$ pontban, ha az
  $a$ tetszőlegesen kicsiny környezete a sorozat véges sok elemét tartalmazza.
\end{definition}

\begin{note}
  A sorozat határértéke egyben torlódási pont is, viszont egy torlódási pont nem
  feltétlenül határérték. Pl.: $a_n = (-1)^n$ sorozatnak két torlódási pontja is
  van ($-1$ és $1$), viszont egyik sem határérték, hiszen a sorozat divergens.
\end{note}

\begin{definition}[Sorozat korlátossága]
  Az $(a_n)$-t \textbf{alulról korlátos}nak nevezzük, ha $\forall n$ esetén
  $a_n > k$, vagyis értékkészlete alulról korlátos.

  Az $(a_n)$-t \textbf{felülről korlátos}nak nevezzük, ha $\forall n$ esetén
  $a_n < K$, vagyis értékkészlete felülről korlátos.

  Az $(a_n)$ sorozat \textbf{korlátos}, ha alulról és felülről is korlátos.
\end{definition}

\begin{definition}[Sorozat monotonitása]
  Az $(a_n)$ sorozat monotonitása:
  \begin{itemize}
    \item monoton növekvő, ha $a_n \geq a_{n-1}$,
    \item monoton csökkenő, ha $a_n \leq a_{n-1}$,
    \item szigorúan monoton növekvő, ha $a_n > a_{n-1}$,
    \item szigorúan monoton csökkenő, ha $a_n < a_{n-1}$.
  \end{itemize}
\end{definition}

\begin{note}
  Konvergens sorozat mindig korlátos.

  Monoton korlátos sorozat mindig konvergens.
\end{note}

\begin{blueBox}
  \sftitle{Nevezetes határértékek}:

  \begin{itemize}
    \item $
            a^n \rightarrow \begin{cases}
              \;0,                & \text{ha } |a| < 1,   \\
              \;1,                & \text{ha } a = 1,     \\
              \;\infty,           & \text{ha } a > 1,     \\
              \;\text{divergens}, & \text{ha } a \leq -1.
            \end{cases}
          $

    \item $
            \sqrt[n]{a} \rightarrow 1
          $

    \item $
            a^n \cdot n^k \rightarrow 0
          $
          \text{, ha }
          $
            |a|<1
          $
          \text{ és }
          $
            k \in \mathbb N
          $

    \item  $
            \sqrt[n]{n} \rightarrow 1
          $

    \item $
            \dfrac{a^n}{n!} \rightarrow 0
          $

    \item $
            \left(1 + \dfrac{1}{n}\right)^n \rightarrow e
          $

    \item $
            \left(1 + \dfrac{r}{n}\right)^n \rightarrow e^r
          $
  \end{itemize}
\end{blueBox}

\begin{blueBox}
  \sftitle{Dominancia elv}:
  \[
    \log_n a <
    \sqrt[n]{a} <
    \log_a n <
    \sqrt[a]{n} <
    n <
    n^a <
    a^n <
    n! <
    n^n
  \]
\end{blueBox}

\begin{note}
  A dominancia elvet olyan esetekben érdemes használnunk, amikor egy
  \[
    a_n = \frac{p_n}{q_n}
  \]
  alakú sorozat határértékét keressük, hiszen segítségével megállapíthatjuk,
  hogy a nevező vagy a számláló fog gyorsabban nőni.
\end{note}

\begin{theorem}[Rendőrelv]
  Tegyük fel, hogy $(a_n)$, $(b_n)$ és $(x_n)$ sorozatokra teljesül, hogy $a_n
    \leq x_n \leq b_n : \forall n$-re vagy $n > N_0$, továbbá
  \[
    \lim_{n \rightarrow \infty} a_n = \lim_{n \rightarrow \infty} b_n = a
    \text{, ekkor }
    \lim_{n \rightarrow \infty} x_n = a
    \text.
  \]
\end{theorem}

\subsection{Feladatok}

\begin{enumerate}
  % 1
  \item A konvergencia definíciója segítségével bizonyítsa be, hogy az alábbi
        sorozatok kon\-ver\-gens\-ek-e.
        \begin{enumerate}
          \item $a_n = \left|
                  \dfrac{n + 1}{3n - 8}
                  \right|$

          \item $b_n = \dfrac{n \cdot (-1)^n - 1}{2n}$
        \end{enumerate}

        % 2
  \item Határozza meg az alábbi sorozatok határértékét!
        \begin{enumerate}
          \item $a_n = \dfrac{n^2 - 6n + 7}{n^2 + 12n + 49}$

          \item $b_n = \dfrac{1 + 2 + 3 + \dots + n}{n + 2} - \dfrac{n}{2}$

          \item $c_n = \dfrac{(-2)^n + 3^n}{(-2)^{n + 1} + 3^{n + 1}}$

          \item $d_n = \dfrac{
                    \sqrt{n^3 + 3n^2} + \sqrt[3]{n^4 + 1}
                  }{
                    \sqrt[4]{5n^6 + 2} + \sqrt[5]{n^7 + 3n^3}
                  } + \dfrac{n!}{(n+1)! + 3^{2n}}$

          \item $e_n = \sqrt{n^2 + n + 1} - \sqrt{n^2 - n + 1}$

          \item $f_n = \sqrt[3]{n^2 - n^3} + n$
        \end{enumerate}

        % 3
  \item Igazolja a rendőrelv segítségével, hogy
        $
          \dfrac{n}{3^n} \rightarrow 0
          \text{, ha }
          n \rightarrow \infty
          \text.
        $

        % 4
  \item Határozza meg az alábbi határértékeket!
        \begin{multicols}{2}
          \begin{enumerate}
            \item $\displaystyle
                    \lim_{n \rightarrow \infty} \sqrt[n]{5n^2 - 30n - 21}
                  $

            \item $\displaystyle
                    \lim_{n \rightarrow \infty} \left(
                    \frac{\cos n^3}{2n} - \frac{3n}{6n+1}
                    \right)
                  $

            \item $\displaystyle
                    \lim_{n \rightarrow \infty} \left(
                    1 + \frac{1}{n^2}
                    \right)^n
                  $

            \item $\displaystyle
                    \lim_{n \rightarrow \infty} \left(
                    \frac{3n - 1}{3n + 2}
                    \right)^{2n}
                  $

            \item $\displaystyle
                    \lim_{n \rightarrow \infty} \left(
                    1 + \frac{1}{n}
                    \right)^{\ln n}
                  $

            \item $\displaystyle
                    \lim_{n \rightarrow \infty} \left(
                    \frac{n^2 - n + 1}{n^2 + n + 1}
                    \right)^{2n + 5}
                  $
          \end{enumerate}
        \end{multicols}

        % 5
  \item Bizonyítsa be, hogy bármely $k \geq 0$ egész számra
        \\[3mm]
        $\displaystyle
          \lim_{n \rightarrow \infty} \left(
          1 + \frac{k}{n}
          \right)^n = e^k
          \text.
        $
\end{enumerate}

% \\underline\{(\w)\}

\end{document}