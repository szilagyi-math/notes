\documentclass[a4paper, 12pt]{scrartcl}

\usepackage{math-practice}

\title{Folytonosság}
\area{Függvények}
\subject{Matematika G1}
\subjectCode{BMETE94BG01}
\date{Utoljára frissítve: \today}
\docno{6}

\begin{document}
\maketitle

\subsection{Elméleti Áttekintő}

\begin{definition}[Cauchy-féle határérték definíció]
  Azt mondjuk, hogy az $f$ függvény határértéke az $a$ pontban $A$, ha $\forall
    \varepsilon > 0$ esetén $\exists \delta(\varepsilon) > 0$, hogy $|f(x) - A|
    < \varepsilon$, ha $0 < |x - a| < \delta(\varepsilon)$. Jele:
  \[
    \lim_{x \rightarrow a} f(x) = A
    \text.
  \]
\end{definition}

\begin{definition}[Heine-féle határérték definíció]
  Az $f$ függvény határértéke az $a$ pontban akkor és csak akkor $A$, ha
  $\forall x_n \rightarrow a$ sorozat esetén $f(x_n) \rightarrow A$.
\end{definition}

\begin{note}
  A két definíció teljesen ekvivalens egymással.
\end{note}

\begin{theorem}[Nevezetes határérték a nullában]
  \[
    \lim_{x \rightarrow 0} \frac{\sin x}{x} = 1
  \]
\end{theorem}

\begin{definition}[Folytonosság]
  Egy $f : \Domain_f \rightarrow \Reals$ függvény folytonos egy $a \in
    \Domain_f$ pontban, ha $\forall \varepsilon > 0$ esetén $\exists \delta(
    \varepsilon) > 0$, hogy $|f(x) - f(a)| < \varepsilon$, ha $|x - a| < \delta(
    \varepsilon)$.
\end{definition}

\begin{statement}
  A folytonosság definíciója ekvivalens a következővel: $f$ függvény folytonos
  egy ${a \in \Domain_f}$ pontban, ha
  \[
    \lim_{x \to a} f(x) = f(a)
    \text.
  \]
\end{statement}

\begin{note}
  Ha ez nem teljesül, akkor a függvénynek az adott pontban szakadása van. Ez
  lehet
  \begin{itemize}
    \item \textbf{megszüntethető}, tehát a függvény az adott pontban nincsen
          értelmezve, viszont a pontbeli határértéke létezik,
    \item \textbf{nem megszüntethető}, vagyis nem létezik az adott pontbeli
          határértéke.
  \end{itemize}
\end{note}

\clearpage
\subsection{Feladatok}

\begin{enumerate}
  \item A függvényhatárérték két definíciója segítségével bizonyítsa be, hogy
        \[
          \lim_{x \rightarrow 2} \frac{3x + 1}{5x + 4} = \frac12
          \text.
        \]

  \item Számítsa ki az alábbi határértékeket!
        \begin{multicols}{2}
          \begin{enumerate}
            \item $\displaystyle
                    \lim_{x \rightarrow -\infty} \left(
                    \frac{x^2}{2x + 1} +
                    \frac{x^3 + 4x^2 - 2}{1 - 2x^2}
                    \right)
                  $

            \item $\displaystyle
                    \lim_{x \rightarrow 0} \frac{3}{2^{\sfrac{\sqrt2}{x}} + 1}
                  $


            \item $\displaystyle
                    \lim_{x \rightarrow -1} \frac{x + 1}{\sqrt{6x^2 + 3} + 3x}
                  $

            \item $\displaystyle
                    \lim_{x \rightarrow 1} \left(
                    \frac{1 + x}{2 + x}
                    \right)^{\frac{1 - \sqrt x}{1 - x}}
                  $

            \item $\displaystyle
                    \lim_{x \rightarrow \sfrac{\pi}{6}} \frac{
                      2 \sin^2 x + \sin x - 1
                    }{
                      2 \sin^2 x - 3 \sin x + 1
                    }
                  $

            \item $\displaystyle
                    \lim_{x \rightarrow \sfrac{\pi}{2}} \frac{
                      \cos x - \sin x + 1
                    }{
                      \cos x + \sin x - 1
                    }
                  $

            \item $\displaystyle
                    \lim_{x \rightarrow 0} \frac{\sin 5x}{x}
                  $

            \item $\displaystyle
                    \lim_{x \rightarrow 0} \frac{\tan x - \sin x}{x^3}
                  $

            \item $\displaystyle
                    \lim_{x \rightarrow 0} \frac{\sin 8x}{\tan 5x}
                  $

            \item $\displaystyle
                    \lim_{x \rightarrow \sfrac{\pi}{2}} (\sfrac{\pi}{2} - x) \tan x
                  $

            \item $\displaystyle
                    \lim_{x \rightarrow 0} \frac{1 - \cos \sin x}{\sin^2 x}
                  $
          \end{enumerate}
        \end{multicols}

  \item Vizsgálja meg az alábbi függvényt folytonosság szempontjából!
        \[
          f(x) = \frac{x^2 + 3x - 10}{x^2 - 3x + 2}
        \]

  \item Vizsgálja meg az alábbi függvényt folytonossát a nullában!
        \[
          f(x) = x \sin \frac1x
        \]

  \item Határozza meg az $a$ és $b$ paraméterek értékét úgy, hogy $f$ függvény
        folytonos legyen!
        \[
          f(x) = \begin{cases}
            x \text,            & \text{ha } |x| < 1    \\
            x^2 + ax + b \text, & \text{ha } |x| \geq 1
          \end{cases}
        \]

  \item Határozza meg az alábbi komplexebb határértékeket!
        \begin{multicols}{2}
          \begin{enumerate}
            \item $\displaystyle
                    \lim_{x \rightarrow 0} \frac{\ln \cos 2x}{x^2}
                  $

            \item $\displaystyle
                    \lim_{x \rightarrow \sfrac{\pi}{4}} \tan^{\tan 2x} x
                  $
          \end{enumerate}
        \end{multicols}
\end{enumerate}

% \\underline\{(\w)\}

\end{document}