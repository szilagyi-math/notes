\documentclass[a4paper, 12pt]{scrartcl}

\usepackage{math-practise}

\title{Integrálszámítás I}
\area{Integrálszámítás}
\subject{Matematika G1}
\subjectCode{BMETE94BG01}
\date{Utoljára frissítve: \today}
\docno{11}

\begin{document}
\maketitle

\subsection{Elméleti Áttekintő}

\clearpage
\subsection{Feladatok}

\begin{enumerate}
  \item Határozzuk meg az alábbi integrálok értékét!
        \begin{enumerate}
          \item $\displaystyle
                  \int x \cos x \dd x
                $

          \item $\displaystyle
                  \int (x^2 - 1) \sin 3x \dd x
                $

          \item $\displaystyle
                  \int \ln x \dd x
                $

          \item $\displaystyle
                  \int x \arctan x \dd x
                $

          \item $\displaystyle
                  \int e^x \sin x \dd x
                $

          \item $\displaystyle
                  \int \sin^2 x \dd x
                $

          \item $\displaystyle
                  \int e^{\arccos x} \dd x
                $
        \end{enumerate}

  \item Integráljuk az alábbi racionális törtfüggvényeket!
        \begin{enumerate}
          \item $\displaystyle
                  \int \frac{x^3 - 9x^2 + 27x - 26}{x^2 - 7x + 12} \dd x
                $

          \item $\displaystyle
                  \int \frac{x^3 - 2x^2 + 4}{x^3 (x - 2)^2} \dd x
                $

          \item $\displaystyle
                  \int \frac{3x - 2}{x^2 + 4x + 8} \dd x
                $
        \end{enumerate}

  \item Végezzük el az alábbi integrálokat!
        \begin{enumerate}
          \item $\displaystyle
                  \int \frac{1}{5 + 3 \cos x} \dd x
                $

          \item $\displaystyle
                  \int \sqrt{\frac{x}{1 - x}} \dd x
                $

          \item $\displaystyle
                  \int \frac{1}{\sqrt x (1 + \sqrt[3] x)}
                $

          \item $\displaystyle
                  \int \frac{e^x + 2}{e^x + e^{2x}} \dd x
                $
        \end{enumerate}
\end{enumerate}

% \\underline\{(\w)\}

\end{document}
