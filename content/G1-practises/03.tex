\documentclass[a4paper, 12pt]{scrartcl}

\usepackage{math-practise}

\title{Komplex számok}
\area{Komplex számok}
\subject{Matematika G1}
\subjectCode{BMETE94BG01}
\date{Utoljára frissítve: \today}
\docno{3}

\begin{document}
\maketitle

\subsection{Elméleti Áttekintő}

\subsubsection{A komplex számok megadási módjai}

\begin{blueBox}
  \sftitle{Algebrai alak}:

  \begin{minipage}{.6\textwidth}
    A komplex számok ($\mathbb C$) algebrai alakja:
    \[
      z = a + b\iu
      \text,
    \]
    ahol $a, b \in \Reals$ valós számok, és $\iu = \sqrt{-1}$ az
    \textbf{imaginárius egység}. A komplex szám \textbf{valós rész}e
    $\iRe \{z\} = a$, \textbf{képzetes rész}e pedig $\iIm \{z\} = b$.
  \end{minipage}\begin{minipage}{.4\textwidth}
    \flushright
    \begin{tikzpicture}[thick, scale=0.75]
      \draw[->] (-.5, 0) -- (5, 0) node[right] {$\Re$};
      \draw[->] (0, -.5) -- (0, 3) node[above] {$\Im$};

      \draw[draw=gray, dashed]
      (1.5,0) node[below] {$a$}
      -- (1.5, 2.5)
      -- (0, 2.5) node[left] {$b$};

      \draw[fill=primaryColor] (1.5, 2.5) circle (0.1)
      node[above right] {$z = a + b\iu$};
    \end{tikzpicture}
  \end{minipage}
\end{blueBox}

\begin{note}
  Két komplex szám akkor és csak akkor egyenlő, ha valós és képzetes részük is
  megegyezik. ($z_1 = z_2 \; \Leftrightarrow \; \iRe \{z_1\} = \iRe \{z_2\} \;
    \land \; \iIm \{z_1\} = \iIm\{z_2\}$)
\end{note}

\begin{blueBox}
  \begin{minipage}{.6\textwidth}
    \sftitle{Trigonometrikus alak}:\\[3mm]
    Térjünk át az eddigi Descartes-féle koordinátarendszerről a
    \textbf{polárkoordináta-rendszer}re, amelyben
    \begin{itemize}
      \item a komplex szám hossza: $r = |z| = \sqrt{a^2 + b^2}$,
      \item valós tengellyel bezárt szöge: $\varphi = \arg z$.
    \end{itemize}
    Ebből az alábbi alakot kapjuk:
    \[
      z = \rcisrad*{r}{\varphi}
      \text.
    \]
  \end{minipage}\begin{minipage}{.4\textwidth}
    \flushright
    \begin{tikzpicture}[scale=2/3]
      % Draw the lines at multiples of pi/12
      \foreach \ang in {0,...,31} {
          \draw [lightgray] (0,0) -- (\ang * 180 / 16:3.1);
        }

      % Concentric circles
      \foreach \s in {0, 1, 2} {
          \draw [lightgray] (0,0) circle (\s + 0.5);
          \draw [primaryColor] (0,0) circle (\s);
        }

      % Add the labels at multiples of pi/4
      \foreach \ang/\lab/\dir in {
      % 0/0/right,
      1/{\pi/4}/{above right},
      2/{\pi/2}/above,
      3/{3\pi/4}/{above left},
      4/{\pi}/left,
      5/{5\pi/4}/{below left},
      7/{7\pi/4}/{below right},
      6/{3\pi/2}/below} {
      \draw (0,0) -- (\ang * 180 / 4:3.5);
      \node [fill=cyan!10] at (\ang * 180 / 4:3.15) [\dir] {\scriptsize$\lab$};
      }

      % Outer circle
      \draw (0,0) circle (3);

      % 0 angle line
      \draw [-to, ultra thick] (0,0) -- (3.75,0) node [above right] {$r$};
      \node [above right] at (3,0) {\scriptsize$0$};

      % Example
      \draw[secondaryColor, ultra thick] (0,0) -- (60:2.5);
      \draw[secondaryColor, thick] (0,0) -- ++(150:.5);
      \draw[secondaryColor, thick] (60:2.5) -- ++(150:.5);

      \draw[draw=primaryColor, ultra thick, to-to] (150:.35) -- ++(60:2.5)
      node[midway, above, fill=cyan!10, rotate=60, inner sep=2pt, outer sep=2pt]
      {$r$};

      \draw[ultra thick, primaryColor, -to] (1.4,0) arc
        [start angle=0, end angle=60, radius=1.4cm];

      \node[fill=cyan!10, inner sep=2pt, outer sep=2pt] at (30:.85) {$\varphi$};

      \draw[fill=primaryColor] (60:2.5) circle (0.1);
    \end{tikzpicture}
  \end{minipage}
\end{blueBox}

\begin{blueBox}
  \sftitle{Exponeciális alak}:

  Induljunk ki a trigonometrikus alakból, és használjuk fel az alábbi
  azonosságokat:
  \[
    \cos \varphi     = \cosh \iu \varphi = \frac{e^{\iu \varphi} + e^{-\iu \varphi}}{2}
    \text{, és }
    \iu \sin \varphi = \sinh \iu \varphi = \frac{e^{\iu \varphi} - e^{-\iu \varphi}}{2}
    \text.
  \]
  A trigonometrikus alakba helyettesítve megkapjuk az exponeciális alakot:
  \[
    z
    = \rcisrad*{r}{\varphi}
    = r \left(
    \frac{e^{\iu \varphi} + e^{-\iu \varphi}}{2}
    + \frac{e^{\iu \varphi} - e^{-\iu \varphi}}{2}
    \right)
    = r e^{\iu \varphi}
    \text.
  \]
\end{blueBox}

\subsection{Műveletek komplex számokkal}

\begin{blueBox}
  \begin{minipage}{.6\textwidth}
    \sftitle{Konjugált}:
    \\[3mm]
    Egy $z = a + b \iu$ komplex szám konjugáltját úgy kapjuk meg, hogy
    tükrözzük a valós tengelyre, vagyis
    \[
      \overline z
      = \overline{a + b \iu}
      = a - b\iu
      \text.
    \]
  \end{minipage}\begin{minipage}{.4\textwidth}
    \centering
    \begin{tikzpicture}[ultra thick]
      \draw[-to, draw=ternaryColor, thick] (-1, 0) -- (3, 0) node [below left] {$\Re$};
      \draw[-to, draw=ternaryColor, thick] (0, -1.5) -- (0, 1.5) node [below left] {$\Im$};

      \draw[-to, draw=secondaryColor] (0,0) -- (+30:2.5) coordinate (A)
      node[above, pos=.7] {$z$};
      \draw[-to, draw=primaryColor]   (0,0) -- (-30:2.5) coordinate (B)
      node[below, pos=.7] {$\overline z$};

      \coordinate (O) at (0,0);
      \coordinate (X) at (1,0);

      \draw pic[to-to, "$\varphi$", draw=ternaryColor, angle eccentricity=0.7, angle radius=1.2cm, thick]
        {angle=X--O--A};
      \draw pic[to-to, "$\varphi$", draw=ternaryColor, angle eccentricity=0.7, angle radius=1.2cm, thick]
        {angle=B--O--X};
    \end{tikzpicture}
  \end{minipage}
\end{blueBox}

\begin{blueBox}
  \sftitle{Inverz}:

  \begin{minipage}{.6\textwidth}
    Egy $z = a + b \iu$ komplex szám inverze:
    \[
      z^{-1}
      = \frac{1}{z}
      = \frac{\overline z}{z \cdot \overline z}
      = \frac{a - b\iu}{a^2 + b^2}
      \text.
    \]
    Komplex szám inverzének hossza az eredeti szám hosszának a reciproka.
  \end{minipage}\begin{minipage}{.4\textwidth}
    \centering
    \begin{tikzpicture}[ultra thick]
      \draw[-to, draw=ternaryColor, thick] (-1, 0) -- (3, 0) node [below left] {$\Re$};
      \draw[-to, draw=ternaryColor, thick] (0, -1.5) -- (0, 1.5) node [below left] {$\Im$};

      \draw[-to, draw=secondaryColor] (0,0) -- (+30:2.5) coordinate (A)
      node[above, pos=.7] {$z$};
      \draw[-to, draw=primaryColor]   (0,0) -- (-30:1.5) coordinate (B)
      node[below, pos=.7] {$z^{-1}\phantom{z}$};

      \coordinate (O) at (0,0);
      \coordinate (X) at (1,0);

      \draw pic[to-to, "$\varphi$", draw=ternaryColor, angle eccentricity=0.7, angle radius=1.2cm, thick]
        {angle=X--O--A};
      \draw pic[to-to, "$\varphi$", draw=ternaryColor, angle eccentricity=0.7, angle radius=1.2cm, thick]
        {angle=B--O--X};
    \end{tikzpicture}
  \end{minipage}
\end{blueBox}

\begin{blueBox}
  \begin{minipage}{.6\textwidth}
    \sftitle{Összeadás és kivonás}:
    \\[3mm]
    Algebrai alakban, a vektorműveletekhez hasonlóan:
    \[
      z_1 + z_2 = (a_1 + a_2) + \iu (b_1 + b_2)
      \text.
    \]
  \end{minipage}\begin{minipage}{.4\textwidth}
    \centering
    \begin{tikzpicture}[scale=1/2, ultra thick, rotate=-20]
      \node at(-2.5,0) {};

      % vectors
      \draw[draw=ternaryColor, ->]
      (0,0) -- (2,3) node[midway,anchor=east]{$z_1$};
      \draw[draw=ternaryColor, ->]
      (0,0) -- (5,1) node[midway,anchor=north]{$z_2$};

      % helper lines
      \draw[dashed, draw=gray, thick]
      (2,3) -- (7,4)
      (5,1) -- (7,4)
      ;

      % sum
      \draw[primaryColor, ->]
      (0,0) -- (7,4) node[anchor=south]{$z_1 + z_2$};

      % difference
      \draw[secondaryColor, ->]
      (5,1) -- (2,3) node[anchor=south]{$z_1 - z_2$};
    \end{tikzpicture}
  \end{minipage}
\end{blueBox}

\begin{blueBox}
  \begin{minipage}{.6\textwidth}
    \sftitle{Szorzás}:
    \\[3mm]
    Algebrai alakban:
    \begin{align*}
      z_1 \cdot z_2
      %  & = (a_1 + b_1\iu) \cdot (a_2 + b_2\iu)              \\
      %  & = a_1 a_2 + a_1 b_2\iu + a_2 b_1\iu + b_1 b_2\iu^2 \\
       & = (a_1 a_2 - b_1 b_2) + (a_1 b_2 + a_2 b_1)\iu
    \end{align*}

    Trigonometrikus alakban:
    \[
      z_1 \cdot z_2 = r_1 r_2 \left(
      \cos(\varphi_1 + \varphi_2) + \iu \sin(\varphi_1 + \varphi_2)
      \right)
    \]

    Exponenciális alakban:
    \[
      z_1 \cdot z_2 = r_1 r_2 e^{\iu (\varphi_1 + \varphi_2)}
    \]
  \end{minipage}\begin{minipage}{.4\textwidth}
    \centering
    \begin{tikzpicture}[ultra thick]
      \draw[-to, draw=ternaryColor, thick] (-1, 0) -- (3, 0) node [below left] {$\Re$};
      \draw[-to, draw=ternaryColor, thick] (0, -1) -- (0,4.5) node [below left] {$\Im$};

      \draw[-to, draw=secondaryColor] (0,0) -- (15:1.5) coordinate (A)
      node[right] {$z_1$};
      \draw[-to, draw=secondaryColor] (0,0) -- (45:3) coordinate (B)
      node[right] {$z_2$};

      \draw[-to, draw=primaryColor] (0,0) -- (60:4.5) coordinate (A)
      node[right] {$z_1 \cdot z_2$};
    \end{tikzpicture}
  \end{minipage}
\end{blueBox}

\begin{blueBox}
  \begin{minipage}{.6\textwidth}
    \sftitle{Osztás}:
    \\[3mm]
    Trigonometrikus alakban:
    \[
      \frac{z_1}{z_2} = \frac{r_1}{r_2} \left(
      \cos(\varphi_1 - \varphi_2) + \iu \sin(\varphi_1 - \varphi_2)
      \right)
    \]

    Exponenciális alakban:
    \[
      \frac{z_1}{z_2} = \frac{r_1}{r_2} e^{\iu (\varphi_1 - \varphi_2)}
    \]
  \end{minipage}\begin{minipage}{.4\textwidth}
    \centering
    \begin{tikzpicture}[ultra thick]
      \draw[-to, draw=ternaryColor, thick] (-1, 0) -- (3, 0) node [below left] {$\Re$};
      \draw[-to, draw=ternaryColor, thick] (0, -1) -- (0,3.5) node [below left] {$\Im$};

      \draw[-to, draw=secondaryColor] (0,0) -- (60:3.5) coordinate (A)
      node[right] {$z_1$};
      \draw[-to, draw=secondaryColor] (0,0) -- (45:2.5) coordinate (B)
      node[right] {$z_2$};

      \draw[-to, draw=primaryColor] (0,0) -- (15:1.4) coordinate (A)
      node[right] {$z_1 / z_2$};
    \end{tikzpicture}
  \end{minipage}
\end{blueBox}

\begin{blueBox}
  \begin{minipage}{.6\textwidth}
    \sftitle{Hatványozás}:
    \\[3mm]
    Trigonometrikus alakban:
    \[
      z^n = \rcisrad*{r^n}{n\varphi}
    \]
    Exponenciális alakban:
    \[
      z^n = r^n e^{\iu \cdot n \varphi}
    \]
  \end{minipage}\begin{minipage}{.4\textwidth}
    \centering
    \begin{tikzpicture}[ultra thick]
      \draw[-to, draw=ternaryColor, thick] (-1.5, 0) -- (2.5, 0) node [below left] {$\Re$};
      \draw[-to, draw=ternaryColor, thick] (0, -1) -- (0,2.75) node [right] {$\Im$};

      \draw[-to, draw=secondaryColor] (0,0) -- (20:1.25)
      node[right] {$z$};

      \draw[-to, draw=primaryColor] (0,0) -- (50:1.5625)
      node[right] {$z^2$};
      \draw[-to, draw=primaryColor] (0,0) -- (75:1.953125)
      node[right] {$z^3$};
      \draw[-to, draw=primaryColor] (0,0) -- (100:2.44140625)
      node[left] {$z^4$};
      \draw[-to, draw=primaryColor] (0,0) -- (125:3.05175781)
      node[left] {$z^5$};
    \end{tikzpicture}
  \end{minipage}
\end{blueBox}

\begin{note}
  Ha egy komplex számot az $n$-edik hatványra emelünk, akkor
  \begin{itemize}
    \item hossza az $n$-szeresére nő,
    \item argumentuma is az $n$-szeresére nő.
  \end{itemize}
\end{note}

\begin{blueBox}
  \sftitle{Gyökvonás}:

  Trigonometrikus alakban:
  \[
    \sqrt[n]{z} = \rcisrad{\sqrt[n]{r}}{
      \frac{\varphi + 2k\pi}{n}
    }
    \text{, ahol }
    k \in \{0; 1; \ldots; n-1\}
  \]

  Exponenciális alakban:
  \[
    \sqrt[n]{z} = \sqrt[n]{r} e^{
        \iu \cdot \frac{\varphi + 2k\pi}{n}
      }
    \text{, ahol }
    k \in \{0; 1; \ldots; n-1\}
  \]
\end{blueBox}

\begin{note}
  Tetszőleges komplex szám $n$-edik gyökei egy olyan szabályos $n$-szög csúcsai,
  amelynek középpontja az origó.
  \begin{center}
    \begin{tikzpicture}[ultra thick, scale=.9]
      % COORDINATE SYSTEM
      \draw[-to, draw=primaryColor] (-2, 0) -- (2, 0) node [right] {$\Re$};
      \draw[-to, draw=primaryColor] (0, -2) -- (0, 2) node [above] {$\Im$};

      \coordinate (P) at (318:1.5);
      \foreach \deg in {30,102,174,246,318} {
          \draw[draw=primaryColor] (\deg:1.5) coordinate(C) circle (0.1);
          \draw[dashed, gray] (P) -- (C);
          \coordinate (P) at (C);
        }
      \draw[dashed, gray] (0,0) circle (1.5);
      \node[below] at (0,-2) {$\sqrt[5]{z}$};

      \begin{scope}[xshift=-6cm]
        % COORDINATE SYSTEM
        \draw[-to, draw=primaryColor] (-2, 0) -- (2, 0) node [right] {$\Re$};
        \draw[-to, draw=primaryColor] (0, -2) -- (0, 2) node [above] {$\Im$};

        \coordinate (P) at (300:1.5);
        \foreach \deg in {30,120,210,300} {
            \draw[draw=primaryColor] (\deg:1.5) coordinate(C) circle (0.1);
            \draw[dashed, gray] (P) -- (C);
            \coordinate (P) at (C);
          }

        \draw[dashed, gray] (0,0) circle (1.5);
        \node[below] at (0,-2) {$\sqrt[4]{z}$};
      \end{scope}

      \begin{scope}[xshift=-12cm]
        \draw[-to, draw=primaryColor] (-2, 0) -- (2, 0) node [right] {$\Re$};
        \draw[-to, draw=primaryColor] (0, -2) -- (0, 2) node [above] {$\Im$};

        \coordinate (P) at (315:1.5);
        \foreach \deg in {75,195,315} {
            \draw[draw=primaryColor] (\deg:1.5) coordinate(C) circle (0.1);
            \draw[dashed, gray] (P) -- (C);
            \coordinate (P) at (C);
          }

        \draw[dashed, gray] (0,0) circle (1.5);
        \node[below] at (0,-2) {$\sqrt[3]{z}$};
      \end{scope}
    \end{tikzpicture}
  \end{center}
\end{note}

\clearpage
\subsection{Feladatok}

\begin{enumerate}
  % 1
  \item Hozza egyszerűbb alakra az alábbi kifejezéseket!
        \begin{enumerate}
          \item $\overline{\left(\dfrac{
                      2 - \iu
                    }{
                      e^{\iu \sfrac{\pi}{3}}
                    }\right)}$
          \item $\dfrac{5 + \iu}{3 - 2i} \cdot \overline{
                    \rcisdeg*{3}{45} \cdot
                    \rcisdeg*{2}{270} \cdot
                    e^{\iu \sfrac{5\pi}{12}}
                  }$

          \item $\dfrac{
                    5 e^{\iu \sfrac{7\pi}{13}}
                  }{
                    \rcisdeg*{4}{135}
                  } \cdot \overline{\left(
                    \dfrac{1}{2\iu}
                    \right)} \cdot \left(
                  2 \sqrt{3} + 2 \iu
                  \right)
                $
        \end{enumerate}

        % 2
  \item Végezze el az alábbi hatványozásokat!
        \begin{enumerate}
          \item $(\iu - 1)^{16}$
          \item $(3 + 5\iu)^4 \cdot
                  (21 - 35\iu)^5 \cdot
                  \left(\dfrac{1 + \iu}{1 - \iu}\right)^4$
        \end{enumerate}

        % 3
  \item Végezze el az alábbi gyökvonásokat!
        \begin{enumerate}
          \item $\sqrt[3]{-8}$
          \item $\sqrt[4]{1}$
          \item $\sqrt{3 + 4\iu}$
        \end{enumerate}

        % 4
  \item Oldja meg a következő egyenleteket!
        \begin{enumerate}
          \item $z^4 - 81 \iu = 0$
          \item $z^2 - 6z + 13 = 0$
        \end{enumerate}

        % 5
  \item Oldja meg az alábbi egyenletrendszert!\\[2mm]
        \hspace{2ex}$
          \begin{cases}
            z_1 + 2z_2 = 1 + \iu    \\
            3z_1 + \iu z_2 = 2 - 3i \\
          \end{cases}
        $

        % 6
  \item Adja meg a geometriai helyét azoknak a komplex számoknak, amelyekre
        \dots
        \begin{enumerate}
          \item $\iIm \{ z \} > 0$,
          \item $|z - a| = |z - b|$, ahol $a, b \in \mathbb C$,
          \item $|z| < 1 - \iRe \{ z \}$.
        \end{enumerate}

        % 7
  \item Egy négyzet két szomszédos csúcsát jelölje a $z_1 = 3 + 2 \iu$ és a
        $z_2 = 5 + 4 \iu$ komplex szám. Hol található a többi csúcs?

        % 8
  \item Írja fel a $(-2; 1)$ középpontú, $4$ sugarú kör egyenletét a komplex
        számsíkon!
\end{enumerate}

% \\underline\{(\w)\}

\end{document}