\documentclass[a4paper, 12pt]{scrartcl}

\usepackage{math-practise}

\title{Numerikus sorozatok II}
\area{Sorozatok}
\subject{Matematika G1}
\subjectCode{BMETE94BG01}
\date{Utoljára frissítve: \today}
\docno{5}

\begin{document}
\maketitle

\subsection{Elméleti Áttekintő}

\clearpage
\subsection{Feladatok}

\begin{enumerate}
  % 1
  \item A konvergencia definíciója alapján bizonyítsuk be, hogy az alábbi
        számsorozatok konvergensek, úgy hogy minden $\varepsilon$ esetén adjunk
        meg egy $N$ köszöbszámort, amelytől kezdve minden $n > N$ esetén
        $|a_n - a| < \varepsilon$, ahol $a$ az adott sorozat határértéke!
        \begin{enumerate}
          \item $\displaystyle
                  a_n = \frac{2n + 5}{n - 1}
                  \text,\quad
                  \varepsilon = 10^{-6}
                $

          \item $\displaystyle
                  b_n = \frac{2n + 3 \sqrt{n}}{3n + 1}
                  \text,\quad
                  \varepsilon = 10^{-3}
                $
        \end{enumerate}

        % 2
  \item Határozzuk meg az alábbi sorozatok torlódási pontjainak halmazát!
        \begin{enumerate}
          \item $\displaystyle
                  a_n = \frac{\left(
                    \sqrt{n^2 + 1} + n
                    \right)^2}{
                    \sqrt[3]{n^6 + 1}
                  } \cdot \cos\left(
                  \frac{n\pi}{3}
                  \right)
                $

          \item $\displaystyle
                  b_n = \frac{(-1)^n \cdot n + 2n}{3n} \cdot \sin\left(
                  \frac{2n\pi}{3}
                  \right)
                $
        \end{enumerate}

        % 3
  \item Vizsgáljuk meg monotonitás és korlátosság szempontjából az alábbi
        sorozatokat!
        \begin{enumerate}
          \item $\displaystyle
                  a_n = \frac{2n^2 + 1}{n^2 - n + 1}
                $

          \item $\displaystyle
                  b_n = \frac{n^5}{n!}
                $
        \end{enumerate}

  \item Adjuk meg azon sorozat határértékét, melynek első eleme $a_1 = 1$,
        $n$-edik eleme pedig $a_n = a_{n - 1} + \sfrac{1}{2^{n - 1}}$.

  \item Vizsgáljuk meg azt a sorozatot konvergencia, monotonitás és korlátosság
        szempontjából, amelynek első eleme $a_1 = 1$, $n$-edik eleme pedig
        $a_n = \sqrt{1 + a_{n - 1}}$.

  \item Határozzuk meg az alábbi komplex elemű sorozatok határértékeit!
        \begin{enumerate}
          \item $\displaystyle
                  a_n = \frac{n^2 - \iu(n^2 - 1)}{n^2 - \iu^n}
                $

          \item $\displaystyle
                  b_n = \frac{\iu^n}{3^n + \iu^n}
                $

          \item $\displaystyle
                  c_n = (1 - \iu)^n
                $
        \end{enumerate}

  \item Bizonyítsuk be, hogy ha $n \geq 3$, akkor $n^{n + 1} \geq (n + 1)^n$!

  \item Vizsgáljuk az $a_n = \scalebox{1.25}{
        $\sfrac{\binom{n}{2}}{\binom{n}{3}}$
          }$ sorozatot monotonitás és korlátosság szempontjából!
\end{enumerate}

% \\underline\{(\w)\}

\end{document}