\documentclass[a4paper, 12pt]{scrartcl}

\usepackage{math-practise}

\title{Integrálszámítás II}
\area{Integrálszámítás}
\subject{Matematika G1}
\subjectCode{BMETE94BG01}
\date{Utoljára frissítve: \today}
\docno{12}

\begin{document}
\maketitle

\subsection{Elméleti Áttekintő}

\clearpage
\subsection{Feladatok}

\begin{enumerate}
  \item Határozzuk meg az alábbi trigonometrikus integrálok értékét!
        \begin{multicols}{3}
          \begin{enumerate}
            \item $\displaystyle
                    \int \cos^3 x \sin x \dd x
                  $

            \item $\displaystyle
                    \int \cos^5 x \dd x
                  $

            \item $\displaystyle
                    \int \sin^4 x \cos^2 x \dd x
                  $
          \end{enumerate}
        \end{multicols}

  \item Oldjuk meg az alábbi összetett integrálási feladatokat!
        \begin{enumerate}
          \item $\displaystyle
                  \int \sin \sqrt x \dd x
                $

          \item $\displaystyle
                  \int \frac{\ln \ln x}{x} \dd x
                $

          \item $\displaystyle
                  \int |x| \dd x
                $

          \item $\displaystyle
                  \int \frac{\ln x + 1}{x^x - 1} \dd x
                $

          \item $\displaystyle
                  \int (x^2 - 3x + 2) \sqrt{2x - 1} \dd x
                $
        \end{enumerate}

  \item Határozzuk meg az alábbi határozott integrálok értékét!
        \begin{enumerate}
          \item $\displaystyle
                  \int_0^{2\pi} \cos x \dd x
                $

          \item $\displaystyle
                  \int_0^1 x \sinh x \dd x
                $

          \item $\displaystyle
                  \int_{-3}^3 \sqrt{9 - x^2} \dd x
                $
        \end{enumerate}

  \item Határozzuk meg az $f(x) = (x + 1) x  (x - 2)$ függvény és az $x$-tengely
        által bezárt geometriai területet!

  \item Adjuk meg az $f(x) = x^4$ és a $g(x) = 3x^2 - 2$ függvények által bezárt
        terület nagyságát!

  \item Adjuk meg egy $a$ sugarú körvonal ($x(t) = a \cos t$, $y(t) = a \sin t$)
        alapján a $t \in [0; 2\pi]$ intervallumhoz tartozó görbevonalú trapéz
        területét!

\end{enumerate}

% \\underline\{(\w)\}

\end{document}
